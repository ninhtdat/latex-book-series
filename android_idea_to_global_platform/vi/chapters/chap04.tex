\chapter{Các phiên bản Android đầu tiên và giai đoạn định hình}

Sự ra đời của Android gắn liền với tham vọng xây dựng một hệ điều hành di động mở, linh hoạt và có khả năng mở rộng trên nhiều loại thiết bị khác nhau. Không giống các nền tảng di động đương thời vốn kiểm soát chặt chẽ phần cứng lẫn phần mềm, Android ngay từ đầu đã được định hướng như một nền tảng chung cho nhiều nhà sản xuất. Giai đoạn từ Android 1.0 đến Android 2.x vì vậy mang tính chất đặt nền móng: các quyết định kỹ thuật được đưa ra trong thời kỳ này chưa hoàn thiện về mặt trải nghiệm, nhưng lại có ảnh hưởng lâu dài đến cấu trúc hệ thống, hệ sinh thái ứng dụng và cách Android phát triển trong hơn một thập kỷ sau đó.

\section{Android 1.0–1.1: các tính năng cơ bản, triết lý thiết kế ban đầu và những thiếu sót rõ rệt}

Android 1.0 chính thức ra mắt năm 2008, đánh dấu lần đầu tiên Google tham gia trực tiếp vào thị trường hệ điều hành di động. Ở phiên bản này, Android tập trung cung cấp những chức năng thiết yếu nhất cho một thiết bị thông minh: gọi điện, nhắn tin SMS/MMS, trình duyệt web dựa trên WebKit, ứng dụng email, danh bạ, lịch và đặc biệt là các dịch vụ trực tuyến như Gmail và Google Maps. Đây là những thành phần tối thiểu để Android có thể được xem là một nền tảng smartphone đúng nghĩa, thay vì chỉ là một hệ điều hành thử nghiệm.

Triết lý thiết kế ban đầu của Android 1.0 thể hiện khá rõ qua cấu trúc hệ thống. Thứ nhất, Android được xây dựng dựa trên nhân Linux, tận dụng khả năng quản lý tiến trình, bộ nhớ và bảo mật ở mức hệ điều hành. Điều này cho phép Android hỗ trợ đa nhiệm thực sự, tức nhiều ứng dụng có thể chạy song song ở nền, thay vì bị tạm dừng hoàn toàn. Thứ hai, Google định hướng Android như một nền tảng dịch vụ, trong đó tài khoản Google đóng vai trò trung tâm cho việc đồng bộ email, danh bạ và dữ liệu người dùng. Cách tiếp cận này khác biệt rõ rệt so với nhiều hệ điều hành di động cùng thời vốn còn thiên về sử dụng cục bộ.

Một yếu tố quan trọng khác là tính mở. Android 1.0 được phát hành kèm bộ công cụ phát triển phần mềm (SDK), cho phép lập trình viên bên thứ ba xây dựng ứng dụng và phân phối thông qua Android Market. Dù Android Market lúc này còn rất sơ khai, số lượng ứng dụng ít và cơ chế kiểm soát chưa rõ ràng, nhưng việc cho phép phát triển ứng dụng ngay từ phiên bản đầu tiên đã đặt nền móng cho một hệ sinh thái mở, nơi phần mềm không bị giới hạn bởi một nhà cung cấp duy nhất.

Tuy nhiên, những hạn chế của Android 1.0 cũng rất rõ rệt. Giao diện người dùng mang tính chức năng nhiều hơn thẩm mỹ, thiếu sự nhất quán trong thiết kế và chưa có ngôn ngữ giao diện rõ ràng. Hiệu năng hệ thống phụ thuộc nặng vào phần cứng, trong khi các thiết bị đời đầu thường có cấu hình thấp, dẫn đến trải nghiệm chậm và thiếu mượt mà. Ngoài ra, API của Android 1.0 còn nghèo nàn, nhiều thành phần hệ thống chưa cho phép lập trình viên truy cập sâu, khiến việc phát triển các ứng dụng phức tạp gặp khó khăn.

Android 1.1 được phát hành không lâu sau đó, chủ yếu nhằm sửa lỗi và cải thiện độ ổn định. Phiên bản này không mang lại thay đổi lớn về tính năng hay giao diện, nhưng cho thấy cách Google tiếp cận phát triển Android: phát hành nhanh, sửa sai sớm và cải tiến liên tục thay vì chờ đến khi hệ thống đạt độ hoàn thiện cao. Đây là một lựa chọn mang tính chiến lược, chấp nhận những thiếu sót ban đầu để nhanh chóng đưa Android ra thị trường và thu thập phản hồi thực tế.

Nhìn chung, Android 1.0–1.1 có thể xem là giai đoạn thử nghiệm công khai. Các phiên bản này chưa tạo được trải nghiệm người dùng hấp dẫn, nhưng đã xác lập những nguyên tắc cốt lõi: nền tảng mở, tích hợp chặt chẽ dịch vụ trực tuyến, đa nhiệm và khả năng mở rộng. Chính những nguyên tắc này trở thành cơ sở để Android tiếp tục được cải tiến mạnh mẽ ở các phiên bản sau.

\section{Android 1.5–1.6: cải tiến UI, hỗ trợ widget và mở rộng khả năng cho nhà phát triển}

Sau giai đoạn thử nghiệm ban đầu với Android 1.0–1.1, Google bắt đầu chuyển trọng tâm từ việc “chứng minh Android có thể hoạt động” sang việc “khiến Android trở nên hữu dụng và hấp dẫn hơn”. Android 1.5 (Cupcake) và Android 1.6 (Donut) vì vậy được xem là các phiên bản đặt nền móng thực tế cho trải nghiệm người dùng và cho cộng đồng phát triển ứng dụng.

Android 1.5 đánh dấu nhiều thay đổi mang tính bản lề. Trước hết là sự xuất hiện của bàn phím ảo trên màn hình cảm ứng. Ở các phiên bản trước, Android phụ thuộc nhiều vào bàn phím cứng, điều này giới hạn đáng kể thiết kế phần cứng của thiết bị. Việc hỗ trợ bàn phím ảo không chỉ cải thiện trải nghiệm nhập liệu mà còn mở đường cho các mẫu smartphone mỏng, gọn, ít phím vật lý – một xu hướng trở thành tiêu chuẩn trong những năm sau đó.

Một cải tiến quan trọng khác là cơ chế widget trên màn hình chính. Lần đầu tiên, người dùng có thể đặt các thành phần hiển thị động như đồng hồ, lịch, thời tiết hoặc điều khiển nhạc trực tiếp lên màn hình chính mà không cần mở ứng dụng. Widget nhanh chóng trở thành đặc trưng nhận diện của Android, giúp hệ điều hành này khác biệt so với các nền tảng cạnh tranh vốn còn hạn chế khả năng tùy biến giao diện.

Về phía lập trình viên, Android 1.5 mở rộng đáng kể bộ API. Các ứng dụng bắt đầu có quyền truy cập tốt hơn vào tài nguyên hệ thống, vòng đời ứng dụng được định nghĩa rõ ràng hơn, và framework đồ họa được cải thiện để hỗ trợ giao diện linh hoạt. Những thay đổi này giúp việc phát triển ứng dụng ổn định và có cấu trúc hơn, giảm bớt tính thử nghiệm vốn tồn tại ở Android 1.0–1.1.

Android 1.6 tiếp tục xu hướng hoàn thiện nền tảng thay vì bổ sung tính năng hào nhoáng. Điểm nổi bật nhất của Donut là khả năng hỗ trợ nhiều độ phân giải và kích thước màn hình khác nhau. Trước đó, Android gần như chỉ nhắm đến một cấu hình màn hình nhất định. Việc mở rộng hỗ trợ này cho phép Android chạy trên nhiều loại thiết bị hơn, từ smartphone giá rẻ đến các mẫu có màn hình lớn. Đây là bước đi chiến lược nhằm thu hút các nhà sản xuất phần cứng, nhưng đồng thời cũng đặt nền móng cho vấn đề phân mảnh sau này.

Ngoài ra, Android 1.6 cải thiện hiệu năng tìm kiếm, tích hợp sâu hơn các chức năng tìm kiếm hệ thống và tối ưu Android Market để hỗ trợ nhiều thiết bị hơn. Trải nghiệm người dùng tuy chưa thực sự mượt mà, nhưng đã có sự nhất quán hơn so với các phiên bản đầu tiên. Các thao tác cơ bản trở nên rõ ràng, ít lỗi hơn và phù hợp hơn với người dùng phổ thông.

Xét tổng thể, Android 1.5–1.6 là giai đoạn Android bắt đầu “định hình bản sắc”. Hệ điều hành không còn chỉ là một thử nghiệm kỹ thuật, mà dần trở thành một nền tảng có giá trị thực tế cho cả người dùng lẫn lập trình viên. Những quyết định như hỗ trợ widget, bàn phím ảo và đa dạng phần cứng cho thấy Google sẵn sàng đánh đổi sự đồng nhất để đạt được khả năng mở rộng nhanh. Đây chính là tiền đề để Android bước sang giai đoạn phát triển mạnh mẽ hơn ở các phiên bản 2.x sau đó.

\section{Android 2.0–2.1: tối ưu hiệu năng, hỗ trợ màn hình lớn hơn và tích hợp sâu dịch vụ Google}

Android 2.0 và 2.1, thường được gọi chung với tên mã Eclair, đánh dấu bước chuyển quan trọng của Android từ một nền tảng đang hoàn thiện sang một hệ điều hành có khả năng cạnh tranh thực sự trên thị trường smartphone. Nếu Android 1.5–1.6 tập trung vào việc mở rộng khả năng và định hình bản sắc, thì Android 2.0–2.1 hướng đến việc mở rộng quy mô sử dụng và nâng cao mức độ hoàn thiện tổng thể của hệ thống.

Một trong những thay đổi quan trọng nhất của Android 2.x là khả năng hỗ trợ tốt hơn cho các thiết bị có màn hình lớn và độ phân giải cao. Google bắt đầu chú trọng đến việc tách biệt rõ ràng giữa tài nguyên giao diện và logic ứng dụng, khuyến khích lập trình viên thiết kế ứng dụng có khả năng thích ứng với nhiều kích thước màn hình. Điều này phản ánh chiến lược dài hạn của Android: không bị giới hạn trong một kiểu thiết bị duy nhất, mà sẵn sàng mở rộng sang nhiều phân khúc phần cứng khác nhau.

Về giao diện người dùng, Android 2.0 mang đến những tinh chỉnh đáng kể. Các thành phần giao diện được sắp xếp hợp lý hơn, biểu tượng và typography rõ ràng hơn, tạo cảm giác hiện đại và nhất quán hơn so với các phiên bản trước. Thanh thông báo và hệ thống menu được cải thiện, giúp người dùng truy cập thông tin và thao tác nhanh hơn. Dù chưa đạt đến mức hoàn thiện cao về mặt thẩm mỹ, Android 2.x đã thu hẹp đáng kể khoảng cách về trải nghiệm người dùng so với các đối thủ lớn trên thị trường.

Một điểm nhấn khác của Android 2.0–2.1 là sự tích hợp ngày càng sâu các dịch vụ của Google vào hệ điều hành. Gmail, Google Maps và hệ thống đồng bộ tài khoản được nâng cấp mạnh mẽ, cho phép người dùng quản lý nhiều tài khoản, đồng bộ dữ liệu ổn định hơn và sử dụng các dịch vụ định vị, dẫn đường trực tiếp trên thiết bị. Google Maps Navigation, xuất hiện trong giai đoạn này, là ví dụ điển hình cho chiến lược sử dụng phần mềm và dịch vụ để tạo lợi thế cạnh tranh cho Android.

Về mặt kỹ thuật, Android 2.x tiếp tục cải thiện hiệu năng và độ ổn định. Quản lý bộ nhớ được tối ưu hơn, các tiến trình nền được kiểm soát chặt chẽ hơn để giảm tình trạng ứng dụng bị dừng đột ngột. API cũng được mở rộng, cung cấp thêm công cụ cho lập trình viên phát triển ứng dụng có giao diện phức tạp và khả năng tương tác cao hơn. Đây là giai đoạn mà Android bắt đầu trở nên “thân thiện với lập trình viên” theo nghĩa thực tế, chứ không chỉ dừng ở mức cho phép phát triển ứng dụng.

Nhìn chung, Android 2.0–2.1 có vai trò như một cầu nối. Nó kế thừa các nền tảng kỹ thuật từ giai đoạn trước, đồng thời chuẩn bị cho bước nhảy vọt về hiệu năng và trải nghiệm ở các phiên bản sau. Android trong giai đoạn này đã chứng minh rằng mô hình hệ điều hành mở, hỗ trợ đa dạng phần cứng và tích hợp chặt chẽ dịch vụ trực tuyến không chỉ khả thi về mặt kỹ thuật, mà còn có tiềm năng mở rộng trên quy mô toàn cầu.

\section{Android 2.2–2.3: cải thiện tốc độ, quản lý bộ nhớ và trải nghiệm người dùng}

Android 2.2 (Froyo) và Android 2.3 (Gingerbread) được xem là giai đoạn Android chuyển từ “có thể sử dụng” sang “sử dụng tốt trong thực tế”. Trọng tâm phát triển không còn nằm ở việc bổ sung tính năng mới cho bằng được, mà tập trung mạnh vào hiệu năng, độ ổn định và cảm giác sử dụng hằng ngày. Đây là bước đi cần thiết khi Android bắt đầu được triển khai rộng rãi trên nhiều dòng thiết bị với cấu hình phần cứng rất khác nhau.

Cải tiến kỹ thuật quan trọng nhất của Android 2.2 là việc giới thiệu trình biên dịch Just-In-Time (JIT) cho máy ảo Dalvik. Thay vì chỉ thông dịch bytecode như trước, JIT cho phép biên dịch động các đoạn mã được sử dụng thường xuyên sang mã máy, giúp tăng đáng kể tốc độ thực thi ứng dụng. Trong thực tế, nhiều ứng dụng trên Froyo chạy nhanh hơn rõ rệt so với Android 2.1, đặc biệt là các ứng dụng nặng về xử lý logic và tính toán. Đây là lần đầu tiên người dùng phổ thông có thể cảm nhận rõ ràng sự cải thiện về hiệu năng giữa các phiên bản Android.

Song song với đó, Android 2.2 cũng cải thiện cơ chế quản lý bộ nhớ và tiến trình. Hệ thống kiểm soát tốt hơn các ứng dụng chạy nền, giảm tình trạng chiếm dụng tài nguyên quá mức và hạn chế hiện tượng treo máy. Những thay đổi này tuy không luôn dễ nhận thấy về mặt giao diện, nhưng có tác động trực tiếp đến độ ổn định và thời lượng pin – hai yếu tố then chốt đối với trải nghiệm smartphone.

Android 2.3 tiếp tục hướng tối ưu này nhưng với phạm vi rộng hơn. Giao diện người dùng được tinh chỉnh để đơn giản và nhất quán hơn, sử dụng tông màu tối nhằm giảm tiêu thụ năng lượng trên các màn hình phổ biến thời điểm đó. Các thao tác cảm ứng được cải thiện độ chính xác và độ phản hồi, giúp hệ thống cho cảm giác “nhanh” hơn ngay cả trên phần cứng không quá mạnh. Android 2.3 cũng tối ưu cho các thiết bị có độ phân giải cao hơn và tỷ lệ màn hình đa dạng hơn, phản ánh xu hướng phát triển nhanh của phần cứng smartphone.

Về khả năng hỗ trợ phần cứng, Android 2.3 mở rộng đáng kể các API liên quan đến cảm biến, âm thanh và đồ họa. Điều này đặc biệt quan trọng đối với game và các ứng dụng đa phương tiện, vốn đang bắt đầu phát triển mạnh trong giai đoạn này. Android từ đây không chỉ phục vụ nhu cầu liên lạc và thông tin cơ bản, mà dần trở thành một nền tảng giải trí và ứng dụng phong phú.

Xét tổng thể, Android 2.2–2.3 là giai đoạn “củng cố”. Google tập trung trả lời những phàn nàn thực tế từ người dùng và nhà sản xuất: hệ thống phải nhanh hơn, ổn định hơn và dễ sử dụng hơn. Thành công của các phiên bản này giúp Android xây dựng được niềm tin về mặt kỹ thuật, tạo tiền đề cho những thay đổi lớn hơn về giao diện và kiến trúc hệ thống trong các thế hệ Android tiếp theo.

\section{Đánh giá giai đoạn định hình: cách Google phản hồi thị trường và điều chỉnh chiến lược kỹ thuật}

Giai đoạn từ Android 1.0 đến Android 2.x có thể xem là thời kỳ định hình toàn diện cả về kỹ thuật lẫn chiến lược phát triển. Thay vì theo đuổi sự hoàn thiện ngay từ đầu, Google lựa chọn cách tiếp cận mang tính thử nghiệm và thích nghi nhanh với thị trường. Các phiên bản Android đầu tiên liên tục được phát hành trong thời gian ngắn, mỗi bản đều phản ánh trực tiếp những phản hồi từ người dùng, lập trình viên và đối tác phần cứng.

Về mặt kỹ thuật, Google sớm nhận ra rằng hiệu năng và độ ổn định là điều kiện tiên quyết để Android có thể tồn tại trên các thiết bị phần cứng đa dạng. Việc tập trung tối ưu máy ảo Dalvik, quản lý bộ nhớ và tiến trình trong Android 2.2–2.3 cho thấy Google sẵn sàng điều chỉnh ưu tiên phát triển khi các hạn chế cốt lõi bắt đầu ảnh hưởng đến trải nghiệm thực tế. Thay vì chỉ bổ sung tính năng mới, các bản cập nhật sau này chú trọng giải quyết các vấn đề nền tảng, giúp Android trở nên đáng tin cậy hơn trong sử dụng hằng ngày.

Ở góc độ hệ sinh thái, chiến lược “mở” được Google theo đuổi nhất quán. Android chấp nhận chạy trên nhiều cấu hình phần cứng, nhiều kích thước và độ phân giải màn hình, qua đó thu hút được đông đảo nhà sản xuất. Điều này giúp Android mở rộng thị phần rất nhanh, nhưng đồng thời cũng tạo ra thách thức về phân mảnh. Đáng chú ý là Google không tìm cách né tránh vấn đề này, mà lựa chọn cung cấp công cụ, hướng dẫn và API để lập trình viên tự thích ứng, đặt trách nhiệm tương thích lên cả nền tảng lẫn cộng đồng phát triển.

Về trải nghiệm người dùng, giai đoạn định hình cho thấy Android tiến hóa theo hướng thực dụng. Giao diện không được ưu tiên trau chuốt ngay từ đầu, nhưng từng bước được cải thiện dựa trên hành vi sử dụng thực tế. Các tính năng như widget, thanh thông báo, đa nhiệm và tích hợp dịch vụ trực tuyến dần trở thành điểm mạnh cốt lõi, tạo nên bản sắc riêng của Android so với các hệ điều hành di động khác.

Tổng kết lại, giai đoạn Android 1.0–2.x phản ánh rõ triết lý phát triển của Google: phát hành sớm, cải tiến nhanh, chấp nhận thiếu sót để đổi lấy tốc độ và khả năng mở rộng. Những quyết định kỹ thuật trong thời kỳ này không chỉ giải quyết vấn đề trước mắt, mà còn đặt nền móng cho kiến trúc Android về sau. Chính nhờ giai đoạn định hình này, Android có đủ nền tảng để bước sang các thế hệ tiếp theo với quy mô lớn hơn, giao diện hoàn thiện hơn và vai trò trung tâm trong hệ sinh thái di động toàn cầu.
