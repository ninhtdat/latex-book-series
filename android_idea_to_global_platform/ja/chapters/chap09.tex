\chapter{Android trong kỷ nguyên đa nền tảng và IoT}
\ruby{マルチ}{まるち}\ruby{プラットフォーム}{ぷらっとふぉーむ}およびIoT\ruby{時代}{じだい}におけるAndroid

Android ban đầu được phát triển như một hệ điều hành dành cho điện thoại thông minh, nhưng theo thời gian đã mở rộng sang nhiều loại thiết bị khác nhau. Trong số đó, tablet và các thiết bị màn hình lớn là lĩnh vực thể hiện rõ sự thay đổi trong tư duy thiết kế và định hướng chiến lược của Android. Việc Android quay trở lại mạnh mẽ trên phân khúc này cho thấy nỗ lực của Google nhằm xây dựng một nền tảng thống nhất, có khả năng đáp ứng cả nhu cầu tiêu dùng lẫn năng suất làm việc trong bối cảnh đa thiết bị.

Androidは\ruby{当初}{とうしょ}\ruby{スマートフォン}{すまーとふぉん}\ruby{向}{む}けの\ruby{OS}{おーえす}として\ruby{開発}{かいはつ}されたが、\ruby{時間}{じかん}の\ruby{経過}{けいか}とともに\ruby{多様}{たよう}な\ruby{デバイス}{でばいす}へと\ruby{拡張}{かくちょう}されてきた。その\ruby{中}{なか}でも、tabletや\ruby{大型}{おおがた}\ruby{画面}{がめん}\ruby{端末}{たんまつ}は、Androidの\ruby{設計}{せっけい}\ruby{思想}{しそう}と\ruby{戦略}{せんりゃく}\ruby{方針}{ほうしん}の\ruby{変化}{へんか}を\ruby{最}{もっと}も\ruby{顕著}{けんちょ}に\ruby{示}{しめ}す\ruby{分野}{ぶんや}である。この\ruby{領域}{りょういき}での\ruby{再}{ふたた}びの\ruby{注力}{ちゅうりょく}は、Googleが\ruby{消費}{しょうひ}\ruby{用途}{ようと}と\ruby{生産性}{せいさんせい}\ruby{向上}{こうじょう}の\ruby{双方}{そうほう}に\ruby{対応}{たいおう}する\ruby{統一}{とういつ}プラットフォームを\ruby{構築}{こうちく}しようとする\ruby{意図}{いと}を\ruby{示唆}{しさ}している。

\section{Android trên tablet và thiết bị màn hình lớn: tối ưu giao diện và trải nghiệm đa nhiệm}
tabletおよび\ruby{大型}{おおがた}\ruby{画面}{がめん}\ruby{端末}{たんまつ}におけるAndroid:\ruby{インターフェース}{いんたーふぇーす}\ruby{最適化}{さいてきか}と\ruby{マルチタスク}{まるちたすく}\ruby{体験}{たいけん}

Sự phát triển của tablet, thiết bị màn hình gập và các thiết bị Android có kích thước màn hình lớn đặt ra nhiều thách thức mới cho nền tảng Android. Khác với điện thoại thông minh, nơi giao diện chủ yếu được thiết kế cho thao tác một tay và không gian hiển thị hạn chế, tablet và màn hình lớn yêu cầu cách tiếp cận linh hoạt hơn để tận dụng tối đa diện tích hiển thị và hỗ trợ các kịch bản sử dụng phức tạp.

tablet、\ruby{折}{お}り\ruby{畳}{たた}み\ruby{画面}{がめん}\ruby{端末}{たんまつ}、および\ruby{大型}{おおがた}\ruby{画面}{がめん}を\ruby{備}{そな}えたAndroid\ruby{デバイス}{でばいす}の\ruby{普及}{ふきゅう}は、Android\ruby{プラットフォーム}{ぷらっとふぉーむ}に\ruby{新}{あら}たな\ruby{課題}{かだい}を\ruby{提起}{ていき}している。\ruby{片手}{かたて}\ruby{操作}{そうさ}と\ruby{限}{かぎ}られた\ruby{表示}{ひょうじ}\ruby{空間}{くうかん}を\ruby{前提}{ぜんてい}とする\ruby{スマートフォン}{すまーとふぉん}とは\ruby{異}{こと}なり、tabletや\ruby{大型}{おおがた}\ruby{画面}{がめん}では\ruby{広}{ひろ}い\ruby{表示}{ひょうじ}\ruby{領域}{りょういき}を\ruby{最大限}{さいだいげん}\ruby{活用}{かつよう}し、\ruby{複雑}{ふくざつ}な\ruby{利用}{りよう}\ruby{シナリオ}{しなりお}を\ruby{支援}{しえん}するための\ruby{柔軟}{じゅうなん}な\ruby{設計}{せっけい}が\ruby{求}{もと}められる。

Trước hết, vấn đề quan trọng nhất là tối ưu giao diện người dùng. Ứng dụng Android không thể chỉ đơn thuần phóng to giao diện của điện thoại để chạy trên tablet, vì điều này dẫn đến lãng phí không gian và trải nghiệm kém hiệu quả. Thay vào đó, Android khuyến khích sử dụng các bố cục thích ứng, cho phép giao diện tự động điều chỉnh theo kích thước, độ phân giải và tỷ lệ màn hình. Nội dung có thể được trình bày theo dạng nhiều cột, bảng điều khiển hoặc bố cục chia vùng, giúp người dùng tiếp cận thông tin nhanh hơn và trực quan hơn.

\ruby{第一}{だいいち}に、\ruby{ユーザー}{ゆーざー}\ruby{インターフェース}{いんたーふぇーす}の\ruby{最適化}{さいてきか}が\ruby{最}{もっと}も\ruby{重要}{じゅうよう}な\ruby{課題}{かだい}である。Android\ruby{アプリケーション}{あぷりけーしょん}は、\ruby{スマートフォン}{すまーとふぉん}の\ruby{画面}{がめん}を\ruby{単純}{たんじゅん}に\ruby{拡大}{かくだい}してtabletで\ruby{表示}{ひょうじ}するだけでは、\ruby{空間}{くうかん}を\ruby{浪費}{ろうひ}し、\ruby{効率}{こうりつ}の\ruby{低}{ひく}い\ruby{体験}{たいけん}を\ruby{招}{まね}く。そこでAndroidは、\ruby{画面}{がめん}\ruby{サイズ}{さいず}、\ruby{解像度}{かいぞうど}、\ruby{縦横比}{じゅうおうひ}に\ruby{応}{おう}じて\ruby{自動}{じどう}\ruby{調整}{ちょうせい}される\ruby{適応}{てきおう}\ruby{レイアウト}{れいあうと}の\ruby{利用}{りよう}を\ruby{推奨}{すいしょう}する。\ruby{複数}{ふくすう}\ruby{列}{れつ}、\ruby{ダッシュボード}{だっしゅぼーど}、\ruby{分割}{ぶんかつ}\ruby{配置}{はいち}などの\ruby{構成}{こうせい}により、\ruby{情報}{じょうほう}は\ruby{迅速}{じんそく}かつ\ruby{直感的}{ちょっかんてき}に\ruby{把握}{はあく}できる。

Bên cạnh đó, Android trên màn hình lớn phải hỗ trợ tốt hơn cho các phương thức nhập liệu đa dạng. Ngoài cảm ứng, người dùng tablet thường sử dụng bàn phím vật lý, chuột hoặc bút cảm ứng. Do đó, hệ điều hành và ứng dụng cần được thiết kế để phản hồi chính xác với con trỏ chuột, hỗ trợ phím tắt, thao tác kéo–thả và nhận diện nét vẽ. Đây là yếu tố then chốt để Android có thể đáp ứng các nhu cầu làm việc, học tập và sáng tạo nội dung vốn trước đây chủ yếu thuộc về máy tính cá nhân.

さらに、\ruby{大型}{おおがた}\ruby{画面}{がめん}におけるAndroidは\ruby{多様}{たよう}な\ruby{入力}{にゅうりょく}\ruby{手段}{しゅだん}への\ruby{対応}{たいおう}が\ruby{不可欠}{ふかけつ}である。\ruby{タッチ}{たっち}に\ruby{加}{くわ}え、tablet\ruby{利用}{りよう}\ruby{者}{しゃ}は\ruby{物理}{ぶつり}\ruby{キーボード}{きーぼーど}、\ruby{マウス}{まうす}、または\ruby{スタイラス}{すたいらす}を\ruby{用}{もち}いる\ruby{場合}{ばあい}が\ruby{多}{おお}い。そのため、OSおよび\ruby{アプリケーション}{あぷりけーしょん}は\ruby{ポインタ}{ぽいんた}\ruby{操作}{そうさ}への\ruby{正確}{せいかく}な\ruby{反応}{はんのう}、\ruby{ショートカット}{しょーとかっと}\ruby{キー}{きー}、\ruby{ドラッグ}{どらっぐ}&\ruby{ドロップ}{どろっぷ}、および\ruby{筆跡}{ひっせき}\ruby{認識}{にんしき}を\ruby{支援}{しえん}する\ruby{設計}{せっけい}が\ruby{求}{もと}められる。これらは、Androidが\ruby{学習}{がくしゅう}、\ruby{業務}{ぎょうむ}、\ruby{創作}{そうさく}\ruby{活動}{かつどう}に\ruby{対応}{たいおう}するための\ruby{鍵}{かぎ}となる。

Một khía cạnh quan trọng khác là trải nghiệm đa nhiệm. Trên điện thoại, đa nhiệm thường mang tính luân phiên, tức là người dùng chuyển đổi giữa các ứng dụng. Trong khi đó, trên tablet và thiết bị màn hình lớn, đa nhiệm cần mang tính song song. Android đã phát triển các cơ chế như chia đôi màn hình, cửa sổ nổi và khả năng chạy nhiều ứng dụng cùng lúc. Điều này cho phép người dùng vừa đọc tài liệu vừa ghi chú, vừa xem video vừa duyệt web, hoặc thao tác đồng thời trên nhiều nguồn thông tin. Khả năng đa nhiệm hiệu quả là yếu tố quyết định để Android trên tablet không chỉ là thiết bị giải trí mà còn là công cụ làm việc thực thụ.

もう\ruby{一}{ひと}つの\ruby{重要}{じゅうよう}な\ruby{側面}{そくめん}は\ruby{マルチタスク}{まるちたすく}\ruby{体験}{たいけん}である。\ruby{スマートフォン}{すまーとふぉん}では、\ruby{多}{おお}くの\ruby{場合}{ばあい}\ruby{アプリケーション}{あぷりけーしょん}の\ruby{切}{き}り\ruby{替}{か}えという\ruby{逐次}{ちくじ}\ruby{的}{てき}な\ruby{形}{かたち}が\ruby{中心}{ちゅうしん}となる。これに\ruby{対}{たい}し、tabletや\ruby{大型}{おおがた}\ruby{画面}{がめん}では\ruby{同時}{どうじ}\ruby{並行}{へいこう}\ruby{実行}{じっこう}が\ruby{求}{もと}められる。Androidは\ruby{画面}{がめん}\ruby{分割}{ぶんかつ}、\ruby{フローティング}{ふろーてぃんぐ}\ruby{ウィンドウ}{うぃんどう}、および\ruby{複数}{ふくすう}\ruby{アプリ}{あぷり}の\ruby{同時}{どうじ}\ruby{起動}{きどう}を\ruby{可能}{かのう}にした。これにより、\ruby{資料}{しりょう}\ruby{閲覧}{えつらん}と\ruby{メモ}{めも}、\ruby{動画}{どうが}\ruby{視聴}{しちょう}と\ruby{ウェブ}{うぇぶ}\ruby{閲覧}{えつらん}などの\ruby{複合}{ふくごう}\ruby{作業}{さぎょう}が\ruby{容易}{ようい}になる。\ruby{効率的}{こうりつてき}な\ruby{マルチタスク}{まるちたすく}は、tabletを\ruby{娯楽}{ごらく}\ruby{端末}{たんまつ}から\ruby{実用}{じつよう}\ruby{的}{てき}な\ruby{作業}{さぎょう}\ruby{ツール}{つーる}へと\ruby{昇華}{しょうか}させる\ruby{決定}{けってい}\ruby{要因}{よういん}である。

Ngoài ra, Android còn phải giải quyết bài toán tính nhất quán trải nghiệm giữa các loại thiết bị. Người dùng kỳ vọng có thể bắt đầu một công việc trên điện thoại, tiếp tục trên tablet và hoàn thiện trên thiết bị màn hình lớn khác mà không bị gián đoạn. Điều này đòi hỏi hệ điều hành hỗ trợ đồng bộ dữ liệu, trạng thái ứng dụng và hành vi người dùng một cách liền mạch. Tablet trong trường hợp này đóng vai trò là cầu nối giữa điện thoại di động và máy tính, kết hợp tính linh hoạt của thiết bị di động với không gian làm việc rộng rãi hơn.

さらに、Androidは\ruby{異}{こと}なる\ruby{デバイス}{でばいす}\ruby{間}{かん}での\ruby{体験}{たいけん}の\ruby{一貫性}{いっかんせい}という\ruby{課題}{かだい}にも\ruby{対処}{たいしょ}する\ruby{必要}{ひつよう}がある。\ruby{利用}{りよう}\ruby{者}{しゃ}は、\ruby{スマートフォン}{すまーとふぉん}で\ruby{開始}{かいし}した\ruby{作業}{さぎょう}をtabletで\ruby{継続}{けいぞく}し、\ruby{別}{べつ}の\ruby{大型}{おおがた}\ruby{画面}{がめん}\ruby{端末}{たんまつ}で\ruby{完了}{かんりょう}させることを\ruby{期待}{きたい}する。そのため、OSは\ruby{データ}{でーた}、\ruby{アプリケーション}{あぷりけーしょん}\ruby{状態}{じょうたい}、および\ruby{ユーザー}{ゆーざー}\ruby{操作}{そうさ}を\ruby{シームレス}{しーむれす}に\ruby{同期}{どうき}する\ruby{機能}{きのう}を\ruby{備}{そな}える\ruby{必要}{ひつよう}がある。この\ruby{文脈}{ぶんみゃく}でtabletは、\ruby{モバイル}{もばいる}\ruby{端末}{たんまつ}と\ruby{コンピュータ}{こんぴゅーた}の\ruby{橋渡}{はしわた}し\ruby{役}{やく}として\ruby{機能}{きのう}する。

Từ góc độ chiến lược, việc tối ưu Android cho tablet và màn hình lớn không chỉ nhằm cải thiện trải nghiệm người dùng hiện tại, mà còn mở rộng phạm vi ứng dụng của nền tảng. Android có thể được sử dụng trong giáo dục, doanh nghiệp, thiết bị chuyên dụng và các môi trường làm việc lai. Điều này giúp Google củng cố vị thế của Android như một hệ điều hành đa năng, có khả năng thích ứng với nhiều ngữ cảnh sử dụng khác nhau, thay vì bị giới hạn trong vai trò của một nền tảng di động truyền thống.

\ruby{戦略}{せんりゃく}\ruby{的}{てき}な\ruby{観点}{かんてん}からは、tabletおよび\ruby{大型}{おおがた}\ruby{画面}{がめん}への\ruby{最適化}{さいてきか}は、\ruby{既存}{きそん}の\ruby{体験}{たいけん}を\ruby{改善}{かいぜん}するだけでなく、Androidの\ruby{適用}{てきよう}\ruby{範囲}{はんい}を\ruby{拡大}{かくだい}する。\ruby{教育}{きょういく}、\ruby{企業}{きぎょう}、\ruby{専用}{せんよう}\ruby{端末}{たんまつ}、および\ruby{ハイブリッド}{はいぶりっど}\ruby{作業}{さぎょう}\ruby{環境}{かんきょう}においてAndroidが\ruby{活用}{かつよう}されることで、GoogleはAndroidを\ruby{汎用}{はんよう}\ruby{的}{てき}で\ruby{適応}{てきおう}\ruby{力}{りょく}の\ruby{高}{たか}い\ruby{OS}{おーえす}として\ruby{位置}{いち}づけることができる。

Tóm lại, Android trên tablet và thiết bị màn hình lớn là minh chứng cho quá trình tiến hóa của nền tảng này. Việc tập trung vào tối ưu giao diện, hỗ trợ đa nhiệm và nâng cao năng suất đã giúp Android từng bước thu hẹp khoảng cách với các hệ điều hành máy tính, đồng thời tạo nền tảng cho chiến lược đa thiết bị và hệ sinh thái thống nhất trong kỷ nguyên số.

\ruby{総括}{そうかつ}すると、tabletおよび\ruby{大型}{おおがた}\ruby{画面}{がめん}\ruby{端末}{たんまつ}におけるAndroidは、その\ruby{進化}{しんか}\ruby{過程}{かてい}を\ruby{象徴}{しょうちょう}している。\ruby{インターフェース}{いんたーふぇーす}\ruby{最適化}{さいてきか}、\ruby{マルチタスク}{まるちたすく}\ruby{支援}{しえん}、および\ruby{生産性}{せいさんせい}\ruby{向上}{こうじょう}への\ruby{注力}{ちゅうりょく}により、Androidは\ruby{従来}{じゅうらい}の\ruby{モバイル}{もばいる}\ruby{OS}{おーえす}の\ruby{枠}{わく}を\ruby{超}{こ}え、\ruby{多}{た}\ruby{デバイス}{でばいす}\ruby{戦略}{せんりゃく}と\ruby{統一}{とういつ}\ruby{エコシステム}{えこしすてむ}を\ruby{支}{ささ}える\ruby{基盤}{きばん}へと\ruby{発展}{はってん}しつつある。

\section{Android TV và Google TV: mở rộng Android sang lĩnh vực giải trí gia đình}
Android TVとGoogle TV:Androidの\ruby{家庭}{かてい}\ruby{向}{む}け\ruby{娯楽}{ごらく}\ruby{分野}{ぶんや}への\ruby{拡張}{かくちょう}

Việc mở rộng Android sang lĩnh vực TV đánh dấu một bước chuyển quan trọng trong chiến lược đa nền tảng của Google. Khác với tablet hay thiết bị di động vốn mang tính cá nhân cao, TV là thiết bị dùng chung trong gia đình, có thời gian sử dụng dài và tập trung chủ yếu vào tiêu thụ nội dung. Android TV và Google TV được thiết kế nhằm đáp ứng đặc thù này, đồng thời đưa Android trở thành nền tảng cốt lõi trong hệ sinh thái giải trí gia đình.

AndroidをTV\ruby{分野}{ぶんや}へ\ruby{拡張}{かくちょう}することは、Googleの\ruby{多}{た}\ruby{基盤}{きばん}\ruby{戦略}{せんりゃく}における\ruby{重要}{じゅうよう}な\ruby{転換}{てんかん}を\ruby{意味}{いみ}する。tabletや\ruby{携帯}{けいたい}\ruby{端末}{たんまつ}が\ruby{高}{たか}い\ruby{個人}{こじん}\ruby{性}{せい}を\ruby{持}{も}つのに\ruby{対}{たい}し、TVは\ruby{家庭}{かてい}で\ruby{共有}{きょうゆう}され、\ruby{使用}{しよう}\ruby{時間}{じかん}が\ruby{長}{なが}く、\ruby{主}{おも}に\ruby{コンテンツ}{こんてんつ}\ruby{消費}{しょうひ}に\ruby{集中}{しゅうちゅう}する\ruby{装置}{そうち}である。Android TVとGoogle TVは、こうした\ruby{特性}{とくせい}に\ruby{対応}{たいおう}するために\ruby{設計}{せっけい}され、Androidを\ruby{家庭}{かてい}\ruby{娯楽}{ごらく}\ruby{エコシステム}{えこしすてむ}の\ruby{中核}{ちゅうかく}\ruby{基盤}{きばん}として\ruby{位置}{いち}づけている。

Android TV là phiên bản Android được tối ưu hóa cho màn hình lớn đặt ở khoảng cách xa người dùng. Giao diện của Android TV được thiết kế theo hướng đơn giản, dễ quan sát, với các thành phần hiển thị lớn và rõ ràng. Thao tác điều khiển chủ yếu thông qua điều khiển từ xa hoặc giọng nói, thay vì cảm ứng trực tiếp. Do đó, hệ điều hành và ứng dụng phải giảm thiểu các thao tác phức tạp, tập trung vào điều hướng tuyến tính và lựa chọn nhanh nội dung. Điều này tạo ra một mô hình trải nghiệm hoàn toàn khác so với Android trên điện thoại hoặc tablet.

Android TVは、\ruby{利用}{りよう}\ruby{者}{しゃ}から\ruby{距離}{きょり}のある\ruby{大}{おお}きな\ruby{画面}{がめん}に\ruby{最適}{さいてき}\ruby{化}{か}されたAndroid\ruby{版本}{ばんぽん}である。ユーザー\ruby{インターフェース}{いんたーふぇーす}は\ruby{簡潔}{かんけつ}で\ruby{視認}{しにん}\ruby{性}{せい}が\ruby{高}{たか}く、\ruby{大}{おお}きく\ruby{明瞭}{めいりょう}な\ruby{表示}{ひょうじ}\ruby{要素}{ようそ}を\ruby{用}{もち}いる。\ruby{操作}{そうさ}は\ruby{主}{おも}に\ruby{リモート}{りもーと}\ruby{コントロール}{こんとろーる}や\ruby{音声}{おんせい}によって\ruby{行}{おこな}われ、\ruby{直接}{ちょくせつ}\ruby{的}{てき}な\ruby{タッチ}{たっち}\ruby{操作}{そうさ}は\ruby{前提}{ぜんてい}とされない。そのため、\ruby{複雑}{ふくざつ}な\ruby{操作}{そうさ}は\ruby{抑制}{よくせい}され、\ruby{直線}{ちょくせん}\ruby{的}{てき}な\ruby{ナビゲーション}{なびげーしょん}と\ruby{迅速}{じんそく}な\ruby{選択}{せんたく}に\ruby{重点}{じゅうてん}が\ruby{置}{お}かれる。これは\ruby{電話}{でんわ}やtabletのAndroidとは\ruby{異}{こと}なる\ruby{体験}{たいけん}\ruby{モデル}{もでる}を\ruby{形成}{けいせい}する。

Một điểm mạnh của Android TV là hệ sinh thái ứng dụng và nội dung phong phú. Các dịch vụ xem phim, truyền hình trực tuyến, âm nhạc và trò chơi có thể được triển khai thống nhất trên nhiều thiết bị TV khác nhau. Android TV đóng vai trò như một nền tảng trung gian, kết nối người dùng với các nhà cung cấp nội dung số, đồng thời cho phép Google tích hợp sâu các dịch vụ của mình như tìm kiếm, trợ lý giọng nói và quảng cáo. Từ góc độ kỹ thuật, Android TV tận dụng cùng nền tảng Android cốt lõi, nhưng được tinh chỉnh để ưu tiên hiệu năng hiển thị, khả năng phát đa phương tiện và độ ổn định trong thời gian dài.

Android TVの\ruby{強}{つよ}みは、\ruby{豊富}{ほうふ}な\ruby{アプリケーション}{あぷりけーしょん}と\ruby{コンテンツ}{こんてんつ}の\ruby{エコシステム}{えこしすてむ}にある。\ruby{映像}{えいぞう}\ruby{配信}{はいしん}、\ruby{オンライン}{おんらいん}\ruby{放送}{ほうそう}、\ruby{音楽}{おんがく}、\ruby{ゲーム}{げーむ}といった\ruby{サービス}{さーびす}を、\ruby{多様}{たよう}なTV\ruby{端末}{たんまつ}で\ruby{統一}{とういつ}\ruby{的}{てき}に\ruby{提供}{ていきょう}できる。Android TVは\ruby{仲介}{ちゅうかい}\ruby{基盤}{きばん}として、\ruby{利用}{りよう}\ruby{者}{しゃ}と\ruby{デジタル}{でじたる}\ruby{コンテンツ}{こんてんつ}\ruby{提供}{ていきょう}\ruby{者}{しゃ}を\ruby{接続}{せつぞく}し、\ruby{検索}{けんさく}、\ruby{音声}{おんせい}\ruby{アシスタント}{あしすたんと}、\ruby{広告}{こうこく}などGoogleの\ruby{サービス}{さーびす}を\ruby{深}{ふか}く\ruby{統合}{とうごう}する。\ruby{技術}{ぎじゅつ}\ruby{的}{てき}には、Androidの\ruby{中核}{ちゅうかく}\ruby{基盤}{きばん}を\ruby{共有}{きょうゆう}しつつ、\ruby{表示}{ひょうじ}\ruby{性能}{せいのう}、\ruby{マルチ}{まるち}\ruby{メディア}{めでぃあ}\ruby{再生}{さいせい}\ruby{能力}{のうりょく}、\ruby{長時間}{ちょうじかん}の\ruby{安定}{あんてい}\ruby{動作}{どうさ}を\ruby{優先}{ゆうせん}するよう\ruby{調整}{ちょうせい}されている。

Google TV là bước phát triển tiếp theo, tập trung nhiều hơn vào trải nghiệm nội dung thay vì trải nghiệm ứng dụng. Thay vì yêu cầu người dùng mở từng ứng dụng riêng lẻ để tìm phim hoặc chương trình yêu thích, Google TV xây dựng một lớp giao diện tổng hợp, nơi nội dung từ nhiều nguồn khác nhau được gom lại và đề xuất theo ngữ cảnh. Cách tiếp cận này phản ánh sự thay đổi trong hành vi người dùng: họ quan tâm đến nội dung muốn xem hơn là ứng dụng cung cấp nội dung đó. Google TV vì vậy đóng vai trò như một “trình quản lý nội dung” ở cấp hệ điều hành.

Google TVは\ruby{次}{つぎ}の\ruby{発展}{はってん}\ruby{段階}{だんかい}であり、\ruby{アプリ}{あぷり}\ruby{体験}{たいけん}よりも\ruby{コンテンツ}{こんてんつ}\ruby{体験}{たいけん}に\ruby{重点}{じゅうてん}を\ruby{置}{お}く。ユーザーが\ruby{個別}{こべつ}の\ruby{アプリ}{あぷり}を\ruby{開}{ひら}いて\ruby{検索}{けんさく}する\ruby{必要}{ひつよう}はなく、\ruby{複数}{ふくすう}の\ruby{供給}{きょうきゅう}\ruby{源}{げん}から\ruby{集約}{しゅうやく}された\ruby{コンテンツ}{こんてんつ}が\ruby{文脈}{ぶんみゃく}に\ruby{基}{もと}づいて\ruby{提示}{ていじ}される。この\ruby{手法}{しゅほう}は、\ruby{利用}{りよう}\ruby{者}{しゃ}が\ruby{アプリ}{あぷり}ではなく\ruby{視聴}{しちょう}\ruby{対象}{たいしょう}そのものに\ruby{関心}{かんしん}を\ruby{持}{も}つという\ruby{行動}{こうどう}\ruby{変化}{へんか}を\ruby{反映}{はんえい}している。Google TVは\ruby{オペレーティング}{おぺれーてぃんぐ}\ruby{システム}{しすてむ}\ruby{水準}{すいじゅん}の「\ruby{コンテンツ}{こんてんつ}\ruby{管理}{かんり}\ruby{者}{しゃ}」として\ruby{機能}{きのう}する。

Khả năng cá nhân hóa là yếu tố then chốt của Google TV. Dựa trên tài khoản Google, hệ thống có thể đồng bộ lịch sử xem, sở thích và danh sách theo dõi giữa TV và các thiết bị khác như điện thoại hay tablet. Điều này giúp trải nghiệm giải trí trở nên liền mạch và mang tính cá nhân ngay cả trên một thiết bị dùng chung. Đồng thời, việc tích hợp trợ lý giọng nói cho phép người dùng tìm kiếm nội dung, điều khiển thiết bị thông minh trong nhà và thực hiện các tác vụ cơ bản mà không cần rời khỏi màn hình TV.

Google TVの\ruby{中核}{ちゅうかく}は\ruby{個人}{こじん}\ruby{化}{か}である。Google\ruby{アカウント}{あかうんと}を\ruby{基}{もと}に、TVと\ruby{携帯}{けいたい}\ruby{端末}{たんまつ}やtablet\ruby{間}{かん}で\ruby{視聴}{しちょう}\ruby{履歴}{りれき}、\ruby{嗜好}{しこう}、\ruby{ウォッチ}{うぉっち}\ruby{リスト}{りすと}が\ruby{同期}{どうき}される。これにより、\ruby{共有}{きょうゆう}\ruby{端末}{たんまつ}であっても\ruby{連続}{れんぞく}\ruby{的}{てき}かつ\ruby{個人}{こじん}\ruby{的}{てき}な\ruby{娯楽}{ごらく}\ruby{体験}{たいけん}が\ruby{実現}{じつげん}する。さらに、\ruby{音声}{おんせい}\ruby{アシスタント}{あしすたんと}の\ruby{統合}{とうごう}により、\ruby{検索}{けんさく}、\ruby{家庭}{かてい}\ruby{内}{ない}の\ruby{スマート}{すまーと}\ruby{機器}{きき}\ruby{制御}{せいぎょ}、\ruby{基本}{きほん}\ruby{操作}{そうさ}をTV\ruby{画面}{がめん}から\ruby{離}{はな}れることなく\ruby{行}{おこな}える。

Từ góc độ chiến lược hệ sinh thái, Android TV và Google TV giúp Google mở rộng sự hiện diện của Android vào không gian phòng khách, nơi trước đây thường do các nền tảng độc quyền kiểm soát. TV trở thành một nút trung tâm trong hệ sinh thái đa thiết bị, kết nối giải trí, nhà thông minh và các dịch vụ số khác. Điều này không chỉ gia tăng thời gian người dùng tương tác với nền tảng Android mà còn củng cố vai trò của Google trong chuỗi giá trị nội dung số.

\ruby{エコシステム}{えこしすてむ}\ruby{戦略}{せんりゃく}の\ruby{観点}{かんてん}では、Android TVとGoogle TVは、\ruby{従来}{じゅうらい}\ruby{独占}{どくせん}\ruby{的}{てき}な\ruby{基盤}{きばん}が\ruby{支配}{しはい}してきた\ruby{居間}{いま}の\ruby{空間}{くうかん}へAndroidの\ruby{存在}{そんざい}を\ruby{拡張}{かくちょう}する。TVは\ruby{多}{た}\ruby{端末}{たんまつ}\ruby{エコシステム}{えこしすてむ}の\ruby{中枢}{ちゅうすう}となり、\ruby{娯楽}{ごらく}、\ruby{スマート}{すまーと}\ruby{ホーム}{ほーむ}、\ruby{デジタル}{でじたる}\ruby{サービス}{さーびす}を\ruby{結節}{けっせつ}する。これはAndroidとの\ruby{接触}{せっしょく}\ruby{時間}{じかん}を\ruby{増}{ふ}やすだけでなく、\ruby{デジタル}{でじたる}\ruby{コンテンツ}{こんてんつ}\ruby{価値}{かち}\ruby{連鎖}{れんさ}におけるGoogleの\ruby{役割}{やくわり}を\ruby{強化}{きょうか}する。

Tóm lại, Android TV và Google TV thể hiện sự thích nghi của Android với một bối cảnh sử dụng hoàn toàn khác so với thiết bị di động. Thông qua việc tối ưu giao diện, thay đổi cách tiếp cận trải nghiệm và tập trung vào nội dung, Android đã mở rộng thành công sang lĩnh vực giải trí gia đình, đóng góp quan trọng vào chiến lược xây dựng một hệ sinh thái số thống nhất và đa nền tảng.

\ruby{総括}{そうかつ}すると、Android TVとGoogle TVは、\ruby{携帯}{けいたい}\ruby{端末}{たんまつ}とは\ruby{異}{こと}なる\ruby{利用}{りよう}\ruby{文脈}{ぶんみゃく}へのAndroidの\ruby{適応}{てきおう}を\ruby{示}{しめ}す。ユーザー\ruby{インターフェース}{いんたーふぇーす}の\ruby{最適}{さいてき}\ruby{化}{か}、\ruby{体験}{たいけん}\ruby{設計}{せっけい}の\ruby{転換}{てんかん}、\ruby{コンテンツ}{こんてんつ}\ruby{中心}{ちゅうしん}の\ruby{戦略}{せんりゃく}により、Androidは\ruby{家庭}{かてい}\ruby{娯楽}{ごらく}\ruby{分野}{ぶんや}へ\ruby{成功}{せいこう}裏に\ruby{拡張}{かくちょう}された。これは\ruby{統一}{とういつ}\ruby{的}{てき}で\ruby{多}{た}\ruby{基盤}{きばん}な\ruby{デジタル}{でじたる}\ruby{エコシステム}{えこしすてむ}を\ruby{構築}{こうちく}するうえで\ruby{重要}{じゅうよう}な\ruby{貢献}{こうけん}である。

\section{Wear OS và thiết bị đeo: thách thức về pin, hiệu năng và trải nghiệm người dùng}
\ruby{Wear}{うぇあ} \ruby{OS}{おーえす}と\ruby{ウェアラブル}{うぇあらぶる}\ruby{デバイス}{でばいす}:\ruby{バッテリー}{ばってりー}、\ruby{性能}{せいのう}、および\ruby{ユーザー}{ゆーざー}\ruby{体験}{たいけん}の\ruby{課題}{かだい}

Thiết bị đeo thông minh, đặc biệt là đồng hồ thông minh, là một trong những hướng mở rộng quan trọng của Android trong kỷ nguyên đa nền tảng. Không giống điện thoại hay tablet, thiết bị đeo có kích thước nhỏ, tài nguyên phần cứng hạn chế và cách thức tương tác rất ngắn gọn. Điều này buộc Android phải được điều chỉnh mạnh mẽ cả về kiến trúc hệ điều hành lẫn triết lý thiết kế trải nghiệm người dùng. Wear OS ra đời nhằm đáp ứng những yêu cầu đặc thù đó, đồng thời mở rộng hệ sinh thái Android sang lĩnh vực chăm sóc sức khỏe, thể thao và tương tác tức thời.

\ruby{スマート}{すまーと}\ruby{ウェアラブル}{うぇあらぶる}、とりわけ\ruby{スマート}{すまーと}\ruby{ウォッチ}{うぉっち}は、\ruby{マルチ}{まるち}\ruby{プラットフォーム}{ぷらっとふぉーむ}\ruby{時代}{じだい}におけるAndroidの\ruby{重要}{じゅうよう}な\ruby{拡張}{かくちょう}\ruby{方向}{ほうこう}の\ruby{一}{ひと}つである。\ruby{スマートフォン}{すまーとふぉん}や\ruby{タブレット}{たぶれっと}と\ruby{異}{こと}なり、\ruby{ウェアラブル}{うぇあらぶる}\ruby{デバイス}{でばいす}は\ruby{小型}{こがた}で、\ruby{ハードウェア}{はーどうぇあ}\ruby{資源}{しげん}が\ruby{限}{かぎ}られ、\ruby{操作}{そうさ}も\ruby{短時間}{たんじかん}で\ruby{完結}{かんけつ}することが\ruby{前提}{ぜんてい}となる。このためAndroidは、\ruby{OS}{おーえす}\ruby{アーキテクチャ}{あーきてくちゃ}と\ruby{ユーザー}{ゆーざー}\ruby{体験}{たいけん}\ruby{設計}{せっけい}の\ruby{両面}{りょうめん}で\ruby{大幅}{おおはば}な\ruby{調整}{ちょうせい}を\ruby{迫}{せま}られた。Wear OSは、こうした\ruby{特有}{とくゆう}の\ruby{要件}{ようけん}に\ruby{応}{こた}えるために\ruby{誕生}{たんじょう}し、Android\ruby{生態系}{せいたいけい}を\ruby{健康}{けんこう}\ruby{管理}{かんり}、\ruby{スポーツ}{すぽーつ}、および\ruby{即時}{そくじ}\ruby{的}{てき}\ruby{相互}{そうご}\ruby{作用}{さよう}の\ruby{分野}{ぶんや}へと\ruby{拡張}{かくちょう}した。

Thách thức lớn nhất đối với Wear OS là vấn đề pin. Đồng hồ thông minh có dung lượng pin rất nhỏ, trong khi vẫn phải duy trì nhiều chức năng như hiển thị liên tục, cảm biến sinh học, kết nối không dây và xử lý thông báo theo thời gian thực. Do đó, hệ điều hành phải được tối ưu để giảm thiểu tối đa các tác vụ nền, kiểm soát chặt chẽ chu kỳ đánh thức CPU và ưu tiên các cơ chế tiết kiệm năng lượng. Khác với smartphone, nơi người dùng có thể chấp nhận sạc mỗi ngày, thiết bị đeo đòi hỏi thời lượng pin đủ dài để không làm gián đoạn trải nghiệm sử dụng liên tục.

Wear OSにおける\ruby{最大}{さいだい}の\ruby{課題}{かだい}は\ruby{バッテリー}{ばってりー}である。\ruby{スマート}{すまーと}\ruby{ウォッチ}{うぉっち}は\ruby{電池}{でんち}\ruby{容量}{ようりょう}が\ruby{極}{きわ}めて\ruby{小}{ちい}さいにもかかわらず、\ruby{常時}{じょうじ}\ruby{表示}{ひょうじ}、\ruby{生体}{せいたい}\ruby{センサー}{せんさー}、\ruby{無線}{むせん}\ruby{通信}{つうしん}、および\ruby{リアル}{りある}\ruby{タイム}{たいむ}の\ruby{通知}{つうち}\ruby{処理}{しょり}を\ruby{維持}{いじ}する\ruby{必要}{ひつよう}がある。そのため、\ruby{OS}{おーえす}は\ruby{バックグラウンド}{ばっくぐらうんど}\ruby{タスク}{たすく}を\ruby{最小限}{さいしょうげん}に\ruby{抑}{おさ}え、CPUの\ruby{起動}{きどう}\ruby{周期}{しゅうき}を\ruby{厳密}{げんみつ}に\ruby{制御}{せいぎょ}し、\ruby{省}{しょう}\ruby{電力}{でんりょく}\ruby{機構}{きこう}を\ruby{最優先}{さいゆうせん}する\ruby{設計}{せっけい}が\ruby{求}{もと}められる。\ruby{毎日}{まいにち}の\ruby{充電}{じゅうでん}が\ruby{許容}{きょよう}される\ruby{スマートフォン}{すまーとふぉん}と\ruby{異}{こと}なり、\ruby{ウェアラブル}{うぇあらぶる}\ruby{デバイス}{でばいす}では\ruby{連続}{れんぞく}\ruby{使用}{しよう}を\ruby{妨}{さまた}げない\ruby{十分}{じゅうぶん}な\ruby{駆動}{くどう}\ruby{時間}{じかん}が\ruby{不可欠}{ふかけつ}である。

Bên cạnh pin, hiệu năng cũng là một vấn đề mang tính giới hạn. Phần cứng của thiết bị đeo thường sử dụng vi xử lý công suất thấp, bộ nhớ và khả năng lưu trữ hạn chế. Điều này khiến Wear OS không thể chạy các ứng dụng phức tạp hoặc đa nhiệm theo cách truyền thống của Android. Thay vào đó, nền tảng này được thiết kế để xử lý các tác vụ ngắn, phản hồi nhanh và mang tính ngữ cảnh cao. Các ứng dụng trên Wear OS cần được tối giản về logic xử lý, giao diện và luồng tương tác, nhằm đảm bảo hệ thống luôn phản hồi tức thì và ổn định.

\ruby{バッテリー}{ばってりー}に\ruby{加}{くわ}え、\ruby{性能}{せいのう}も\ruby{制約}{せいやく}の\ruby{大}{おお}きな\ruby{要因}{よういん}である。\ruby{ウェアラブル}{うぇあらぶる}\ruby{デバイス}{でばいす}の\ruby{ハードウェア}{はーどうぇあ}は、\ruby{低}{ひく}い\ruby{消費}{しょうひ}\ruby{電力}{でんりょく}の\ruby{プロセッサ}{ぷろせっさ}や、\ruby{限}{かぎ}られた\ruby{メモリ}{めもり}、\ruby{ストレージ}{すとれーじ}を\ruby{前提}{ぜんてい}としている。このためWear OSは、\ruby{従来}{じゅうらい}のAndroidのように\ruby{複雑}{ふくざつ}な\ruby{アプリケーション}{あぷりけーしょん}や\ruby{多}{た}\ruby{重}{じゅう}\ruby{タスク}{たすく}を\ruby{実行}{じっこう}することができない。かわりに、\ruby{短時間}{たんじかん}で\ruby{完了}{かんりょう}し、\ruby{高速}{こうそく}に\ruby{反応}{はんのう}し、\ruby{文脈}{ぶんみゃく}\ruby{依存}{いぞん}の\ruby{高}{たか}い\ruby{処理}{しょり}に\ruby{最適化}{さいてきか}されている。Wear OS向け\ruby{アプリケーション}{あぷりけーしょん}は、\ruby{処理}{しょり}\ruby{ロジック}{ろじっく}、\ruby{UI}{ゆーあい}、および\ruby{操作}{そうさ}\ruby{フロー}{ふろー}を\ruby{徹底的}{てっていてき}に\ruby{簡素}{かんそ}化し、\ruby{即時}{そくじ}かつ\ruby{安定}{あんてい}した\ruby{応答}{おうとう}を\ruby{保証}{ほしょう}する\ruby{必要}{ひつよう}がある。

Trải nghiệm người dùng trên thiết bị đeo cũng khác biệt rõ rệt so với các thiết bị Android khác. Màn hình nhỏ khiến việc hiển thị thông tin phải được chọn lọc và cô đọng. Người dùng thường chỉ liếc nhanh để xem thông báo, thời gian, nhịp tim hoặc trạng thái hoạt động, thay vì tương tác kéo dài. Vì vậy, Wear OS ưu tiên các giao diện dạng thẻ, thông báo ngắn và thao tác một chạm. Mọi yếu tố không cần thiết đều phải được loại bỏ để tránh gây rối hoặc làm chậm quá trình sử dụng.

\ruby{ユーザー}{ゆーざー}\ruby{体験}{たいけん}も、\ruby{他}{ほか}のAndroid\ruby{デバイス}{でばいす}とは\ruby{明確}{めいかく}に\ruby{異}{こと}なる。\ruby{小}{ちい}さな\ruby{画面}{がめん}では、\ruby{情報}{じょうほう}を\ruby{選択的}{せんたくてき}かつ\ruby{凝縮}{ぎょうしゅく}して\ruby{表示}{ひょうじ}する\ruby{必要}{ひつよう}がある。\ruby{利用者}{りようしゃ}は\ruby{長時間}{ちょうじかん}の\ruby{操作}{そうさ}ではなく、\ruby{通知}{つうち}、\ruby{時刻}{じこく}、\ruby{心拍}{しんぱく}、\ruby{活動}{かつどう}\ruby{状態}{じょうたい}を\ruby{一瞬}{いっしゅん}で\ruby{確認}{かくにん}することが\ruby{多}{おお}い。そのためWear OSは、\ruby{カード}{かーど}\ruby{形式}{けいしき}の\ruby{UI}{ゆーあい}、\ruby{短}{みじか}い\ruby{通知}{つうち}、および\ruby{ワン}{わん}\ruby{タップ}{たっぷ}\ruby{操作}{そうさ}を\ruby{重視}{じゅうし}する。\ruby{不要}{ふよう}な\ruby{要素}{ようそ}は\ruby{排除}{はいじょ}され、\ruby{混乱}{こんらん}や\ruby{遅延}{ちえん}を\ruby{招}{まね}かない\ruby{設計}{せっけい}が\ruby{求}{もと}められる。

Một điểm đặc trưng của Wear OS là vai trò bổ trợ cho smartphone. Trong hầu hết các kịch bản, đồng hồ thông minh không hoạt động hoàn toàn độc lập mà đóng vai trò như một thiết bị vệ tinh, mở rộng khả năng của điện thoại. Thông báo, dữ liệu và nhiều tác vụ được đồng bộ từ smartphone sang thiết bị đeo, giúp người dùng tiếp cận thông tin nhanh chóng mà không cần lấy điện thoại ra. Cách tiếp cận này giúp giảm tải cho phần cứng của thiết bị đeo, đồng thời tận dụng sức mạnh xử lý và kết nối của smartphone.

Wear OSの\ruby{特徴}{とくちょう}の\ruby{一}{ひと}つは、\ruby{スマートフォン}{すまーとふぉん}を\ruby{補完}{ほかん}する\ruby{役割}{やくわり}である。\ruby{多}{おお}くの\ruby{場面}{ばめん}で、\ruby{スマート}{すまーと}\ruby{ウォッチ}{うぉっち}は\ruby{完全}{かんぜん}に\ruby{独立}{どくりつ}して\ruby{動作}{どうさ}するのではなく、\ruby{衛星}{えいせい}\ruby{的}{てき}な\ruby{デバイス}{でばいす}として\ruby{電話}{でんわ}の\ruby{機能}{きのう}を\ruby{拡張}{かくちょう}する。\ruby{通知}{つうち}や\ruby{データ}{でーた}、\ruby{各種}{かくしゅ}\ruby{タスク}{たすく}は\ruby{スマートフォン}{すまーとふぉん}から\ruby{同期}{どうき}され、\ruby{利用者}{りようしゃ}は\ruby{端末}{たんまつ}を\ruby{取}{と}り\ruby{出}{だ}すことなく\ruby{情報}{じょうほう}に\ruby{即座}{そくざ}に\ruby{アクセス}{あくせす}できる。この\ruby{設計}{せっけい}は、\ruby{ウェアラブル}{うぇあらぶる}\ruby{側}{がわ}の\ruby{負荷}{ふか}を\ruby{軽減}{けいげん}しつつ、\ruby{スマートフォン}{すまーとふぉん}の\ruby{処理}{しょり}\ruby{能力}{のうりょく}と\ruby{通信}{つうしん}\ruby{性能}{せいのう}を\ruby{活用}{かつよう}する。

Ngoài ra, Wear OS còn đóng vai trò quan trọng trong lĩnh vực theo dõi sức khỏe và hoạt động thể chất. Các cảm biến như đo nhịp tim, theo dõi giấc ngủ, đếm bước chân hay đo mức độ vận động tạo ra lượng dữ liệu lớn và liên tục. Hệ điều hành phải đảm bảo việc thu thập và xử lý dữ liệu này diễn ra chính xác, an toàn và tiết kiệm năng lượng. Đồng thời, dữ liệu cần được đồng bộ với các dịch vụ đám mây và ứng dụng trên smartphone để phục vụ phân tích dài hạn và cá nhân hóa trải nghiệm.

さらにWear OSは、\ruby{健康}{けんこう}\ruby{管理}{かんり}や\ruby{身体}{しんたい}\ruby{活動}{かつどう}\ruby{追跡}{ついせき}の\ruby{分野}{ぶんや}で\ruby{重要}{じゅうよう}な\ruby{役割}{やくわり}を\ruby{担}{にな}う。\ruby{心拍}{しんぱく}\ruby{測定}{そくてい}、\ruby{睡眠}{すいみん}\ruby{追跡}{ついせき}、\ruby{歩数}{ほすう}\ruby{計測}{けいそく}、\ruby{運動}{うんどう}\ruby{量}{りょう}\ruby{評価}{ひょうか}などの\ruby{センサー}{せんさー}は、\ruby{大量}{たいりょう}かつ\ruby{継続的}{けいぞくてき}な\ruby{データ}{でーた}を\ruby{生成}{せいせい}する。\ruby{OS}{おーえす}は、これらを\ruby{正確}{せいかく}、\ruby{安全}{あんぜん}、かつ\ruby{省}{しょう}\ruby{電力}{でんりょく}で\ruby{処理}{しょり}することを\ruby{保証}{ほしょう}しなければならない。また、\ruby{長期}{ちょうき}\ruby{分析}{ぶんせき}や\ruby{個別}{こべつ}\ruby{最適化}{さいてきか}のため、\ruby{クラウド}{くらうど}\ruby{サービス}{さーびす}や\ruby{スマートフォン}{すまーとふぉん}\ruby{アプリ}{あぷり}との\ruby{同期}{どうき}も\ruby{不可欠}{ふかけつ}である。

Từ góc độ hệ sinh thái, Wear OS giúp Android mở rộng hiện diện sang một dạng thiết bị có tần suất tương tác cao và gắn liền với đời sống hằng ngày của người dùng. Mặc dù còn nhiều hạn chế về pin và hiệu năng, Wear OS thể hiện rõ định hướng của Android trong việc thích nghi với các thiết bị chuyên biệt. Thay vì cố gắng sao chép trải nghiệm smartphone, Wear OS tập trung vào các giá trị cốt lõi của thiết bị đeo: nhanh, gọn, ngữ cảnh và luôn sẵn sàng.

\ruby{生態系}{せいたいけい}の\ruby{観点}{かんてん}では、Wear OSはAndroidの\ruby{存在}{そんざい}を\ruby{日常}{にちじょう}\ruby{生活}{せいかつ}に\ruby{密着}{みっちゃく}した\ruby{高頻度}{こうひんど}\ruby{デバイス}{でばいす}へと\ruby{拡張}{かくちょう}する。\ruby{バッテリー}{ばってりー}や\ruby{性能}{せいのう}に\ruby{制約}{せいやく}はあるものの、Wear OSは\ruby{専用}{せんよう}\ruby{機器}{きき}への\ruby{適応}{てきおう}というAndroidの\ruby{明確}{めいかく}な\ruby{方向性}{ほうこうせい}を\ruby{示}{しめ}している。\ruby{スマートフォン}{すまーとふぉん}の\ruby{体験}{たいけん}を\ruby{模倣}{もほう}するのではなく、\ruby{迅速}{じんそく}、\ruby{簡潔}{かんけつ}、\ruby{文脈}{ぶんみゃく}\ruby{重視}{じゅうし}、そして\ruby{常}{つね}に\ruby{利用}{りよう}\ruby{可能}{かのう}であることを\ruby{中核}{ちゅうかく}とする。

Tóm lại, Wear OS và thiết bị đeo là minh chứng cho khả năng linh hoạt của Android trong việc mở rộng sang những nền tảng có ràng buộc kỹ thuật khắt khe. Việc giải quyết bài toán pin, hiệu năng và trải nghiệm người dùng không chỉ là thách thức kỹ thuật, mà còn định hình cách Android tham gia vào lĩnh vực chăm sóc sức khỏe và tương tác tức thời trong hệ sinh thái đa thiết bị.

\ruby{総括}{そうかつ}すると、Wear OSと\ruby{ウェアラブル}{うぇあらぶる}\ruby{デバイス}{でばいす}は、\ruby{厳}{きび}しい\ruby{技術的}{ぎじゅつてき}\ruby{制約}{せいやく}を\ruby{伴}{ともな}う\ruby{プラットフォーム}{ぷらっとふぉーむ}へAndroidが\ruby{柔軟}{じゅうなん}に\ruby{拡張}{かくちょう}できることを\ruby{示}{しめ}す\ruby{好例}{こうれい}である。\ruby{バッテリー}{ばってりー}、\ruby{性能}{せいのう}、\ruby{ユーザー}{ゆーざー}\ruby{体験}{たいけん}という\ruby{課題}{かだい}への\ruby{対応}{たいおう}は、\ruby{技術}{ぎじゅつ}\ruby{問題}{もんだい}に\ruby{留}{とど}まらず、Androidが\ruby{健康}{けんこう}\ruby{管理}{かんり}や\ruby{即時}{そくじ}\ruby{的}{てき}\ruby{相互}{そうご}\ruby{作用}{さよう}の\ruby{分野}{ぶんや}でどのような\ruby{役割}{やくわり}を\ruby{担}{にな}うかを\ruby{規定}{きてい}している。

\section{Android Automotive: tích hợp hệ điều hành vào ô tô và hệ thống điều khiển}
Android Automotive:\ruby{自動車}{じどうしゃ}への\ruby{オペレーティングシステム}{おぺれーてぃんぐしすてむ}\ruby{統合}{とうごう}と\ruby{制御}{せいぎょ}\ruby{システム}{しすてむ}

Sự phát triển của ô tô thông minh đã làm thay đổi vai trò của phần mềm trong ngành công nghiệp ô tô truyền thống. Từ chỗ chỉ đóng vai trò hỗ trợ, phần mềm ngày nay trở thành yếu tố trung tâm trong trải nghiệm lái xe, giải trí và quản lý phương tiện. Android Automotive xuất hiện trong bối cảnh đó như một bước tiến chiến lược của Android, đưa hệ điều hành này trực tiếp chạy trên phần cứng của xe, thay vì chỉ đóng vai trò phản chiếu từ điện thoại như các giải pháp trước đây.

\ruby{スマート}{すまーと}\ruby{自動車}{じどうしゃ}の\ruby{発展}{はってん}は、\ruby{従来}{じゅうらい}の\ruby{自動車}{じどうしゃ}\ruby{産業}{さんぎょう}における\ruby{ソフトウェア}{そふとうぇあ}の\ruby{役割}{やくわり}を\ruby{大}{おお}きく\ruby{変}{か}えた。かつては\ruby{補助的}{ほじょてき}な\ruby{存在}{そんざい}であった\ruby{ソフトウェア}{そふとうぇあ}は、\ruby{現在}{げんざい}では\ruby{運転}{うんてん}\ruby{体験}{たいけん}、\ruby{娯楽}{ごらく}、および\ruby{車両}{しゃりょう}\ruby{管理}{かんり}における\ruby{中核}{ちゅうかく}となっている。この\ruby{文脈}{ぶんみゃく}において、Android Automotiveは、\ruby{従来}{じゅうらい}の\ruby{スマートフォン}{すまーとふぉん}からの\ruby{画面}{がめん}\ruby{投影}{とうえい}に\ruby{依存}{いぞん}する\ruby{方式}{ほうしき}とは\ruby{異}{こと}なり、\ruby{車両}{しゃりょう}の\ruby{ハードウェア}{はーどうぇあ}上で\ruby{直接}{ちょくせつ}\ruby{動作}{どうさ}する\ruby{戦略的}{せんりゃくてき}な\ruby{進化}{しんか}として\ruby{登場}{とうじょう}した。

Android Automotive là một hệ điều hành nhúng, được thiết kế để tích hợp sâu với các hệ thống của ô tô. Không giống Android Auto, vốn phụ thuộc vào smartphone, Android Automotive hoạt động độc lập và có quyền truy cập trực tiếp vào các thành phần phần cứng của xe. Điều này cho phép hệ điều hành điều khiển các chức năng cốt lõi như hệ thống giải trí, định vị, điều hòa, cũng như hiển thị thông tin vận hành của phương tiện. Android trong trường hợp này không còn là một nền tảng ứng dụng thuần túy, mà trở thành một phần của hệ thống điều khiển tổng thể.

Android Automotiveは\ruby{組}{く}み\ruby{込}{こ}み\ruby{型}{がた}の\ruby{オペレーティングシステム}{おぺれーてぃんぐしすてむ}であり、\ruby{自動車}{じどうしゃ}の\ruby{各種}{かくしゅ}\ruby{システム}{しすてむ}と\ruby{深}{ふか}く\ruby{統合}{とうごう}されるよう\ruby{設計}{せっけい}されている。スマートフォンに\ruby{依存}{いぞん}するAndroid Autoとは\ruby{異}{こと}なり、Android Automotiveは\ruby{独立}{どくりつ}して\ruby{動作}{どうさ}し、\ruby{車両}{しゃりょう}の\ruby{ハードウェア}{はーどうぇあ}\ruby{構成要素}{こうせいようそ}へ\ruby{直接}{ちょくせつ}\ruby{アクセス}{あくせす}できる。これにより、\ruby{エンターテインメント}{えんたーていんめんと}\ruby{システム}{しすてむ}、\ruby{ナビゲーション}{なびげーしょん}、\ruby{空調}{くうちょう}、および\ruby{車両}{しゃりょう}\ruby{運行}{うんこう}\ruby{情報}{じょうほう}の\ruby{表示}{ひょうじ}といった\ruby{中核}{ちゅうかく}\ruby{機能}{きのう}を\ruby{制御}{せいぎょ}できる。この\ruby{場合}{ばあい}、Androidは\ruby{単}{たん}なる\ruby{アプリケーション}{あぷりけーしょん}\ruby{基盤}{きばん}ではなく、\ruby{全体}{ぜんたい}\ruby{制御}{せいぎょ}\ruby{システム}{しすてむ}の\ruby{一部}{いちぶ}となる。

Một đặc điểm quan trọng của Android Automotive là khả năng tích hợp chặt chẽ giữa giao diện người dùng và bối cảnh lái xe. Giao diện phải được thiết kế sao cho giảm thiểu sự phân tâm, ưu tiên an toàn và dễ quan sát trong mọi điều kiện ánh sáng. Các thao tác phức tạp bị hạn chế, thay vào đó là điều khiển bằng giọng nói, nút vật lý hoặc các tương tác đơn giản. Điều này đòi hỏi Android phải điều chỉnh mạnh mẽ so với phiên bản dành cho thiết bị cá nhân, nơi sự linh hoạt và tự do tương tác được đặt lên hàng đầu.

Android Automotiveの\ruby{重要}{じゅうよう}な\ruby{特徴}{とくちょう}は、\ruby{ユーザー}{ゆーざー}\ruby{インターフェース}{いんたーふぇーす}と\ruby{運転}{うんてん}\ruby{状況}{じょうきょう}との\ruby{緊密}{きんみつ}な\ruby{統合}{とうごう}にある。UIは\ruby{注意}{ちゅうい}\ruby{散漫}{さんまん}を\ruby{最小限}{さいしょうげん}に\ruby{抑}{おさ}え、\ruby{安全}{あんぜん}を\ruby{最優先}{さいゆうせん}し、\ruby{あらゆる}{あらゆる}\ruby{照明}{しょうめい}\ruby{条件}{じょうけん}で\ruby{視認}{しにん}しやすいよう\ruby{設計}{せっけい}される\ruby{必要}{ひつよう}がある。\ruby{複雑}{ふくざつ}な\ruby{操作}{そうさ}は\ruby{制限}{せいげん}され、\ruby{音声}{おんせい}\ruby{操作}{そうさ}、\ruby{物理}{ぶつり}\ruby{ボタン}{ぼたん}、または\ruby{簡潔}{かんけつ}な\ruby{インタラクション}{いんたらくしょん}が\ruby{中心}{ちゅうしん}となる。この\ruby{要件}{ようけん}は、\ruby{個人}{こじん}\ruby{端末}{たんまつ}向けのAndroidと\ruby{比}{くら}べて、\ruby{大幅}{おおはば}な\ruby{調整}{ちょうせい}を\ruby{求}{もと}める。

Về mặt kỹ thuật, Android Automotive phải đáp ứng các yêu cầu rất khắt khe về độ ổn định và độ tin cậy. Hệ thống trên ô tô được kỳ vọng hoạt động liên tục trong thời gian dài, chịu được các điều kiện môi trường khắc nghiệt như nhiệt độ cao, rung động và nhiễu điện từ. Bất kỳ lỗi phần mềm nào cũng có thể ảnh hưởng trực tiếp đến an toàn của người sử dụng. Do đó, Android Automotive được xây dựng với các cơ chế kiểm soát nghiêm ngặt, tách biệt rõ ràng giữa các thành phần giải trí và các hệ thống quan trọng liên quan đến vận hành xe.

\ruby{技術的}{ぎじゅつてき}には、Android Automotiveは\ruby{極}{きわ}めて\ruby{厳格}{げんかく}な\ruby{安定性}{あんていせい}と\ruby{信頼性}{しんらいせい}の\ruby{要件}{ようけん}を\ruby{満}{み}たす\ruby{必要}{ひつよう}がある。\ruby{車載}{しゃさい}\ruby{システム}{しすてむ}は\ruby{長時間}{ちょうじかん}の\ruby{連続}{れんぞく}\ruby{稼働}{かどう}が\ruby{期待}{きたい}され、\ruby{高温}{こうおん}、\ruby{振動}{しんどう}、\ruby{電磁}{でんじ}\ruby{ノイズ}{のいず}といった\ruby{過酷}{かこく}な\ruby{環境}{かんきょう}に\ruby{耐}{た}えなければならない。いかなる\ruby{ソフトウェア}{そふとうぇあ}\ruby{不具合}{ふぐあい}も、\ruby{利用者}{りようしゃ}の\ruby{安全}{あんぜん}に\ruby{直接}{ちょくせつ}\ruby{影響}{えいきょう}しうる。そのため、Android Automotiveは、\ruby{娯楽}{ごらく}\ruby{系}{けい}の\ruby{機能}{きのう}と\ruby{走行}{そうこう}に\ruby{関}{かん}わる\ruby{重要}{じゅうよう}\ruby{システム}{しすてむ}を\ruby{明確}{めいかく}に\ruby{分離}{ぶんり}する\ruby{厳密}{げんみつ}な\ruby{制御}{せいぎょ}\ruby{機構}{きこう}のもとで\ruby{構築}{こうちく}されている。

Một lợi thế lớn của Android Automotive là khả năng cập nhật phần mềm qua mạng. Trong khi các hệ thống ô tô truyền thống thường bị cố định theo vòng đời sản phẩm, Android Automotive cho phép nhà sản xuất cập nhật tính năng, vá lỗi và cải thiện trải nghiệm người dùng sau khi xe đã được bán ra thị trường. Điều này làm thay đổi mô hình phát triển sản phẩm trong ngành ô tô, chuyển từ tư duy phần cứng cố định sang phần mềm có thể tiến hóa theo thời gian, tương tự như các thiết bị công nghệ tiêu dùng.

Android Automotiveの\ruby{大}{おお}きな\ruby{利点}{りてん}の一つは、\ruby{ネットワーク}{ねっとわーく}を\ruby{通}{つう}じた\ruby{ソフトウェア}{そふとうぇあ}\ruby{更新}{こうしん}が\ruby{可能}{かのう}である\ruby{点}{てん}である。\ruby{従来}{じゅうらい}の\ruby{自動車}{じどうしゃ}\ruby{システム}{しすてむ}が\ruby{製品}{せいひん}\ruby{寿命}{じゅみょう}に\ruby{固定}{こてい}されがちだったのに\ruby{対}{たい}し、Android Automotiveは\ruby{販売}{はんばい}\ruby{後}{ご}でも\ruby{機能}{きのう}\ruby{追加}{ついか}、\ruby{不具合}{ふぐあい}\ruby{修正}{しゅうせい}、\ruby{体験}{たいけん}\ruby{改善}{かいぜん}を\ruby{可能}{かのう}にする。これは、\ruby{自動車}{じどうしゃ}\ruby{産業}{さんぎょう}における\ruby{製品}{せいひん}\ruby{開発}{かいはつ}の\ruby{発想}{はっそう}を、\ruby{固定}{こてい}\ruby{的}{てき}な\ruby{ハードウェア}{はーどうぇあ}から、\ruby{時間}{じかん}とともに\ruby{進化}{しんか}する\ruby{ソフトウェア}{そふとうぇあ}へと\ruby{転換}{てんかん}させる。

Từ góc độ hệ sinh thái, Android Automotive mở ra một không gian mới cho các nhà phát triển ứng dụng. Các ứng dụng dẫn đường, giải trí, thông tin giao thông và dịch vụ xe có thể được triển khai trực tiếp trên hệ thống của xe, tận dụng dữ liệu cảm biến và khả năng kết nối. Đồng thời, việc tích hợp tài khoản người dùng giúp cá nhân hóa trải nghiệm lái xe, từ cấu hình giao diện đến gợi ý nội dung và dịch vụ phù hợp với thói quen sử dụng.

\ruby{エコシステム}{えこしすてむ}の\ruby{観点}{かんてん}では、Android Automotiveは\ruby{開発者}{かいはつしゃ}に\ruby{新}{あたら}しい\ruby{領域}{りょういき}を\ruby{提供}{ていきょう}する。\ruby{ナビゲーション}{なびげーしょん}、\ruby{娯楽}{ごらく}、\ruby{交通}{こうつう}\ruby{情報}{じょうほう}、\ruby{車両}{しゃりょう}\ruby{サービス}{さーびす}といった\ruby{アプリケーション}{あぷりけーしょん}は、\ruby{車両}{しゃりょう}\ruby{システム}{しすてむ}上で\ruby{直接}{ちょくせつ}\ruby{動作}{どうさ}し、\ruby{センサー}{せんさー}\ruby{データ}{でーた}や\ruby{接続}{せつぞく}\ruby{機能}{きのう}を\ruby{活用}{かつよう}できる。さらに、\ruby{利用者}{りようしゃ}\ruby{アカウント}{あかうんと}の\ruby{統合}{とうごう}により、\ruby{画面}{がめん}\ruby{設定}{せってい}から\ruby{内容}{ないよう}や\ruby{サービス}{さーびす}の\ruby{提案}{ていあん}に\ruby{至}{いた}るまで、\ruby{個別}{こべつ}\ruby{最適化}{さいてきか}された\ruby{運転}{うんてん}\ruby{体験}{たいけん}が\ruby{実現}{じつげん}される。

Tuy nhiên, việc đưa Android vào ô tô cũng đặt ra nhiều thách thức về pháp lý và tiêu chuẩn. Ngành công nghiệp ô tô có chu kỳ phát triển dài và chịu sự quản lý chặt chẽ của các quy định an toàn. Android Automotive phải thích ứng với các tiêu chuẩn này, đồng thời đảm bảo khả năng tùy biến cho từng nhà sản xuất xe. Điều này khiến quá trình triển khai Android Automotive phức tạp hơn so với các thiết bị Android truyền thống.

しかし、Androidを\ruby{自動車}{じどうしゃ}に\ruby{導入}{どうにゅう}することは、\ruby{法規}{ほうき}および\ruby{標準}{ひょうじゅん}の\ruby{面}{めん}で\ruby{多}{おお}くの\ruby{課題}{かだい}を\ruby{伴}{ともな}う。\ruby{自動車}{じどうしゃ}\ruby{産業}{さんぎょう}は\ruby{開発}{かいはつ}\ruby{周期}{しゅうき}が\ruby{長}{なが}く、\ruby{安全}{あんぜん}\ruby{規制}{きせい}のもとで\ruby{厳}{きび}しく\ruby{管理}{かんり}されている。Android Automotiveは、これらの\ruby{基準}{きじゅん}に\ruby{適合}{てきごう}しつつ、\ruby{各}{かく}\ruby{自動車}{じどうしゃ}\ruby{メーカー}{めーかー}ごとの\ruby{カスタマイズ}{かすたまいず}も\ruby{確保}{かくほ}しなければならない。この\ruby{点}{てん}で、Android Automotiveの\ruby{展開}{てんかい}は\ruby{従来}{じゅうらい}のAndroid\ruby{端末}{たんまつ}よりも\ruby{複雑}{ふくざつ}である。

Tóm lại, Android Automotive thể hiện bước mở rộng táo bạo của Android sang lĩnh vực giao thông thông minh. Việc tích hợp hệ điều hành vào ô tô không chỉ nâng cao trải nghiệm người dùng mà còn tái định hình vai trò của phần mềm trong ngành công nghiệp xe hơi. Android, trong bối cảnh này, trở thành nền tảng kết nối giữa con người, phương tiện và các dịch vụ số trong một hệ sinh thái ngày càng thông minh và liên thông.

\ruby{総括}{そうかつ}すると、Android Automotiveは、Androidが\ruby{知能}{ちのう}\ruby{交通}{こうつう}の\ruby{分野}{ぶんや}へ\ruby{踏}{ふ}み\ruby{出}{だ}した\ruby{大胆}{だいたん}な\ruby{拡張}{かくちょう}を\ruby{示}{しめ}している。\ruby{オペレーティングシステム}{おぺれーてぃんぐしすてむ}を\ruby{自動車}{じどうしゃ}に\ruby{統合}{とうごう}することは、\ruby{利用者}{りようしゃ}\ruby{体験}{たいけん}を\ruby{向上}{こうじょう}させるだけでなく、\ruby{自動車}{じどうしゃ}\ruby{産業}{さんぎょう}における\ruby{ソフトウェア}{そふとうぇあ}の\ruby{位置}{いち}づけを\ruby{再定義}{さいていぎ}する。この\ruby{文脈}{ぶんみゃく}において、Androidは、\ruby{人}{ひと}、\ruby{車両}{しゃりょう}、および\ruby{デジタル}{でじたる}\ruby{サービス}{さーびす}を\ruby{結}{むす}ぶ\ruby{中核}{ちゅうかく}\ruby{基盤}{きばん}として、ますます\ruby{高度}{こうど}で\ruby{相互}{そうご}\ruby{接続}{せつぞく}された\ruby{エコシステム}{えこしすてむ}を\ruby{形成}{けいせい}していく。

\section{Android và IoT: vai trò trong hệ sinh thái đa thiết bị và kết nối thông minh}
\ruby{Android}{あんどろいど}とIoT:\ruby{多}{た}\ruby{機器}{きき}\ruby{生態系}{せいたいけい}および\ruby{知的}{ちてき}\ruby{接続}{せつぞく}における\ruby{役割}{やくわり}

Internet of Things (IoT) đánh dấu giai đoạn mà các thiết bị vật lý được kết nối, trao đổi dữ liệu và phối hợp hoạt động thông qua mạng. Trong bối cảnh đó, Android không còn chỉ là hệ điều hành chạy trên các thiết bị đầu cuối như điện thoại hay tablet, mà dần trở thành một thành phần trung tâm trong hệ sinh thái đa thiết bị. Vai trò của Android trong IoT thể hiện rõ ở khả năng kết nối, điều phối và cung cấp giao diện tương tác giữa con người với thế giới thiết bị thông minh.

Internet of Things(IoT)は、\ruby{物理}{ぶつり}\ruby{機器}{きき}が\ruby{ネットワーク}{ねっとわーく}を\ruby{通}{とお}じて\ruby{接続}{せつぞく}され、\ruby{データ}{でーた}を\ruby{交換}{こうかん}し、\ruby{協調}{きょうちょう}して\ruby{動作}{どうさ}する\ruby{段階}{だんかい}を\ruby{示}{しめ}す。この\ruby{文脈}{ぶんみゃく}において、Androidは\ruby{電話}{でんわ}や\ruby{タブレット}{たぶれっと}といった\ruby{端末}{たんまつ}のみで\ruby{動作}{どうさ}する\ruby{オペレーティングシステム}{おぺれーてぃんぐしすてむ}にとどまらず、\ruby{多}{た}\ruby{機器}{きき}\ruby{生態系}{せいたいけい}の\ruby{中核}{ちゅうかく}を\ruby{担}{にな}う\ruby{存在}{そんざい}となりつつある。IoTにおけるAndroidの\ruby{役割}{やくわり}は、\ruby{接続}{せつぞく}、\ruby{制御}{せいぎょ}、および\ruby{人}{ひと}と\ruby{知的}{ちてき}\ruby{機器}{きき}との\ruby{相互}{そうご}\ruby{作用}{さよう}を\ruby{担保}{たんぽ}する\ruby{インターフェース}{いんたーふぇーす}の\ruby{提供}{ていきょう}に\ruby{明確}{めいかく}に\ruby{表}{あらわ}れている。

Trong hệ sinh thái IoT, Android thường không trực tiếp chạy trên các thiết bị siêu nhỏ có tài nguyên hạn chế, mà đảm nhiệm vai trò trung tâm điều khiển và tương tác. Smartphone, tablet, TV hoặc màn hình thông minh chạy Android trở thành điểm truy cập chính để người dùng giám sát, cấu hình và điều khiển các thiết bị IoT như đèn chiếu sáng, khóa cửa, cảm biến môi trường hay thiết bị gia dụng thông minh. Android vì vậy đóng vai trò như lớp giao diện và điều phối, kết nối người dùng với một mạng lưới thiết bị đa dạng phía sau.

IoTの\ruby{生態系}{せいたいけい}において、Androidは\ruby{資源}{しげん}が\ruby{制限}{せいげん}された\ruby{超}{ちょう}\ruby{小型}{こがた}\ruby{機器}{きき}に\ruby{直接}{ちょくせつ}\ruby{搭載}{とうさい}されることは\ruby{少}{すく}なく、\ruby{制御}{せいぎょ}および\ruby{対話}{たいわ}\ruby{中枢}{ちゅうすう}としての\ruby{役割}{やくわり}を\ruby{担}{にな}うことが\ruby{多}{おお}い。Androidを\ruby{搭載}{とうさい}した\ruby{スマートフォン}{すまーとふぉん}、\ruby{タブレット}{たぶれっと}、\ruby{テレビ}{てれび}、\ruby{知的}{ちてき}\ruby{ディスプレイ}{でぃすぷれい}は、\ruby{照明}{しょうめい}、\ruby{施錠}{せじょう}\ruby{装置}{そうち}、\ruby{環境}{かんきょう}\ruby{センサー}{せんさー}、\ruby{家電}{かでん}\ruby{機器}{きき}といったIoT\ruby{装置}{そうち}を\ruby{監視}{かんし}・\ruby{設定}{せってい}・\ruby{操作}{そうさ}するための\ruby{主要}{しゅよう}な\ruby{接点}{せってん}となる。したがってAndroidは、\ruby{利用者}{りようしゃ}と\ruby{背後}{はいご}に\ruby{存在}{そんざい}する\ruby{多様}{たよう}な\ruby{機器}{きき}\ruby{群}{ぐん}を\ruby{結}{むす}びつける\ruby{インターフェース}{いんたーふぇーす}および\ruby{調停}{ちょうてい}\ruby{層}{そう}として\ruby{機能}{きのう}する。

Một lợi thế lớn của Android trong lĩnh vực IoT là khả năng tích hợp sâu với các dịch vụ đám mây và hạ tầng dữ liệu. Thông qua tài khoản người dùng và các dịch vụ trực tuyến, Android cho phép đồng bộ trạng thái thiết bị, lịch sử hoạt động và các kịch bản tự động hóa trên nhiều thiết bị khác nhau. Người dùng có thể điều khiển hệ thống nhà thông minh từ xa, theo dõi dữ liệu theo thời gian thực và nhận cảnh báo khi có sự kiện bất thường. Khả năng kết nối này giúp Android trở thành cầu nối giữa thiết bị IoT cục bộ và hệ sinh thái dịch vụ số quy mô lớn.

IoT\ruby{分野}{ぶんや}におけるAndroidの\ruby{大}{おお}きな\ruby{利点}{りてん}の\ruby{一}{ひと}つは、\ruby{クラウド}{くらうど}\ruby{サービス}{さーびす}および\ruby{データ}{でーた}\ruby{基盤}{きばん}との\ruby{深}{ふか}い\ruby{統合}{とうごう}にある。\ruby{利用者}{りようしゃ}\ruby{アカウント}{あかうんと}や\ruby{オンライン}{おんらいん}\ruby{サービス}{さーびす}を\ruby{通}{とお}じて、Androidは\ruby{機器}{きき}の\ruby{状態}{じょうたい}、\ruby{動作}{どうさ}\ruby{履歴}{りれき}、および\ruby{自動化}{じどうか}\ruby{シナリオ}{しなりお}を\ruby{複数}{ふくすう}の\ruby{端末}{たんまつ}で\ruby{同期}{どうき}できる。\ruby{利用者}{りようしゃ}は\ruby{遠隔}{えんかく}から\ruby{スマート}{すまーと}\ruby{ホーム}{ほーむ}を\ruby{制御}{せいぎょ}し、\ruby{リアルタイム}{りあるたいむ}で\ruby{データ}{でーた}を\ruby{監視}{かんし}し、\ruby{異常}{いじょう}\ruby{発生}{はっせい}\ruby{時}{じ}に\ruby{通知}{つうち}を\ruby{受}{う}け\ruby{取}{と}ることができる。この\ruby{接続}{せつぞく}\ruby{能力}{のうりょく}により、Androidは\ruby{局所的}{きょくしょてき}なIoT\ruby{装置}{そうち}と\ruby{大規模}{だいきぼ}な\ruby{デジタル}{でじたる}\ruby{サービス}{さーびす}\ruby{生態系}{せいたいけい}を\ruby{結}{むす}ぶ\ruby{橋渡}{はしわた}し\ruby{役}{やく}となる。

Android cũng đóng vai trò quan trọng trong việc chuẩn hóa trải nghiệm người dùng trong môi trường IoT. Thay vì mỗi thiết bị có một giao diện và cách vận hành riêng, Android cung cấp một mô hình tương tác thống nhất, giúp người dùng dễ tiếp cận và quản lý hệ thống phức tạp. Các ứng dụng IoT trên Android có thể chia sẻ nguyên tắc thiết kế, cơ chế cấp quyền và mô hình bảo mật, từ đó giảm độ phức tạp và tăng tính nhất quán trong trải nghiệm sử dụng.

Androidは、IoT\ruby{環境}{かんきょう}における\ruby{利用者}{りようしゃ}\ruby{体験}{たいけん}の\ruby{標準化}{ひょうじゅんか}にも\ruby{重要}{じゅうよう}な\ruby{役割}{やくわり}を\ruby{果}{は}たす。\ruby{各}{かく}\ruby{機器}{きき}が\ruby{個別}{こべつ}の\ruby{操作}{そうさ}\ruby{方法}{ほうほう}や\ruby{インターフェース}{いんたーふぇーす}を\ruby{持}{も}つのではなく、Androidは\ruby{統一}{とういつ}された\ruby{対話}{たいわ}\ruby{モデル}{もでる}を\ruby{提供}{ていきょう}し、\ruby{複雑}{ふくざつ}な\ruby{システム}{しすてむ}の\ruby{管理}{かんり}を\ruby{容易}{ようい}にする。Android上のIoT\ruby{アプリケーション}{あぷりけーしょん}は、\ruby{設計}{せっけい}\ruby{原則}{げんそく}、\ruby{権限}{けんげん}\ruby{付与}{ふよ}の\ruby{仕組}{しく}み、\ruby{セキュリティ}{せきゅりてぃ}\ruby{モデル}{もでる}を\ruby{共有}{きょうゆう}でき、\ruby{複雑性}{ふくざつせい}の\ruby{低減}{ていげん}と\ruby{一貫性}{いっかんせい}の\ruby{向上}{こうじょう}に\ruby{寄与}{きよ}する。

Tuy nhiên, Android không phải là giải pháp phù hợp cho mọi lớp thiết bị IoT. Các thiết bị yêu cầu thời gian thực cao, tiêu thụ năng lượng cực thấp hoặc có phần cứng rất hạn chế thường sử dụng các hệ điều hành nhúng chuyên biệt. Trong những trường hợp này, Android tham gia gián tiếp, đóng vai trò như trung tâm điều khiển và hiển thị, thay vì là nền tảng chạy trực tiếp trên thiết bị. Cách tiếp cận này phản ánh chiến lược linh hoạt của Android trong hệ sinh thái IoT, nơi nhiều nền tảng cùng tồn tại và bổ trợ lẫn nhau.

しかし、Androidは\ruby{全}{すべ}てのIoT\ruby{機器}{きき}に\ruby{適}{てき}した\ruby{解決策}{かいけつさく}ではない。\ruby{高}{たか}い\ruby{リアルタイム}{りあるたいむ}\ruby{性}{せい}、\ruby{極低}{きょくてい}\ruby{消費}{しょうひ}\ruby{電力}{でんりょく}、または\ruby{著}{いちじる}しく\ruby{制限}{せいげん}された\ruby{ハードウェア}{はーどうぇあ}を\ruby{要}{よう}する\ruby{装置}{そうち}では、\ruby{専用}{せんよう}の\ruby{組込}{くみこみ}\ruby{オペレーティングシステム}{おぺれーてぃんぐしすてむ}が\ruby{採用}{さいよう}されることが\ruby{多}{おお}い。このような\ruby{場合}{ばあい}、Androidは\ruby{直接}{ちょくせつ}\ruby{動作}{どうさ}する\ruby{基盤}{きばん}ではなく、\ruby{制御}{せいぎょ}および\ruby{表示}{ひょうじ}の\ruby{中枢}{ちゅうすう}として\ruby{間接的}{かんせつてき}に\ruby{関与}{かんよ}する。この\ruby{柔軟}{じゅうなん}な\ruby{役割}{やくわり}\ruby{分担}{ぶんたん}は、\ruby{複数}{ふくすう}の\ruby{基盤}{きばん}が\ruby{共存}{きょうぞん}し\ruby{相互}{そうご}に\ruby{補完}{ほかん}するIoT\ruby{生態系}{せいたいけい}におけるAndroidの\ruby{戦略}{せんりゃく}を\ruby{反映}{はんえい}している。

Từ góc độ chiến lược hệ sinh thái, sự hiện diện của Android trong IoT giúp Google mở rộng ảnh hưởng sang các không gian vật lý như nhà ở, văn phòng và đô thị thông minh. Android trở thành điểm hội tụ của dữ liệu, dịch vụ và trải nghiệm người dùng, kết nối các thiết bị rời rạc thành một hệ thống thông minh và có khả năng học hỏi theo hành vi sử dụng. Điều này không chỉ gia tăng giá trị của từng thiết bị riêng lẻ, mà còn củng cố vai trò trung tâm của Android trong đời sống số hàng ngày.

\ruby{生態系}{せいたいけい}\ruby{戦略}{せんりゃく}の\ruby{視点}{してん}から、IoTにおけるAndroidの\ruby{存在}{そんざい}は、Googleの\ruby{影響力}{えいきょうりょく}を\ruby{住宅}{じゅうたく}、\ruby{オフィス}{おふぃす}、\ruby{知的}{ちてき}\ruby{都市}{とし}といった\ruby{物理}{ぶつり}\ruby{空間}{くうかん}へ\ruby{拡張}{かくちょう}することを\ruby{可能}{かのう}にする。Androidは\ruby{データ}{でーた}、\ruby{サービス}{さーびす}、および\ruby{利用者}{りようしゃ}\ruby{体験}{たいけん}の\ruby{結節点}{けっせつてん}となり、\ruby{分散}{ぶんさん}した\ruby{機器}{きき}を\ruby{知的}{ちてき}で\ruby{学習}{がくしゅう}\ruby{可能}{かのう}な\ruby{システム}{しすてむ}へと\ruby{統合}{とうごう}する。これは\ruby{個々}{ここ}の\ruby{装置}{そうち}の\ruby{価値}{かち}を\ruby{高}{たか}めるだけでなく、\ruby{日常}{にちじょう}\ruby{生活}{せいかつ}におけるAndroidの\ruby{中核的}{ちゅうかくてき}\ruby{地位}{ちい}を\ruby{強化}{きょうか}する。

Tóm lại, trong kỷ nguyên IoT, Android giữ vai trò then chốt như một nền tảng kết nối và điều phối trong hệ sinh thái đa thiết bị. Dù không hiện diện trực tiếp trên mọi thiết bị IoT, Android vẫn đóng vai trò trung tâm trong việc kết nối con người với thế giới thiết bị thông minh, góp phần hiện thực hóa tầm nhìn về một môi trường số liền mạch, thông minh và ngày càng gắn bó với đời sống thực.

\ruby{総括}{そうかつ}すると、IoT\ruby{時代}{じだい}においてAndroidは、\ruby{多}{た}\ruby{機器}{きき}\ruby{生態系}{せいたいけい}における\ruby{接続}{せつぞく}および\ruby{調整}{ちょうせい}の\ruby{要}{かなめ}となる\ruby{基盤}{きばん}である。IoT\ruby{装置}{そうち}の\ruby{全}{すべ}てに\ruby{直接}{ちょくせつ}\ruby{存在}{そんざい}するわけではないが、Androidは\ruby{人}{ひと}と\ruby{知的}{ちてき}\ruby{機器}{きき}の\ruby{世界}{せかい}を\ruby{結}{むす}ぶ\ruby{中心}{ちゅうしん}として\ruby{機能}{きのう}し、\ruby{連続的}{れんぞくてき}で\ruby{知的}{ちてき}な\ruby{デジタル}{でじたる}\ruby{環境}{かんきょう}の\ruby{実現}{じつげん}に\ruby{寄与}{きよ}している。
