\chapter{Android Inc. và thương vụ với Google}
\ruby{Android}{あんどろいど}\ruby{社}{しゃ}とGoogleによる\ruby{買収}{ばいしゅう}

Sự ra đời của Android không phải là một câu chuyện ngẫu nhiên hay bộc phát, mà là kết quả của những thay đổi sâu sắc trong ngành công nghiệp thiết bị di động đầu những năm 2000. Trước khi Android trở thành hệ điều hành phổ biến nhất thế giới, nó khởi đầu như một công ty nhỏ với tầm nhìn kỹ thuật khác biệt, đi ngược lại tư duy thống trị của thị trường lúc bấy giờ.

\ruby{Android}{あんどろいど}の\ruby{誕生}{たんじょう}は\ruby{偶然}{ぐうぜん}や\ruby{突発的}{とっぱつてき}な\ruby{出来事}{できごと}ではなく、2000\ruby{年代}{ねんだい}\ruby{初頭}{しょとう}における\ruby{モバイル}{もばいる}\ruby{端末}{たんまつ}\ruby{産業}{さんぎょう}の\ruby{構造的}{こうぞうてき}な\ruby{変化}{へんか}の\ruby{結果}{けっか}である。\ruby{世界}{せかい}で\ruby{最}{もっと}も\ruby{普及}{ふきゅう}した\ruby{オペレーティング}{おぺれーてぃんぐ}\ruby{システム}{しすてむ}となる\ruby{以前}{いぜん}、Androidは\ruby{当時}{とうじ}の\ruby{市場}{しじょう}を\ruby{支配}{しはい}していた\ruby{発想}{はっそう}とは\ruby{異}{こと}なる\ruby{技術的}{ぎじゅつてき}\ruby{ビジョン}{びじょん}を\ruby{持}{も}つ\ruby{小規模}{しょうきぼ}な\ruby{企業}{きぎょう}として\ruby{出発}{しゅっぱつ}した。

\section{Sự ra đời của Android Inc.: mục tiêu ban đầu, đội ngũ sáng lập và định hướng xây dựng hệ điều hành cho thiết bị di động}
\ruby{Android}{あんどろいど}\ruby{社}{しゃ}の\ruby{設立}{せつりつ}:\ruby{初期}{しょき}\ruby{目標}{もくひょう}、\ruby{創業}{そうぎょう}\ruby{メンバー}{めんばー}、および\ruby{モバイル}{もばいる}\ruby{OS}{おーえす}\ruby{構想}{こうそう}

Android Inc. được thành lập vào năm 2003 tại Palo Alto, California, trong bối cảnh thị trường thiết bị di động đang bị phân mảnh và kiểm soát chặt chẽ bởi một số hệ điều hành đóng. Thời điểm này, Symbian chiếm ưu thế trên điện thoại phổ thông, Palm OS thống trị PDA, còn Windows Mobile hướng đến nhóm khách hàng doanh nghiệp. Điểm chung của các nền tảng này là tính đóng, khả năng tùy biến hạn chế và phụ thuộc nặng nề vào nhà sản xuất hoặc nhà mạng.

Android Inc.は2003\ruby{年}{ねん}にカリフォルニア\ruby{州}{しゅう}パロアルトで\ruby{設立}{せつりつ}された。\ruby{当時}{とうじ}の\ruby{モバイル}{もばいる}\ruby{端末}{たんまつ}\ruby{市場}{しじょう}は\ruby{分断}{ぶんだん}され、\ruby{少数}{しょうすう}の\ruby{閉鎖的}{へいさてき}な\ruby{OS}{おーえす}によって\ruby{厳}{きび}しく\ruby{支配}{しはい}されていた。Symbianは\ruby{一般}{いっぱん}\ruby{携帯}{けいたい}\ruby{電話}{でんわ}で\ruby{優位}{ゆうい}を\ruby{占}{し}め、Palm OSはPDA\ruby{分野}{ぶんや}を\ruby{統治}{とうち}し、Windows Mobileは\ruby{企業}{きぎょう}\ruby{向}{む}け\ruby{市場}{しじょう}を\ruby{対象}{たいしょう}としていた。これらの\ruby{共通点}{きょうつうてん}は、\ruby{閉鎖性}{へいさせい}、\ruby{限定的}{げんていてき}な\ruby{カスタマイズ}{かすたまいず}\ruby{能力}{のうりょく}、そして\ruby{製造}{せいぞう}\ruby{業者}{ぎょうしゃ}や\ruby{通信}{つうしん}\ruby{事業者}{じぎょうしゃ}への\ruby{強}{つよ}い\ruby{依存}{いぞん}であった。

Mục tiêu ban đầu của Android Inc. không trực tiếp nhắm đến điện thoại thông minh. Ý tưởng khởi đầu là xây dựng một hệ điều hành thông minh cho các thiết bị điện tử cầm tay có kết nối mạng, trước hết là máy ảnh kỹ thuật số. Nhóm sáng lập nhận định rằng các thiết bị phần cứng sẽ ngày càng cần kết nối Internet, đồng bộ dữ liệu và khả năng mở rộng phần mềm, nhưng thị trường lúc đó chưa có một hệ điều hành chung đáp ứng được yêu cầu này.

Android Inc.の\ruby{初期}{しょき}\ruby{目標}{もくひょう}は、\ruby{直接}{ちょくせつ}\ruby{スマートフォン}{すまーとふぉん}を\ruby{対象}{たいしょう}とするものではなかった。\ruby{当初}{とうしょ}の\ruby{構想}{こうそう}は、\ruby{ネットワーク}{ねっとわーく}\ruby{接続}{せつぞく}を\ruby{備}{そな}えた\ruby{携帯}{けいたい}\ruby{電子}{でんし}\ruby{機器}{きき}、とりわけ\ruby{デジタル}{でじたる}\ruby{カメラ}{かめら}のための\ruby{知的}{ちてき}な\ruby{OS}{おーえす}を\ruby{構築}{こうちく}することであった。\ruby{創業}{そうぎょう}\ruby{チーム}{ちーむ}は、\ruby{ハードウェア}{はーどうぇあ}が\ruby{今後}{こんご}ますます\ruby{インターネット}{いんたーねっと}\ruby{接続}{せつぞく}、\ruby{データ}{でーた}\ruby{同期}{どうき}、および\ruby{ソフトウェア}{そふとうぇあ}\ruby{拡張}{かくちょう}\ruby{能力}{のうりょく}を\ruby{必要}{ひつよう}とすると\ruby{判断}{はんだん}していたが、\ruby{当時}{とうじ}の\ruby{市場}{しじょう}にはこれらの\ruby{要件}{ようけん}を\ruby{満}{み}たす\ruby{共通}{きょうつう}\ruby{OS}{おーえす}が\ruby{存在}{そんざい}していなかった。

Đội ngũ sáng lập Android Inc. gồm những cá nhân có nền tảng kỹ thuật và kinh nghiệm thực tiễn trong lĩnh vực thiết bị di động và viễn thông. Andy Rubin đóng vai trò trung tâm về kiến trúc hệ điều hành, từng tham gia phát triển các thiết bị di động có kết nối Internet từ rất sớm. Các đồng sáng lập khác mang đến góc nhìn về sản phẩm, thị trường và mối quan hệ với nhà mạng. Sự kết hợp này giúp Android Inc. không chỉ là một dự án kỹ thuật thuần túy, mà còn gắn với thực tế triển khai trên thị trường.

Android Inc.の\ruby{創業}{そうぎょう}\ruby{メンバー}{めんばー}は、\ruby{モバイル}{もばいる}\ruby{端末}{たんまつ}および\ruby{通信}{つうしん}\ruby{分野}{ぶんや}における\ruby{技術的}{ぎじゅつてき}\ruby{背景}{はいけい}と\ruby{実務}{じつむ}\ruby{経験}{けいけん}を\ruby{有}{ゆう}する\ruby{人材}{じんざい}で\ruby{構成}{こうせい}されていた。Andy Rubinは\ruby{OS}{おーえす}\ruby{アーキテクチャ}{あーきてくちゃ}の\ruby{中核}{ちゅうかく}を\ruby{担}{にな}い、\ruby{早期}{そうき}から\ruby{インターネット}{いんたーねっと}\ruby{接続}{せつぞく}を\ruby{備}{そな}えた\ruby{モバイル}{もばいる}\ruby{端末}{たんまつ}の\ruby{開発}{かいはつ}に\ruby{関与}{かんよ}していた。\ruby{他}{ほか}の\ruby{共同}{きょうどう}\ruby{創業}{そうぎょう}\ruby{者}{しゃ}は、\ruby{製品}{せいひん}、\ruby{市場}{しじょう}、および\ruby{通信}{つうしん}\ruby{事業者}{じぎょうしゃ}との\ruby{関係}{かんけい}に\ruby{関}{かん}する\ruby{視点}{してん}を\ruby{提供}{ていきょう}した。この\ruby{組}{く}み\ruby{合}{あ}わせにより、Android Inc.は\ruby{純粋}{じゅんすい}な\ruby{技術}{ぎじゅつ}\ruby{プロジェクト}{ぷろじぇくと}にとどまらず、\ruby{市場}{しじょう}での\ruby{実装}{じっそう}を\ruby{見据}{みす}えた\ruby{存在}{そんざい}となった。

Về mặt kỹ thuật, Android Inc. sớm xác định Linux là nền tảng cốt lõi. Việc lựa chọn nhân Linux không chỉ vì yếu tố chi phí bản quyền bằng không, mà quan trọng hơn là khả năng tùy biến, tính ổn định và cộng đồng phát triển rộng lớn. Trái với các hệ điều hành di động đóng thời bấy giờ, Android được thiết kế để cho phép nhà sản xuất phần cứng can thiệp sâu vào hệ thống, điều chỉnh theo nhiều cấu hình thiết bị khác nhau.

\ruby{技術的}{ぎじゅつてき}には、Android Inc.は\ruby{早期}{そうき}にLinuxを\ruby{中核}{ちゅうかく}\ruby{基盤}{きばん}として\ruby{採用}{さいよう}した。Linux\ruby{カーネル}{かーねる}の\ruby{選択}{せんたく}は、\ruby{無償}{むしょう}の\ruby{ライセンス}{らいせんす}\ruby{費用}{ひよう}という\ruby{理由}{りゆう}にとどまらず、\ruby{高}{たか}い\ruby{柔軟性}{じゅうなんせい}、\ruby{安定性}{あんていせい}、および\ruby{広範}{こうはん}な\ruby{開発}{かいはつ}\ruby{コミュニティ}{こみゅにてぃ}を\ruby{重視}{じゅうし}したためである。\ruby{当時}{とうじ}の\ruby{閉鎖的}{へいさてき}な\ruby{モバイル}{もばいる}\ruby{OS}{おーえす}とは\ruby{対照的}{たいしょうてき}に、Androidは\ruby{ハードウェア}{はーどうぇあ}\ruby{メーカー}{めーかー}が\ruby{システム}{しすてむ}に\ruby{深}{ふか}く\ruby{介入}{かいにゅう}し、\ruby{多様}{たよう}な\ruby{構成}{こうせい}に\ruby{適応}{てきおう}できるよう\ruby{設計}{せっけい}された。

Một định hướng quan trọng khác là tách hệ điều hành khỏi phần cứng cụ thể. Android không được xây dựng cho một thiết bị duy nhất, mà hướng đến khả năng chạy trên nhiều loại phần cứng với cấu hình đa dạng. Đây là tư duy mang tính nền tảng, đặt tiền đề cho việc mở rộng quy mô trong tương lai, dù ở thời điểm đó Android Inc. chưa có đối tác sản xuất cụ thể.

もう\ruby{一}{ひと}つの\ruby{重要}{じゅうよう}な\ruby{方針}{ほうしん}は、\ruby{OS}{おーえす}を\ruby{特定}{とくてい}の\ruby{ハードウェア}{はーどうぇあ}から\ruby{切}{き}り\ruby{離}{はな}すことであった。Androidは\ruby{単一}{たんいつ}の\ruby{端末}{たんまつ}のために\ruby{構築}{こうちく}されたのではなく、\ruby{多様}{たよう}な\ruby{構成}{こうせい}を\ruby{持}{も}つ\ruby{複数}{ふくすう}の\ruby{ハードウェア}{はーどうぇあ}での\ruby{動作}{どうさ}を\ruby{想定}{そうてい}していた。この\ruby{基盤的}{きばんてき}\ruby{発想}{はっそう}は、\ruby{将来}{しょうらい}の\ruby{規模}{きぼ}\ruby{拡大}{かくだい}に\ruby{向}{む}けた\ruby{前提}{ぜんてい}を\ruby{形成}{けいせい}したが、\ruby{当時}{とうじ}のAndroid Inc.には\ruby{具体的}{ぐたいてき}な\ruby{製造}{せいぞう}\ruby{パートナー}{ぱーとなー}は\ruby{存在}{そんざい}していなかった。

Tuy nhiên, Android Inc. cũng đối mặt với nhiều thách thức nghiêm trọng. Thị trường máy ảnh kỹ thuật số nhanh chóng bão hòa, trong khi smartphone bắt đầu manh nha nhưng chưa rõ hình hài. Công ty không có sản phẩm thương mại hoàn chỉnh, không có nguồn doanh thu ổn định và phụ thuộc vào vốn đầu tư bên ngoài. Trong bối cảnh đó, Android Inc. buộc phải điều chỉnh hướng đi, chuyển trọng tâm sang hệ điều hành cho điện thoại di động, nơi tiềm năng kết nối Internet và dịch vụ trực tuyến lớn hơn nhiều.

しかし、Android Inc.は\ruby{深刻}{しんこく}な\ruby{課題}{かだい}にも\ruby{直面}{ちょくめん}していた。\ruby{デジタル}{でじたる}\ruby{カメラ}{かめら}\ruby{市場}{しじょう}は\ruby{急速}{きゅうそく}に\ruby{飽和}{ほうわ}し、\ruby{スマートフォン}{すまーとふぉん}は\ruby{兆}{きざ}しを\ruby{見}{み}せつつも\ruby{形}{かたち}は\ruby{定}{さだ}まっていなかった。\ruby{同社}{どうしゃ}は\ruby{完成}{かんせい}した\ruby{商用}{しょうよう}\ruby{製品}{せいひん}を\ruby{持}{も}たず、\ruby{安定}{あんてい}した\ruby{収益}{しゅうえき}\ruby{源}{げん}もなく、\ruby{外部}{がいぶ}\ruby{投資}{とうし}に\ruby{依存}{いぞん}していた。このような\ruby{状況}{じょうきょう}の\ruby{中}{なか}で、Android Inc.は\ruby{方針}{ほうしん}を\ruby{修正}{しゅうせい}し、\ruby{より}{より}\ruby{大}{おお}きな\ruby{可能性}{かのうせい}を\ruby{持}{も}つ\ruby{携帯}{けいたい}\ruby{電話}{でんわ}\ruby{向}{む}け\ruby{OS}{おーえす}へと\ruby{重点}{じゅうてん}を\ruby{移}{うつ}した。

Chính giai đoạn này đã hình thành nên bản sắc cốt lõi của Android: một hệ điều hành mở, linh hoạt, không bị khóa chặt bởi một nhà sản xuất hay nhà mạng cụ thể. Dù còn ở quy mô nhỏ và chưa được thị trường chú ý, Android Inc. đã đặt những viên gạch đầu tiên cho một nền tảng di động có khả năng thay đổi cục diện ngành công nghiệp trong tương lai.

この\ruby{時期}{じき}こそが、Androidの\ruby{中核的}{ちゅうかくてき}\ruby{アイデンティティ}{あいでんてぃてぃ}を\ruby{形成}{けいせい}した。すなわち、\ruby{特定}{とくてい}の\ruby{製造}{せいぞう}\ruby{業者}{ぎょうしゃ}や\ruby{通信}{つうしん}\ruby{事業者}{じぎょうしゃ}に\ruby{拘束}{こうそく}されない、\ruby{開放的}{かいほうてき}で\ruby{柔軟}{じゅうなん}な\ruby{OS}{おーえす}である。\ruby{規模}{きぼ}は\ruby{小}{ちい}さく、\ruby{市場}{しじょう}からの\ruby{注目}{ちゅうもく}も\ruby{限}{かぎ}られていたが、Android Inc.は\ruby{将来}{しょうらい}の\ruby{産業}{さんぎょう}\ruby{構図}{こうず}を\ruby{変}{か}えうる\ruby{モバイル}{もばいる}\ruby{プラットフォーム}{ぷらっとふぉーむ}の\ruby{第一歩}{だいいっぽ}を\ruby{築}{きず}いた。

\section{Tầm nhìn kỹ thuật của Andy Rubin: hệ điều hành linh hoạt, gắn chặt với dịch vụ và dữ liệu trực tuyến}
Andy Rubinの\ruby{技術}{ぎじゅつ}\ruby{的}{てき}\ruby{構想}{こうそう}:\ruby{柔軟}{じゅうなん}な\ruby{オペレーティング}{おぺれーてぃんぐ}\ruby{システム}{しすてむ}と\ruby{オンライン}{おんらいん}\ruby{サービス}{さーびす}・\ruby{データ}{でーた}との\ruby{密接}{みっせつ}な\ruby{連携}{れんけい}

Andy Rubin nhìn nhận thiết bị di động không chỉ là một công cụ liên lạc hay xử lý dữ liệu cục bộ, mà là một điểm truy cập thường trực vào Internet. Theo quan điểm này, giá trị cốt lõi của hệ điều hành di động không nằm ở số lượng tính năng sẵn có, mà ở khả năng kết nối liên tục với dịch vụ và dữ liệu trực tuyến. Đây là một cách tiếp cận mang tính đột phá trong bối cảnh đầu những năm 2000, khi phần lớn thiết bị di động vẫn hoạt động như những hệ thống độc lập, ít phụ thuộc vào mạng.

Andy Rubinは、\ruby{携帯}{けいたい}\ruby{端末}{たんまつ}を\ruby{単}{たん}なる\ruby{通信}{つうしん}\ruby{手段}{しゅだん}や\ruby{ローカル}{ろーかる}な\ruby{データ}{でーた}\ruby{処理}{しょり}\ruby{装置}{そうち}としてではなく、\ruby{常時}{じょうじ}\ruby{インターネット}{いんたーねっと}へ\ruby{接続}{せつぞく}する\ruby{アクセスポイント}{あくせすぽいんと}として\ruby{捉}{とら}えていた。この\ruby{観点}{かんてん}において、\ruby{携帯}{けいたい}\ruby{端末}{たんまつ}\ruby{向}{む}け\ruby{オペレーティング}{おぺれーてぃんぐ}\ruby{システム}{しすてむ}の\ruby{中核}{ちゅうかく}\ruby{的}{てき}\ruby{価値}{かち}は、\ruby{搭載}{とうさい}された\ruby{機能}{きのう}の\ruby{数}{かず}ではなく、\ruby{オンライン}{おんらいん}\ruby{サービス}{さーびす}や\ruby{データ}{でーた}と\ruby{継続}{けいぞく}\ruby{的}{てき}に\ruby{接続}{せつぞく}できる\ruby{能力}{のうりょく}に\ruby{存}{そん}すると\ruby{考}{かんが}えられていた。これは、\ruby{2000}{にせん}\ruby{年代}{ねんだい}\ruby{初頭}{しょとう}において、\ruby{多}{おお}くの\ruby{携帯}{けいたい}\ruby{端末}{たんまつ}が\ruby{独立}{どくりつ}した\ruby{システム}{しすてむ}として\ruby{動作}{どうさ}し、\ruby{ネットワーク}{ねっとわーく}への\ruby{依存}{いぞん}が\ruby{低}{ひく}かった\ruby{状況}{じょうきょう}の\ruby{中}{なか}では、\ruby{革新}{かくしん}\ruby{的}{てき}な\ruby{発想}{はっそう}であった。

Từ góc độ kỹ thuật, Rubin chủ trương xây dựng một hệ điều hành có khả năng thích ứng cao với nhiều loại phần cứng khác nhau. Điều này đòi hỏi kiến trúc hệ thống phải được phân tầng rõ ràng, tách biệt giữa nhân hệ điều hành, lớp trừu tượng phần cứng và các thành phần ứng dụng. Cách tiếp cận này giúp giảm sự phụ thuộc vào một cấu hình thiết bị cụ thể, đồng thời tạo điều kiện cho việc mở rộng sang nhiều phân khúc sản phẩm trong tương lai.

\ruby{技術}{ぎじゅつ}\ruby{的}{てき}な\ruby{観点}{かんてん}から、Rubinは\ruby{多様}{たよう}な\ruby{ハードウェア}{はーどうぇあ}に\ruby{高}{たか}い\ruby{適応}{てきおう}\ruby{性}{せい}を\ruby{持}{も}つ\ruby{オペレーティング}{おぺれーてぃんぐ}\ruby{システム}{しすてむ}の\ruby{構築}{こうちく}を\ruby{主張}{しゅちょう}した。そのためには、\ruby{システム}{しすてむ}\ruby{アーキテクチャ}{あーきてくちゃ}を\ruby{明確}{めいかく}に\ruby{階層}{かいそう}\ruby{化}{か}し、\ruby{カーネル}{かーねる}、\ruby{ハードウェア}{はーどうぇあ}\ruby{抽象}{ちゅうしょう}\ruby{化}{か}\ruby{層}{そう}、\ruby{アプリケーション}{あぷりけーしょん}\ruby{構成}{こうせい}\ruby{要素}{ようそ}を\ruby{分離}{ぶんり}する\ruby{必要}{ひつよう}があった。この\ruby{手法}{しゅほう}により、\ruby{特定}{とくてい}の\ruby{端末}{たんまつ}\ruby{構成}{こうせい}への\ruby{依存}{いぞん}が\ruby{低減}{ていげん}され、\ruby{将来}{しょうらい}における\ruby{複数}{ふくすう}の\ruby{製品}{せいひん}\ruby{分野}{ぶんや}への\ruby{展開}{てんかい}が\ruby{容易}{ようい}になると\ruby{考}{かんが}えられていた。

Một trụ cột quan trọng trong tầm nhìn của Rubin là tính mở. Hệ điều hành không nên bị kiểm soát chặt chẽ bởi một công ty duy nhất, cũng không nên bị khóa bởi nhà mạng hay nhà sản xuất. Thay vào đó, Android cần cho phép bên thứ ba phát triển ứng dụng, tùy biến giao diện và tích hợp dịch vụ theo nhu cầu riêng. Về mặt kỹ thuật, điều này đồng nghĩa với việc cung cấp bộ công cụ phát triển đầy đủ, tài liệu rõ ràng và cơ chế phân phối phần mềm không mang tính độc quyền.

Rubinの\ruby{構想}{こうそう}における\ruby{重要}{じゅうよう}な\ruby{柱}{はしら}の\ruby{一}{ひと}つは\ruby{開放}{かいほう}\ruby{性}{せい}であった。\ruby{オペレーティング}{おぺれーてぃんぐ}\ruby{システム}{しすてむ}は、\ruby{単一}{たんいつ}の\ruby{企業}{きぎょう}によって\ruby{厳格}{げんかく}に\ruby{管理}{かんり}されるべきではなく、\ruby{通信}{つうしん}\ruby{事業}{じぎょう}\ruby{者}{しゃ}や\ruby{製造}{せいぞう}\ruby{業者}{ぎょうしゃ}によって\ruby{固定}{こてい}されるべきでもない。その\ruby{代}{か}わりに、Androidは\ruby{第三}{だいさん}\ruby{者}{しゃ}が\ruby{アプリケーション}{あぷりけーしょん}を\ruby{開発}{かいはつ}し、\ruby{ユーザー}{ゆーざー}\ruby{インターフェース}{いんたーふぇーす}を\ruby{カスタマイズ}{かすたまいず}し、\ruby{独自}{どくじ}の\ruby{サービス}{さーびす}を\ruby{統合}{とうごう}できるように\ruby{設計}{せっけい}される\ruby{必要}{ひつよう}があった。\ruby{技術}{ぎじゅつ}\ruby{的}{てき}には、\ruby{充実}{じゅうじつ}した\ruby{開発}{かいはつ}\ruby{ツール}{つーる}、\ruby{明確}{めいかく}な\ruby{ドキュメント}{どきゅめんと}、および\ruby{排他}{はいた}\ruby{的}{てき}でない\ruby{ソフトウェア}{そふとうぇあ}\ruby{配布}{はいふ}\ruby{仕組}{しく}みの\ruby{提供}{ていきょう}を\ruby{意味}{いみ}していた。

Rubin cũng nhấn mạnh vai trò trung tâm của dữ liệu người dùng. Email, danh bạ, lịch làm việc, vị trí địa lý và hành vi sử dụng đều cần được đồng bộ hóa thông qua Internet, thay vì lưu trữ rời rạc trên từng thiết bị. Hệ điều hành, theo cách hiểu này, chỉ là lớp trung gian giúp người dùng truy cập dữ liệu của mình một cách liền mạch, bất kể họ đang sử dụng thiết bị nào. Đây là tiền đề cho mô hình trải nghiệm dựa trên tài khoản và dịch vụ đám mây sau này.

Rubinはまた、\ruby{ユーザー}{ゆーざー}\ruby{データ}{でーた}の\ruby{中心}{ちゅうしん}\ruby{的}{てき}\ruby{役割}{やくわり}を\ruby{強調}{きょうちょう}した。\ruby{電子}{でんし}\ruby{メール}{めーる}、\ruby{連絡}{れんらく}\ruby{先}{さき}、\ruby{予定}{よてい}\ruby{表}{ひょう}、\ruby{位置}{いち}\ruby{情報}{じょうほう}、\ruby{利用}{りよう}\ruby{行動}{こうどう}は、\ruby{個々}{ここ}の\ruby{端末}{たんまつ}に\ruby{分散}{ぶんさん}して\ruby{保存}{ほぞん}されるのではなく、\ruby{インターネット}{いんたーねっと}を\ruby{通}{つう}じて\ruby{同期}{どうき}されるべきであるとされた。この\ruby{理解}{りかい}において、\ruby{オペレーティング}{おぺれーてぃんぐ}\ruby{システム}{しすてむ}は、\ruby{利用}{りよう}\ruby{者}{しゃ}が\ruby{使用}{しよう}する\ruby{端末}{たんまつ}に\ruby{関係}{かんけい}なく、\ruby{自身}{じしん}の\ruby{データ}{でーた}へ\ruby{円滑}{えんかつ}に\ruby{アクセス}{あくせす}するための\ruby{中間}{ちゅうかん}\ruby{層}{そう}に\ruby{過}{す}ぎない。これは、\ruby{後}{のち}の\ruby{アカウント}{あかうんと}および\ruby{クラウド}{くらうど}\ruby{サービス}{さーびす}に\ruby{基}{もと}づく\ruby{体験}{たいけん}\ruby{モデル}{もでる}の\ruby{前提}{ぜんてい}となった。

Một điểm đáng chú ý khác trong tầm nhìn kỹ thuật của Rubin là thái độ đối với nhà mạng. Thời điểm đó, nhà mạng có quyền kiểm soát lớn đối với phần mềm cài đặt trên điện thoại, từ ứng dụng mặc định đến khả năng truy cập Internet. Rubin cho rằng mô hình này kìm hãm đổi mới và làm giảm giá trị của thiết bị đối với người dùng cuối. Android vì vậy được thiết kế để hạn chế sự can thiệp ở mức hệ thống, chuyển trọng tâm kiểm soát sang phía người dùng và nhà phát triển.

Rubinの\ruby{技術}{ぎじゅつ}\ruby{的}{てき}\ruby{構想}{こうそう}における\ruby{別}{べつ}の\ruby{注目}{ちゅうもく}\ruby{点}{てん}は、\ruby{通信}{つうしん}\ruby{事業}{じぎょう}\ruby{者}{しゃ}に\ruby{対}{たい}する\ruby{姿勢}{しせい}である。\ruby{当時}{とうじ}、\ruby{通信}{つうしん}\ruby{事業}{じぎょう}\ruby{者}{しゃ}は、\ruby{標準}{ひょうじゅん}\ruby{アプリケーション}{あぷりけーしょん}から\ruby{インターネット}{いんたーねっと}\ruby{接続}{せつぞく}\ruby{機能}{きのう}に\ruby{至}{いた}るまで、\ruby{端末}{たんまつ}に\ruby{搭載}{とうさい}される\ruby{ソフトウェア}{そふとうぇあ}を\ruby{強}{つよ}く\ruby{管理}{かんり}していた。Rubinは、この\ruby{構造}{こうぞう}が\ruby{革新}{かくしん}を\ruby{阻害}{そがい}し、\ruby{最終}{さいしゅう}\ruby{利用}{りよう}\ruby{者}{しゃ}にとっての\ruby{端末}{たんまつ}\ruby{価値}{かち}を\ruby{低下}{ていか}させると\ruby{考}{かんが}えた。そのためAndroidは、\ruby{システム}{しすてむ}\ruby{水準}{すいじゅん}での\ruby{介入}{かいにゅう}を\ruby{抑制}{よくせい}し、\ruby{制御}{せいぎょ}の\ruby{中心}{ちゅうしん}を\ruby{利用}{りよう}\ruby{者}{しゃ}および\ruby{開発}{かいはつ}\ruby{者}{しゃ}へ\ruby{移}{うつ}すよう\ruby{設計}{せっけい}された。

Tuy nhiên, tầm nhìn này cũng kéo theo những thách thức kỹ thuật đáng kể. Việc hỗ trợ nhiều cấu hình phần cứng khác nhau làm tăng độ phức tạp trong phát triển và kiểm thử. Mô hình mở tiềm ẩn rủi ro về bảo mật và phân mảnh hệ thống. Ngoài ra, việc phụ thuộc mạnh vào kết nối Internet đòi hỏi hạ tầng mạng đủ tốt, điều chưa phổ biến ở nhiều thị trường thời điểm đó.

しかし、この\ruby{構想}{こうそう}は\ruby{重大}{じゅうだい}な\ruby{技術}{ぎじゅつ}\ruby{的}{てき}\ruby{課題}{かだい}も\ruby{伴}{ともな}っていた。\ruby{多様}{たよう}な\ruby{ハードウェア}{はーどうぇあ}\ruby{構成}{こうせい}への\ruby{対応}{たいおう}は、\ruby{開発}{かいはつ}および\ruby{テスト}{てすと}の\ruby{複雑}{ふくざつ}\ruby{性}{せい}を\ruby{増大}{ぞうだい}させる。\ruby{開放}{かいほう}\ruby{的}{てき}\ruby{モデル}{もでる}は、\ruby{セキュリティ}{せきゅりてぃ}や\ruby{システム}{しすてむ}の\ruby{分断}{ぶんだん}に\ruby{関}{かん}する\ruby{潜在}{せんざい}\ruby{的}{てき}\ruby{リスク}{りすく}を\ruby{内包}{ないほう}していた。さらに、\ruby{インターネット}{いんたーねっと}\ruby{接続}{せつぞく}への\ruby{強}{つよ}い\ruby{依存}{いぞん}は、\ruby{十分}{じゅうぶん}な\ruby{通信}{つうしん}\ruby{基盤}{きばん}を\ruby{必要}{ひつよう}とするが、それは\ruby{当時}{とうじ}の\ruby{多}{おお}くの\ruby{市場}{しじょう}では\ruby{一般}{いっぱん}\ruby{的}{てき}ではなかった。

Dù vậy, Andy Rubin chấp nhận những rủi ro này như một phần tất yếu của chiến lược dài hạn. Ông tin rằng xu hướng kết nối liên tục là không thể đảo ngược, và hệ điều hành nào được thiết kế xoay quanh Internet ngay từ đầu sẽ có lợi thế bền vững. Tầm nhìn kỹ thuật này không chỉ định hình Android trong giai đoạn khởi đầu, mà còn ảnh hưởng sâu sắc đến cách nền tảng này phát triển sau khi được Google mua lại.

それでもAndy Rubinは、これらの\ruby{リスク}{りすく}を\ruby{長期}{ちょうき}\ruby{的}{てき}\ruby{戦略}{せんりゃく}に\ruby{不可欠}{ふかけつ}な\ruby{要素}{ようそ}として\ruby{受容}{じゅよう}した。\ruby{常時}{じょうじ}\ruby{接続}{せつぞく}の\ruby{潮流}{ちょうりゅう}は\ruby{不可逆}{ふかぎゃく}であり、\ruby{当初}{とうしょ}から\ruby{インターネット}{いんたーねっと}を\ruby{中心}{ちゅうしん}に\ruby{設計}{せっけい}された\ruby{オペレーティング}{おぺれーてぃんぐ}\ruby{システム}{しすてむ}こそが、\ruby{持続}{じぞく}\ruby{的}{てき}な\ruby{優位}{ゆうい}\ruby{性}{せい}を\ruby{得}{え}ると\ruby{信}{しん}じていた。この\ruby{技術}{ぎじゅつ}\ruby{的}{てき}\ruby{構想}{こうそう}は、Androidの\ruby{初期}{しょき}\ruby{段階}{だんかい}を\ruby{形作}{かたちづく}っただけでなく、\ruby{後}{のち}にGoogleに\ruby{買収}{ばいしゅう}された\ruby{後}{のち}の\ruby{発展}{はってん}\ruby{方向}{ほうこう}にも\ruby{深}{ふか}い\ruby{影響}{えいきょう}を\ruby{与}{あた}えた。

\section{Bối cảnh Google trước khi mua Android: nhu cầu mở rộng hệ sinh thái tìm kiếm và dịch vụ sang nền tảng di động}
\ruby{Android}{あんどろいど}\ruby{買収}{ばいしゅう}\ruby{以前}{いぜん}の\ruby{Google}{ぐーぐる}の\ruby{背景}{はいけい}――\ruby{検索}{けんさく}および\ruby{サービス}{さーびす}\ruby{生態系}{せいたいけい}を\ruby{モバイル}{もばいる}\ruby{基盤}{きばん}へ\ruby{拡張}{かくちょう}する\ruby{必要性}{ひつようせい}

Trước năm 2005, Google là công ty thống trị gần như tuyệt đối trong lĩnh vực tìm kiếm trên web máy tính. Mô hình kinh doanh của Google dựa trên quảng cáo gắn với truy vấn tìm kiếm, và giá trị cốt lõi của công ty nằm ở khả năng thu thập dữ liệu hành vi người dùng ở quy mô lớn. Tuy nhiên, sự thống trị này chủ yếu giới hạn trong môi trường trình duyệt trên máy tính cá nhân, nơi Google kiểm soát gần như toàn bộ “cổng vào” Internet.

2005\ruby{年}{ねん}\ruby{以前}{いぜん}、\ruby{Google}{ぐーぐる}は\ruby{デスクトップ}{ですくとっぷ}\ruby{Web}{うぇぶ}における\ruby{検索}{けんさく}\ruby{分野}{ぶんや}で、ほぼ\ruby{絶対的}{ぜったいてき}な\ruby{支配的}{しはいてき}\ruby{地位}{ちい}を\ruby{確立}{かくりつ}していた。\ruby{Google}{ぐーぐる}の\ruby{ビジネス}{びじねす}\ruby{モデル}{もでる}は\ruby{検索}{けんさく}\ruby{クエリ}{くえり}に\ruby{連動}{れんどう}した\ruby{広告}{こうこく}に\ruby{基}{もと}づいており、\ruby{企業}{きぎょう}の\ruby{中核的}{ちゅうかくてき}\ruby{価値}{かち}は、\ruby{大規模}{だいきぼ}に\ruby{利用者}{りようしゃ}の\ruby{行動}{こうどう}\ruby{データ}{でーた}を\ruby{収集}{しゅうしゅう}する\ruby{能力}{のうりょく}にあった。しかし、この\ruby{支配}{しはい}は\ruby{主}{おも}に\ruby{個人}{こじん}\ruby{コンピュータ}{こんぴゅーた}の\ruby{ブラウザ}{ぶらうざ}\ruby{環境}{かんきょう}に\ruby{限定}{げんてい}されており、そこでは\ruby{Google}{ぐーぐる}が\ruby{インターネット}{いんたーねっと}への「\ruby{入口}{いりぐち}」をほぼ\ruby{完全}{かんぜん}に\ruby{掌握}{しょうあく}していた。

Trong cùng giai đoạn đó, thiết bị di động bắt đầu phát triển nhanh chóng, không chỉ về số lượng người dùng mà còn về khả năng truy cập Internet. Dù trải nghiệm còn hạn chế, xu hướng người dùng tiếp cận email, tin tức và tìm kiếm thông tin qua điện thoại đã trở nên rõ ràng. Điều này đặt ra một vấn đề chiến lược nghiêm trọng cho Google: nếu Internet di động phát triển trên các nền tảng mà Google không kiểm soát, vị thế cốt lõi của công ty sẽ bị đe dọa.

\ruby{同時期}{どうじき}に、\ruby{モバイル}{もばいる}\ruby{端末}{たんまつ}は\ruby{利用者}{りようしゃ}\ruby{数}{すう}だけでなく、\ruby{インターネット}{いんたーねっと}\ruby{接続}{せつぞく}\ruby{能力}{のうりょく}の\ruby{面}{めん}でも\ruby{急速}{きゅうそく}に\ruby{発展}{はってん}し\ruby{始}{はじ}めていた。\ruby{体験}{たいけん}はまだ\ruby{限定的}{げんていてき}であったものの、\ruby{電子}{でんし}\ruby{メール}{めーる}、\ruby{ニュース}{にゅーす}、および\ruby{情報}{じょうほう}\ruby{検索}{けんさく}を\ruby{電話}{でんわ}から\ruby{利用}{りよう}する\ruby{傾向}{けいこう}は\ruby{明確}{めいかく}になりつつあった。これは\ruby{Google}{ぐーぐる}にとって\ruby{重大}{じゅうだい}な\ruby{戦略的}{せんりゃくてき}\ruby{問題}{もんだい}を\ruby{提起}{ていき}した。すなわち、\ruby{モバイル}{もばいる}\ruby{インターネット}{いんたーねっと}が\ruby{Google}{ぐーぐる}の\ruby{管理}{かんり}できない\ruby{プラットフォーム}{ぷらっとふぉーむ}上で\ruby{発展}{はってん}した\ruby{場合}{ばあい}、\ruby{企業}{きぎょう}の\ruby{中核的}{ちゅうかくてき}\ruby{地位}{ちい}が\ruby{脅}{おびや}かされるという\ruby{危険}{きけん}である。

Vấn đề không nằm ở việc Google có thể xây dựng ứng dụng cho điện thoại hay không, mà ở chỗ Google không kiểm soát hệ điều hành. Các hệ điều hành di động thời điểm đó đều mang tính đóng và chịu ảnh hưởng lớn từ nhà sản xuất hoặc nhà mạng. Điều này cho phép họ lựa chọn công cụ tìm kiếm mặc định, kiểm soát ứng dụng cài sẵn và thậm chí chặn hoặc hạn chế dịch vụ của bên thứ ba. Với Google, đây là rủi ro mang tính hệ thống.

\ruby{問題}{もんだい}は、\ruby{Google}{ぐーぐる}が\ruby{携帯電話}{けいたいでんわ}向けの\ruby{アプリケーション}{あぷりけーしょん}を\ruby{開発}{かいはつ}できるかどうかではなく、\ruby{オペレーティング}{おぺれーてぃんぐ}\ruby{システム}{しすてむ}を\ruby{支配}{しはい}していない\ruby{点}{てん}にあった。当時の\ruby{モバイル}{もばいる}\ruby{オペレーティング}{おぺれーてぃんぐ}\ruby{システム}{しすてむ}は、いずれも\ruby{閉鎖的}{へいさてき}で、\ruby{端末}{たんまつ}\ruby{メーカー}{めーかー}や\ruby{通信}{つうしん}\ruby{事業者}{じぎょうしゃ}の\ruby{影響}{えいきょう}を\ruby{強}{つよ}く\ruby{受}{う}けていた。これにより、\ruby{既定}{きてい}の\ruby{検索}{けんさく}\ruby{エンジン}{えんじん}の\ruby{選択}{せんたく}、\ruby{プリインストール}{ぷりいんすとーる}\ruby{アプリ}{あぷり}の\ruby{管理}{かんり}、さらには\ruby{第三者}{だいさんしゃ}\ruby{サービス}{さーびす}の\ruby{遮断}{しゃだん}や\ruby{制限}{せいげん}さえも\ruby{可能}{かのう}であった。\ruby{Google}{ぐーぐる}にとって、これは\ruby{構造的}{こうぞうてき}\ruby{リスク}{りすく}であった。

Nếu Google phụ thuộc hoàn toàn vào các nền tảng của đối thủ, công ty có thể bị đẩy ra khỏi vị trí trung tâm trong trải nghiệm Internet di động. Khi đó, dữ liệu người dùng sẽ bị phân mảnh, doanh thu quảng cáo suy giảm, và khả năng đổi mới bị hạn chế bởi các quyết định từ bên ngoài. Đây là kịch bản mà Google cần tránh bằng mọi giá.

もし\ruby{Google}{ぐーぐる}が\ruby{競合}{きょうごう}\ruby{他社}{たしゃ}の\ruby{プラットフォーム}{ぷらっとふぉーむ}に\ruby{完全}{かんぜん}に\ruby{依存}{いぞん}した\ruby{場合}{ばあい}、\ruby{モバイル}{もばいる}\ruby{インターネット}{いんたーねっと}の\ruby{体験}{たいけん}における\ruby{中心的}{ちゅうしんてき}\ruby{位置}{いち}から\ruby{排除}{はいじょ}される\ruby{可能性}{かのうせい}があった。その\ruby{結果}{けっか}、\ruby{利用者}{りようしゃ}\ruby{データ}{でーた}は\ruby{分散}{ぶんさん}し、\ruby{広告}{こうこく}\ruby{収益}{しゅうえき}は\ruby{減少}{げんしょう}し、\ruby{革新}{かくしん}\ruby{能力}{のうりょく}は\ruby{外部}{がいぶ}の\ruby{意思}{いし}\ruby{決定}{けってい}によって\ruby{制約}{せいやく}される。このような\ruby{シナリオ}{しなりお}は、\ruby{Google}{ぐーぐる}が\ruby{何}{なん}としても\ruby{回避}{かいひ}すべきものであった。

Bên cạnh rủi ro chiến lược, Google cũng nhận thấy cơ hội dài hạn từ thiết bị di động. Điện thoại cá nhân hóa hơn máy tính, luôn gắn liền với người dùng và mang theo dữ liệu về vị trí, thói quen và bối cảnh sử dụng. Đối với một công ty sống dựa trên dữ liệu và quảng cáo như Google, đây là nguồn giá trị tiềm năng khổng lồ nếu có thể khai thác đúng cách.

\ruby{戦略的}{せんりゃくてき}\ruby{リスク}{りすく}に\ruby{加}{くわ}え、\ruby{Google}{ぐーぐる}は\ruby{モバイル}{もばいる}\ruby{端末}{たんまつ}に\ruby{長期的}{ちょうきてき}な\ruby{機会}{きかい}を\ruby{見出}{みいだ}していた。\ruby{携帯電話}{けいたいでんわ}は\ruby{コンピュータ}{こんぴゅーた}よりも\ruby{個人}{こじん}\ruby{化}{か}が\ruby{進}{すす}んでおり、\ruby{常}{つね}に\ruby{利用者}{りようしゃ}と\ruby{共}{とも}にあり、\ruby{位置}{いち}、\ruby{習慣}{しゅうかん}、および\ruby{利用}{りよう}\ruby{文脈}{ぶんみゃく}に\ruby{関}{かん}する\ruby{データ}{でーた}を\ruby{伴}{ともな}う。\ruby{データ}{でーた}と\ruby{広告}{こうこく}に\ruby{依存}{いぞん}する\ruby{企業}{きぎょう}である\ruby{Google}{ぐーぐる}にとって、これは\ruby{適切}{てきせつ}に\ruby{活用}{かつよう}できれば\ruby{莫大}{ばくだい}な\ruby{潜在的}{せんざいてき}\ruby{価値}{かち}を\ruby{持}{も}つ\ruby{資源}{しげん}であった。

Tuy nhiên, để tận dụng cơ hội này, Google cần một nền tảng mà họ có thể đảm bảo sự hiện diện mặc định của dịch vụ tìm kiếm và các sản phẩm cốt lõi khác. Việc chỉ phát triển ứng dụng riêng lẻ là không đủ, vì ứng dụng có thể bị thay thế, gỡ bỏ hoặc hạn chế bởi hệ điều hành. Cách tiếp cận bền vững hơn là tham gia trực tiếp vào tầng nền tảng, nơi các quyết định kiến trúc ảnh hưởng đến toàn bộ hệ sinh thái.

しかし、この\ruby{機会}{きかい}を\ruby{活}{い}かすためには、\ruby{Google}{ぐーぐる}が\ruby{検索}{けんさく}\ruby{サービス}{さーびす}や\ruby{他}{ほか}の\ruby{中核}{ちゅうかく}\ruby{製品}{せいひん}の\ruby{既定}{きてい}\ruby{的}{てき}\ruby{存在}{そんざい}を\ruby{保証}{ほしょう}できる\ruby{プラットフォーム}{ぷらっとふぉーむ}が\ruby{必要}{ひつよう}であった。\ruby{個別}{こべつ}の\ruby{アプリケーション}{あぷりけーしょん}を\ruby{開発}{かいはつ}するだけでは\ruby{不十分}{ふじゅうぶん}であり、\ruby{オペレーティング}{おぺれーてぃんぐ}\ruby{システム}{しすてむ}によって\ruby{置換}{ちかん}、\ruby{削除}{さくじょ}、あるいは\ruby{制限}{せいげん}される\ruby{可能性}{かのうせい}があった。より\ruby{持続的}{じぞくてき}な\ruby{アプローチ}{あぷろーち}は、\ruby{設計}{せっけい}\ruby{上}{じょう}の\ruby{決定}{けってい}が\ruby{生態系}{せいたいけい}\ruby{全体}{ぜんたい}に\ruby{影響}{えいきょう}を\ruby{及}{およ}ぼす\ruby{基盤}{きばん}\ruby{層}{そう}に\ruby{直接}{ちょくせつ}\ruby{関与}{かんよ}することであった。

Trong bối cảnh đó, Google đứng trước ba lựa chọn: hợp tác sâu với các hệ điều hành hiện có, tự phát triển một hệ điều hành di động từ đầu, hoặc mua lại một công ty đã có nền tảng và tầm nhìn phù hợp. Phương án đầu tiên tiềm ẩn rủi ro phụ thuộc, phương án thứ hai tốn nhiều thời gian và chi phí. Phương án thứ ba, dù mang tính mạo hiểm, lại cho phép Google rút ngắn đáng kể con đường tiến vào thị trường di động.

そのような\ruby{状況}{じょうきょう}の\ruby{中}{なか}で、\ruby{Google}{ぐーぐる}は\ruby{三}{みっ}つの\ruby{選択肢}{せんたくし}に\ruby{直面}{ちょくめん}した。すなわち、\ruby{既存}{きそん}の\ruby{オペレーティング}{おぺれーてぃんぐ}\ruby{システム}{しすてむ}と\ruby{深}{ふか}く\ruby{協力}{きょうりょく}すること、\ruby{一}{いち}から\ruby{モバイル}{もばいる}\ruby{オペレーティング}{おぺれーてぃんぐ}\ruby{システム}{しすてむ}を\ruby{自社}{じしゃ}\ruby{開発}{かいはつ}すること、または\ruby{基盤}{きばん}と\ruby{ビジョン}{びじょん}を\ruby{持}{も}つ\ruby{企業}{きぎょう}を\ruby{買収}{ばいしゅう}することである。\ruby{第一}{だいいち}の\ruby{選択肢}{せんたくし}は\ruby{依存}{いぞん}\ruby{リスク}{りすく}を\ruby{伴}{ともな}い、\ruby{第二}{だいに}の\ruby{選択肢}{せんたくし}は\ruby{時間}{じかん}と\ruby{コスト}{こすと}を\ruby{要}{よう}する。\ruby{第三}{だいさん}の\ruby{選択肢}{せんたくし}は\ruby{冒険的}{ぼうけんてき}ではあるが、\ruby{Google}{ぐーぐる}が\ruby{モバイル}{もばいる}\ruby{市場}{しじょう}へ\ruby{進出}{しんしゅつ}する\ruby{道}{みち}を\ruby{大幅}{おおはば}に\ruby{短縮}{たんしゅく}することを\ruby{可能}{かのう}にした。

Chính tại thời điểm này, Android Inc. xuất hiện như một lời giải phù hợp. Một hệ điều hành đang được xây dựng xoay quanh Internet, mang triết lý mở và chưa bị ràng buộc bởi lợi ích của bất kỳ nhà sản xuất hay nhà mạng nào. Với Google, Android không chỉ là một sản phẩm phần mềm, mà là cơ hội để bảo vệ và mở rộng hệ sinh thái tìm kiếm và dịch vụ của mình sang kỷ nguyên di động.

まさにこの\ruby{時点}{じてん}で、\ruby{Android}{あんどろいど}\ruby{社}{しゃ}が\ruby{有力}{ゆうりょく}な\ruby{解答}{かいとう}として\ruby{浮上}{ふじょう}した。\ruby{インターネット}{いんたーねっと}を\ruby{中心}{ちゅうしん}に\ruby{設計}{せっけい}され、\ruby{オープン}{おーぷん}な\ruby{理念}{りねん}を\ruby{持}{も}ち、いかなる\ruby{メーカー}{めーかー}や\ruby{通信}{つうしん}\ruby{事業者}{じぎょうしゃ}の\ruby{利害}{りがい}にも\ruby{拘束}{こうそく}されていない\ruby{オペレーティング}{おぺれーてぃんぐ}\ruby{システム}{しすてむ}である。\ruby{Google}{ぐーぐる}にとって、\ruby{Android}{あんどろいど}は\ruby{単}{たん}なる\ruby{ソフトウェア}{そふとうぇあ}\ruby{製品}{せいひん}ではなく、\ruby{検索}{けんさく}および\ruby{サービス}{さーびす}\ruby{生態系}{せいたいけい}を\ruby{モバイル}{もばいる}\ruby{時代}{じだい}へ\ruby{保護}{ほご}し、さらに\ruby{拡張}{かくちょう}するための\ruby{決定的}{けっていてき}な\ruby{機会}{きかい}であった。

\section{Thương vụ mua lại Android Inc.: động cơ chiến lược, rủi ro và cơ hội từ góc nhìn kinh doanh lẫn kỹ thuật}
Android Inc.\ruby{買収}{ばいしゅう}:\ruby{戦略的}{せんりゃくてき}\ruby{動機}{どうき}、\ruby{リスク}{りすく}と\ruby{機会}{きかい}―\ruby{経営}{けいえい}および\ruby{技術}{ぎじゅつ}の\ruby{視点}{してん}から

Năm 2005, Google chính thức mua lại Android Inc. với mức giá tương đối thấp so với quy mô của Google tại thời điểm đó. Thương vụ này không gây nhiều chú ý trên truyền thông, bởi Android khi ấy chưa có sản phẩm hoàn chỉnh, chưa có thị phần và cũng chưa chứng minh được mô hình kinh doanh rõ ràng. Tuy nhiên, từ góc nhìn chiến lược, đây là một quyết định mang tính phòng thủ và chuẩn bị dài hạn.

2005\ruby{年}{ねん}、GoogleはAndroid Inc.を\ruby{当時}{とうじ}のGoogleの\ruby{規模}{きぼ}に\ruby{比}{くら}べて\ruby{比較的}{ひかくてき}\ruby{低}{ひく}い\ruby{価格}{かかく}で\ruby{正式}{せいしき}に\ruby{買収}{ばいしゅう}した。この\ruby{取引}{とりひき}は、Androidが\ruby{完成}{かんせい}した\ruby{製品}{せいひん}を\ruby{有}{ゆう}しておらず、\ruby{市場}{しじょう}\ruby{シェア}{しぇあ}もなく、\ruby{明確}{めいかく}な\ruby{ビジネス}{びじねす}\ruby{モデル}{もでる}を\ruby{示}{しめ}していなかったため、\ruby{メディア}{めでぃあ}から\ruby{大}{おお}きな\ruby{注目}{ちゅうもく}を\ruby{集}{あつ}めなかった。しかし、\ruby{戦略的}{せんりゃくてき}\ruby{視点}{してん}から\ruby{見}{み}れば、これは\ruby{防御的}{ぼうぎょてき}で\ruby{長期的}{ちょうきてき}な\ruby{準備}{じゅんび}を\ruby{目的}{もくてき}とした\ruby{意思決定}{いしけってい}であった。

Động cơ chiến lược quan trọng nhất của Google là bảo vệ vị thế trung tâm trong việc truy cập Internet. Google nhận thức rõ rằng trong kỷ nguyên di động, hệ điều hành sẽ trở thành “người gác cổng” quyết định dịch vụ nào được tiếp cận người dùng. Việc sở hữu Android cho phép Google chủ động đảm bảo sự hiện diện của công cụ tìm kiếm, trình duyệt và các dịch vụ cốt lõi trên thiết bị di động, thay vì phụ thuộc vào thiện chí của đối tác.

Googleにとって\ruby{最}{もっと}も\ruby{重要}{じゅうよう}な\ruby{戦略的}{せんりゃくてき}\ruby{動機}{どうき}は、\ruby{インターネット}{いんたーねっと}への\ruby{アクセス}{あくせす}における\ruby{中心的}{ちゅうしんてき}\ruby{地位}{ちい}を\ruby{守}{まも}ることであった。Googleは、\ruby{モバイル}{もばいる}\ruby{時代}{じだい}において、\ruby{オペレーティングシステム}{おぺれーてぃんぐしすてむ}がどの\ruby{サービス}{さーびす}が\ruby{利用者}{りようしゃ}に\ruby{届}{とど}くかを\ruby{左右}{さゆう}する「\ruby{門番}{もんばん}」となることを\ruby{明確}{めいかく}に\ruby{認識}{にんしき}していた。Androidを\ruby{所有}{しょゆう}することで、Googleは\ruby{検索}{けんさく}\ruby{エンジン}{えんじん}、\ruby{ブラウザ}{ぶらうざ}、および\ruby{中核}{ちゅうかく}\ruby{サービス}{さーびす}の\ruby{存在}{そんざい}を\ruby{モバイル}{もばいる}\ruby{端末}{たんまつ}上で\ruby{主体的}{しゅたいてき}に\ruby{確保}{かくほ}でき、\ruby{提携先}{ていけいさき}の\ruby{善意}{ぜんい}に\ruby{依存}{いぞん}する\ruby{必要}{ひつよう}がなくなった。

Từ góc độ kinh doanh, Android không được mua để tạo doanh thu trực tiếp. Google không có ý định bán giấy phép hệ điều hành hay thu phí sử dụng. Thay vào đó, Android được xem là một nền tảng chiến lược giúp mở rộng quy mô sử dụng dịch vụ Google, từ đó gián tiếp gia tăng doanh thu quảng cáo và dữ liệu người dùng. Đây là một cách tiếp cận khác biệt so với các mô hình kinh doanh phần mềm truyền thống.

\ruby{経営}{けいえい}の\ruby{観点}{かんてん}から、Androidは\ruby{直接的}{ちょくせつてき}な\ruby{収益}{しゅうえき}を\ruby{生}{う}むために\ruby{購入}{こうにゅう}されたわけではない。Googleは\ruby{オペレーティングシステム}{おぺれーてぃんぐしすてむ}の\ruby{ライセンス}{らいせんす}を\ruby{販売}{はんばい}したり、\ruby{利用}{りよう}\ruby{料金}{りょうきん}を\ruby{徴収}{ちょうしゅう}したりする\ruby{意図}{いと}を\ruby{持}{も}っていなかった。その\ruby{代}{か}わりに、AndroidはGoogleの\ruby{サービス}{さーびす}の\ruby{利用}{りよう}\ruby{規模}{きぼ}を\ruby{拡大}{かくだい}するための\ruby{戦略的}{せんりゃくてき}\ruby{基盤}{きばん}と\ruby{位置}{いち}づけられ、\ruby{結果}{けっか}として\ruby{広告}{こうこく}\ruby{収益}{しゅうえき}や\ruby{利用者}{りようしゃ}\ruby{データ}{でーた}の\ruby{間接的}{かんせつてき}\ruby{増加}{ぞうか}に\ruby{寄与}{きよ}した。これは\ruby{従来}{じゅうらい}の\ruby{ソフトウェア}{そふとうぇあ}\ruby{ビジネス}{びじねす}\ruby{モデル}{もでる}とは\ruby{異}{こと}なる\ruby{アプローチ}{あぷろーち}であった。

Về mặt kỹ thuật, Google nhìn thấy ở Android một kiến trúc phù hợp với triết lý phát triển nhanh, mở và dựa trên Internet. Android sử dụng Linux kernel, có khả năng tùy biến cao và chưa bị ràng buộc bởi các quyết định thiết kế cũ. Điều này cho phép Google định hình lại hệ điều hành theo hướng tích hợp chặt chẽ với hạ tầng dịch vụ của mình, từ tìm kiếm, email đến bản đồ và đồng bộ dữ liệu.

\ruby{技術的}{ぎじゅつてき}には、GoogleはAndroidに\ruby{迅速}{じんそく}で\ruby{オープン}{おーぷん}、かつ\ruby{インターネット}{いんたーねっと}を\ruby{基盤}{きばん}とする\ruby{開発}{かいはつ}\ruby{哲学}{てつがく}に\ruby{適合}{てきごう}した\ruby{アーキテクチャ}{あーきてくちゃ}を\ruby{見出}{みいだ}していた。AndroidはLinux\ruby{カーネル}{かーねる}を\ruby{採用}{さいよう}し、\ruby{高}{たか}い\ruby{柔軟性}{じゅうなんせい}を\ruby{備}{そな}え、\ruby{過去}{かこ}の\ruby{設計}{せっけい}\ruby{判断}{はんだん}に\ruby{縛}{しば}られていなかった。これにより、Googleは\ruby{検索}{けんさく}、\ruby{メール}{めーる}、\ruby{地図}{ちず}、\ruby{データ}{でーた}\ruby{同期}{どうき}に\ruby{至}{いた}るまで、\ruby{自社}{じしゃ}の\ruby{サービス}{さーびす}\ruby{基盤}{きばん}と\ruby{緊密}{きんみつ}に\ruby{統合}{とうごう}された\ruby{形}{かたち}で\ruby{オペレーティングシステム}{おぺれーてぃんぐしすてむ}を\ruby{再}{ふたた}び\ruby{設計}{せっけい}することが\ruby{可能}{かのう}となった。

Tuy nhiên, thương vụ này cũng đi kèm với những rủi ro đáng kể. Trước hết, Android là một dự án chưa hoàn thiện, yêu cầu đầu tư lớn về nhân lực và thời gian để có thể ra mắt sản phẩm thương mại. Google phải chấp nhận khả năng thất bại trong một thị trường mà họ chưa có nhiều kinh nghiệm trực tiếp về phần cứng và quan hệ với nhà mạng.

しかし、この\ruby{取引}{とりひき}には\ruby{無視}{むし}できない\ruby{リスク}{りすく}も\ruby{伴}{ともな}っていた。まず、Androidは\ruby{未完成}{みかんせい}の\ruby{プロジェクト}{ぷろじぇくと}であり、\ruby{商用}{しょうよう}\ruby{製品}{せいひん}として\ruby{投入}{とうにゅう}するためには、\ruby{人材}{じんざい}と\ruby{時間}{じかん}への\ruby{大規模}{だいきぼ}な\ruby{投資}{とうし}が\ruby{必要}{ひつよう}であった。Googleは、\ruby{ハードウェア}{はーどうぇあ}や\ruby{通信事業者}{つうしんじぎょうしゃ}との\ruby{関係}{かんけい}において\ruby{直接的}{ちょくせつてき}な\ruby{経験}{けいけん}が\ruby{乏}{とぼ}しい\ruby{市場}{しじょう}で\ruby{失敗}{しっぱい}する\ruby{可能性}{かのうせい}を\ruby{受}{う}け\ruby{入}{い}れなければならなかった。

Ngoài ra, việc phát triển một hệ điều hành di động đồng nghĩa với việc Google bước vào cuộc cạnh tranh trực tiếp với các công ty lớn như Microsoft và sau này là Apple. Đây không chỉ là cạnh tranh về công nghệ, mà còn về hệ sinh thái, tiêu chuẩn và quyền kiểm soát thị trường. Google cũng phải đối mặt với nguy cơ xung đột lợi ích với các đối tác hiện tại nếu Android bị xem là mối đe dọa.

さらに、\ruby{モバイル}{もばいる}\ruby{オペレーティングシステム}{おぺれーてぃんぐしすてむ}の\ruby{開発}{かいはつ}は、GoogleがMicrosoftや\ruby{後}{のち}のAppleといった\ruby{大企業}{だいきぎょう}と\ruby{直接}{ちょくせつ}\ruby{競争}{きょうそう}することを\ruby{意味}{いみ}した。これは\ruby{技術}{ぎじゅつ}のみならず、\ruby{エコシステム}{えこしすてむ}、\ruby{標準}{ひょうじゅん}、および\ruby{市場}{しじょう}\ruby{支配}{しはい}を\ruby{巡}{めぐ}る\ruby{競争}{きょうそう}でもあった。また、Androidが\ruby{脅威}{きょうい}と\ruby{見}{み}なされた\ruby{場合}{ばあい}、Googleは\ruby{既存}{きそん}の\ruby{提携先}{ていけいさき}との\ruby{利害}{りがい}\ruby{衝突}{しょうとつ}という\ruby{リスク}{りすく}にも\ruby{直面}{ちょくめん}した。

Một rủi ro khác nằm ở mô hình mở mà Android theo đuổi. Việc cho phép nhiều nhà sản xuất và nhà phát triển tham gia có thể dẫn đến phân mảnh hệ thống, khó kiểm soát chất lượng và bảo mật. Từ góc độ kỹ thuật, đây là bài toán phức tạp đòi hỏi sự cân bằng giữa tự do và kiểm soát.

もう一つの\ruby{リスク}{りすく}は、Androidが\ruby{採用}{さいよう}する\ruby{オープン}{おーぷん}\ruby{モデル}{もでる}にあった。\ruby{多数}{たすう}の\ruby{製造}{せいぞう}\ruby{業者}{ぎょうしゃ}や\ruby{開発者}{かいはつしゃ}の\ruby{参加}{さんか}を\ruby{許}{ゆる}すことは、\ruby{システム}{しすてむ}の\ruby{分断}{ぶんだん}を\ruby{招}{まね}き、\ruby{品質}{ひんしつ}や\ruby{セキュリティ}{せきゅりてぃ}の\ruby{管理}{かんり}を\ruby{困難}{こんなん}にする\ruby{可能性}{かのうせい}があった。\ruby{技術的}{ぎじゅつてき}\ruby{観点}{かんてん}からは、これは\ruby{自由}{じゆう}と\ruby{統制}{とうせい}の\ruby{均衡}{きんこう}を\ruby{求}{もと}められる\ruby{複雑}{ふくざつ}な\ruby{課題}{かだい}であった。

Bù lại, cơ hội mà Android mang lại là rất lớn. Nếu thành công, Google sẽ sở hữu một nền tảng di động phổ biến toàn cầu, đóng vai trò trung tâm trong trải nghiệm Internet của hàng trăm triệu, thậm chí hàng tỷ người dùng. Việc giữ lại đội ngũ sáng lập Android, đặc biệt là Andy Rubin, và cho phép họ hoạt động tương đối độc lập cho thấy Google hiểu rằng giá trị lớn nhất của thương vụ không chỉ nằm ở mã nguồn, mà ở tầm nhìn và tư duy kỹ thuật phía sau nó.

その\ruby{一方}{いっぽう}で、Androidが\ruby{もたら}{もたら}す\ruby{機会}{きかい}は\ruby{非常}{ひじょう}に\ruby{大}{おお}きかった。\ruby{成功}{せいこう}すれば、Googleは\ruby{世界的}{せかいてき}に\ruby{普及}{ふきゅう}した\ruby{モバイル}{もばいる}\ruby{プラットフォーム}{ぷらっとふぉーむ}を\ruby{所有}{しょゆう}し、\ruby{数億}{すうおく}、さらには\ruby{数十億}{すうじゅうおく}の\ruby{利用者}{りようしゃ}にとっての\ruby{インターネット}{いんたーねっと}\ruby{体験}{たいけん}の\ruby{中心}{ちゅうしん}を\ruby{担}{にな}うことになる。Androidの\ruby{創業}{そうぎょう}\ruby{チーム}{ちーむ}、とりわけAndy Rubinを\ruby{留任}{りゅうにん}させ、\ruby{比較的}{ひかくてき}\ruby{独立}{どくりつ}した\ruby{活動}{かつどう}を\ruby{認}{みと}めたことは、この\ruby{取引}{とりひき}の\ruby{最大}{さいだい}の\ruby{価値}{かち}が\ruby{ソース}{そーす}\ruby{コード}{こーど}そのものではなく、その\ruby{背後}{はいご}にある\ruby{ビジョン}{びじょん}と\ruby{技術的}{ぎじゅつてき}\ruby{思考}{しこう}にあることをGoogleが\ruby{理解}{りかい}していたことを\ruby{示}{しめ}している。

\section{Định hướng dài hạn của Google: xây dựng một nền tảng di động mở để chi phối hệ sinh thái ứng dụng và dịch vụ}
\ruby{Google}{ぐーぐる}の\ruby{長期的}{ちょうきてき}\ruby{方針}{ほうしん}:\ruby{アプリケーション}{あぷりけーしょん}および\ruby{サービス}{さーびす}の\ruby{生態系}{せいたいけい}を\ruby{主導}{しゅどう}する\ruby{開放的}{かいほうてき}な\ruby{移動}{いどう}\ruby{基盤}{きばん}の\ruby{構築}{こうちく}

Ngay từ đầu, Google xác định Android không phải là một sản phẩm độc lập nhằm tạo doanh thu trực tiếp. Android được thiết kế như một hạ tầng chiến lược, đóng vai trò nền móng cho việc phân phối dịch vụ và thu thập dữ liệu người dùng ở quy mô toàn cầu. Cách tiếp cận này phản ánh rõ mô hình kinh doanh cốt lõi của Google: cung cấp nền tảng miễn phí để tối đa hóa mức độ sử dụng dịch vụ, từ đó tạo ra giá trị kinh tế gián tiếp.

\ruby{当初}{とうしょ}から、\ruby{Google}{ぐーぐる}は\ruby{Android}{あんどろいど}を\ruby{直接的}{ちょくせつてき}な\ruby{収益}{しゅうえき}を\ruby{生}{う}み\ruby{出}{だ}すための\ruby{独立}{どくりつ}した\ruby{製品}{せいひん}とは\ruby{位置付}{いちづ}けていなかった。\ruby{Android}{あんどろいど}は、\ruby{戦略的}{せんりゃくてき}な\ruby{インフラ}{いんふら}として\ruby{設計}{せっけい}され、\ruby{世界規模}{せかいきぼ}での\ruby{サービス}{さーびす}\ruby{配信}{はいしん}と\ruby{利用者}{りようしゃ}\ruby{データ}{でーた}の\ruby{収集}{しゅうしゅう}を\ruby{支}{ささ}える\ruby{基盤}{きばん}の\ruby{役割}{やくわり}を\ruby{担}{にな}っていた。この\ruby{アプローチ}{あぷろーち}は、\ruby{無料}{むりょう}の\ruby{基盤}{きばん}を\ruby{提供}{ていきょう}することで\ruby{サービス}{さーびす}\ruby{利用}{りよう}を\ruby{最大化}{さいだいか}し、\ruby{間接的}{かんせつてき}な\ruby{経済的}{けいざいてき}\ruby{価値}{かち}を\ruby{創出}{そうしゅつ}するという、\ruby{Google}{ぐーぐる}の\ruby{中核的}{ちゅうかくてき}な\ruby{事業}{じぎょう}\ruby{モデル}{もでる}を\ruby{明確}{めいかく}に\ruby{反映}{はんえい}している。

Một trong những định hướng quan trọng nhất của Google là xây dựng Android như một nền tảng mở. Việc không thu phí bản quyền hệ điều hành giúp Android nhanh chóng được các nhà sản xuất phần cứng đón nhận, đặc biệt là những công ty không đủ nguồn lực để tự phát triển hệ điều hành riêng. Mô hình này tạo ra một hệ sinh thái phần cứng đa dạng, từ thiết bị cao cấp đến giá rẻ, giúp Android thâm nhập sâu vào nhiều thị trường khác nhau.

\ruby{Google}{ぐーぐる}の\ruby{最重要}{さいじゅうよう}な\ruby{方針}{ほうしん}の\ruby{一}{ひと}つは、\ruby{Android}{あんどろいど}を\ruby{開放的}{かいほうてき}な\ruby{基盤}{きばん}として\ruby{構築}{こうちく}することである。\ruby{オペレーティングシステム}{おぺれーてぃんぐしすてむ}の\ruby{ライセンス}{らいせんす}\ruby{料}{りょう}を\ruby{徴収}{ちょうしゅう}しないことで、\ruby{Android}{あんどろいど}は\ruby{迅速}{じんそく}に\ruby{ハードウェア}{はーどうぇあ}\ruby{製造者}{せいぞうしゃ}に\ruby{受}{う}け\ruby{入}{い}れられ、とりわけ\ruby{自社}{じしゃ}で\ruby{独自}{どくじ}の\ruby{オペレーティングシステム}{おぺれーてぃんぐしすてむ}を\ruby{開発}{かいはつ}する\ruby{資源}{しげん}を\ruby{持}{も}たない\ruby{企業}{きぎょう}にとって\ruby{有力}{ゆうりょく}な\ruby{選択肢}{せんたくし}となった。この\ruby{モデル}{もでる}は、\ruby{高級}{こうきゅう}から\ruby{低価格}{ていかかく}まで\ruby{多様}{たよう}な\ruby{端末}{たんまつ}を\ruby{含}{ふく}む\ruby{ハードウェア}{はーどうぇあ}\ruby{生態系}{せいたいけい}を\ruby{形成}{けいせい}し、\ruby{Android}{あんどろいど}が\ruby{多様}{たよう}な\ruby{市場}{しじょう}へ\ruby{深}{ふか}く\ruby{浸透}{しんとう}することを\ruby{可能}{かのう}にした。

Song song với đó, Google chủ động định hình trải nghiệm người dùng thông qua các dịch vụ cốt lõi được tích hợp chặt chẽ vào hệ điều hành. Công cụ tìm kiếm, email, bản đồ, trình duyệt và sau này là kho ứng dụng trở thành các thành phần gần như không thể tách rời khỏi Android. Dù hệ điều hành mang danh nghĩa mở, Google vẫn giữ quyền kiểm soát các dịch vụ then chốt, qua đó duy trì vị thế trung tâm trong hệ sinh thái.

\ruby{同時}{どうじ}に、\ruby{Google}{ぐーぐる}は\ruby{オペレーティングシステム}{おぺれーてぃんぐしすてむ}に\ruby{密接}{みっせつ}に\ruby{統合}{とうごう}された\ruby{中核}{ちゅうかく}\ruby{サービス}{さーびす}を\ruby{通}{とお}じて、\ruby{利用者}{りようしゃ}\ruby{体験}{たいけん}を\ruby{主導的}{しゅどうてき}に\ruby{形成}{けいせい}した。\ruby{検索}{けんさく}、\ruby{電子}{でんし}\ruby{メール}{めーる}、\ruby{地図}{ちず}、\ruby{ブラウザ}{ぶらうざ}、そして\ruby{後}{のち}の\ruby{アプリ}{あぷり}\ruby{ストア}{すとあ}は、\ruby{Android}{あんどろいど}から\ruby{切}{き}り\ruby{離}{はな}すことが\ruby{困難}{こんなん}な\ruby{構成要素}{こうせいようそ}となった。\ruby{名目上}{めいもくじょう}は\ruby{開放的}{かいほうてき}な\ruby{基盤}{きばん}でありながら、\ruby{Google}{ぐーぐる}は\ruby{要所}{ようしょ}となる\ruby{サービス}{さーびす}の\ruby{支配}{しはい}を\ruby{維持}{いじ}し、\ruby{生態系}{せいたいけい}における\ruby{中心的}{ちゅうしんてき}\ruby{地位}{ちい}を\ruby{確保}{かくほ}した。

Một mục tiêu dài hạn khác là xây dựng hệ sinh thái ứng dụng phong phú xoay quanh Android. Google hiểu rằng giá trị của một nền tảng không chỉ nằm ở hệ điều hành, mà còn ở số lượng và chất lượng ứng dụng. Việc cung cấp bộ công cụ phát triển miễn phí, cho phép phân phối ứng dụng tương đối tự do và tiếp cận lượng người dùng lớn đã tạo động lực mạnh mẽ cho cộng đồng lập trình viên. Hệ sinh thái ứng dụng càng phát triển, Android càng trở nên khó thay thế.

もう\ruby{一}{ひと}つの\ruby{長期的}{ちょうきてき}\ruby{目標}{もくひょう}は、\ruby{Android}{あんどろいど}を\ruby{中核}{ちゅうかく}とする\ruby{豊富}{ほうふ}な\ruby{アプリケーション}{あぷりけーしょん}\ruby{生態系}{せいたいけい}を\ruby{構築}{こうちく}することである。\ruby{Google}{ぐーぐる}は、\ruby{基盤}{きばん}の\ruby{価値}{かち}が\ruby{オペレーティングシステム}{おぺれーてぃんぐしすてむ}\ruby{自体}{じたい}だけでなく、\ruby{アプリケーション}{あぷりけーしょん}の\ruby{数}{かず}と\ruby{質}{しつ}に\ruby{依存}{いぞん}することを\ruby{理解}{りかい}していた。\ruby{無料}{むりょう}の\ruby{開発}{かいはつ}\ruby{ツール}{つーる}を\ruby{提供}{ていきょう}し、\ruby{比較的}{ひかくてき}\ruby{自由}{じゆう}な\ruby{配信}{はいしん}と\ruby{大規模}{だいきぼ}な\ruby{利用者}{りようしゃ}\ruby{層}{そう}への\ruby{接近}{せっきん}を\ruby{可能}{かのう}にしたことは、\ruby{開発者}{かいはつしゃ}\ruby{コミュニティ}{こみゅにてぃ}に\ruby{強力}{きょうりょく}な\ruby{動機}{どうき}を\ruby{与}{あた}えた。\ruby{生態系}{せいたいけい}が\ruby{発展}{はってん}するほど、\ruby{Android}{あんどろいど}は\ruby{代替}{だいたい}しがたい\ruby{存在}{そんざい}となる。

Từ góc độ dữ liệu, Android cho phép Google tiếp cận trực tiếp với hành vi người dùng trên thiết bị cá nhân nhất của họ. Dữ liệu về tìm kiếm, vị trí, thói quen sử dụng và ngữ cảnh di động mang lại giá trị vượt trội so với dữ liệu từ máy tính cá nhân. Đây là nền tảng để Google cải thiện chất lượng dịch vụ, cá nhân hóa quảng cáo và củng cố lợi thế cạnh tranh trong dài hạn.

\ruby{データ}{でーた}の\ruby{観点}{かんてん}から\ruby{見}{み}ると、\ruby{Android}{あんどろいど}は\ruby{Google}{ぐーぐる}に\ruby{個人}{こじん}\ruby{端末}{たんまつ}における\ruby{利用者}{りようしゃ}\ruby{行動}{こうどう}への\ruby{直接的}{ちょくせつてき}な\ruby{接近}{せっきん}を\ruby{可能}{かのう}にした。\ruby{検索}{けんさく}、\ruby{位置}{いち}、\ruby{利用}{りよう}\ruby{習慣}{しゅうかん}、および\ruby{移動}{いどう}\ruby{文脈}{ぶんみゃく}に\ruby{関}{かん}する\ruby{データ}{でーた}は、\ruby{個人}{こじん}\ruby{用}{よう}\ruby{コンピュータ}{こんぴゅーた}から\ruby{得}{え}られる\ruby{情報}{じょうほう}よりも\ruby{高}{たか}い\ruby{価値}{かち}を\ruby{持}{も}つ。これは、\ruby{Google}{ぐーぐる}が\ruby{サービス}{さーびす}\ruby{品質}{ひんしつ}を\ruby{向上}{こうじょう}させ、\ruby{広告}{こうこく}を\ruby{個別化}{こべつか}し、\ruby{長期的}{ちょうきてき}な\ruby{競争}{きょうそう}\ruby{優位}{ゆうい}を\ruby{強化}{きょうか}するための\ruby{基盤}{きばん}となる。

Tuy nhiên, việc theo đuổi mô hình nền tảng mở cũng buộc Google phải chấp nhận những đánh đổi. Phân mảnh hệ điều hành, khác biệt về giao diện và tốc độ cập nhật giữa các nhà sản xuất là hệ quả khó tránh khỏi. Google lựa chọn cách tiếp cận thực dụng: chấp nhận mức độ không đồng nhất nhất định để đổi lấy quy mô và độ phủ thị trường. Trong chiến lược này, quy mô được ưu tiên hơn sự kiểm soát tuyệt đối.

しかし、\ruby{開放的}{かいほうてき}な\ruby{基盤}{きばん}\ruby{モデル}{もでる}を\ruby{追求}{ついきゅう}することは、\ruby{Google}{ぐーぐる}に\ruby{一定}{いってい}の\ruby{トレードオフ}{とれーどおふ}を\ruby{受}{う}け\ruby{入}{い}れることを\ruby{強}{つよ}いた。\ruby{オペレーティングシステム}{おぺれーてぃんぐしすてむ}の\ruby{断片化}{だんぺんか}、\ruby{製造者}{せいぞうしゃ}ごとの\ruby{インターフェース}{いんたーふぇーす}や\ruby{更新}{こうしん}\ruby{速度}{そくど}の\ruby{差異}{さい}は、\ruby{避}{さ}けがたい\ruby{結果}{けっか}である。\ruby{Google}{ぐーぐる}は\ruby{実務的}{じつむてき}な\ruby{方針}{ほうしん}を\ruby{選択}{せんたく}し、\ruby{市場}{しじょう}の\ruby{規模}{きぼ}と\ruby{普及度}{ふきゅうど}を\ruby{得}{え}るために、\ruby{一定}{いってい}の\ruby{不統一}{ふとういつ}を\ruby{許容}{きょよう}した。この\ruby{戦略}{せんりゃく}においては、\ruby{完全}{かんぜん}な\ruby{統制}{とうせい}よりも\ruby{規模}{きぼ}が\ruby{優先}{ゆうせん}される。

Về bản chất, định hướng dài hạn của Google với Android không nhằm thống trị thị trường bằng phần cứng hay bán phần mềm, mà bằng việc trở thành lớp hạ tầng không thể thiếu của Internet di động. Khi Android hiện diện trên phần lớn thiết bị, Google có thể đảm bảo rằng các dịch vụ của mình luôn nằm ở trung tâm trải nghiệm người dùng. Đây chính là giá trị chiến lược lớn nhất mà thương vụ Android mang lại.

\ruby{本質的}{ほんしつてき}に、\ruby{Android}{あんどろいど}に\ruby{対}{たい}する\ruby{Google}{ぐーぐる}の\ruby{長期}{ちょうき}\ruby{方針}{ほうしん}は、\ruby{ハードウェア}{はーどうぇあ}や\ruby{ソフトウェア}{そふとうぇあ}の\ruby{販売}{はんばい}によって\ruby{市場}{しじょう}を\ruby{支配}{しはい}することではなく、\ruby{移動}{いどう}\ruby{インターネット}{いんたーねっと}における\ruby{不可欠}{ふかけつ}な\ruby{インフラ}{いんふら}と\ruby{化}{か}すことにある。\ruby{Android}{あんどろいど}が\ruby{多数}{たすう}の\ruby{端末}{たんまつ}に\ruby{存在}{そんざい}することで、\ruby{Google}{ぐーぐる}は\ruby{自社}{じしゃ}\ruby{サービス}{さーびす}が\ruby{常}{つね}に\ruby{利用者}{りようしゃ}\ruby{体験}{たいけん}の\ruby{中心}{ちゅうしん}に\ruby{位置}{いち}することを\ruby{保証}{ほしょう}できる。これこそが、\ruby{Android}{あんどろいど}\ruby{事業}{じぎょう}が\ruby{もたら}{もたら}した\ruby{最大}{さいだい}の\ruby{戦略的}{せんりゃくてき}\ruby{価値}{かち}である。

Kết thúc chương này, có thể thấy rằng việc Google mua lại Android Inc. không phải là một quyết định mang tính cơ hội ngắn hạn, mà là bước đi có tính toán nhằm tái định vị Google trong bối cảnh Internet đang dịch chuyển từ máy tính cá nhân sang thiết bị di động. Android, từ một công ty khởi nghiệp nhỏ, đã trở thành nền tảng then chốt trong chiến lược dài hạn đó.

\ruby{本章}{ほんしょう}の\ruby{結}{むす}びとして、\ruby{Google}{ぐーぐる}による\ruby{Android}{あんどろいど}\ruby{Inc.}{いんく}の\ruby{買収}{ばいしゅう}は、\ruby{短期的}{たんきてき}な\ruby{機会主義的}{きかいしゅぎてき}\ruby{判断}{はんだん}ではなく、\ruby{インターネット}{いんたーねっと}が\ruby{個人}{こじん}\ruby{用}{よう}\ruby{コンピュータ}{こんぴゅーた}から\ruby{移動}{いどう}\ruby{端末}{たんまつ}へと\ruby{移行}{いこう}する\ruby{状況}{じょうきょう}において、\ruby{Google}{ぐーぐる}を\ruby{再定位}{さいていい}するための\ruby{計算}{けいさん}された\ruby{一手}{いって}であったことが\ruby{分}{わ}かる。\ruby{小規模}{しょうきぼ}な\ruby{スタートアップ}{すたーとあっぷ}であった\ruby{Android}{あんどろいど}は、この\ruby{長期}{ちょうき}\ruby{戦略}{せんりゃく}の\ruby{中核}{ちゅうかく}を\ruby{成}{な}す\ruby{基盤}{きばん}へと\ruby{成長}{せいちょう}したのである。
