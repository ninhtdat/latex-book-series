\chapter{Bối cảnh ra đời của Android}
\ruby{Android}{あんどろいど}の\ruby{誕生}{たんじょう}の\ruby{背景}{はいけい}

Để hiểu rõ vì sao Android xuất hiện và nhanh chóng trở thành nền tảng di động phổ biến nhất thế giới, cần đặt nó vào đúng bối cảnh lịch sử và công nghệ của đầu những năm 2000. Đây là giai đoạn thị trường thiết bị di động đang ở bước chuyển mình quan trọng, từ các thiết bị liên lạc đơn thuần sang những thiết bị có khả năng xử lý thông tin và dữ liệu. Tuy nhiên, những giới hạn nghiêm trọng về phần cứng, phần mềm và mô hình phát triển lúc bấy giờ đã tạo ra một khoảng trống lớn, đặt nền móng cho sự ra đời của một nền tảng di động mới.
\ruby{Android}{あんどろいど}が\ruby{登場}{とうじょう}し、\ruby{急速}{きゅうそく}に\ruby{世界}{せかい}で\ruby{最}{もっと}も\ruby{普及}{ふきゅう}した\ruby{モバイル}{もばいる}\ruby{プラットフォーム}{ぷらっとふぉーむ}となった\ruby{理由}{りゆう}を\ruby{理解}{りかい}するためには、2000\ruby{年代}{ねんだい}\ruby{初頭}{しょとう}の\ruby{歴史的}{れきしてき}・\ruby{技術的}{ぎじゅつてき}\ruby{背景}{はいけい}の\ruby{中}{なか}で\ruby{位置}{いち}づける\ruby{必要}{ひつよう}がある。この\ruby{時期}{じき}は、\ruby{モバイル}{もばいる}\ruby{端末}{たんまつ}\ruby{市場}{しじょう}が、\ruby{単純}{たんじゅん}な\ruby{通信}{つうしん}\ruby{機器}{きき}から、\ruby{情報}{じょうほう}や\ruby{データ}{でーた}を\ruby{処理}{しょり}できる\ruby{端末}{たんまつ}へと\ruby{重要}{じゅうよう}な\ruby{転換}{てんかん}を\ruby{遂}{と}げつつあった\ruby{段階}{だんかい}である。しかし、\ruby{当時}{とうじ}の\ruby{ハードウェア}{はーどうぇあ}、\ruby{ソフトウェア}{そふとうぇあ}、および\ruby{開発}{かいはつ}\ruby{モデル}{もでる}における\ruby{深刻}{しんこく}な\ruby{制約}{せいやく}が\ruby{大}{おお}きな\ruby{空白}{くうはく}を\ruby{生}{う}み、\ruby{新}{あたら}しい\ruby{モバイル}{もばいる}\ruby{プラットフォーム}{ぷらっとふぉーむ}の\ruby{誕生}{たんじょう}に\ruby{向}{む}けた\ruby{基盤}{きばん}を\ruby{形成}{けいせい}した。

\section{Thị trường thiết bị di động đầu những năm 2000}
2000\ruby{年代}{ねんだい}\ruby{初頭}{しょとう}の\ruby{モバイル}{もばいる}\ruby{端末}{たんまつ}\ruby{市場}{しじょう}

Vào đầu thế kỷ XXI, thị trường thiết bị di động chủ yếu xoay quanh điện thoại di động truyền thống. Mục đích sử dụng chính của các thiết bị này là gọi điện và nhắn tin SMS. Các chức năng bổ sung như danh bạ, lịch, báo thức hay trò chơi đơn giản chỉ đóng vai trò phụ trợ, không phải trọng tâm thiết kế.
21\ruby{世紀}{せいき}\ruby{初頭}{しょとう}において、\ruby{モバイル}{もばいる}\ruby{端末}{たんまつ}\ruby{市場}{しじょう}は\ruby{主}{おも}に\ruby{従来}{じゅうらい}の\ruby{携帯}{けいたい}\ruby{電話}{でんわ}を\ruby{中心}{ちゅうしん}としていた。これらの\ruby{端末}{たんまつ}の\ruby{主}{おも}な\ruby{利用}{りよう}\ruby{目的}{もくてき}は、\ruby{通話}{つうわ}およびSMSによる\ruby{メッセージ}{めっせーじ}\ruby{送信}{そうしん}であった。\ruby{連絡}{れんらく}\ruby{先}{さき}、\ruby{カレンダー}{かれんだー}、\ruby{アラーム}{あらーむ}、\ruby{簡単}{かんたん}な\ruby{ゲーム}{げーむ}などの\ruby{付加}{ふか}\ruby{機能}{きのう}は、\ruby{設計}{せっけい}の\ruby{主眼}{しゅがん}ではなく、\ruby{補助的}{ほじょてき}な\ruby{役割}{やくわり}にとどまっていた。

Về mặt phần cứng, các thiết bị di động thời kỳ này bị giới hạn nghiêm trọng. Bộ xử lý thường là đơn nhân với xung nhịp thấp, chỉ đủ đáp ứng các tác vụ cơ bản. Bộ nhớ RAM rất nhỏ, thường chỉ vài megabyte, khiến khả năng chạy ứng dụng phức tạp gần như không khả thi. Không gian lưu trữ hạn chế, chủ yếu dùng cho danh bạ, tin nhắn và một số dữ liệu hệ thống. Màn hình có kích thước nhỏ, độ phân giải thấp, khả năng hiển thị màu còn hạn chế, ảnh hưởng trực tiếp đến trải nghiệm người dùng.
\ruby{ハードウェア}{はーどうぇあ}の\ruby{観点}{かんてん}からは、\ruby{当時}{とうじ}の\ruby{モバイル}{もばいる}\ruby{端末}{たんまつ}は\ruby{深刻}{しんこく}な\ruby{制限}{せいげん}を\ruby{受}{う}けていた。\ruby{プロセッサ}{ぷろせっさ}は\ruby{低}{ひく}い\ruby{クロック}{くろっく}の\ruby{単}{たん}\ruby{コア}{こあ}が\ruby{一般的}{いっぱんてき}で、\ruby{基本的}{きほんてき}な\ruby{処理}{しょり}のみを\ruby{担}{にな}う\ruby{水準}{すいじゅん}であった。RAMは\ruby{極}{きわ}めて\ruby{少}{すく}なく、\ruby{多}{おお}くの場合\ruby{数}{すう}MB\ruby{程度}{ていど}であり、\ruby{複雑}{ふくざつ}な\ruby{アプリケーション}{あぷりけーしょん}の\ruby{実行}{じっこう}は\ruby{事実上}{じじつじょう}\ruby{困難}{こんなん}であった。\ruby{ストレージ}{すとれーじ}も\ruby{限定的}{げんていてき}で、\ruby{主}{おも}に\ruby{連絡}{れんらく}\ruby{先}{さき}、\ruby{メッセージ}{めっせーじ}、および\ruby{一部}{いちぶ}の\ruby{システム}{しすてむ}\ruby{データ}{でーた}に\ruby{使用}{しよう}された。\ruby{画面}{がめん}は\ruby{小型}{こがた}で\ruby{解像度}{かいぞうど}も\ruby{低}{ひく}く、\ruby{色}{いろ}\ruby{表示}{ひょうじ}\ruby{能力}{のうりょく}も\ruby{限定}{げんてい}されており、\ruby{ユーザー}{ゆーざー}\ruby{体験}{たいけん}に\ruby{直接}{ちょくせつ}\ruby{影響}{えいきょう}を\ruby{与}{あた}えていた。

Pin cũng là một yếu tố ràng buộc quan trọng. Do công nghệ pin chưa phát triển, các nhà sản xuất buộc phải tối ưu thiết bị theo hướng tiết kiệm năng lượng tối đa. Điều này dẫn đến việc cắt giảm nhiều khả năng xử lý và hiển thị, làm cho thiết bị di động khó có thể đảm nhận vai trò của một nền tảng tính toán đa năng.
\ruby{バッテリー}{ばってりー}も\ruby{重要}{じゅうよう}な\ruby{制約}{せいやく}\ruby{要因}{よういん}であった。\ruby{電池}{でんち}\ruby{技術}{ぎじゅつ}が\ruby{未成熟}{みじゅく}であったため、\ruby{製造}{せいぞう}\ruby{業者}{ぎょうしゃ}は\ruby{最大限}{さいだいげん}の\ruby{省電力}{しょうでんりょく}\ruby{設計}{せっけい}を\ruby{余儀}{よぎ}なくされた。その\ruby{結果}{けっか}、\ruby{処理}{しょり}および\ruby{表示}{ひょうじ}\ruby{能力}{のうりょく}の\ruby{多}{おお}くが\ruby{削減}{さくげん}され、\ruby{モバイル}{もばいる}\ruby{端末}{たんまつ}が\ruby{汎用}{はんよう}\ruby{計算}{けいさん}\ruby{プラットフォーム}{ぷらっとふぉーむ}としての\ruby{役割}{やくわり}を\ruby{果}{は}たすことは\ruby{困難}{こんなん}であった。

Về xu hướng sử dụng, người dùng chưa có thói quen cài đặt và sử dụng nhiều ứng dụng. Phần lớn phần mềm được cài sẵn từ nhà sản xuất và gần như không thay đổi trong suốt vòng đời của thiết bị. Khái niệm cửa hàng ứng dụng, cập nhật phần mềm thường xuyên hay cá nhân hóa sâu gần như chưa tồn tại. Điện thoại di động được xem là sản phẩm tiêu dùng đóng, không phải là một nền tảng mở để mở rộng chức năng.
\ruby{利用}{りよう}\ruby{傾向}{けいこう}の\ruby{面}{めん}では、\ruby{ユーザー}{ゆーざー}は\ruby{多数}{たすう}の\ruby{アプリケーション}{あぷりけーしょん}を\ruby{インストール}{いんすとーる}して\ruby{使用}{しよう}する\ruby{習慣}{しゅうかん}を\ruby{持}{も}っていなかった。\ruby{ソフトウェア}{そふとうぇあ}の\ruby{大半}{たいはん}は\ruby{メーカー}{めーかー}によって\ruby{事前}{じぜん}に\ruby{搭載}{とうさい}され、\ruby{端末}{たんまつ}の\ruby{寿命}{じゅみょう}を\ruby{通}{とお}じて\ruby{ほとんど}{ほとんど}\ruby{変更}{へんこう}されなかった。\ruby{アプリ}{あぷり}\ruby{ストア}{すとあ}、\ruby{頻繁}{ひんぱん}な\ruby{ソフトウェア}{そふとうぇあ}\ruby{更新}{こうしん}、あるいは\ruby{高度}{こうど}な\ruby{個人化}{こじんか}といった\ruby{概念}{がいねん}は\ruby{ほぼ}{ほぼ}\ruby{存在}{そんざい}していなかった。\ruby{携帯}{けいたい}\ruby{電話}{でんわ}は\ruby{閉鎖的}{へいさてき}な\ruby{消費}{しょうひ}\ruby{製品}{せいひん}と\ruby{見}{み}なされ、\ruby{機能}{きのう}を\ruby{拡張}{かくちょう}するための\ruby{オープン}{おーぷん}な\ruby{プラットフォーム}{ぷらっとふぉーむ}ではなかった。

Ngoài ra, sự phân mảnh về thiết bị là một đặc điểm nổi bật của thị trường. Mỗi nhà sản xuất theo đuổi thiết kế phần cứng và phần mềm riêng, từ kích thước màn hình, bàn phím vật lý cho đến cách bố trí phím chức năng. Điều này khiến việc xây dựng các ứng dụng có thể chạy đồng nhất trên nhiều thiết bị trở nên rất khó khăn.
\ruby{さらに}{さらに}、\ruby{端末}{たんまつ}の\ruby{断片化}{だんぺんか}は\ruby{市場}{しじょう}の\ruby{顕著}{けんちょ}な\ruby{特徴}{とくちょう}であった。\ruby{各}{かく}\ruby{メーカー}{めーかー}は、\ruby{画面}{がめん}\ruby{サイズ}{さいず}、\ruby{物理}{ぶつり}\ruby{キーボード}{きーぼーど}、\ruby{機能}{きのう}\ruby{キー}{きー}の\ruby{配置}{はいち}に\ruby{至}{いた}るまで、\ruby{独自}{どくじ}の\ruby{ハードウェア}{はーどうぇあ}および\ruby{ソフトウェア}{そふとうぇあ}\ruby{設計}{せっけい}を\ruby{採用}{さいよう}していた。そのため、\ruby{複数}{ふくすう}の\ruby{端末}{たんまつ}で\ruby{同一}{どういつ}に\ruby{動作}{どうさ}する\ruby{アプリケーション}{あぷりけーしょん}を\ruby{開発}{かいはつ}することは\ruby{非常}{ひじょう}に\ruby{困難}{こんなん}であった。

Tóm lại, thị trường thiết bị di động đầu những năm 2000 được đặc trưng bởi phần cứng yếu, tài nguyên hạn chế, mục tiêu sử dụng đơn giản và mô hình thiết kế khép kín. Chính những đặc điểm này đã tạo ra rào cản lớn cho sự phát triển của các ứng dụng phong phú và đặt ra nhu cầu về một cách tiếp cận mới trong thiết kế nền tảng di động.
\ruby{要約}{ようやく}すると、2000\ruby{年代}{ねんだい}\ruby{初頭}{しょとう}の\ruby{モバイル}{もばいる}\ruby{端末}{たんまつ}\ruby{市場}{しじょう}は、\ruby{非力}{ひりき}な\ruby{ハードウェア}{はーどうぇあ}、\ruby{限}{かぎ}られた\ruby{資源}{しげん}、\ruby{単純}{たんじゅん}な\ruby{利用}{りよう}\ruby{目的}{もくてき}、および\ruby{閉鎖的}{へいさてき}な\ruby{設計}{せっけい}\ruby{モデル}{もでる}によって\ruby{特徴}{とくちょう}づけられていた。これらの\ruby{特性}{とくせい}が、\ruby{多様}{たよう}な\ruby{アプリケーション}{あぷりけーしょん}の\ruby{発展}{はってん}に\ruby{大}{おお}きな\ruby{障壁}{しょうへき}を\ruby{生}{う}み、\ruby{モバイル}{もばいる}\ruby{プラットフォーム}{ぷらっとふぉーむ}\ruby{設計}{せっけい}における\ruby{新}{あたら}しい\ruby{アプローチ}{あぷろーち}の\ruby{必要性}{ひつようせい}を\ruby{提示}{ていじ}した。

\section{Các hệ điều hành di động phổ biến trước Android}
\ruby{Android}{あんどろいど} \ruby{登場}{とうじょう} \ruby{以前}{いぜん}の\ruby{主要}{しゅよう}な\ruby{携帯}{けいたい}\ruby{端末}{たんまつ}\ruby{向}{む}け\ruby{オペレーティング}{おぺれーてぃんぐ}\ruby{システム}{しすてむ}

Trước khi Android xuất hiện, thị trường hệ điều hành di động bị chi phối bởi một số nền tảng lớn như Symbian, Windows Mobile và BlackBerry OS. Mỗi hệ điều hành này gắn liền với một hoặc một nhóm nhà sản xuất cụ thể, phản ánh rõ tư duy phát triển phần mềm di động mang tính đóng và phân mảnh của thời kỳ đầu.

\ruby{Android}{あんどろいど}が\ruby{登場}{とうじょう}する\ruby{以前}{いぜん}、\ruby{携帯}{けいたい}\ruby{端末}{たんまつ}\ruby{向}{む}け\ruby{オペレーティング}{おぺれーてぃんぐ}\ruby{システム}{しすてむ}の\ruby{市場}{しじょう}は、Symbian、Windows Mobile、BlackBerry OSといった\ruby{少数}{しょうすう}の\ruby{大規模}{だいきぼ}な\ruby{プラットフォーム}{ぷらっとふぉーむ}によって\ruby{支配}{しはい}されていた。これら\ruby{各}{かく}\ruby{オペレーティング}{おぺれーてぃんぐ}\ruby{システム}{しすてむ}は、\ruby{特定}{とくてい}の\ruby{製造}{せいぞう}\ruby{業者}{ぎょうしゃ}、または\ruby{特定}{とくてい}の\ruby{メーカー}{めーかー}\ruby{群}{ぐん}と\ruby{密接}{みっせつ}に\ruby{結}{むす}び\ruby{付}{つ}いており、\ruby{初期}{しょき}の\ruby{携帯}{けいたい}\ruby{ソフトウェア}{そふとうぇあ}\ruby{開発}{かいはつ}に\ruby{見}{み}られた\ruby{閉鎖的}{へいさてき}かつ\ruby{分断}{ぶんだん}された\ruby{思想}{しそう}を\ruby{明確}{めいかく}に\ruby{反映}{はんえい}していた。

Symbian là hệ điều hành di động phổ biến nhất trong giai đoạn này, đặc biệt trên các thiết bị của Nokia. Symbian được thiết kế với mục tiêu tối ưu cho phần cứng hạn chế, tiêu thụ ít tài nguyên và tiết kiệm năng lượng. Tuy nhiên, kiến trúc của Symbian rất phức tạp, khó tiếp cận đối với lập trình viên. Việc phát triển ứng dụng yêu cầu kiến thức sâu về hệ thống, quản lý bộ nhớ thủ công và tuân thủ nhiều ràng buộc kỹ thuật. Ngoài ra, Symbian tồn tại nhiều biến thể khác nhau tùy theo nhà sản xuất, dẫn đến sự không tương thích giữa các thiết bị, làm gia tăng chi phí phát triển và bảo trì ứng dụng.

Symbianは、この\ruby{時期}{じき}において\ruby{最}{もっと}も\ruby{普及}{ふきゅう}した\ruby{携帯}{けいたい}\ruby{端末}{たんまつ}\ruby{向}{む}け\ruby{オペレーティング}{おぺれーてぃんぐ}\ruby{システム}{しすてむ}であり、とりわけNokia\ruby{製}{せい}の\ruby{端末}{たんまつ}で\ruby{広}{ひろ}く\ruby{採用}{さいよう}されていた。Symbianは、\ruby{制約}{せいやく}の\ruby{多}{おお}い\ruby{ハードウェア}{はーどうぇあ}に\ruby{最適化}{さいてきか}することを\ruby{目的}{もくてき}として\ruby{設計}{せっけい}され、\ruby{資源}{しげん}の\ruby{消費}{しょうひ}を\ruby{抑}{おさ}え、\ruby{省}{しょう}\ruby{電力}{でんりょく}を\ruby{実現}{じつげん}していた。しかしながら、Symbianの\ruby{アーキテクチャ}{あーきてくちゃ}は\ruby{極}{きわ}めて\ruby{複雑}{ふくざつ}であり、\ruby{開発}{かいはつ}\ruby{者}{しゃ}にとって\ruby{習得}{しゅうとく}が\ruby{困難}{こんなん}であった。\ruby{アプリケーション}{あぷりけーしょん}の\ruby{開発}{かいはつ}には、\ruby{システム}{しすてむ}に\ruby{関}{かん}する\ruby{高度}{こうど}な\ruby{知識}{ちしき}、\ruby{手動}{しゅどう}による\ruby{メモリ}{めもり}\ruby{管理}{かんり}、および\ruby{多数}{たすう}の\ruby{技術的}{ぎじゅつてき}な\ruby{制約}{せいやく}の\ruby{遵守}{じゅんしゅ}が\ruby{求}{もと}められた。さらに、Symbianは\ruby{メーカー}{めーかー}ごとに\ruby{異}{こと}なる\ruby{複数}{ふくすう}の\ruby{派生}{はせい}\ruby{形態}{けいたい}を\ruby{有}{ゆう}しており、\ruby{端末}{たんまつ}\ruby{間}{かん}の\ruby{互換性}{ごかんせい}の\ruby{欠如}{けつじょ}を\ruby{招}{まね}き、\ruby{開発}{かいはつ}および\ruby{保守}{ほしゅ}に\ruby{要}{よう}する\ruby{コスト}{こすと}を\ruby{増大}{ぞうだい}させていた。

Windows Mobile là nỗ lực của Microsoft nhằm đưa trải nghiệm quen thuộc của Windows lên thiết bị di động. Hệ điều hành này hướng tới người dùng doanh nghiệp, tập trung vào email, lịch và các ứng dụng văn phòng. Mặc dù cung cấp môi trường phát triển tương đối tốt so với mặt bằng chung thời kỳ đó, Windows Mobile vẫn gặp hạn chế lớn về hiệu năng và khả năng tối ưu cho thiết bị cầm tay. Giao diện và triết lý thiết kế chịu ảnh hưởng nặng từ máy tính để bàn, khiến trải nghiệm người dùng trên thiết bị di động trở nên kém tự nhiên và khó sử dụng.

Windows Mobileは、Microsoftが\ruby{従来}{じゅうらい}のWindowsの\ruby{操作}{そうさ}\ruby{体験}{たいけん}を\ruby{携帯}{けいたい}\ruby{端末}{たんまつ}へ\ruby{移植}{いしょく}しようとした\ruby{試}{こころ}みである。この\ruby{オペレーティング}{おぺれーてぃんぐ}\ruby{システム}{しすてむ}は、\ruby{企業}{きぎょう}\ruby{利用}{りよう}\ruby{者}{しゃ}を\ruby{主}{おも}な\ruby{対象}{たいしょう}とし、\ruby{電子}{でんし}\ruby{メール}{めーる}、\ruby{予定}{よてい}\ruby{管理}{かんり}、\ruby{オフィス}{おふぃす}\ruby{アプリケーション}{あぷりけーしょん}に\ruby{重点}{じゅうてん}を\ruby{置}{お}いていた。\ruby{当時}{とうじ}の\ruby{平均的}{へいきんてき}な\ruby{水準}{すいじゅん}と\ruby{比較}{ひかく}すれば、\ruby{比較的}{ひかくてき}\ruby{良好}{りょうこう}な\ruby{開発}{かいはつ}\ruby{環境}{かんきょう}を\ruby{提供}{ていきょう}していたものの、Windows Mobileは\ruby{性能}{せいのう}および\ruby{携帯}{けいたい}\ruby{端末}{たんまつ}への\ruby{最適化}{さいてきか}の\ruby{点}{てん}で\ruby{重大}{じゅうだい}な\ruby{制限}{せいげん}を\ruby{抱}{かか}えていた。\ruby{ユーザー}{ゆーざー}\ruby{インターフェース}{いんたーふぇーす}や\ruby{設計}{せっけい}\ruby{思想}{しそう}は、\ruby{デスクトップ}{ですくとっぷ}\ruby{コンピュータ}{こんぴゅーた}からの\ruby{影響}{えいきょう}が\ruby{強}{つよ}く、\ruby{携帯}{けいたい}\ruby{端末}{たんまつ}における\ruby{利用}{りよう}\ruby{体験}{たいけん}を\ruby{不自然}{ふしぜん}で\ruby{扱}{あつか}いにくいものとしていた。

BlackBerry OS lại đi theo hướng khác, tập trung mạnh vào bảo mật và dịch vụ email thời gian thực. Đây là nền tảng được ưa chuộng trong giới doanh nghiệp và chính phủ. Tuy nhiên, BlackBerry kiểm soát chặt chẽ cả phần cứng lẫn phần mềm, tạo ra một hệ sinh thái khép kín. Việc phát triển và phân phối ứng dụng bị giới hạn nghiêm ngặt, không khuyến khích sự tham gia rộng rãi của cộng đồng lập trình viên độc lập.

BlackBerry OSは、\ruby{異}{こと}なる\ruby{方向性}{ほうこうせい}を\ruby{採}{と}り、\ruby{高度}{こうど}な\ruby{セキュリティ}{せきゅりてぃ}と\ruby{リアルタイム}{りあるたいむ}の\ruby{電子}{でんし}\ruby{メール}{めーる}\ruby{サービス}{さーびす}に\ruby{重点}{じゅうてん}を\ruby{置}{お}いていた。この\ruby{プラットフォーム}{ぷらっとふぉーむ}は、\ruby{企業}{きぎょう}や\ruby{政府}{せいふ}\ruby{機関}{きかん}において\ruby{高}{たか}い\ruby{評価}{ひょうか}を\ruby{受}{う}けていた。しかし、BlackBerryは\ruby{ハードウェア}{はーどうぇあ}と\ruby{ソフトウェア}{そふとうぇあ}の\ruby{双方}{そうほう}を\ruby{厳格}{げんかく}に\ruby{管理}{かんり}し、\ruby{閉鎖的}{へいさてき}な\ruby{エコシステム}{えこしすてむ}を\ruby{形成}{けいせい}していた。\ruby{アプリケーション}{あぷりけーしょん}の\ruby{開発}{かいはつ}および\ruby{配布}{はいふ}は\ruby{厳}{きび}しく\ruby{制限}{せいげん}され、\ruby{独立}{どくりつ}した\ruby{開発}{かいはつ}\ruby{者}{しゃ}\ruby{コミュニティ}{こみゅにてぃ}の\ruby{広範}{こうはん}な\ruby{参加}{さんか}を\ruby{促}{うなが}すものではなかった。

Điểm chung của các hệ điều hành di động trước Android là mô hình phát triển đóng và phụ thuộc lớn vào nhà sản xuất. Mã nguồn không được công khai hoặc chỉ mở ở mức rất hạn chế, khiến việc tùy biến, cải tiến và mở rộng nền tảng trở nên khó khăn. Các quyết định về tính năng, API và định hướng phát triển chủ yếu do nhà cung cấp kiểm soát, không dựa trên nhu cầu đa dạng của cộng đồng phát triển.

Android\ruby{以前}{いぜん}の\ruby{携帯}{けいたい}\ruby{端末}{たんまつ}\ruby{向}{む}け\ruby{オペレーティング}{おぺれーてぃんぐ}\ruby{システム}{しすてむ}に\ruby{共通}{きょうつう}していた\ruby{特徴}{とくちょう}は、\ruby{閉鎖的}{へいさてき}な\ruby{開発}{かいはつ}\ruby{モデル}{もでる}であり、\ruby{製造}{せいぞう}\ruby{業者}{ぎょうしゃ}への\ruby{強}{つよ}い\ruby{依存}{いぞん}であった。\ruby{ソース}{そーす}\ruby{コード}{こーど}は\ruby{公開}{こうかい}されていないか、あるいは\ruby{極}{きわ}めて\ruby{限定的}{げんていてき}にのみ\ruby{開放}{かいほう}されており、\ruby{カスタマイズ}{かすたまいず}、\ruby{改良}{かいりょう}、\ruby{拡張}{かくちょう}を\ruby{行}{おこな}うことが\ruby{困難}{こんなん}であった。\ruby{機能}{きのう}、API、\ruby{開発}{かいはつ}\ruby{方針}{ほうしん}に\ruby{関}{かん}する\ruby{意思}{いし}\ruby{決定}{けってい}は、\ruby{主}{おも}として\ruby{提供}{ていきょう}\ruby{者}{しゃ}によって\ruby{管理}{かんり}され、\ruby{開発}{かいはつ}\ruby{者}{しゃ}\ruby{コミュニティ}{こみゅにてぃ}の\ruby{多様}{たよう}な\ruby{需要}{じゅよう}に\ruby{基}{もと}づくものではなかった。

Bên cạnh đó, khả năng mở rộng của các hệ điều hành này rất hạn chế. Việc tích hợp công nghệ mới, đặc biệt là các dịch vụ Internet và ứng dụng trực tuyến, diễn ra chậm chạp. Mỗi nền tảng sử dụng bộ API riêng, thiếu tiêu chuẩn chung, làm gia tăng sự phân mảnh của thị trường ứng dụng. Một ứng dụng được phát triển cho hệ điều hành hoặc thiết bị này thường không thể chạy trên hệ điều hành hoặc thiết bị khác mà không cần chỉnh sửa đáng kể.

さらに、これらの\ruby{オペレーティング}{おぺれーてぃんぐ}\ruby{システム}{しすてむ}は\ruby{拡張性}{かくちょうせい}に\ruby{乏}{とぼ}しかった。\ruby{新技術}{しんぎじゅつ}、とりわけ\ruby{インターネット}{いんたーねっと}\ruby{サービス}{さーびす}や\ruby{オンライン}{おんらいん}\ruby{アプリケーション}{あぷりけーしょん}の\ruby{統合}{とうごう}は\ruby{緩慢}{かんまん}に\ruby{進}{すす}んだ。\ruby{各}{かく}\ruby{プラットフォーム}{ぷらっとふぉーむ}は\ruby{独自}{どくじ}のAPIを\ruby{採用}{さいよう}し、\ruby{共通}{きょうつう}の\ruby{標準}{ひょうじゅん}を\ruby{欠}{か}いていたため、\ruby{アプリケーション}{あぷりけーしょん}\ruby{市場}{しじょう}の\ruby{分断}{ぶんだん}が\ruby{拡大}{かくだい}した。ある\ruby{オペレーティング}{おぺれーてぃんぐ}\ruby{システム}{しすてむ}や\ruby{端末}{たんまつ}のために\ruby{開発}{かいはつ}された\ruby{アプリケーション}{あぷりけーしょん}は、\ruby{大幅}{おおはば}な\ruby{修正}{しゅうせい}を\ruby{行}{おこな}わなければ、\ruby{他}{ほか}の\ruby{オペレーティング}{おぺれーてぃんぐ}\ruby{システム}{しすてむ}や\ruby{端末}{たんまつ}で\ruby{動作}{どうさ}しないことが\ruby{多}{おお}かった。

Những hạn chế này không chỉ ảnh hưởng đến lập trình viên mà còn tác động trực tiếp đến người dùng cuối. Số lượng ứng dụng ít, chất lượng không đồng đều và khó tiếp cận khiến thiết bị di động chưa thể trở thành một nền tảng phong phú cho công việc và giải trí. Chính trong bối cảnh đó, nhu cầu về một hệ điều hành di động mới, có mô hình mở, dễ mở rộng và thân thiện với hệ sinh thái phát triển, ngày càng trở nên rõ ràng.

これらの\ruby{制約}{せいやく}は、\ruby{開発}{かいはつ}\ruby{者}{しゃ}のみならず、\ruby{最終}{さいしゅう}\ruby{利用}{りよう}\ruby{者}{しゃ}にも\ruby{直接}{ちょくせつ}的な\ruby{影響}{えいきょう}を\ruby{及}{およ}ぼした。\ruby{アプリケーション}{あぷりけーしょん}の\ruby{数}{かず}は\ruby{少}{すく}なく、\ruby{品質}{ひんしつ}も\ruby{均一}{きんいつ}ではなく、\ruby{入手}{にゅうしゅ}も\ruby{容易}{ようい}ではなかったため、\ruby{携帯}{けいたい}\ruby{端末}{たんまつ}は\ruby{業務}{ぎょうむ}や\ruby{娯楽}{ごらく}のための\ruby{豊富}{ほうふ}な\ruby{プラットフォーム}{ぷらっとふぉーむ}とは\ruby{言}{い}えなかった。このような\ruby{状況}{じょうきょう}の\ruby{中}{なか}で、\ruby{開放的}{かいほうてき}な\ruby{モデル}{もでる}を\ruby{持}{も}ち、\ruby{拡張}{かくちょう}が\ruby{容易}{ようい}で、\ruby{開発}{かいはつ}\ruby{エコシステム}{えこしすてむ}に\ruby{親和的}{しんわてき}な\ruby{新}{あたら}しい\ruby{携帯}{けいたい}\ruby{端末}{たんまつ}\ruby{向}{む}け\ruby{オペレーティング}{おぺれーてぃんぐ}\ruby{システム}{しすてむ}への\ruby{需要}{じゅよう}が、\ruby{次第}{しだい}に\ruby{明確}{めいかく}となっていった。

\section{Khó khăn của lập trình viên khi phát triển ứng dụng di động thời kỳ đầu}
\ruby{初期}{しょき}の\ruby{モバイル}{もばいる}\ruby{アプリケーション}{あぷりけーしょん}\ruby{開発}{かいはつ}における\ruby{プログラマー}{ぷろぐらまー}の\ruby{困難}{こんなん}

Trong giai đoạn đầu của thị trường di động, lập trình viên phải đối mặt với nhiều rào cản kỹ thuật và phi kỹ thuật khi phát triển ứng dụng. Những khó khăn này không chỉ làm chậm tốc độ đổi mới mà còn hạn chế nghiêm trọng sự hình thành của một hệ sinh thái phần mềm phong phú.

\ruby{モバイル}{もばいる}\ruby{市場}{しじょう}の\ruby{初期}{しょき}\ruby{段階}{だんかい}において、\ruby{プログラマー}{ぷろぐらまー}は\ruby{アプリケーション}{あぷりけーしょん}\ruby{開発}{かいはつ}の\ruby{過程}{かてい}で、\ruby{技術的}{ぎじゅつてき}および\ruby{非}{ひ}\ruby{技術的}{ぎじゅつてき}な\ruby{多数}{たすう}の\ruby{障壁}{しょうへき}に\ruby{直面}{ちょくめん}した。これらの\ruby{困難}{こんなん}は、\ruby{革新}{かくしん}の\ruby{速度}{そくど}を\ruby{低下}{ていか}させただけでなく、\ruby{豊富}{ほうふ}な\ruby{ソフトウェア}{そふとうぇあ}\ruby{生態系}{せいたいけい}の\ruby{形成}{けいせい}を\ruby{深刻}{しんこく}に\ruby{制限}{せいげん}した。

Trước hết, công cụ phát triển ứng dụng di động còn rất nghèo nàn. Mỗi hệ điều hành cung cấp một bộ công cụ riêng biệt, thường thiếu tính hoàn thiện và không được cập nhật thường xuyên. Tài liệu kỹ thuật ít, rời rạc và không thống nhất, khiến quá trình học tập và tiếp cận nền tảng trở nên tốn thời gian. Việc gỡ lỗi ứng dụng trên thiết bị thật cũng gặp nhiều khó khăn do thiếu công cụ hỗ trợ hiệu quả.

\ruby{まず}{まず}、\ruby{モバイル}{もばいる}\ruby{アプリケーション}{あぷりけーしょん}\ruby{開発}{かいはつ}\ruby{ツール}{つーる}は\ruby{極}{きわ}めて\ruby{貧弱}{ひんじゃく}であった。各\ruby{オペレーティング}{おぺれーてぃんぐ}\ruby{システム}{しすてむ}は\ruby{独自}{どくじ}の\ruby{ツール}{つーる}\ruby{セット}{せっと}を\ruby{提供}{ていきょう}していたが、\ruby{完成度}{かんせいど}に\ruby{欠}{か}け、\ruby{頻繁}{ひんぱん}に\ruby{更新}{こうしん}されることはなかった。\ruby{技術}{ぎじゅつ}\ruby{文書}{ぶんしょ}は\ruby{少}{すく}なく、\ruby{断片的}{だんぺんてき}で\ruby{統一性}{とういつせい}を\ruby{欠}{か}いており、\ruby{学習}{がくしゅう}および\ruby{基盤}{きばん}への\ruby{理解}{りかい}に\ruby{多}{おお}くの\ruby{時間}{じかん}を\ruby{要}{よう}した。\ruby{実機}{じっき}での\ruby{デバッグ}{でばっぐ}も、\ruby{有効}{ゆうこう}な\ruby{支援}{しえん}\ruby{ツール}{つーる}の\ruby{不足}{ふそく}により、\ruby{困難}{こんなん}を\ruby{伴}{ともな}った。

Một vấn đề lớn khác là sự thiếu thống nhất của API. Các hệ điều hành di động khác nhau sử dụng các ngôn ngữ lập trình, thư viện và mô hình phát triển khác nhau. Ngay cả trong cùng một hệ điều hành, các nhà sản xuất cũng có thể tùy biến sâu, dẫn đến việc API hoạt động khác nhau trên từng thiết bị. Điều này buộc lập trình viên phải viết mã đặc thù cho từng dòng máy, làm giảm khả năng tái sử dụng và gia tăng độ phức tạp của dự án.

もう一つの\ruby{大}{おお}きな\ruby{問題}{もんだい}は、\ruby{API}{えーぴーあい}の\ruby{非}{ひ}\ruby{統一}{とういつ}であった。異なる\ruby{モバイル}{もばいる}\ruby{オペレーティング}{おぺれーてぃんぐ}\ruby{システム}{しすてむ}は、それぞれ\ruby{異}{こと}なる\ruby{プログラミング}{ぷろぐらみんぐ}\ruby{言語}{げんご}、\ruby{ライブラリ}{らいぶらり}、および\ruby{開発}{かいはつ}\ruby{モデル}{もでる}を\ruby{採用}{さいよう}していた。同一の\ruby{オペレーティング}{おぺれーてぃんぐ}\ruby{システム}{しすてむ}であっても、\ruby{メーカー}{めーかー}ごとの\ruby{深}{ふか}い\ruby{カスタマイズ}{かすたまいず}により、\ruby{API}{えーぴーあい}の\ruby{挙動}{きょどう}が\ruby{端末}{たんまつ}ごとに\ruby{異}{こと}なることがあった。このため、\ruby{プログラマー}{ぷろぐらまー}は\ruby{機種}{きしゅ}ごとに\ruby{専用}{せんよう}の\ruby{コード}{こーど}を\ruby{記述}{きじゅつ}する\ruby{必要}{ひつよう}があり、\ruby{再利用性}{さいりようせい}の\ruby{低下}{ていか}と\ruby{プロジェクト}{ぷろじぇくと}の\ruby{複雑性}{ふくざつせい}の\ruby{増大}{ぞうだい}を\ruby{招}{まね}いた。

Chi phí phát triển ứng dụng di động thời kỳ này cũng rất cao. Nhiều nền tảng yêu cầu lập trình viên phải mua bộ công cụ, giấy phép phát triển hoặc tham gia các chương trình đối tác với chi phí đáng kể. Bên cạnh đó, để kiểm thử ứng dụng trên nhiều thiết bị khác nhau, lập trình viên hoặc doanh nghiệp phải đầu tư phần cứng với giá thành cao, trong khi số lượng thiết bị trên thị trường ngày càng đa dạng.

この\ruby{時期}{じき}における\ruby{モバイル}{もばいる}\ruby{アプリケーション}{あぷりけーしょん}\ruby{開発}{かいはつ}\ruby{コスト}{こすと}も\ruby{非常}{ひじょう}に\ruby{高}{たか}かった。多くの\ruby{プラットフォーム}{ぷらっとふぉーむ}では、\ruby{プログラマー}{ぷろぐらまー}に\ruby{開発}{かいはつ}\ruby{ツール}{つーる}や\ruby{ライセンス}{らいせんす}の\ruby{購入}{こうにゅう}、あるいは\ruby{提携}{ていけい}\ruby{プログラム}{ぷろぐらむ}への\ruby{参加}{さんか}を\ruby{要求}{ようきゅう}し、\ruby{相当}{そうとう}な\ruby{費用}{ひよう}が\ruby{発生}{はっせい}した。さらに、\ruby{多様}{たよう}な\ruby{端末}{たんまつ}での\ruby{テスト}{てすと}を\ruby{行}{おこな}うため、\ruby{プログラマー}{ぷろぐらまー}や\ruby{企業}{きぎょう}は\ruby{高価}{こうか}な\ruby{ハードウェア}{はーどうぇあ}への\ruby{投資}{とうし}を\ruby{余儀}{よぎ}なくされた。

Quy trình phân phối ứng dụng là một rào cản đáng kể khác. Trước khi khái niệm cửa hàng ứng dụng tập trung trở nên phổ biến, việc đưa phần mềm đến tay người dùng thường thông qua các kênh phân phối phức tạp và thiếu minh bạch. Một số nền tảng yêu cầu ứng dụng phải trải qua quá trình kiểm duyệt khắt khe, kéo dài và thiếu tiêu chí rõ ràng. Điều này làm giảm tốc độ phát hành và gây khó khăn cho các lập trình viên độc lập hoặc nhóm nhỏ.

\ruby{アプリケーション}{あぷりけーしょん}の\ruby{配布}{はいふ}\ruby{プロセス}{ぷろせす}も、\ruby{重要}{じゅうよう}な\ruby{障壁}{しょうへき}であった。\ruby{集中}{しゅうちゅう}\ruby{型}{がた}の\ruby{アプリ}{あぷり}\ruby{ストア}{すとあ}という\ruby{概念}{がいねん}が\ruby{普及}{ふきゅう}する\ruby{以前}{いぜん}、\ruby{ソフトウェア}{そふとうぇあ}を\ruby{利用者}{りようしゃ}に\ruby{届ける}{とどける}には、\ruby{複雑}{ふくざつ}で\ruby{透明性}{とうめいせい}に\ruby{欠}{か}ける\ruby{流通}{りゅうつう}\ruby{経路}{けいろ}を\ruby{経}{へ}る\ruby{必要}{ひつよう}があった。一部の\ruby{プラットフォーム}{ぷらっとふぉーむ}では、\ruby{厳格}{げんかく}で\ruby{長期}{ちょうき}にわたる\ruby{審査}{しんさ}を\ruby{要求}{ようきゅう}し、\ruby{明確}{めいかく}な\ruby{基準}{きじゅん}が\ruby{示}{しめ}されないこともあった。これにより、\ruby{リリース}{りりーす}\ruby{速度}{そくど}が\ruby{低下}{ていか}し、\ruby{独立}{どくりつ}した\ruby{プログラマー}{ぷろぐらまー}や\ruby{小規模}{しょうきぼ}な\ruby{チーム}{ちーむ}にとって\ruby{大}{おお}きな\ruby{負担}{ふたん}となった。

Ngoài ra, thị trường ứng dụng di động còn rất hạn chế về quy mô. Số lượng người dùng smartphone chưa nhiều, khả năng thanh toán trực tuyến còn thấp và mô hình kinh doanh ứng dụng chưa rõ ràng. Lập trình viên khó thu hồi chi phí đầu tư, dẫn đến việc thiếu động lực phát triển các ứng dụng chất lượng cao và sáng tạo.

さらに、\ruby{モバイル}{もばいる}\ruby{アプリケーション}{あぷりけーしょん}\ruby{市場}{しじょう}は\ruby{規模}{きぼ}の\ruby{面}{めん}でも\ruby{限定的}{げんていてき}であった。\ruby{スマートフォン}{すまーとふぉん}\ruby{利用者}{りようしゃ}の\ruby{数}{かず}は\ruby{少}{すく}なく、\ruby{オンライン}{おんらいん}\ruby{決済}{けっさい}\ruby{能力}{のうりょく}も\ruby{低}{ひく}く、\ruby{アプリケーション}{あぷりけーしょん}の\ruby{ビジネス}{びじねす}\ruby{モデル}{もでる}は\ruby{未成熟}{みせいじゅく}であった。\ruby{プログラマー}{ぷろぐらまー}は\ruby{投資}{とうし}\ruby{コスト}{こすと}を\ruby{回収}{かいしゅう}しにくく、\ruby{高品質}{こうひんしつ}で\ruby{創造的}{そうぞうてき}な\ruby{アプリケーション}{あぷりけーしょん}\ruby{開発}{かいはつ}への\ruby{動機}{どうき}が\ruby{不足}{ふそく}した。

Tổng hợp các yếu tố trên cho thấy, lập trình viên trong giai đoạn đầu của kỷ nguyên di động phải làm việc trong một môi trường thiếu tiêu chuẩn, chi phí cao và nhiều rủi ro. Những khó khăn này đã kìm hãm sự phát triển của phần mềm di động và góp phần tạo ra nhu cầu cấp thiết về một nền tảng mới, cung cấp công cụ tốt hơn, API thống nhất và một mô hình phát triển thân thiện với cộng đồng.

\ruby{以上}{いじょう}の\ruby{要因}{よういん}を\ruby{総合}{そうごう}すると、\ruby{モバイル}{もばいる}\ruby{時代}{じだい}の\ruby{初期}{しょき}における\ruby{プログラマー}{ぷろぐらまー}は、\ruby{標準}{ひょうじゅん}の\ruby{欠如}{けつじょ}、\ruby{高}{たか}い\ruby{コスト}{こすと}、および\ruby{多数}{たすう}の\ruby{リスク}{りすく}を\ruby{伴}{ともな}う\ruby{環境}{かんきょう}で\ruby{作業}{さぎょう}していたことが\ruby{分}{わ}かる。これらの\ruby{困難}{こんなん}は\ruby{モバイル}{もばいる}\ruby{ソフトウェア}{そふとうぇあ}の\ruby{発展}{はってん}を\ruby{抑制}{よくせい}し、\ruby{より}{より}\ruby{優}{すぐ}れた\ruby{ツール}{つーる}、\ruby{統一}{とういつ}された\ruby{API}{えーぴーあい}、および\ruby{コミュニティ}{こみゅにてぃ}に\ruby{親和的}{しんわてき}な\ruby{開発}{かいはつ}\ruby{モデル}{もでる}を\ruby{提供}{ていきょう}する\ruby{新}{あたら}しい\ruby{プラットフォーム}{ぷらっとふぉーむ}への\ruby{切実}{せつじつ}な\ruby{需要}{じゅよう}を\ruby{生}{う}み\ruby{出}{だ}した。

\section{Sự trỗi dậy của Internet di động và nhu cầu tích hợp dịch vụ trực tuyến}
\ruby{モバイル}{もばいる}\ruby{インターネット}{いんたーねっと}の\ruby{台頭}{たいとう}と\ruby{オンライン}{おんらいん}\ruby{サービス}{さーびす}\ruby{統合}{とうごう}の\ruby{必要性}{ひつようせい}

Song song với sự phát triển của thiết bị di động, Internet cũng bước vào giai đoạn mở rộng mạnh mẽ. Từ đầu những năm 2000, các công nghệ mạng di động như GPRS, EDGE và sau đó là 3G bắt đầu được triển khai rộng rãi, cho phép thiết bị di động kết nối Internet liên tục thay vì chỉ phục vụ liên lạc thoại. Đây là một bước ngoặt quan trọng, làm thay đổi kỳ vọng của người dùng đối với điện thoại di động.

\ruby{モバイル}{もばいる}\ruby{端末}{たんまつ}の\ruby{発展}{はってん}と\ruby{並行}{へいこう}して、\ruby{インターネット}{いんたーねっと}も\ruby{急速}{きゅうそく}な\ruby{拡大}{かくだい}の\ruby{段階}{だんかい}に\ruby{入}{はい}った。2000\ruby{年代}{ねんだい}\ruby{初頭}{しょとう}から、GPRSやEDGE、さらに3Gといった\ruby{モバイル}{もばいる}\ruby{通信}{つうしん}\ruby{技術}{ぎじゅつ}が\ruby{広範}{こうはん}に\ruby{導入}{どうにゅう}され、\ruby{音声}{おんせい}\ruby{通話}{つうわ}のみを\ruby{目的}{もくてき}とするのではなく、\ruby{モバイル}{もばいる}\ruby{端末}{たんまつ}が\ruby{継続的}{けいぞくてき}に\ruby{インターネット}{いんたーねっと}へ\ruby{接続}{せつぞく}できるようになった。これは\ruby{利用者}{りようしゃ}の\ruby{携帯}{けいたい}\ruby{電話}{でんわ}に\ruby{対}{たい}する\ruby{期待}{きたい}を\ruby{変化}{へんか}させた\ruby{重要}{じゅうよう}な\ruby{転換点}{てんかんてん}であった。

Người dùng không còn xem điện thoại chỉ là công cụ gọi điện mà bắt đầu mong đợi khả năng truy cập thông tin mọi lúc, mọi nơi. Các nhu cầu như kiểm tra email, đọc tin tức, tìm kiếm thông tin, sử dụng bản đồ và đồng bộ dữ liệu cá nhân dần trở nên phổ biến. Thiết bị di động được kỳ vọng sẽ trở thành một phần mở rộng của Internet, gắn liền với các dịch vụ trực tuyến và dữ liệu thời gian thực.

\ruby{利用者}{りようしゃ}は、\ruby{携帯}{けいたい}\ruby{電話}{でんわ}を\ruby{単}{たん}なる\ruby{通話}{つうわ}\ruby{手段}{しゅだん}として\ruby{捉}{とら}えるのではなく、\ruby{いつでも}{いつでも}\ruby{どこでも}{どこでも}\ruby{情報}{じょうほう}に\ruby{アクセス}{あくせす}できる\ruby{能力}{のうりょく}を\ruby{期待}{きたい}するようになった。メールの\ruby{確認}{かくにん}、\ruby{ニュース}{にゅーす}の\ruby{閲覧}{えつらん}、\ruby{情報}{じょうほう}\ruby{検索}{けんさく}、\ruby{地図}{ちず}の\ruby{利用}{りよう}、\ruby{個人}{こじん}\ruby{データ}{でーた}の\ruby{同期}{どうき}といった\ruby{需要}{じゅよう}が\ruby{次第}{しだい}に\ruby{一般化}{いっぱんか}した。\ruby{モバイル}{もばいる}\ruby{端末}{たんまつ}は、\ruby{インターネット}{いんたーねっと}の\ruby{拡張}{かくちょう}\ruby{的一部}{てきいちぶ}として、\ruby{オンライン}{おんらいん}\ruby{サービス}{さーびす}や\ruby{リアルタイム}{りあるたいむ}\ruby{データ}{でーた}と\ruby{密接}{みっせつ}に\ruby{結}{むす}びつくことが\ruby{期待}{きたい}されるようになった。

Tuy nhiên, các hệ điều hành di động trước Android không được thiết kế với Internet làm trung tâm. Khả năng hỗ trợ trình duyệt web còn hạn chế, tốc độ chậm và trải nghiệm người dùng kém. Việc tích hợp các dịch vụ web vào ứng dụng gặp nhiều khó khăn do thiếu API chuẩn hóa cho mạng, bảo mật và xử lý dữ liệu. Các ứng dụng thường hoạt động độc lập, ít khả năng trao đổi dữ liệu với nhau hoặc với các dịch vụ bên ngoài.

しかし、Android\ruby{以前}{いぜん}の\ruby{モバイル}{もばいる}\ruby{オペレーティングシステム}{おぺれーてぃんぐしすてむ}は、\ruby{インターネット}{いんたーねっと}を\ruby{中心}{ちゅうしん}に\ruby{据}{す}えて\ruby{設計}{せっけい}されていなかった。\ruby{ウェブ}{うぇぶ}\ruby{ブラウザ}{ぶらうざ}の\ruby{対応}{たいおう}\ruby{能力}{のうりょく}は\ruby{限定的}{げんていてき}で、\ruby{速度}{そくど}も\ruby{遅}{おそ}く、\ruby{ユーザー}{ゆーざー}\ruby{体験}{たいけん}は\ruby{不十分}{ふじゅうぶん}であった。\ruby{ネットワーク}{ねっとわーく}、\ruby{セキュリティ}{せきゅりてぃ}、\ruby{データ}{でーた}\ruby{処理}{しょり}に\ruby{関}{かん}する\ruby{標準化}{ひょうじゅんか}されたAPIが\ruby{不足}{ふそく}していたため、\ruby{ウェブ}{うぇぶ}\ruby{サービス}{さーびす}を\ruby{アプリケーション}{あぷりけーしょん}に\ruby{統合}{とうごう}することは\ruby{困難}{こんなん}であった。\ruby{アプリケーション}{あぷりけーしょん}は\ruby{多}{おお}くの\ruby{場合}{ばあい}、\ruby{独立}{どくりつ}して\ruby{動作}{どうさ}し、\ruby{相互}{そうご}に、あるいは\ruby{外部}{がいぶ}\ruby{サービス}{さーびす}と\ruby{データ}{でーた}を\ruby{交換}{こうかん}する\ruby{能力}{のうりょく}が\ruby{乏}{とぼ}しかった。

Một vấn đề quan trọng khác là mô hình dữ liệu. Khi Internet di động phát triển, nhu cầu đồng bộ dữ liệu giữa nhiều thiết bị và nền tảng trở nên cấp thiết. Người dùng muốn danh bạ, lịch, email và dữ liệu cá nhân được cập nhật nhất quán. Tuy nhiên, các hệ điều hành cũ thường xử lý dữ liệu theo cách cục bộ, thiếu cơ chế đồng bộ linh hoạt và mở rộng. Điều này làm giảm đáng kể giá trị sử dụng của thiết bị di động trong bối cảnh Internet ngày càng đóng vai trò trung tâm.

もう一つの\ruby{重要}{じゅうよう}な\ruby{課題}{かだい}は\ruby{データ}{でーた}\ruby{モデル}{もでる}である。\ruby{モバイル}{もばいる}\ruby{インターネット}{いんたーねっと}の\ruby{発展}{はってん}に\ruby{伴}{ともな}い、\ruby{複数}{ふくすう}の\ruby{端末}{たんまつ}や\ruby{プラットフォーム}{ぷらっとふぉーむ}間で\ruby{データ}{でーた}を\ruby{同期}{どうき}する\ruby{需要}{じゅよう}が\ruby{急速}{きゅうそく}に\ruby{高}{たか}まった。\ruby{利用者}{りようしゃ}は、\ruby{連絡先}{れんらくさき}、\ruby{カレンダー}{かれんだー}、\ruby{メール}{めーる}、\ruby{個人}{こじん}\ruby{データ}{でーた}が\ruby{一貫}{いっかん}して\ruby{更新}{こうしん}されることを\ruby{望}{のぞ}んだ。しかし、\ruby{従来}{じゅうらい}の\ruby{オペレーティングシステム}{おぺれーてぃんぐしすてむ}は、\ruby{データ}{でーた}を\ruby{ローカル}{ろーかる}に\ruby{処理}{しょり}する\ruby{傾向}{けいこう}が\ruby{強}{つよ}く、\ruby{柔軟}{じゅうなん}かつ\ruby{拡張性}{かくちょうせい}のある\ruby{同期}{どうき}\ruby{機構}{きこう}を\ruby{欠}{か}いていた。その\ruby{結果}{けっか}、\ruby{インターネット}{いんたーねっと}が\ruby{中心的}{ちゅうしんてき}な\ruby{役割}{やくわり}を\ruby{果}{は}たす\ruby{状況}{じょうきょう}において、\ruby{モバイル}{もばいる}\ruby{端末}{たんまつ}の\ruby{利用価値}{りようかち}は\ruby{大幅}{おおはば}に\ruby{低下}{ていか}した。

Đối với lập trình viên, sự trỗi dậy của Internet di động vừa là cơ hội vừa là thách thức. Nhu cầu xây dựng các ứng dụng dựa trên dịch vụ trực tuyến tăng nhanh, nhưng nền tảng kỹ thuật lại không đáp ứng được yêu cầu này. Việc xử lý kết nối mạng không ổn định, bảo mật dữ liệu và hiệu năng ứng dụng trở nên phức tạp trong khi công cụ hỗ trợ còn hạn chế. Các nền tảng hiện có không cung cấp một kiến trúc rõ ràng để xây dựng các ứng dụng hướng dịch vụ và dữ liệu.

\ruby{開発者}{かいはつしゃ}にとって、\ruby{モバイル}{もばいる}\ruby{インターネット}{いんたーねっと}の\ruby{台頭}{たいとう}は\ruby{機会}{きかい}であると\ruby{同時}{どうじ}に\ruby{課題}{かだい}でもあった。\ruby{オンライン}{おんらいん}\ruby{サービス}{さーびす}に\ruby{基}{もと}づく\ruby{アプリケーション}{あぷりけーしょん}を\ruby{構築}{こうちく}する\ruby{需要}{じゅよう}は\ruby{急増}{きゅうぞう}したが、\ruby{技術的}{ぎじゅつてき}\ruby{基盤}{きばん}はその\ruby{要件}{ようけん}を\ruby{十分}{じゅうぶん}に\ruby{満}{み}たしていなかった。\ruby{不安定}{ふあんてい}な\ruby{ネットワーク}{ねっとわーく}\ruby{接続}{せつぞく}、\ruby{データ}{でーた}\ruby{保護}{ほご}、\ruby{アプリケーション}{あぷりけーしょん}\ruby{性能}{せいのう}の\ruby{処理}{しょり}は\ruby{複雑}{ふくざつ}である一方、\ruby{支援}{しえん}\ruby{ツール}{つーる}は\ruby{限定的}{げんていてき}であった。\ruby{既存}{きそん}の\ruby{プラットフォーム}{ぷらっとふぉーむ}は、\ruby{サービス}{さーびす}および\ruby{データ}{でーた}\ruby{指向}{しこう}の\ruby{アプリケーション}{あぷりけーしょん}を\ruby{構築}{こうちく}するための\ruby{明確}{めいかく}な\ruby{アーキテクチャ}{あーきてくちゃ}を\ruby{提供}{ていきょう}していなかった。

Thực tế này cho thấy một khoảng cách lớn giữa nhu cầu sử dụng Internet di động ngày càng tăng và khả năng đáp ứng của các hệ điều hành di động truyền thống. Thị trường cần một nền tảng mới, được thiết kế ngay từ đầu để tích hợp chặt chẽ Internet, dịch vụ trực tuyến và dữ liệu, đồng thời cung cấp cho lập trình viên các công cụ và API phù hợp để khai thác tối đa tiềm năng của kết nối di động.

これらの\ruby{現実}{げんじつ}は、\ruby{増大}{ぞうだい}し\ruby{続}{つづ}ける\ruby{モバイル}{もばいる}\ruby{インターネット}{いんたーねっと}の\ruby{利用}{りよう}\ruby{需要}{じゅよう}と、\ruby{従来}{じゅうらい}の\ruby{モバイル}{もばいる}\ruby{オペレーティングシステム}{おぺれーてぃんぐしすてむ}の\ruby{対応能力}{たいおうのうりょく}との\ruby{間}{あいだ}に\ruby{大}{おお}きな\ruby{乖離}{かいり}が\ruby{存在}{そんざい}することを\ruby{示}{しめ}している。\ruby{市場}{しじょう}は、\ruby{インターネット}{いんたーねっと}、\ruby{オンライン}{おんらいん}\ruby{サービス}{さーびす}、\ruby{データ}{でーた}を\ruby{緊密}{きんみつ}に\ruby{統合}{とうごう}することを\ruby{前提}{ぜんてい}として\ruby{設計}{せっけい}された\ruby{新}{あたら}しい\ruby{プラットフォーム}{ぷらっとふぉーむ}を\ruby{必要}{ひつよう}としており、\ruby{同時}{どうじ}に\ruby{開発者}{かいはつしゃ}が\ruby{モバイル}{もばいる}\ruby{接続}{せつぞく}の\ruby{潜在力}{せんざいりょく}を\ruby{最大限}{さいだいげん}に\ruby{活用}{かつよう}できる\ruby{適切}{てきせつ}な\ruby{ツール}{つーる}とAPIの\ruby{提供}{ていきょう}が\ruby{求}{もと}められている。

\section{Nhu cầu tất yếu về một nền tảng di động mã nguồn mở}
\ruby{必然的}{ひつぜんてき}な\ruby{要請}{ようせい}としての\ruby{オープンソース}{おーぷんそーす}\ruby{型}{がた}\ruby{移動}{いどう}\ruby{基盤}{きばん}

Từ những phân tích về phần cứng, hệ điều hành và môi trường phát triển ứng dụng di động đầu những năm 2000, có thể thấy rõ thị trường đang tồn tại những giới hạn mang tính cấu trúc. Các nền tảng di động khi đó không còn đáp ứng được tốc độ phát triển của công nghệ, cũng như nhu cầu ngày càng cao của người dùng và lập trình viên. Trong bối cảnh đó, nhu cầu về một nền tảng di động mới không chỉ xuất hiện ngẫu nhiên, mà mang tính tất yếu.

2000\ruby{年代}{ねんだい}\ruby{初頭}{しょとう}における\ruby{ハードウェア}{はーどうぇあ}、\ruby{オペレーティングシステム}{おぺれーてぃんぐしすてむ}、および\ruby{モバイル}{もばいる}\ruby{アプリケーション}{あぷりけーしょん}\ruby{開発}{かいはつ}\ruby{環境}{かんきょう}に\ruby{関}{かん}する\ruby{分析}{ぶんせき}から、\ruby{市場}{しじょう}には\ruby{構造的}{こうぞうてき}な\ruby{制約}{せいやく}が\ruby{存在}{そんざい}していたことが\ruby{明}{あき}らかである。当時の\ruby{移動}{いどう}\ruby{基盤}{きばん}は、\ruby{技術}{ぎじゅつ}の\ruby{発展}{はってん}\ruby{速度}{そくど}や、\ruby{利用者}{りようしゃ}および\ruby{開発者}{かいはつしゃ}の\ruby{需要}{じゅよう}の\ruby{高度化}{こうどか}に\ruby{対応}{たいおう}できなくなっていた。このような\ruby{状況}{じょうきょう}において、\ruby{新}{あたら}しい\ruby{移動}{いどう}\ruby{基盤}{きばん}への\ruby{需要}{じゅよう}は\ruby{偶発的}{ぐうはつてき}に\ruby{生}{しょう}じたものではなく、\ruby{必然的}{ひつぜんてき}なものであった。

Trước hết, mô hình phát triển đóng của các hệ điều hành truyền thống đã bộc lộ nhiều bất cập. Việc mã nguồn bị kiểm soát chặt chẽ bởi một số ít nhà cung cấp khiến quá trình đổi mới diễn ra chậm và thiếu linh hoạt. Các nhà sản xuất thiết bị, lập trình viên và doanh nghiệp phụ thuộc nặng nề vào quyết định của chủ sở hữu nền tảng. Điều này đi ngược lại xu hướng phát triển phần mềm hiện đại, nơi sự hợp tác rộng rãi và cải tiến liên tục đóng vai trò then chốt.

まず、\ruby{従来}{じゅうらい}の\ruby{オペレーティングシステム}{おぺれーてぃんぐしすてむ}における\ruby{閉鎖的}{へいさてき}な\ruby{開発}{かいはつ}\ruby{モデル}{もでる}は、\ruby{多}{おお}くの\ruby{問題点}{もんだいてん}を\ruby{露呈}{ろてい}した。\ruby{ソースコード}{そーすこーど}が\ruby{少数}{しょうすう}の\ruby{提供者}{ていきょうしゃ}によって\ruby{厳格}{げんかく}に\ruby{管理}{かんり}されることで、\ruby{革新}{かくしん}の\ruby{過程}{かてい}は\ruby{遅延}{ちえん}し、\ruby{柔軟性}{じゅうなんせい}を\ruby{欠}{か}いた。\ruby{機器}{きき}\ruby{製造者}{せいぞうしゃ}、\ruby{開発者}{かいはつしゃ}、および\ruby{企業}{きぎょう}は、\ruby{基盤}{きばん}\ruby{所有者}{しょゆうしゃ}の\ruby{決定}{けってい}に\ruby{大}{おお}きく\ruby{依存}{いぞん}していた。これは、\ruby{広範}{こうはん}な\ruby{協働}{きょうどう}と\ruby{継続的}{けいぞくてき}な\ruby{改善}{かいぜん}を\ruby{中核}{ちゅうかく}とする\ruby{現代的}{げんだいてき}な\ruby{ソフトウェア}{そふとうぇあ}\ruby{開発}{かいはつ}の\ruby{潮流}{ちょうりゅう}に\ruby{反}{はん}するものである。

Một nền tảng di động mã nguồn mở được kỳ vọng sẽ giải quyết vấn đề này bằng cách cho phép nhiều bên cùng tham gia phát triển và cải tiến hệ thống. Việc công khai mã nguồn giúp giảm rào cản gia nhập thị trường, cho phép nhà sản xuất tùy biến hệ điều hành phù hợp với phần cứng của mình, đồng thời tạo điều kiện để cộng đồng phát hiện và khắc phục lỗi nhanh chóng. Mô hình này cũng thúc đẩy tính minh bạch và giảm sự phụ thuộc vào một nhà cung cấp duy nhất.

\ruby{オープンソース}{おーぷんそーす}\ruby{型}{がた}の\ruby{移動}{いどう}\ruby{基盤}{きばん}は、\ruby{複数}{ふくすう}の\ruby{主体}{しゅたい}が\ruby{開発}{かいはつ}および\ruby{改良}{かいりょう}に\ruby{参加}{さんか}することを\ruby{可能}{かのう}にすることで、この\ruby{問題}{もんだい}を\ruby{解決}{かいけつ}すると\ruby{期待}{きたい}される。\ruby{ソースコード}{そーすこーど}の\ruby{公開}{こうかい}は、\ruby{市場}{しじょう}\ruby{参入}{さんにゅう}の\ruby{障壁}{しょうへき}を\ruby{低減}{ていげん}し、\ruby{製造者}{せいぞうしゃ}が\ruby{自社}{じしゃ}の\ruby{ハードウェア}{はーどうぇあ}に\ruby{適合}{てきごう}した\ruby{オペレーティングシステム}{おぺれーてぃんぐしすてむ}を\ruby{カスタマイズ}{かすたまいず}することを\ruby{可能}{かのう}にする。\ruby{同時}{どうじ}に、\ruby{コミュニティ}{こみゅにてぃ}が\ruby{迅速}{じんそく}に\ruby{欠陥}{けっかん}を\ruby{発見}{はっけん}し、\ruby{修正}{しゅうせい}するための\ruby{環境}{かんきょう}を\ruby{提供}{ていきょう}する。この\ruby{モデル}{もでる}は、\ruby{透明性}{とうめいせい}を\ruby{促進}{そくしん}し、\ruby{単一}{たんいつ}の\ruby{提供者}{ていきょうしゃ}への\ruby{依存}{いぞん}を\ruby{軽減}{けいげん}する。

Bên cạnh tính mở, sự linh hoạt và khả năng mở rộng là yêu cầu quan trọng khác. Nền tảng di động mới cần thích ứng nhanh với sự tiến hóa của phần cứng, từ bộ xử lý mạnh hơn, màn hình lớn hơn cho đến các cảm biến và kết nối mới. Đồng thời, hệ điều hành phải được thiết kế để tích hợp sâu Internet, hỗ trợ ứng dụng dựa trên dịch vụ trực tuyến và xử lý dữ liệu hiệu quả trong môi trường kết nối liên tục.

\ruby{開放性}{かいほうせい}に\ruby{加}{くわ}えて、\ruby{柔軟性}{じゅうなんせい}および\ruby{拡張性}{かくちょうせい}も\ruby{重要}{じゅうよう}な\ruby{要件}{ようけん}である。\ruby{新}{あたら}しい\ruby{移動}{いどう}\ruby{基盤}{きばん}は、\ruby{高性能}{こうせいのう}な\ruby{プロセッサ}{ぷろせっさ}、\ruby{大型}{おおがた}の\ruby{表示装置}{ひょうじそうち}、さらには\ruby{各種}{かくしゅ}の\ruby{センサー}{せんさー}や\ruby{通信}{つうしん}\ruby{技術}{ぎじゅつ}など、\ruby{ハードウェア}{はーどうぇあ}の\ruby{進化}{しんか}に\ruby{迅速}{じんそく}に\ruby{適応}{てきおう}する\ruby{必要}{ひつよう}がある。\ruby{同時}{どうじ}に、\ruby{オペレーティングシステム}{おぺれーてぃんぐしすてむ}は\ruby{インターネット}{いんたーねっと}との\ruby{深}{ふか}い\ruby{統合}{とうごう}を\ruby{前提}{ぜんてい}として\ruby{設計}{せっけい}され、\ruby{オンライン}{おんらいん}\ruby{サービス}{さーびす}に\ruby{基}{もと}づく\ruby{アプリケーション}{あぷりけーしょん}を\ruby{支援}{しえん}し、\ruby{常時}{じょうじ}\ruby{接続}{せつぞく}\ruby{環境}{かんきょう}において\ruby{効率的}{こうりつてき}な\ruby{データ}{でーた}\ruby{処理}{しょり}を\ruby{実現}{じつげん}しなければならない。

Đối với lập trình viên, nhu cầu về một nền tảng thống nhất, dễ tiếp cận và thân thiện là yếu tố quyết định. Một bộ API rõ ràng, nhất quán, cùng công cụ phát triển mạnh mẽ sẽ giúp giảm chi phí phát triển và tăng khả năng tái sử dụng mã nguồn. Khi rào cản kỹ thuật và chi phí được hạ thấp, nhiều lập trình viên và nhóm nhỏ có thể tham gia, từ đó thúc đẩy sự đa dạng và phong phú của ứng dụng.

\ruby{開発者}{かいはつしゃ}にとって、\ruby{統一}{とういつ}され、\ruby{アクセス}{あくせす}しやすく、\ruby{使}{つか}いやすい\ruby{基盤}{きばん}への\ruby{需要}{じゅよう}は\ruby{決定的}{けっていてき}な\ruby{要因}{よういん}である。\ruby{明確}{めいかく}で\ruby{一貫}{いっかん}した\ruby{API}{えーぴーあい}と、\ruby{強力}{きょうりょく}な\ruby{開発}{かいはつ}\ruby{ツール}{つーる}は、\ruby{開発}{かいはつ}\ruby{コスト}{こすと}を\ruby{削減}{さくげん}し、\ruby{ソースコード}{そーすこーど}の\ruby{再利用性}{さいりようせい}を\ruby{高}{たか}める。\ruby{技術的}{ぎじゅつてき}および\ruby{経済的}{けいざいてき}な\ruby{障壁}{しょうへき}が\ruby{低下}{ていか}することで、\ruby{多}{おお}くの\ruby{開発者}{かいはつしゃ}や\ruby{小規模}{しょうきぼ}な\ruby{チーム}{ちーむ}が\ruby{参入}{さんにゅう}でき、\ruby{結果}{けっか}として\ruby{アプリケーション}{あぷりけーしょん}の\ruby{多様性}{たようせい}と\ruby{豊富}{ほうふ}さが\ruby{促進}{そくしん}される。

Cuối cùng, để thiết bị di động thực sự trở thành một nền tảng phổ biến, cần hình thành một hệ sinh thái lớn và bền vững. Hệ sinh thái này bao gồm nhà sản xuất thiết bị, nhà phát triển phần mềm, nhà cung cấp dịch vụ và người dùng cuối. Một nền tảng mở, linh hoạt và dễ mở rộng là điều kiện cần để kết nối các thành phần này, tạo ra hiệu ứng mạng và thúc đẩy tăng trưởng dài hạn.

\ruby{最終的}{さいしゅうてき}に、\ruby{移動}{いどう}\ruby{機器}{きき}が\ruby{真}{しん}に\ruby{普及}{ふきゅう}した\ruby{基盤}{きばん}となるためには、\ruby{大規模}{だいきぼ}で\ruby{持続可能}{じぞくかのう}な\ruby{生態系}{せいたいけい}の\ruby{形成}{けいせい}が\ruby{不可欠}{ふかけつ}である。この\ruby{生態系}{せいたいけい}には、\ruby{機器}{きき}\ruby{製造者}{せいぞうしゃ}、\ruby{ソフトウェア}{そふとうぇあ}\ruby{開発者}{かいはつしゃ}、\ruby{サービス}{さーびす}\ruby{提供者}{ていきょうしゃ}、および\ruby{最終}{さいしゅう}\ruby{利用者}{りようしゃ}が\ruby{含}{ふく}まれる。\ruby{開放的}{かいほうてき}で\ruby{柔軟}{じゅうなん}かつ\ruby{拡張}{かくちょう}しやすい\ruby{基盤}{きばん}は、これらの\ruby{構成要素}{こうせいようそ}を\ruby{結}{むす}びつけ、\ruby{ネットワーク}{ねっとわーく}\ruby{効果}{こうか}を\ruby{創出}{そうしゅつ}し、\ruby{長期的}{ちょうきてき}な\ruby{成長}{せいちょう}を\ruby{促進}{そくしん}するための\ruby{必要条件}{ひつようじょうけん}である。

Chính từ những nhu cầu mang tính tất yếu đó, Android đã ra đời như một lời đáp cho bài toán của thị trường di động. Android không chỉ là một hệ điều hành mới, mà là sự thay đổi căn bản trong cách tiếp cận việc xây dựng và phát triển nền tảng di động, đặt nền móng cho kỷ nguyên smartphone hiện đại.

このような\ruby{必然的}{ひつぜんてき}な\ruby{需要}{じゅよう}を\ruby{背景}{はいけい}として、\ruby{Android}{あんどろいど}は\ruby{移動}{いどう}\ruby{市場}{しじょう}の\ruby{課題}{かだい}に\ruby{対}{たい}する\ruby{解答}{かいとう}として\ruby{誕生}{たんじょう}した。\ruby{Android}{あんどろいど}は\ruby{新}{あたら}しい\ruby{オペレーティングシステム}{おぺれーてぃんぐしすてむ}にとどまらず、\ruby{移動}{いどう}\ruby{基盤}{きばん}の\ruby{構築}{こうちく}と\ruby{発展}{はってん}に\ruby{関}{かん}する\ruby{根本的}{こんぽんてき}な\ruby{発想}{はっそう}の\ruby{転換}{てんかん}を\ruby{示}{しめ}し、\ruby{現代}{げんだい}の\ruby{スマートフォン}{すまーとふぉん}\ruby{時代}{じだい}の\ruby{礎}{いしずえ}を\ruby{築}{きず}いたのである。
