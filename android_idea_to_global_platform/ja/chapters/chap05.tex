\chapter{Sự trưởng thành của Android và bùng nổ hệ sinh thái}
\ruby{Android}{あんどろいど}の\ruby{成熟}{せいじゅく}と\ruby{エコシステム}{えこしすてむ}の\ruby{拡大}{かくだい}

Giai đoạn Android 3.x–5.x đánh dấu bước ngoặt quan trọng trong lịch sử phát triển của nền tảng Android. Sau thời kỳ đầu tập trung vào việc nhanh chóng chiếm lĩnh thị trường, Android bước vào giai đoạn tái cấu trúc và hoàn thiện nhằm giải quyết các vấn đề cốt lõi về trải nghiệm người dùng, tính nhất quán giao diện và khả năng mở rộng trên nhiều loại thiết bị. Đây là thời kỳ Android chuyển mình từ một hệ điều hành còn mang tính thử nghiệm sang một nền tảng di động trưởng thành, đủ khả năng hỗ trợ hệ sinh thái phần cứng và phần mềm ở quy mô toàn cầu.

Android 3.x~5.xの\ruby{時期}{じき}は、Android\ruby{プラットフォーム}{ぷらっとふぉーむ}の\ruby{発展}{はってん}\ruby{史}{し}における\ruby{重要}{じゅうよう}な\ruby{転換点}{てんかんてん}である。\ruby{初期}{しょき}に\ruby{市場}{しじょう}の\ruby{獲得}{かくとく}を\ruby{優先}{ゆうせん}していた\ruby{段階}{だんかい}を\ruby{経}{へ}て、Androidは\ruby{ユーザー}{ゆーざー}\ruby{体験}{たいけん}、\ruby{インターフェース}{いんたーふぇーす}の\ruby{一貫性}{いっかんせい}、および\ruby{多様}{たよう}な\ruby{端末}{たんまつ}への\ruby{拡張性}{かくちょうせい}といった\ruby{中核}{ちゅうかく}\ruby{課題}{かだい}を\ruby{解決}{かいけつ}するための\ruby{再構築}{さいこうちく}と\ruby{完成}{かんせい}の\ruby{段階}{だんかい}に\ruby{入}{はい}った。この\ruby{期間}{きかん}において、Androidは\ruby{試験的}{しけんてき}な\ruby{OS}{おーえす}から、\ruby{世界}{せかい}\ruby{規模}{きぼ}の\ruby{ハードウェア}{はーどうぇあ}および\ruby{ソフトウェア}{そふとうぇあ}\ruby{エコシステム}{えこしすてむ}を\ruby{支}{ささ}える\ruby{成熟}{せいじゅく}した\ruby{モバイル}{もばいる}\ruby{プラットフォーム}{ぷらっとふぉーむ}へと\ruby{変貌}{へんぼう}した。

\section{Android 3.x và 4.x: tái thiết kế giao diện, chuẩn hóa trải nghiệm trên nhiều kích thước màn hình}
Android 3.x・4.x:\ruby{インターフェース}{いんたーふぇーす}の\ruby{再設計}{さいせっけい}と\ruby{画面}{がめん}\ruby{サイズ}{さいず}\ruby{横断}{おうだん}の\ruby{体験}{たいけん}\ruby{標準化}{ひょうじゅんか}

Android 3.x (Honeycomb) ra đời trong bối cảnh thị trường máy tính bảng bắt đầu phát triển mạnh. Trước đó, Android chủ yếu được thiết kế cho điện thoại thông minh với kích thước màn hình nhỏ, dẫn đến nhiều hạn chế khi mở rộng sang các thiết bị có màn hình lớn. Honeycomb là phiên bản Android đầu tiên được thiết kế chuyên biệt cho máy tính bảng, thể hiện nỗ lực của Google trong việc thích ứng giao diện và mô hình tương tác với không gian hiển thị rộng hơn.

Android 3.x(Honeycomb)は、\ruby{タブレット}{たぶれっと}\ruby{市場}{しじょう}が\ruby{本格的}{ほんかくてき}に\ruby{成長}{せいちょう}し\ruby{始}{はじ}めた\ruby{時期}{じき}に\ruby{登場}{とうじょう}した。それまでのAndroidは\ruby{主}{おも}に\ruby{小型}{こがた}\ruby{画面}{がめん}の\ruby{スマートフォン}{すまーとふぉん}を\ruby{前提}{ぜんてい}として\ruby{設計}{せっけい}されており、\ruby{大型}{おおがた}\ruby{画面}{がめん}\ruby{端末}{たんまつ}への\ruby{拡張}{かくちょう}には\ruby{多}{おお}くの\ruby{制限}{せいやく}があった。Honeycombは\ruby{タブレット}{たぶれっと}に\ruby{特化}{とっか}して\ruby{設計}{せっけい}された\ruby{初}{はじ}めてのAndroid\ruby{バージョン}{ばーじょん}であり、\ruby{広}{ひろ}い\ruby{表示}{ひょうじ}\ruby{空間}{くうかん}に\ruby{適応}{てきおう}するための\ruby{インターフェース}{いんたーふぇーす}および\ruby{操作}{そうさ}\ruby{モデル}{もでる}を\ruby{模索}{もさく}するGoogleの\ruby{試}{こころ}みを\ruby{示}{しめ}していた。

Về mặt giao diện, Android 3.x giới thiệu nhiều khái niệm mới như system bar, action bar và bố cục đa khung (multi-pane). Các thành phần này cho phép người dùng thao tác hiệu quả hơn trên màn hình lớn, đồng thời tạo tiền đề cho các chuẩn giao diện sau này. Tuy nhiên, việc Android 3.x chỉ tồn tại trên máy tính bảng đã khiến hệ sinh thái bị chia tách, gây khó khăn cho lập trình viên khi phải duy trì hai nhánh giao diện khác nhau cho điện thoại và tablet.

\ruby{インターフェース}{いんたーふぇーす}の\ruby{面}{めん}では、Android 3.xはsystem barやaction bar、\ruby{複数}{ふくすう}\ruby{ペイン}{ぺいん}からなる\ruby{レイアウト}{れいあうと}(multi-pane)といった\ruby{新}{あたら}しい\ruby{概念}{がいねん}を\ruby{導入}{どうにゅう}した。これらの\ruby{要素}{ようそ}は\ruby{大型}{おおがた}\ruby{画面}{がめん}における\ruby{操作}{そうさ}\ruby{効率}{こうりつ}を\ruby{高}{たか}め、\ruby{後続}{こうぞく}の\ruby{デザイン}{でざいん}\ruby{標準}{ひょうじゅん}の\ruby{基礎}{きそ}となった。しかし、Android 3.xが\ruby{タブレット}{たぶれっと}に\ruby{限定}{げんてい}されたことで、\ruby{エコシステム}{えこしすてむ}は\ruby{分断}{ぶんだん}され、\ruby{開発}{かいはつ}\ruby{者}{しゃ}は\ruby{スマートフォン}{すまーとふぉん}と\ruby{タブレット}{たぶれっと}の\ruby{両方}{りょうほう}に\ruby{対応}{たいおう}する\ruby{異}{こと}なる\ruby{UI}{ゆーあい}を\ruby{維持}{いじ}する\ruby{必要}{ひつよう}に\ruby{迫}{せま}られた。

Những hạn chế này được giải quyết ở Android 4.0 (Ice Cream Sandwich), phiên bản hợp nhất lại nền tảng Android cho cả điện thoại và máy tính bảng. Đây là bước đi mang tính chiến lược, nhằm chuẩn hóa trải nghiệm người dùng trên toàn bộ dải thiết bị. Android 4.x giới thiệu ngôn ngữ thiết kế Holo, lần đầu tiên mang đến một phong cách giao diện thống nhất, hiện đại và dễ nhận diện cho Android. Các yếu tố như màu sắc, biểu tượng, phông chữ và hành vi tương tác được định nghĩa rõ ràng, giúp giảm sự rời rạc vốn tồn tại ở các phiên bản trước.

これらの\ruby{問題}{もんだい}は、Android 4.0(Ice Cream Sandwich)において\ruby{解消}{かいしょう}された。この\ruby{バージョン}{ばーじょん}は\ruby{スマートフォン}{すまーとふぉん}と\ruby{タブレット}{たぶれっと}を\ruby{単一}{たんいつ}の\ruby{プラットフォーム}{ぷらっとふぉーむ}に\ruby{統合}{とうごう}し、\ruby{全}{すべ}ての\ruby{端末}{たんまつ}に\ruby{共通}{きょうつう}の\ruby{ユーザー}{ゆーざー}\ruby{体験}{たいけん}を\ruby{提供}{ていきょう}することを\ruby{目的}{もくてき}とした\ruby{戦略的}{せんりゃくてき}\ruby{決断}{けつだん}であった。Android 4.xではHoloという\ruby{デザイン}{でざいん}\ruby{言語}{げんご}が\ruby{導入}{どうにゅう}され、\ruby{色彩}{しきさい}、\ruby{アイコン}{あいこん}、\ruby{フォント}{ふぉんと}、および\ruby{操作}{そうさ}\ruby{挙動}{きょどう}が\ruby{明確}{めいかく}に\ruby{定義}{ていぎ}された。これにより、\ruby{従来}{じゅうらい}の\ruby{バージョン}{ばーじょん}に\ruby{見}{み}られた\ruby{不統一}{ふとういつ}さが\ruby{大幅}{おおはば}に\ruby{改善}{かいぜん}された。

Một trong những vấn đề lớn của Android giai đoạn đầu là sự đa dạng về kích thước màn hình và độ phân giải. Android 4.x giải quyết vấn đề này bằng cách cung cấp các cơ chế linh hoạt cho lập trình viên, bao gồm hệ thống tài nguyên theo cấu hình (resource qualifiers), đơn vị đo lường độc lập với mật độ điểm ảnh (dp), và các layout có khả năng co giãn. Nhờ đó, cùng một ứng dụng có thể thích ứng với nhiều loại màn hình khác nhau mà không cần viết lại toàn bộ giao diện.

Android\ruby{初期}{しょき}の\ruby{大}{おお}きな\ruby{課題}{かだい}の\ruby{一}{ひと}つは、\ruby{画面}{がめん}\ruby{サイズ}{さいず}や\ruby{解像度}{かいぞうど}の\ruby{多様性}{たようせい}であった。Android 4.xは、resource qualifiersによる\ruby{構成}{こうせい}\ruby{別}{べつ}\ruby{資源}{しげん}\ruby{管理}{かんり}、dpといった\ruby{密度}{みつど}に\ruby{依存}{いぞん}しない\ruby{単位}{たんい}、および\ruby{柔軟}{じゅうなん}な\ruby{レイアウト}{れいあうと}を\ruby{提供}{ていきょう}することで、この\ruby{問題}{もんだい}に\ruby{対処}{たいしょ}した。その\ruby{結果}{けっか}、\ruby{同一}{どういつ}の\ruby{アプリケーション}{あぷりけーしょん}が\ruby{多様}{たよう}な\ruby{画面}{がめん}\ruby{環境}{かんきょう}に\ruby{適応}{てきおう}でき、\ruby{全面的}{ぜんめんてき}な\ruby{書}{か}き\ruby{直}{なお}しを\ruby{必要}{ひつよう}としなくなった。

Bên cạnh cải tiến về thiết kế, Android 4.x cũng tập trung mạnh vào hiệu năng và độ mượt của hệ thống. Dự án Project Butter được triển khai nhằm cải thiện tốc độ phản hồi giao diện và độ mượt của các hiệu ứng chuyển động. Các tối ưu này giúp Android tiệm cận hơn với trải nghiệm liền mạch mà người dùng kỳ vọng ở một hệ điều hành di động hiện đại.

\ruby{設計}{せっけい}の\ruby{改善}{かいぜん}に\ruby{加}{くわ}え、Android 4.xは\ruby{性能}{せいのう}と\ruby{滑}{なめ}らかさの\ruby{向上}{こうじょう}にも\ruby{注力}{ちゅうりょく}した。Project Butterは\ruby{画面}{がめん}\ruby{応答}{おうとう}\ruby{速度}{そくど}と\ruby{アニメーション}{あにめーしょん}の\ruby{滑}{なめ}らかさを\ruby{改善}{かいぜん}するために\ruby{導入}{どうにゅう}され、Androidの\ruby{体験}{たいけん}は\ruby{現代的}{げんだいてき}な\ruby{モバイル}{もばいる}\ruby{OS}{おーえす}に\ruby{求}{もと}められる\ruby{水準}{すいじゅん}に\ruby{近}{ちか}づいた。

Từ góc độ hệ sinh thái, Android 3.x và 4.x là giai đoạn đặt nền móng cho sự chuẩn hóa trải nghiệm. Dù tình trạng phân mảnh phiên bản vẫn tồn tại, Android đã xây dựng được một bộ nguyên tắc giao diện và kỹ thuật đủ rõ ràng để lập trình viên và nhà sản xuất thiết bị cùng tuân theo. Điều này tạo tiền đề cho các bước tiến lớn hơn ở các phiên bản sau, đặc biệt là về ngôn ngữ thiết kế và kiến trúc hệ thống.

\ruby{エコシステム}{えこしすてむ}の\ruby{観点}{かんてん}から見ると、Android 3.xおよび4.xは\ruby{体験}{たいけん}\ruby{標準化}{ひょうじゅんか}の\ruby{基盤}{きばん}を\ruby{築}{きず}いた\ruby{時期}{じき}である。\ruby{バージョン}{ばーじょん}\ruby{分断}{ぶんだん}の\ruby{問題}{もんだい}は\ruby{依然}{いぜん}として\ruby{存在}{そんざい}していたものの、Androidは\ruby{開発}{かいはつ}\ruby{者}{しゃ}と\ruby{端末}{たんまつ}\ruby{メーカー}{めーかー}が\ruby{共有}{きょうゆう}できる\ruby{明確}{めいかく}な\ruby{インターフェース}{いんたーふぇーす}および\ruby{技術}{ぎじゅつ}\ruby{原則}{げんそく}を\ruby{確立}{かくりつ}した。これが\ruby{後続}{こうぞく}の\ruby{バージョン}{ばーじょん}における\ruby{設計}{せっけい}\ruby{言語}{げんご}や\ruby{システム}{しすてむ}\ruby{アーキテクチャ}{あーきてくちゃ}の\ruby{大}{おお}きな\ruby{進化}{しんか}を\ruby{支}{ささ}える\ruby{前提}{ぜんてい}となった。

\section{Android 5.x: Material Design và bước tiến lớn về ngôn ngữ thiết kế và hiệu năng hệ thống}
Android 5.x:Material Designと\ruby{設計}{せっけい}\ruby{言語}{げんご}および\ruby{システム}{しすてむ}\ruby{性能}{せいのう}における\ruby{大}{おお}きな\ruby{進展}{しんてん}

Android 5.x (Lollipop) được xem là một trong những bản phát hành quan trọng nhất trong lịch sử Android, không chỉ vì thay đổi về giao diện mà còn do những cải tổ sâu rộng ở tầng kiến trúc hệ thống. Đây là thời điểm Google thể hiện rõ định hướng xây dựng Android như một nền tảng lâu dài, nhất quán và có khả năng mở rộng vượt ra ngoài phạm vi điện thoại thông minh.

Android 5.x(Lollipop)は、Androidの\ruby{歴史}{れきし}において\ruby{最}{もっと}も\ruby{重要}{じゅうよう}な\ruby{版本}{ばんぽん}の\ruby{一}{ひと}つと\ruby{評価}{ひょうか}されている。これは\ruby{外観}{がいかん}の\ruby{変更}{へんこう}にとどまらず、\ruby{システム}{しすてむ}\ruby{アーキテクチャ}{あーきてくちゃ}の\ruby{層}{そう}に\ruby{及}{およ}ぶ\ruby{抜本}{ばっぽん}\ruby{的}{てき}な\ruby{改革}{かいかく}が\ruby{行}{おこな}われたためである。この\ruby{時期}{じき}は、GoogleがAndroidを\ruby{長期}{ちょうき}\ruby{的}{てき}で\ruby{一貫}{いっかん}\ruby{性}{せい}を\ruby{持}{も}ち、\ruby{スマートフォン}{すまーとふぉん}の\ruby{枠}{わく}を\ruby{超}{こ}えて\ruby{拡張}{かくちょう}しうる\ruby{基盤}{きばん}として\ruby{構築}{こうちく}する\ruby{方針}{ほうしん}を\ruby{明確}{めいかく}に\ruby{示}{しめ}した\ruby{転換}{てんかん}\ruby{点}{てん}であった。

Trọng tâm lớn nhất của Android 5.x là việc giới thiệu Material Design. Khác với Holo UI vốn chủ yếu mang tính chuẩn hóa hình thức, Material Design là một ngôn ngữ thiết kế hoàn chỉnh, được xây dựng dựa trên các nguyên lý vật lý mô phỏng thế giới thực như chiều sâu, ánh sáng và chuyển động. Giao diện không còn là tập hợp các thành phần tĩnh, mà trở thành một không gian có lớp (layers), nơi các đối tượng tương tác với nhau thông qua chuyển động có chủ đích.

Android 5.xの\ruby{最大}{さいだい}の\ruby{焦点}{しょうてん}は、Material Designの\ruby{導入}{どうにゅう}である。Holo UIが\ruby{主}{おも}に\ruby{視覚}{しかく}\ruby{的}{てき}な\ruby{統一}{とういつ}を\ruby{目的}{もくてき}としていたのに\ruby{対}{たい}し、Material Designは、\ruby{奥行}{おくゆ}き、\ruby{光}{ひかり}、\ruby{動}{うご}きといった\ruby{現実}{げんじつ}\ruby{世界}{せかい}を\ruby{模倣}{もほう}した\ruby{物理}{ぶつり}\ruby{的}{てき}\ruby{原理}{げんり}に\ruby{基}{もと}づく\ruby{完成}{かんせい}\ruby{度}{ど}の\ruby{高}{たか}い\ruby{設計}{せっけい}\ruby{言語}{げんご}である。ユーザー\ruby{インターフェース}{いんたーふぇーす}はもはや\ruby{静的}{せいてき}な\ruby{要素}{ようそ}の\ruby{集合}{しゅうごう}ではなく、\ruby{層}{そう}を\ruby{持}{も}つ\ruby{空間}{くうかん}として\ruby{構成}{こうせい}され、\ruby{意図}{いと}された\ruby{動}{うご}きによって\ruby{要素}{ようそ}\ruby{同士}{どうし}が\ruby{相互}{そうご}\ruby{作用}{さよう}する\ruby{環境}{かんきょう}となった。

Material Design mang lại nhiều lợi ích rõ rệt. Thứ nhất, nó tạo ra sự nhất quán cao giữa các ứng dụng và giữa các nền tảng khác nhau của Google, từ Android, web cho đến tablet. Thứ hai, các chuyển động và hiệu ứng được tiêu chuẩn hóa giúp người dùng dễ hiểu hơn về trạng thái hệ thống và kết quả của thao tác. Thứ ba, Material Design cung cấp cho lập trình viên một bộ quy tắc và thành phần UI rõ ràng, giảm đáng kể sự tùy tiện trong thiết kế ứng dụng.

Material Designは\ruby{多}{おお}くの\ruby{明確}{めいかく}な\ruby{利点}{りてん}を\ruby{もたら}{もたら}した。\ruby{第一}{だいいち}に、Android、web、tabletといったGoogleの\ruby{複数}{ふくすう}の\ruby{プラットフォーム}{ぷらっとふぉーむ}\ruby{間}{かん}で\ruby{高}{たか}い\ruby{一貫}{いっかん}\ruby{性}{せい}を\ruby{実現}{じつげん}した。\ruby{第二}{だいに}に、\ruby{動}{うご}きや\ruby{視覚}{しかく}\ruby{効果}{こうか}が\ruby{標準}{ひょうじゅん}\ruby{化}{か}されることで、\ruby{利用}{りよう}\ruby{者}{しゃ}は\ruby{システム}{しすてむ}の\ruby{状態}{じょうたい}や\ruby{操作}{そうさ}の\ruby{結果}{けっか}を\ruby{直感}{ちょっかん}\ruby{的}{てき}に\ruby{理解}{りかい}しやすくなった。\ruby{第三}{だいさん}に、Material Designは\ruby{開発}{かいはつ}\ruby{者}{しゃ}に\ruby{明確}{めいかく}な\ruby{規則}{きそく}とUI\ruby{部品}{ぶひん}を\ruby{提供}{ていきょう}し、\ruby{設計}{せっけい}における\ruby{恣意}{しい}\ruby{性}{せい}を\ruby{大幅}{おおはば}に\ruby{低減}{ていげん}した。

Song song với thay đổi về giao diện, Android 5.x thực hiện một bước chuyển quan trọng về hiệu năng khi thay thế hoàn toàn Dalvik bằng Android Runtime (ART). ART sử dụng cơ chế biên dịch trước (ahead-of-time compilation), cho phép ứng dụng được biên dịch sang mã máy ngay khi cài đặt thay vì trong lúc chạy. Điều này giúp cải thiện hiệu năng thực thi, giảm độ trễ và tối ưu việc sử dụng tài nguyên hệ thống. Dù thời gian cài đặt ứng dụng có thể tăng nhẹ, lợi ích về độ mượt và ổn định là rõ ràng, đặc biệt trên các thiết bị có cấu hình trung bình và thấp.

\ruby{外観}{がいかん}の\ruby{変更}{へんこう}と\ruby{並行}{へいこう}して、Android 5.xは\ruby{性能}{せいのう}においても\ruby{重要}{じゅうよう}な\ruby{転換}{てんかん}を\ruby{行}{おこな}い、Dalvikを\ruby{完全}{かんぜん}にAndroid Runtime(ART)へと\ruby{置}{お}き\ruby{換}{か}えた。ARTは\ruby{事前}{じぜん}\ruby{コンパイル}{こんぱいる}を\ruby{採用}{さいよう}し、\ruby{アプリケーション}{あぷりけーしょん}を\ruby{実行}{じっこう}\ruby{時}{じ}ではなく\ruby{インストール}{いんすとーる}\ruby{時}{じ}に\ruby{機械}{きかい}\ruby{語}{ご}へ\ruby{変換}{へんかん}する。この\ruby{方式}{ほうしき}により、\ruby{実行}{じっこう}\ruby{性能}{せいのう}が\ruby{向上}{こうじょう}し、\ruby{遅延}{ちえん}が\ruby{低減}{ていげん}され、\ruby{システム}{しすてむ}\ruby{資源}{しげん}の\ruby{利用}{りよう}も\ruby{最適}{さいてき}\ruby{化}{か}された。\ruby{インストール}{いんすとーる}\ruby{時間}{じかん}が\ruby{若干}{じゃっかん}\ruby{増加}{ぞうか}する\ruby{可能}{かのう}\ruby{性}{せい}はあるものの、\ruby{動作}{どうさ}の\ruby{滑}{なめ}らかさと\ruby{安定}{あんてい}\ruby{性}{せい}の\ruby{向上}{こうじょう}は\ruby{明白}{めいはく}であり、\ruby{中}{ちゅう}・\ruby{低}{てい}\ruby{性能}{せいのう}の\ruby{端末}{たんまつ}で\ruby{特}{とく}に\ruby{効果}{こうか}を\ruby{発揮}{はっき}した。

Android 5.x cũng mở rộng hỗ trợ kiến trúc 64-bit, phản ánh xu hướng phát triển phần cứng của ngành công nghiệp di động. Việc hỗ trợ 64-bit không chỉ cho phép tận dụng tốt hơn bộ nhớ và khả năng xử lý của thiết bị, mà còn thể hiện định hướng dài hạn của Android trong việc tiệm cận các nền tảng máy tính truyền thống về mặt năng lực kỹ thuật.

Android 5.xはまた、64-bit\ruby{アーキテクチャ}{あーきてくちゃ}への\ruby{対応}{たいおう}を\ruby{拡張}{かくちょう}し、\ruby{移動}{いどう}\ruby{体}{たい}\ruby{産業}{さんぎょう}における\ruby{ハードウェア}{はーどうぇあ}\ruby{発展}{はってん}の\ruby{潮流}{ちょうりゅう}を\ruby{反映}{はんえい}した。64-bit\ruby{対応}{たいおう}は、\ruby{メモリ}{めもり}や\ruby{処理}{しょり}\ruby{能力}{のうりょく}を\ruby{効率}{こうりつ}\ruby{的}{てき}に\ruby{活用}{かつよう}できるだけでなく、Androidが\ruby{技術}{ぎじゅつ}\ruby{的}{てき}な\ruby{能力}{のうりょく}の\ruby{面}{めん}で\ruby{従来}{じゅうらい}の\ruby{計算}{けいさん}\ruby{機}{き}\ruby{基盤}{きばん}に\ruby{近}{ちか}づくという\ruby{長期}{ちょうき}\ruby{的}{てき}な\ruby{方向}{ほうこう}を\ruby{示}{しめ}している。

Hệ thống thông báo trong Android 5.x được tái thiết kế toàn diện. Thông báo có thể hiển thị trực tiếp trên màn hình khóa, được phân loại rõ ràng hơn và cho phép tương tác nhanh. Những cải tiến này giúp Android trở nên thực dụng hơn trong sử dụng hằng ngày, đồng thời tăng khả năng kiểm soát của người dùng đối với ứng dụng.

Android 5.xでは\ruby{通知}{つうち}\ruby{システム}{しすてむ}も\ruby{全面}{ぜんめん}\ruby{的}{てき}に\ruby{再設計}{さいせっけい}された。\ruby{通知}{つうち}は\ruby{ロック}{ろっく}\ruby{画面}{がめん}に\ruby{直接}{ちょくせつ}\ruby{表示}{ひょうじ}され、より\ruby{明確}{めいかく}に\ruby{分類}{ぶんるい}されるとともに、\ruby{迅速}{じんそく}な\ruby{操作}{そうさ}を\ruby{可能}{かのう}にした。これらの\ruby{改善}{かいぜん}により、Androidは\ruby{日常}{にちじょう}\ruby{利用}{りよう}において\ruby{実用}{じつよう}\ruby{性}{せい}を\ruby{高}{たか}め、\ruby{利用}{りよう}\ruby{者}{しゃ}が\ruby{アプリケーション}{あぷりけーしょん}を\ruby{制御}{せいぎょ}する\ruby{能力}{のうりょく}も\ruby{向上}{こうじょう}した。

Từ góc độ hệ sinh thái, Android 5.x là bước chuyển từ giai đoạn “vá lỗi và tối ưu” sang giai đoạn “định hình bản sắc”. Android không còn chỉ là một nền tảng mở và linh hoạt, mà bắt đầu có tiêu chuẩn rõ ràng về thiết kế, hiệu năng và hành vi hệ thống. Điều này tạo ra sự ổn định cần thiết cho OEM trong việc phát triển thiết bị, cũng như cho lập trình viên trong việc xây dựng ứng dụng lâu dài.

\ruby{エコシステム}{えこしすてむ}の\ruby{観点}{かんてん}から\ruby{見}{み}ると、Android 5.xは「\ruby{修正}{しゅうせい}と\ruby{最適}{さいてき}\ruby{化}{か}」の\ruby{段階}{だんかい}から、「\ruby{本質}{ほんしつ}を\ruby{形作}{かたちづく}る」\ruby{段階}{だんかい}への\ruby{移行}{いこう}を\ruby{意味}{いみ}する。Androidはもはや\ruby{開放}{かいほう}\ruby{的}{てき}で\ruby{柔軟}{じゅうなん}な\ruby{基盤}{きばん}であるだけでなく、\ruby{設計}{せっけい}、\ruby{性能}{せいのう}、\ruby{システム}{しすてむ}\ruby{挙動}{きょどう}に\ruby{関}{かん}する\ruby{明確}{めいかく}な\ruby{基準}{きじゅん}を\ruby{持}{も}つ\ruby{存在}{そんざい}となった。これは、OEMが\ruby{端末}{たんまつ}を\ruby{開発}{かいはつ}するうえでも、\ruby{開発}{かいはつ}\ruby{者}{しゃ}が\ruby{長期}{ちょうき}\ruby{的}{てき}な\ruby{アプリケーション}{あぷりけーしょん}を\ruby{構築}{こうちく}するうえでも、\ruby{必要}{ひつよう}な\ruby{安定}{あんてい}\ruby{性}{せい}を\ruby{提供}{ていきょう}した。

Tóm lại, Android 5.x không chỉ cải thiện trải nghiệm người dùng trước mắt, mà còn tái cấu trúc nền tảng Android để sẵn sàng cho sự mở rộng quy mô trong tương lai. Những thay đổi ở phiên bản này đóng vai trò then chốt trong việc đưa Android từ một hệ điều hành phổ biến trở thành một nền tảng di động trưởng thành và bền vững.

\ruby{総括}{そうかつ}すると、Android 5.xは\ruby{短期}{たんき}\ruby{的}{てき}な\ruby{利用}{りよう}\ruby{者}{しゃ}\ruby{体験}{たいけん}の\ruby{改善}{かいぜん}にとどまらず、\ruby{将来}{しょうらい}の\ruby{規模}{きぼ}\ruby{拡大}{かくだい}に\ruby{備}{そな}えてAndroid\ruby{基盤}{きばん}を\ruby{再構築}{さいこうちく}した\ruby{版本}{ばんぽん}である。この\ruby{版本}{ばんぽん}における\ruby{変化}{へんか}は、Androidを\ruby{単}{たん}なる\ruby{普及}{ふきゅう}した\ruby{携帯}{けいたい}\ruby{向}{む}け\ruby{オペレーティング}{おぺれーてぃんぐ}\ruby{システム}{しすてむ}から、\ruby{成熟}{せいじゅく}し\ruby{持続}{じぞく}\ruby{可能}{かのう}な\ruby{モバイル}{もばいる}\ruby{プラットフォーム}{ぷらっとふぉーむ}へと\ruby{押}{お}し\ruby{上}{あ}げる\ruby{決定}{けってい}\ruby{的}{てき}な\ruby{役割}{やくわり}を\ruby{果}{は}たした。

\section{Sự phát triển của Google Play: phân phối ứng dụng, cập nhật và mô hình kinh doanh cho lập trình viên}
\ruby{Google}{ぐーぐる}\ruby{Play}{ぷれい}の\ruby{発展}{はってん}:\ruby{アプリケーション}{あぷりけーしょん}の\ruby{配布}{はいふ}、\ruby{更新}{こうしん}、および\ruby{開発者}{かいはつしゃ}のための\ruby{ビジネス}{びじねす}\ruby{モデル}{もでる}

Trong quá trình Android trưởng thành, Google Play nổi lên như một trụ cột không thể thiếu của hệ sinh thái. Nếu hệ điều hành là nền tảng kỹ thuật, thì Google Play chính là hạ tầng kinh tế và phân phối, quyết định mức độ hấp dẫn của Android đối với lập trình viên và doanh nghiệp.

\ruby{Android}{あんどろいど}の\ruby{成熟}{せいじゅく}とともに、\ruby{Google}{ぐーぐる}\ruby{Play}{ぷれい}は\ruby{生態系}{せいたいけい}における\ruby{不可欠}{ふかけつ}な\ruby{中核}{ちゅうかく}として\ruby{台頭}{たいとう}した。\ruby{オペレーティング}{おぺれーてぃんぐ}\ruby{システム}{しすてむ}が\ruby{技術}{ぎじゅつ}\ruby{基盤}{きばん}であるとすれば、\ruby{Google}{ぐーぐる}\ruby{Play}{ぷれい}は\ruby{経済的}{けいざいてき}かつ\ruby{流通}{りゅうつう}の\ruby{インフラ}{いんふら}であり、\ruby{開発者}{かいはつしゃ}や\ruby{企業}{きぎょう}にとってのAndroidの\ruby{魅力}{みりょく}を\ruby{左右}{さゆう}する\ruby{存在}{そんざい}である。

Ban đầu, Android Market chỉ đóng vai trò là nơi tải ứng dụng cơ bản. Tuy nhiên, cùng với sự phát triển của Android 4.x và đặc biệt là giai đoạn Android 5.x, Google Play được mở rộng thành một hệ sinh thái dịch vụ hoàn chỉnh. Google Play không chỉ phân phối ứng dụng mà còn đảm nhiệm cập nhật phần mềm, quản lý giấy phép, xử lý thanh toán và cung cấp công cụ phân tích cho nhà phát triển.

\ruby{当初}{とうしょ}、Android Marketは\ruby{基本的}{きほんてき}な\ruby{アプリケーション}{あぷりけーしょん}\ruby{配布}{はいふ}の\ruby{場}{ば}に\ruby{過}{す}ぎなかった。しかし、Android 4.x、とりわけAndroid 5.xの\ruby{時期}{じき}に\ruby{入}{はい}ると、\ruby{Google}{ぐーぐる}\ruby{Play}{ぷれい}は\ruby{包括的}{ほうかつてき}な\ruby{サービス}{さーびす}\ruby{生態系}{せいたいけい}へと\ruby{拡張}{かくちょう}された。\ruby{Google}{ぐーぐる}\ruby{Play}{ぷれい}は、\ruby{アプリケーション}{あぷりけーしょん}の\ruby{配布}{はいふ}に\ruby{留}{とど}まらず、\ruby{ソフトウェア}{そふとうぇあ}の\ruby{更新}{こうしん}、\ruby{ライセンス}{らいせんす}\ruby{管理}{かんり}、\ruby{決済}{けっさい}\ruby{処理}{しょり}、さらには\ruby{開発者}{かいはつしゃ}向けの\ruby{分析}{ぶんせき}\ruby{ツール}{つーる}を\ruby{提供}{ていきょう}するようになった。

Một trong những đóng góp quan trọng nhất của Google Play là cơ chế cập nhật ứng dụng tập trung. Người dùng không còn phải tải lại ứng dụng thủ công từ nhiều nguồn khác nhau, trong khi lập trình viên có thể nhanh chóng sửa lỗi, bổ sung tính năng và cải thiện bảo mật. Cơ chế này giúp giảm đáng kể chi phí bảo trì phần mềm, đồng thời nâng cao chất lượng chung của hệ sinh thái ứng dụng Android.

\ruby{Google}{ぐーぐる}\ruby{Play}{ぷれい}の\ruby{最}{もっと}も\ruby{重要}{じゅうよう}な\ruby{貢献}{こうけん}の\ruby{一}{ひと}つが、\ruby{集中}{しゅうちゅう}\ruby{型}{がた}の\ruby{アプリケーション}{あぷりけーしょん}\ruby{更新}{こうしん}\ruby{機構}{きこう}である。\ruby{利用者}{りようしゃ}は\ruby{複数}{ふくすう}の\ruby{配布元}{はいふもと}から\ruby{手動}{しゅどう}で\ruby{再}{さい}\ruby{ダウンロード}{だうんろーど}する\ruby{必要}{ひつよう}がなくなり、\ruby{開発者}{かいはつしゃ}は\ruby{不具合}{ふぐあい}の\ruby{修正}{しゅうせい}、\ruby{機能}{きのう}の\ruby{追加}{ついか}、および\ruby{セキュリティ}{せきゅりてぃ}の\ruby{改善}{かいぜん}を\ruby{迅速}{じんそく}に\ruby{行}{おこな}えるようになった。この\ruby{仕組}{しく}みは、\ruby{保守}{ほしゅ}\ruby{コスト}{こすと}を\ruby{大幅}{おおはば}に\ruby{削減}{さくげん}すると同時に、Android\ruby{生態系}{せいたいけい}\ruby{全体}{ぜんたい}の\ruby{品質}{ひんしつ}を\ruby{向上}{こうじょう}させた。

Về mặt kinh tế, Google Play cung cấp mô hình kinh doanh rõ ràng và linh hoạt. Lập trình viên có thể phân phối ứng dụng miễn phí, ứng dụng trả phí, hoặc kết hợp với các hình thức kiếm tiền như mua hàng trong ứng dụng (in-app purchase) và đăng ký thuê bao (subscription). Việc tích hợp sẵn hệ thống thanh toán giúp giảm rào cản kỹ thuật và pháp lý, đặc biệt đối với lập trình viên cá nhân và doanh nghiệp nhỏ. Nhờ đó, Android trở thành một nền tảng có chi phí gia nhập thấp nhưng khả năng tiếp cận thị trường rất rộng.

\ruby{経済的}{けいざいてき}な\ruby{側面}{そくめん}において、\ruby{Google}{ぐーぐる}\ruby{Play}{ぷれい}は\ruby{明確}{めいかく}で\ruby{柔軟}{じゅうなん}な\ruby{ビジネス}{びじねす}\ruby{モデル}{もでる}を\ruby{提供}{ていきょう}している。\ruby{開発者}{かいはつしゃ}は、\ruby{無料}{むりょう}\ruby{アプリケーション}{あぷりけーしょん}、\ruby{有料}{ゆうりょう}\ruby{アプリケーション}{あぷりけーしょん}、あるいは\ruby{アプリ}{あぷり}\ruby{内}{ない}\ruby{課金}{かきん}や\ruby{定期}{ていき}\ruby{購読}{こうどく}といった\ruby{収益}{しゅうえき}\ruby{手法}{しゅほう}を\ruby{組}{く}み\ruby{合}{あ}わせることができる。\ruby{決済}{けっさい}\ruby{システム}{しすてむ}が\ruby{標準}{ひょうじゅん}で\ruby{統合}{とうごう}されていることで、\ruby{個人}{こじん}\ruby{開発者}{かいはつしゃ}や\ruby{中小}{ちゅうしょう}\ruby{企業}{きぎょう}にとっての\ruby{技術的}{ぎじゅつてき}・\ruby{法的}{ほうてき}\ruby{障壁}{しょうへき}が\ruby{低減}{ていげん}された。その\ruby{結果}{けっか}、Androidは\ruby{参入}{さんにゅう}\ruby{コスト}{こすと}が\ruby{低}{ひく}く、\ruby{市場}{しじょう}\ruby{到達}{とうたつ}\ruby{範囲}{はんい}の\ruby{広}{ひろ}い\ruby{プラットフォーム}{ぷらっとふぉーむ}となった。

Google Play cũng đóng vai trò quan trọng trong việc chuẩn hóa chất lượng ứng dụng. Thông qua các chính sách phát hành, kiểm duyệt và đánh giá tự động, Google từng bước nâng cao yêu cầu về bảo mật, quyền riêng tư và trải nghiệm người dùng. Dù không kiểm soát chặt chẽ như một số nền tảng khác, Google Play vẫn tạo ra một mức chuẩn tối thiểu, giúp người dùng tin tưởng hơn khi cài đặt ứng dụng.

\ruby{Google}{ぐーぐる}\ruby{Play}{ぷれい}は、\ruby{アプリケーション}{あぷりけーしょん}\ruby{品質}{ひんしつ}の\ruby{標準化}{ひょうじゅんか}においても\ruby{重要}{じゅうよう}な\ruby{役割}{やくわり}を\ruby{果}{は}たしている。\ruby{公開}{こうかい}\ruby{ポリシー}{ぽりしー}、\ruby{審査}{しんさ}、および\ruby{自動}{じどう}\ruby{評価}{ひょうか}を\ruby{通}{とお}じて、\ruby{セキュリティ}{せきゅりてぃ}、\ruby{プライバシー}{ぷらいばしー}、\ruby{利用者}{りようしゃ}\ruby{体験}{たいけん}に\ruby{関}{かん}する\ruby{要件}{ようけん}が\ruby{段階的}{だんかいてき}に\ruby{引}{ひ}き\ruby{上}{あ}げられてきた。\ruby{他}{ほか}の\ruby{一部}{いちぶ}\ruby{プラットフォーム}{ぷらっとふぉーむ}ほど\ruby{厳格}{げんかく}ではないものの、\ruby{最低限}{さいていげん}の\ruby{基準}{きじゅん}を\ruby{形成}{けいせい}することで、\ruby{利用者}{りようしゃ}の\ruby{信頼}{しんらい}を\ruby{支}{ささ}えている。

Đối với lập trình viên, Google Play cung cấp các công cụ hỗ trợ phát triển và vận hành ứng dụng ở quy mô lớn. Hệ thống thống kê lượt cài đặt, đánh giá người dùng, báo cáo lỗi và thử nghiệm A/B cho phép nhà phát triển đưa ra quyết định dựa trên dữ liệu thay vì cảm tính. Điều này phản ánh sự chuyển dịch của Android từ một nền tảng mang tính thử nghiệm sang một môi trường phát triển chuyên nghiệp.

\ruby{開発者}{かいはつしゃ}の\ruby{視点}{してん}から\ruby{見}{み}ると、\ruby{Google}{ぐーぐる}\ruby{Play}{ぷれい}は\ruby{大規模}{だいきぼ}な\ruby{アプリケーション}{あぷりけーしょん}\ruby{運用}{うんよう}を\ruby{支援}{しえん}する\ruby{ツール}{つーる}を\ruby{提供}{ていきょう}している。\ruby{インストール}{いんすとーる}\ruby{数}{すう}の\ruby{統計}{とうけい}、\ruby{利用者}{りようしゃ}\ruby{評価}{ひょうか}、\ruby{エラー}{えらー}\ruby{報告}{ほうこく}、および\ruby{A/B}{えーびー}\ruby{テスト}{てすと}により、\ruby{感覚}{かんかく}ではなく\ruby{データ}{でーた}に\ruby{基}{もと}づいた\ruby{意思}{いし}\ruby{決定}{けってい}が\ruby{可能}{かのう}となった。これは、Androidが\ruby{実験的}{じっけんてき}な\ruby{段階}{だんかい}から、\ruby{専門的}{せんもんてき}な\ruby{開発}{かいはつ}\ruby{環境}{かんきょう}へと\ruby{移行}{いこう}したことを\ruby{示}{しめ}している。

Từ góc độ hệ sinh thái, Google Play là yếu tố then chốt giúp Android vượt qua thách thức về phân mảnh. Dù các thiết bị có thể chạy những phiên bản Android khác nhau, Google Play vẫn đảm bảo khả năng phân phối ứng dụng và dịch vụ cốt lõi một cách tương đối thống nhất. Điều này giúp duy trì sự gắn kết của hệ sinh thái ngay cả khi phần cứng và nhà sản xuất rất đa dạng.

\ruby{生態系}{せいたいけい}の\ruby{観点}{かんてん}では、\ruby{Google}{ぐーぐる}\ruby{Play}{ぷれい}はAndroidが\ruby{断片化}{だんぺんか}の\ruby{課題}{かだい}を\ruby{乗}{の}り\ruby{越}{こ}えるための\ruby{中核}{ちゅうかく}\ruby{要素}{ようそ}である。\ruby{端末}{たんまつ}が\ruby{異}{こと}なる\ruby{Android}{あんどろいど}\ruby{バージョン}{ばーじょん}を\ruby{実行}{じっこう}していても、\ruby{Google}{ぐーぐる}\ruby{Play}{ぷれい}は\ruby{アプリケーション}{あぷりけーしょん}と\ruby{中核}{ちゅうかく}\ruby{サービス}{さーびす}の\ruby{配布}{はいふ}を\ruby{比較的}{ひかくてき}\ruby{統一}{とういつ}された\ruby{形}{かたち}で\ruby{保証}{ほしょう}する。これにより、\ruby{多様}{たよう}な\ruby{ハードウェア}{はーどうぇあ}や\ruby{メーカー}{めーかー}が\ruby{存在}{そんざい}しても、\ruby{生態系}{せいたいけい}の\ruby{一体性}{いったいせい}が\ruby{維持}{いじ}される。

Tóm lại, sự phát triển của Google Play đã biến Android từ một hệ điều hành mã nguồn mở thành một nền tảng kinh tế số hoàn chỉnh. Chính Google Play đã tạo động lực tài chính cho lập trình viên, thúc đẩy đổi mới ứng dụng và đóng vai trò trung tâm trong sự bùng nổ của hệ sinh thái Android.

\ruby{総括}{そうかつ}すると、\ruby{Google}{ぐーぐる}\ruby{Play}{ぷれい}の\ruby{発展}{はってん}は、Androidを\ruby{単}{たん}なる\ruby{オープン}{おーぷん}\ruby{ソース}{そーす}の\ruby{オペレーティング}{おぺれーてぃんぐ}\ruby{システム}{しすてむ}から、\ruby{完全}{かんぜん}な\ruby{デジタル}{でじたる}\ruby{経済}{けいざい}\ruby{プラットフォーム}{ぷらっとふぉーむ}へと\ruby{変貌}{へんぼう}させた。\ruby{Google}{ぐーぐる}\ruby{Play}{ぷれい}こそが、\ruby{開発者}{かいはつしゃ}に\ruby{経済的}{けいざいてき}\ruby{動機}{どうき}を\ruby{与}{あた}え、\ruby{アプリケーション}{あぷりけーしょん}の\ruby{革新}{かくしん}を\ruby{促進}{そくしん}し、Android\ruby{生態系}{せいたいけい}の\ruby{爆発的}{ばくはつてき}\ruby{成長}{せいちょう}において\ruby{中心的}{ちゅうしんてき}な\ruby{役割}{やくわり}を\ruby{果}{は}たしてきたのである。

\section{Vai trò của OEM: tùy biến hệ điều hành, cạnh tranh sản phẩm và tác động đến hệ sinh thái}
OEMの\ruby{役割}{やくわり}:\ruby{オペレーティングシステム}{おぺれーてぃんぐしすてむ}の\ruby{カスタマイズ}{かすたまいず}、\ruby{製品}{せいひん}\ruby{競争}{きょうそう}、および\ruby{エコシステム}{えこしすてむ}への\ruby{影響}{えいきょう}

Một trong những đặc điểm cốt lõi làm nên sự khác biệt của Android là khả năng cho phép các nhà sản xuất thiết bị tùy biến sâu hệ điều hành. OEM không chỉ sử dụng Android như một nền tảng có sẵn, mà còn điều chỉnh giao diện, tính năng và dịch vụ để tạo ra bản sắc riêng cho sản phẩm của mình. Điều này đã định hình mạnh mẽ cách Android phát triển và lan rộng trên thị trường toàn cầu.

Androidを\ruby{他}{ほか}の\ruby{プラットフォーム}{ぷらっとふぉーむ}と\ruby{差別化}{さべつか}する\ruby{中核的}{ちゅうかくてき}\ruby{特徴}{とくちょう}の一つは、\ruby{端末}{たんまつ}\ruby{製造業者}{せいぞうぎょうしゃ}が\ruby{オペレーティングシステム}{おぺれーてぃんぐしすてむ}を\ruby{深}{ふか}く\ruby{カスタマイズ}{かすたまいず}できる\ruby{点}{てん}にある。OEMは、Androidを\ruby{既成}{きせい}の\ruby{基盤}{きばん}として\ruby{利用}{りよう}するだけでなく、\ruby{ユーザー}{ゆーざー}\ruby{インターフェース}{いんたーふぇーす}、\ruby{機能}{きのう}、および\ruby{サービス}{さーびす}を\ruby{調整}{ちょうせい}し、\ruby{自社}{じしゃ}\ruby{製品}{せいひん}としての\ruby{独自性}{どくじせい}を\ruby{創出}{そうしゅつ}する。この\ruby{柔軟性}{じゅうなんせい}は、Androidの\ruby{成長}{せいちょう}と\ruby{世界的}{せかいてき}な\ruby{普及}{ふきゅう}の\ruby{在}{あ}り\ruby{方}{かた}を\ruby{大}{おお}きく\ruby{形作}{かたちづく}ってきた。

Về mặt tích cực, sự tham gia của OEM tạo ra mức độ cạnh tranh rất cao trong thị trường thiết bị Android. Các nhà sản xuất liên tục đổi mới về phần cứng như kích thước và chất lượng màn hình, camera, pin, hiệu năng xử lý và thiết kế công nghiệp. Nhờ đó, Android nhanh chóng phủ kín nhiều phân khúc, từ thiết bị giá rẻ cho thị trường mới nổi đến các dòng sản phẩm cao cấp cạnh tranh trực tiếp với các nền tảng khác. Chính sự đa dạng này là yếu tố then chốt giúp Android mở rộng thị phần với tốc độ vượt trội.

\ruby{肯定的}{こうていてき}な\ruby{側面}{そくめん}として、OEMの\ruby{参入}{さんにゅう}はAndroid\ruby{端末}{たんまつ}\ruby{市場}{しじょう}に\ruby{極}{きわ}めて\ruby{高}{たか}い\ruby{競争}{きょうそう}\ruby{水準}{すいじゅん}を\ruby{もたら}{もたら}した。\ruby{製造業者}{せいぞうぎょうしゃ}は、\ruby{画面}{がめん}の\ruby{サイズ}{さいず}や\ruby{品質}{ひんしつ}、\ruby{カメラ}{かめら}、\ruby{バッテリー}{ばってりー}、\ruby{処理}{しょり}\ruby{性能}{せいのう}、\ruby{工業}{こうぎょう}\ruby{デザイン}{でざいん}といった\ruby{ハードウェア}{はーどうぇあ}\ruby{面}{めん}で\ruby{継続的}{けいぞくてき}に\ruby{革新}{かくしん}を\ruby{進}{すす}めてきた。その\ruby{結果}{けっか}、Androidは\ruby{新興}{しんこう}\ruby{市場}{しじょう}の\ruby{低価格}{ていかかく}\ruby{端末}{たんまつ}から、\ruby{他}{ほか}の\ruby{プラットフォーム}{ぷらっとふぉーむ}と\ruby{直接}{ちょくせつ}\ruby{競合}{きょうごう}する\ruby{高級}{こうきゅう}\ruby{機種}{きしゅ}まで、\ruby{多様}{たよう}な\ruby{セグメント}{せぐめんと}を\ruby{迅速}{じんそく}に\ruby{網羅}{もうら}した。この\ruby{多様性}{たようせい}こそが、Androidの\ruby{市場}{しじょう}\ruby{シェア}{しぇあ}を\ruby{飛躍的}{ひやくてき}に\ruby{拡大}{かくだい}させた\ruby{決定的}{けっていてき}\ruby{要因}{よういん}である。

Tùy biến hệ điều hành cũng cho phép OEM tích hợp các tính năng riêng nhằm khác biệt hóa sản phẩm. Các giao diện tùy chỉnh, ứng dụng hệ thống bổ sung và dịch vụ đi kèm giúp OEM xây dựng hệ sinh thái riêng xoay quanh thiết bị của mình. Trong nhiều trường hợp, các tùy biến này mang lại giá trị thực tế cho người dùng, đặc biệt khi đáp ứng tốt nhu cầu địa phương hoặc thói quen sử dụng cụ thể.

\ruby{オペレーティングシステム}{おぺれーてぃんぐしすてむ}の\ruby{カスタマイズ}{かすたまいず}は、OEMが\ruby{独自}{どくじ}の\ruby{機能}{きのう}を\ruby{統合}{とうごう}し、\ruby{製品}{せいひん}を\ruby{差別化}{さべつか}することも\ruby{可能}{かのう}にする。\ruby{独自}{どくじ}の\ruby{インターフェース}{いんたーふぇーす}、\ruby{追加}{ついか}された\ruby{システム}{しすてむ}\ruby{アプリケーション}{あぷりけーしょん}、および\ruby{付随}{ふずい}する\ruby{サービス}{さーびす}は、OEMが\ruby{端末}{たんまつ}を\ruby{中心}{ちゅうしん}とした\ruby{独自}{どくじ}の\ruby{エコシステム}{えこしすてむ}を\ruby{構築}{こうちく}する\ruby{助}{たす}けとなる。\ruby{多}{おお}くの\ruby{場合}{ばあい}、これらの\ruby{カスタマイズ}{かすたまいず}は\ruby{利用者}{りようしゃ}にとって\ruby{実質的}{じっしつてき}な\ruby{価値}{かち}を\ruby{提供}{ていきょう}し、とりわけ\ruby{地域}{ちいき}\ruby{特有}{とくゆう}の\ruby{需要}{じゅよう}や\ruby{使用}{しよう}\ruby{習慣}{しゅうかん}に\ruby{適合}{てきごう}する\ruby{場合}{ばあい}に\ruby{効果}{こうか}を\ruby{発揮}{はっき}する。

Tuy nhiên, mức độ tự do cao của OEM cũng kéo theo những hệ quả tiêu cực. Sự khác biệt về giao diện và tính năng giữa các thiết bị Android dẫn đến trải nghiệm người dùng không đồng nhất. Việc tùy biến sâu còn làm chậm quá trình cập nhật hệ điều hành, khi mỗi phiên bản Android mới cần được điều chỉnh lại để phù hợp với từng thiết bị và từng nhà sản xuất. Điều này góp phần tạo ra hiện tượng phân mảnh phiên bản, một trong những thách thức lớn nhất của hệ sinh thái Android.

しかし、OEMに\ruby{与}{あた}えられた\ruby{高}{たか}い\ruby{自由度}{じゆうど}は、\ruby{否定的}{ひていてき}な\ruby{影響}{えいきょう}も\ruby{伴}{ともな}う。Android\ruby{端末}{たんまつ}\ruby{間}{かん}で\ruby{インターフェース}{いんたーふぇーす}や\ruby{機能}{きのう}が\ruby{大}{おお}きく\ruby{異}{こと}なることで、\ruby{ユーザー}{ゆーざー}\ruby{体験}{たいけん}は\ruby{一貫}{いっかん}しなくなる。\ruby{深度}{しんど}の\ruby{高}{たか}い\ruby{カスタマイズ}{かすたまいず}は、\ruby{新}{あたら}しいAndroid\ruby{バージョン}{ばーじょん}ごとに\ruby{端末}{たんまつ}および\ruby{製造業者}{せいぞうぎょうしゃ}へ\ruby{再調整}{さいちょうせい}を\ruby{必要}{ひつよう}とするため、\ruby{更新}{こうしん}\ruby{速度}{そくど}を\ruby{低下}{ていか}させる。このことが、Android\ruby{エコシステム}{えこしすてむ}における\ruby{最大級}{さいだいきゅう}の\ruby{課題}{かだい}の一つである\ruby{バージョン}{ばーじょん}\ruby{断片化}{だんぺんか}を\ruby{助長}{じょちょう}してきた。

Từ góc độ lập trình viên, vai trò của OEM vừa là cơ hội vừa là thách thức. Một mặt, số lượng thiết bị và người dùng lớn mở ra thị trường tiềm năng rộng chưa từng có. Mặt khác, sự đa dạng về phần cứng và tùy biến hệ thống buộc lập trình viên phải đầu tư nhiều hơn vào kiểm thử và tối ưu ứng dụng, nhằm đảm bảo khả năng hoạt động ổn định trên nhiều cấu hình khác nhau.

\ruby{開発者}{かいはつしゃ}の\ruby{視点}{してん}から\ruby{見}{み}ると、OEMの\ruby{存在}{そんざい}は\ruby{機会}{きかい}であると\ruby{同時}{どうじ}に\ruby{課題}{かだい}でもある。一\ruby{方}{ほう}で、\ruby{膨大}{ぼうだい}な\ruby{端末}{たんまつ}と\ruby{利用者}{りようしゃ}\ruby{数}{すう}は、\ruby{前例}{ぜんれい}のない\ruby{規模}{きぼ}の\ruby{市場}{しじょう}\ruby{機会}{きかい}を\ruby{提供}{ていきょう}する。他\ruby{方}{ほう}で、\ruby{ハードウェア}{はーどうぇあ}の\ruby{多様性}{たようせい}と\ruby{システム}{しすてむ}\ruby{カスタマイズ}{かすたまいず}の\ruby{違}{ちが}いは、\ruby{多数}{たすう}の\ruby{構成}{こうせい}で\ruby{安定}{あんてい}した\ruby{動作}{どうさ}を\ruby{確保}{かくほ}するため、\ruby{検証}{けんしょう}と\ruby{最適化}{さいてきか}への\ruby{追加}{ついか}\ruby{投資}{とうし}を\ruby{要求}{ようきゅう}する。

Dù tồn tại nhiều hạn chế, vai trò của OEM là không thể thay thế trong quá trình bùng nổ của Android. Chính mô hình hợp tác mở giữa Google và các nhà sản xuất thiết bị đã tạo nên sức mạnh quy mô của Android, cho phép nền tảng này hiện diện ở hầu hết mọi phân khúc và khu vực thị trường. Về tổng thể, OEM không chỉ là bên triển khai Android, mà còn là tác nhân chủ động định hình cấu trúc cạnh tranh và đặc tính của toàn bộ hệ sinh thái.

\ruby{多}{おお}くの\ruby{制約}{せいやく}が\ruby{存在}{そんざい}するにもかかわらず、Androidの\ruby{急成長}{きゅうせいちょう}においてOEMの\ruby{役割}{やくわり}は\ruby{代替}{だいたい}\ruby{不可能}{ふかのう}である。Googleと\ruby{端末}{たんまつ}\ruby{製造業者}{せいぞうぎょうしゃ}の\ruby{間}{あいだ}にある\ruby{オープン}{おーぷん}な\ruby{協力}{きょうりょく}\ruby{モデル}{もでる}こそが、Androidに\ruby{規模}{きぼ}の\ruby{優位性}{ゆういせい}を\ruby{与}{あた}え、\ruby{ほぼ}{ほぼ}\ruby{全}{すべ}ての\ruby{市場}{しじょう}\ruby{セグメント}{せぐめんと}と\ruby{地域}{ちいき}における\ruby{展開}{てんかい}を\ruby{可能}{かのう}にした。\ruby{総合的}{そうごうてき}に\ruby{見}{み}れば、OEMはAndroidの\ruby{実装者}{じっそうしゃ}にとどまらず、\ruby{競争}{きょうそう}\ruby{構造}{こうぞう}と\ruby{エコシステム}{えこしすてむ}の\ruby{特性}{とくせい}を\ruby{主体的}{しゅたいてき}に\ruby{形成}{けいせい}する\ruby{重要}{じゅうよう}な\ruby{担}{にな}い\ruby{手}{て}である。

\section{Cộng đồng lập trình viên: động lực chính thúc đẩy Android trở thành nền tảng di động phổ biến nhất}
\ruby{開発者}{かいはつしゃ}\ruby{コミュニティ}{こみゅにてぃ}:Androidを\ruby{最}{もっと}も\ruby{普及}{ふきゅう}した\ruby{移動}{いどう}\ruby{基盤}{きばん}へと\ruby{押}{お}し\ruby{上}{あ}げた\ruby{中核}{ちゅうかく}\ruby{動力}{どうりょく}

Cộng đồng lập trình viên là yếu tố cốt lõi biến Android từ một hệ điều hành mã nguồn mở thành một hệ sinh thái ứng dụng phong phú và có giá trị thực tiễn cao. Không giống các nền tảng đóng với quy trình kiểm soát chặt chẽ, Android được thiết kế với rào cản gia nhập thấp, cho phép lập trình viên cá nhân, nhóm nhỏ và doanh nghiệp ở mọi quy mô tham gia phát triển ứng dụng.

\ruby{開発者}{かいはつしゃ}\ruby{コミュニティ}{こみゅにてぃ}は、Androidを\ruby{オープンソース}{おーぷんそーす}の\ruby{オペレーティングシステム}{おぺれーてぃんぐしすてむ}から、\ruby{実用的}{じつようてき}な\ruby{価値}{かち}を\ruby{持}{も}つ\ruby{豊富}{ほうふ}な\ruby{アプリケーション}{あぷりけーしょん}\ruby{生態系}{せいたいけい}へと\ruby{変}{か}える\ruby{中核的}{ちゅうかくてき}な\ruby{要素}{ようそ}である。\ruby{厳格}{げんかく}な\ruby{統制}{とうせい}\ruby{過程}{かてい}を\ruby{伴}{ともな}う\ruby{閉鎖的}{へいさてき}な\ruby{基盤}{きばん}とは\ruby{異}{こと}なり、Androidは\ruby{参入}{さんにゅう}\ruby{障壁}{しょうへき}が\ruby{低}{ひく}く\ruby{設計}{せっけい}されており、\ruby{個人}{こじん}の\ruby{開発者}{かいはつしゃ}、\ruby{小規模}{しょうきぼ}な\ruby{チーム}{ちーむ}、および\ruby{多様}{たよう}な\ruby{規模}{きぼ}の\ruby{企業}{きぎょう}が\ruby{アプリケーション}{あぷりけーしょん}\ruby{開発}{かいはつ}に\ruby{参加}{さんか}できる。

Một trong những lợi thế lớn nhất của Android đối với lập trình viên là bộ công cụ phát triển ngày càng hoàn thiện. Android SDK, tài liệu kỹ thuật và các thư viện hỗ trợ được cải tiến liên tục trong giai đoạn Android 3.x–5.x, giúp giảm độ phức tạp trong quá trình xây dựng và bảo trì ứng dụng. Việc sử dụng ngôn ngữ Java quen thuộc cũng giúp Android tiếp cận được một lượng lớn lập trình viên đã có nền tảng sẵn từ các lĩnh vực khác.

Androidが\ruby{開発者}{かいはつしゃ}にとって\ruby{持}{も}つ\ruby{最大}{さいだい}の\ruby{利点}{りてん}の\ruby{一}{ひと}つは、\ruby{成熟}{せいじゅく}し\ruby{続}{つづ}ける\ruby{開発}{かいはつ}\ruby{ツール}{つーる}\ruby{群}{ぐん}である。Android 3.xから5.xの\ruby{期間}{きかん}において、Android SDK、\ruby{技術}{ぎじゅつ}\ruby{文書}{ぶんしょ}、および\ruby{支援}{しえん}\ruby{ライブラリ}{らいぶらり}は\ruby{継続的}{けいぞくてき}に\ruby{改善}{かいぜん}され、\ruby{アプリケーション}{あぷりけーしょん}の\ruby{構築}{こうちく}および\ruby{保守}{ほしゅ}における\ruby{複雑性}{ふくざつせい}を\ruby{低減}{ていげん}した。また、\ruby{Java}{じゃば}という\ruby{馴染}{なじ}み\ruby{深}{ぶか}い\ruby{言語}{げんご}の\ruby{採用}{さいよう}は、\ruby{他}{ほか}の\ruby{分野}{ぶんや}で\ruby{基盤}{きばん}を\ruby{持}{も}つ\ruby{多数}{たすう}の\ruby{開発者}{かいはつしゃ}をAndroidへと\ruby{引}{ひ}き\ruby{付}{つ}けた。

Cộng đồng lập trình viên Android phát triển mạnh mẽ nhờ sự kết hợp giữa quy mô người dùng lớn và khả năng tiếp cận thị trường toàn cầu. Một ứng dụng Android có thể nhanh chóng tiếp cận hàng triệu người dùng thông qua Google Play mà không cần các thỏa thuận phân phối phức tạp. Điều này tạo động lực rõ ràng cho việc đầu tư thời gian và nguồn lực vào phát triển ứng dụng Android, ngay cả với những cá nhân hoặc nhóm phát triển nhỏ.

Androidの\ruby{開発者}{かいはつしゃ}\ruby{コミュニティ}{こみゅにてぃ}は、\ruby{大規模}{だいきぼ}な\ruby{利用者}{りようしゃ}\ruby{基盤}{きばん}と\ruby{世界的}{せかいてき}な\ruby{市場}{しじょう}への\ruby{到達性}{とうたつせい}の\ruby{結合}{けつごう}によって\ruby{急速}{きゅうそく}に\ruby{拡大}{かくだい}した。Android\ruby{アプリケーション}{あぷりけーしょん}は、\ruby{複雑}{ふくざつ}な\ruby{配信}{はいしん}\ruby{契約}{けいやく}を\ruby{必要}{ひつよう}とせず、Google Playを\ruby{通}{とお}じて\ruby{数百万}{すうひゃくまん}の\ruby{利用者}{りようしゃ}へ\ruby{迅速}{じんそく}に\ruby{届}{とど}けることができる。この\ruby{特性}{とくせい}は、\ruby{個人}{こじん}や\ruby{小規模}{しょうきぼ}な\ruby{開発}{かいはつ}\ruby{チーム}{ちーむ}にとっても、Android\ruby{開発}{かいはつ}へ\ruby{時間}{じかん}と\ruby{資源}{しげん}を\ruby{投資}{とうし}する\ruby{明確}{めいかく}な\ruby{動機}{どうき}を\ruby{生}{しょう}じさせた。

Bên cạnh các ứng dụng thương mại, cộng đồng Android còn đóng góp mạnh mẽ vào hệ sinh thái mã nguồn mở. Nhiều thư viện, công cụ và framework do cộng đồng phát triển đã trở thành tiêu chuẩn thực tế trong phát triển ứng dụng Android. Những đóng góp này giúp giảm trùng lặp công sức, nâng cao chất lượng phần mềm và rút ngắn chu kỳ phát triển ứng dụng.

\ruby{商用}{しょうよう}\ruby{アプリケーション}{あぷりけーしょん}に\ruby{加}{くわ}え、Android\ruby{コミュニティ}{こみゅにてぃ}は\ruby{オープンソース}{おーぷんそーす}の\ruby{生態系}{せいたいけい}にも\ruby{積極的}{せっきょくてき}に\ruby{貢献}{こうけん}してきた。\ruby{多数}{たすう}の\ruby{ライブラリ}{らいぶらり}、\ruby{ツール}{つーる}、および\ruby{フレームワーク}{ふれーむわーく}が\ruby{コミュニティ}{こみゅにてぃ}によって\ruby{開発}{かいはつ}され、Android\ruby{開発}{かいはつ}における\ruby{事実上}{じじつじょう}の\ruby{標準}{ひょうじゅん}となった。これらの\ruby{貢献}{こうけん}は、\ruby{重複}{ちょうふく}した\ruby{労力}{ろうりょく}を\ruby{削減}{さくげん}し、\ruby{ソフトウェア}{そふとうぇあ}の\ruby{品質}{ひんしつ}を\ruby{向上}{こうじょう}させ、\ruby{開発}{かいはつ}\ruby{サイクル}{さいくる}を\ruby{短縮}{たんしゅく}した。

Cộng đồng lập trình viên cũng đóng vai trò quan trọng trong việc phản hồi và định hình sự phát triển của nền tảng. Thông qua báo lỗi, đề xuất cải tiến và các dự án thử nghiệm, cộng đồng tạo ra áp lực và dữ liệu thực tế để Google điều chỉnh định hướng phát triển Android. Mối quan hệ hai chiều này giúp Android thích nghi nhanh với nhu cầu thực tế của thị trường và người dùng.

\ruby{開発者}{かいはつしゃ}\ruby{コミュニティ}{こみゅにてぃ}は、\ruby{基盤}{きばん}の\ruby{発展}{はってん}を\ruby{形成}{けいせい}するうえでも\ruby{重要}{じゅうよう}な\ruby{役割}{やくわり}を\ruby{果}{は}たす。\ruby{不具合}{ふぐあい}の\ruby{報告}{ほうこく}、\ruby{改善}{かいぜん}\ruby{提案}{ていあん}、および\ruby{実験的}{じっけんてき}な\ruby{プロジェクト}{ぷろじぇくと}を\ruby{通}{とお}じて、\ruby{コミュニティ}{こみゅにてぃ}はGoogleに\ruby{対}{たい}する\ruby{実証的}{じっしょうてき}な\ruby{データ}{でーた}と\ruby{圧力}{あつりょく}を\ruby{提供}{ていきょう}し、Android\ruby{開発}{かいはつ}の\ruby{方向性}{ほうこうせい}を\ruby{調整}{ちょうせい}させる。この\ruby{双方向}{そうほうこう}の\ruby{関係}{かんけい}は、Androidが\ruby{市場}{しじょう}および\ruby{利用者}{りようしゃ}の\ruby{実際的}{じっさいてき}\ruby{需要}{じゅよう}へ\ruby{迅速}{じんそく}に\ruby{適応}{てきおう}することを\ruby{可能}{かのう}にする。

Từ góc độ hệ sinh thái, chính cộng đồng lập trình viên đã biến Android thành một nền tảng có khả năng tự duy trì và mở rộng. Số lượng và sự đa dạng của ứng dụng khiến Android trở nên hấp dẫn hơn với người dùng, từ đó thu hút thêm OEM và tiếp tục mở rộng thị phần. Vòng lặp này tạo ra hiệu ứng mạng lưới, củng cố vị thế thống trị của Android trong thị trường di động toàn cầu.

\ruby{生態系}{せいたいけい}の\ruby{視点}{してん}から\ruby{見}{み}れば、Androidを\ruby{自律的}{じりつてき}に\ruby{維持}{いじ}し\ruby{拡張}{かくちょう}できる\ruby{基盤}{きばん}へと\ruby{変}{か}えたのは、まさに\ruby{開発者}{かいはつしゃ}\ruby{コミュニティ}{こみゅにてぃ}である。\ruby{アプリケーション}{あぷりけーしょん}の\ruby{数量}{すうりょう}と\ruby{多様性}{たようせい}は、Androidの\ruby{魅力}{みりょく}を\ruby{高}{たか}め、\ruby{利用者}{りようしゃ}を\ruby{引}{ひ}き\ruby{寄}{よ}せ、さらに\ruby{OEM}{おーいーえむ}を\ruby{惹}{ひ}き\ruby{付}{つ}けて\ruby{市場}{しじょう}\ruby{シェア}{しぇあ}の\ruby{拡大}{かくだい}を\ruby{促}{うなが}す。この\ruby{循環}{じゅんかん}は\ruby{ネットワーク}{ねっとわーく}\ruby{効果}{こうか}を\ruby{生}{しょう}じさせ、\ruby{世界}{せかい}の\ruby{移動}{いどう}\ruby{市場}{しじょう}におけるAndroidの\ruby{支配的}{しはいてき}\ruby{地位}{ちい}を\ruby{強化}{きょうか}する。

Tóm lại, cộng đồng lập trình viên không chỉ là người sử dụng Android như một công cụ phát triển, mà là động lực trung tâm thúc đẩy sự phổ biến và bền vững của nền tảng. Chính lực lượng này đã biến Android từ một hệ điều hành kỹ thuật thành một hệ sinh thái sống, liên tục phát triển và thích nghi với quy mô toàn cầu.

\ruby{総括}{そうかつ}すると、\ruby{開発者}{かいはつしゃ}\ruby{コミュニティ}{こみゅにてぃ}は、Androidを\ruby{単}{たん}なる\ruby{開発}{かいはつ}\ruby{ツール}{つーる}として\ruby{利用}{りよう}する\ruby{存在}{そんざい}にとどまらず、\ruby{普及}{ふきゅう}と\ruby{持続可能性}{じぞくかのうせい}を\ruby{推進}{すいしん}する\ruby{中心的}{ちゅうしんてき}な\ruby{原動力}{げんどうりょく}である。この\ruby{力}{ちから}こそが、Androidを\ruby{技術的}{ぎじゅつてき}な\ruby{オペレーティングシステム}{おぺれーてぃんぐしすてむ}から、\ruby{世界規模}{せかいきぼ}で\ruby{成長}{せいちょう}し\ruby{続}{つづ}ける\ruby{生}{い}きた\ruby{生態系}{せいたいけい}へと\ruby{変貌}{へんぼう}させたのである。
