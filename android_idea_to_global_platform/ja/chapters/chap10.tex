\chapter{Android hiện đại và xu hướng tương lai}
\ruby{現代}{げんだい}の\ruby{Android}{あんどろいど}と\ruby{将来}{しょうらい}\ruby{動向}{どうこう}

Android trong giai đoạn hiện đại không còn được phát triển theo hướng bổ sung tính năng rời rạc như các phiên bản đầu, mà chuyển sang tối ưu hóa nền tảng cốt lõi. Trọng tâm của Google là xây dựng một hệ điều hành ổn định, hiệu quả, tôn trọng quyền riêng tư và mang lại trải nghiệm nhất quán trên quy mô hàng tỷ thiết bị. Chương này phân tích các đặc điểm kỹ thuật nổi bật của Android hiện đại, các xu hướng kiến trúc quan trọng và dự đoán hướng phát triển trong tương lai dưới góc nhìn của kỹ sư công nghệ thông tin.
\ruby{現代}{げんだい}のAndroidは、\ruby{初期}{しょき}の\ruby{バージョン}{ばーじょん}のように\ruby{個別}{こべつ}の\ruby{機能}{きのう}を\ruby{追加}{ついか}する\ruby{開発}{かいはつ}\ruby{方針}{ほうしん}から、\ruby{中核}{ちゅうかく}\ruby{基盤}{きばん}の\ruby{最適化}{さいてきか}へと\ruby{移行}{いこう}している。Googleの\ruby{焦点}{しょうてん}は、\ruby{安定}{あんてい}し\ruby{効率}{こうりつ}の\ruby{高}{たか}い\ruby{OS}{おーえす}を\ruby{構築}{こうちく}し、\ruby{プライバシー}{ぷらいばしー}を\ruby{尊重}{そんちょう}しつつ、\ruby{数十億}{すうじゅうおく}の\ruby{端末}{たんまつ}で\ruby{一貫}{いっかん}した\ruby{体験}{たいけん}を\ruby{提供}{ていきょう}することにある。本\ruby{章}{しょう}では、\ruby{現代}{げんだい}Androidの\ruby{主要}{しゅよう}な\ruby{技術}{ぎじゅつ}\ruby{的}{てき}\ruby{特徴}{とくちょう}、\ruby{重要}{じゅうよう}な\ruby{アーキテクチャ}{あーきてくちゃ}\ruby{動向}{どうこう}、および\ruby{将来}{しょうらい}の\ruby{発展}{はってん}\ruby{方向}{ほうこう}を、IT\ruby{技術}{ぎじゅつ}\ruby{者}{しゃ}の\ruby{視点}{してん}から\ruby{考察}{こうさつ}する。

\section{Android hiện đại: cải tiến hiệu năng, quyền riêng tư và trải nghiệm người dùng}
\ruby{現代}{げんだい}Android:\ruby{性能}{せいのう}\ruby{改善}{かいぜん}、\ruby{プライバシー}{ぷらいばしー}、および\ruby{ユーザー}{ゆーざー}\ruby{体験}{たいけん}

Trong các phiên bản Android gần đây, đặc biệt từ Android 10 trở đi, Google đã thực hiện nhiều thay đổi mang tính hệ thống nhằm giải quyết các vấn đề tồn tại lâu dài như hiệu năng không ổn định, tiêu thụ tài nguyên cao và lo ngại về quyền riêng tư. Những cải tiến này không phải lúc nào cũng thể hiện rõ qua giao diện, nhưng có ảnh hưởng trực tiếp đến chất lượng vận hành của toàn bộ nền tảng.
\ruby{近年}{きんねん}のAndroid、\ruby{特}{とく}にAndroid 10\ruby{以降}{いこう}では、\ruby{性能}{せいのう}の\ruby{不安定}{ふあんてい}さ、\ruby{高}{たか}い\ruby{資源}{しげん}\ruby{消費}{しょうひ}、および\ruby{プライバシー}{ぷらいばしー}への\ruby{懸念}{けねん}といった\ruby{長年}{ながねん}の\ruby{課題}{かだい}を\ruby{解決}{かいけつ}するため、\ruby{体系}{たいけい}\ruby{的}{てき}な\ruby{変更}{へんこう}が\ruby{行}{おこな}われてきた。これらの\ruby{改善}{かいぜん}は\ruby{必}{かなら}ずしも\ruby{外観}{がいかん}に\ruby{現}{あらわ}れるわけではないが、\ruby{プラットフォーム}{ぷらっとふぉーむ}\ruby{全体}{ぜんたい}の\ruby{運用}{うんよう}\ruby{品質}{ひんしつ}に\ruby{直接}{ちょくせつ}\ruby{影響}{えいきょう}を\ruby{与}{あた}える。

Về hiệu năng, Android hiện đại tập trung tối ưu sâu vào các thành phần lõi của hệ điều hành. Android Runtime (ART) được cải tiến với cơ chế biên dịch dựa trên hồ sơ sử dụng (profile-guided compilation), cho phép ứng dụng được tối ưu hóa theo hành vi thực tế của người dùng thay vì biên dịch đồng loạt. Quản lý bộ nhớ được cải thiện nhằm giảm tình trạng ứng dụng bị hệ thống đóng đột ngột, đặc biệt trên các thiết bị có dung lượng RAM hạn chế. Đồng thời, các chính sách hạn chế tiến trình nền và kiểm soát dịch vụ chạy nền giúp giảm tải cho CPU và tiết kiệm pin một cách rõ rệt.
\ruby{性能}{せいのう}の\ruby{面}{めん}では、\ruby{現代}{げんだい}Androidは\ruby{OS}{おーえす}の\ruby{中核}{ちゅうかく}\ruby{構成}{こうせい}\ruby{要素}{ようそ}に\ruby{対}{たい}する\ruby{深}{ふか}い\ruby{最適化}{さいてきか}に\ruby{注力}{ちゅうりょく}している。Android Runtime(ART)は、\ruby{利用}{りよう}\ruby{履歴}{りれき}に\ruby{基}{もと}づく\ruby{プロファイル}{ぷろふぁいる}\ruby{駆動}{くどう}の\ruby{コンパイル}{こんぱいる}により\ruby{改良}{かいりょう}され、\ruby{一律}{いちりつ}の\ruby{事前}{じぜん}\ruby{最適化}{さいてきか}ではなく、\ruby{実際}{じっさい}の\ruby{使用}{しよう}\ruby{行動}{こうどう}に\ruby{即}{そく}した\ruby{最適化}{さいてきか}を\ruby{可能}{かのう}にする。\ruby{メモリ}{めもり}\ruby{管理}{かんり}も\ruby{改善}{かいぜん}され、\ruby{特}{とく}にRAM\ruby{容量}{ようりょう}が\ruby{限}{かぎ}られた\ruby{端末}{たんまつ}での\ruby{突発}{とっぱつ}\ruby{的}{てき}な\ruby{アプリ}{あぷり}\ruby{終了}{しゅうりょう}が\ruby{抑制}{よくせい}される。さらに、\ruby{バックグラウンド}{ばっくぐらうんど}\ruby{制限}{せいげん}と\ruby{サービス}{さーびす}\ruby{管理}{かんり}の\ruby{強化}{きょうか}により、CPU\ruby{負荷}{ふか}の\ruby{低減}{ていげん}と\ruby{電力}{でんりょく}\ruby{節約}{せつやく}が\ruby{顕著}{けんちょ}となった。

Quản lý năng lượng là một điểm nhấn quan trọng của Android hiện đại. Hệ điều hành áp dụng các cơ chế dự đoán hành vi người dùng để phân bổ tài nguyên hợp lý, hạn chế ứng dụng tiêu thụ pin khi không cần thiết. Các ứng dụng buộc phải tuân thủ vòng đời chặt chẽ hơn, đặc biệt là khi chạy nền, từ đó giảm hiện tượng lạm dụng tài nguyên hệ thống. Với góc nhìn kỹ thuật, đây là sự đánh đổi có chủ đích giữa tính tự do của ứng dụng và hiệu quả tổng thể của hệ thống.
\ruby{電力}{でんりょく}\ruby{管理}{かんり}は、\ruby{現代}{げんだい}Androidの\ruby{重要}{じゅうよう}な\ruby{特徴}{とくちょう}である。\ruby{OS}{おーえす}は\ruby{ユーザー}{ゆーざー}\ruby{行動}{こうどう}の\ruby{予測}{よそく}に\ruby{基}{もと}づき\ruby{資源}{しげん}を\ruby{配分}{はいぶん}し、\ruby{不要}{ふよう}な\ruby{電池}{でんち}\ruby{消費}{しょうひ}を\ruby{抑}{おさ}える。\ruby{アプリ}{あぷり}は\ruby{特}{とく}に\ruby{バックグラウンド}{ばっくぐらうんど}\ruby{実行}{じっこう}において、より\ruby{厳格}{げんかく}な\ruby{ライフサイクル}{らいふさいくる}を\ruby{順守}{じゅんしゅ}する\ruby{必要}{ひつよう}がある。これは、\ruby{アプリ}{あぷり}の\ruby{自由}{じゆう}と\ruby{システム}{しすてむ}\ruby{全体}{ぜんたい}の\ruby{効率}{こうりつ}との\ruby{意図}{いと}\ruby{的}{てき}な\ruby{均衡}{きんこう}である。

Về quyền riêng tư, Android đã có bước chuyển rõ rệt từ mô hình “cấp quyền một lần” sang “cấp quyền theo ngữ cảnh”. Người dùng có thể cho phép ứng dụng truy cập dữ liệu nhạy cảm như vị trí, camera hay microphone chỉ khi đang sử dụng ứng dụng, thay vì cho phép vĩnh viễn. Bên cạnh đó, cơ chế tự động thu hồi quyền đối với các ứng dụng không được sử dụng trong thời gian dài giúp giảm nguy cơ rò rỉ dữ liệu thụ động.
\ruby{プライバシー}{ぷらいばしー}に\ruby{関}{かん}して、Androidは「\ruby{一度}{いちど}\ruby{限}{かぎ}り」の\ruby{権限}{けんげん}\ruby{付与}{ふよ}から、\ruby{文脈}{ぶんみゃく}に\ruby{応}{おう}じた\ruby{権限}{けんげん}\ruby{管理}{かんり}へと\ruby{大}{おお}きく\ruby{転換}{てんかん}した。\ruby{利用}{りよう}\ruby{者}{しゃ}は、\ruby{位置}{いち}\ruby{情報}{じょうほう}、\ruby{カメラ}{かめら}、\ruby{マイク}{まいく}などの\ruby{機微}{きび}な\ruby{データ}{でーた}への\ruby{アクセス}{あくせす}を、\ruby{使用}{しよう}\ruby{中}{ちゅう}の\ruby{時}{とき}に\ruby{限定}{げんてい}して\ruby{許可}{きょか}できる。さらに、\ruby{長期間}{ちょうきかん}\ruby{未使用}{みしよう}の\ruby{アプリ}{あぷり}に\ruby{対}{たい}する\ruby{権限}{けんげん}の\ruby{自動}{じどう}\ruby{回収}{かいしゅう}は、\ruby{受動}{じゅどう}\ruby{的}{てき}な\ruby{情報}{じょうほう}\ruby{漏洩}{ろうえい}の\ruby{リスク}{りすく}を\ruby{低減}{ていげん}する。

Một thay đổi có tác động lớn là cơ chế Scoped Storage, giới hạn khả năng truy cập hệ thống tệp của ứng dụng. Mỗi ứng dụng chỉ có thể truy cập không gian lưu trữ riêng hoặc các dữ liệu được người dùng cho phép rõ ràng. Điều này làm tăng độ an toàn dữ liệu cá nhân, nhưng đồng thời buộc lập trình viên phải thay đổi cách tiếp cận truyền thống trong việc quản lý tệp và chia sẻ dữ liệu.
\ruby{大}{おお}きな\ruby{影響}{えいきょう}を\ruby{与}{あた}えた\ruby{変更}{へんこう}として、Scoped Storageがある。これは\ruby{アプリ}{あぷり}の\ruby{ファイル}{ふぁいる}\ruby{システム}{しすてむ}への\ruby{アクセス}{あくせす}を\ruby{制限}{せいげん}し、\ruby{各}{かく}\ruby{アプリ}{あぷり}が\ruby{自身}{じしん}の\ruby{保存}{ほぞん}\ruby{領域}{りょういき}、または\ruby{利用}{りよう}\ruby{者}{しゃ}が\ruby{明示}{めいじ}した\ruby{データ}{でーた}のみに\ruby{アクセス}{あくせす}することを\ruby{求}{もと}める。この\ruby{仕組}{しく}みは\ruby{個人}{こじん}\ruby{データ}{でーた}の\ruby{安全}{あんぜん}\ruby{性}{せい}を\ruby{高}{たか}める\ruby{一方}{いっぽう}で、\ruby{開発}{かいはつ}\ruby{者}{しゃ}に\ruby{従来}{じゅうらい}の\ruby{ファイル}{ふぁいる}\ruby{管理}{かんり}\ruby{手法}{しゅほう}の\ruby{見直}{みなお}しを\ruby{迫}{せま}る。

Trải nghiệm người dùng trong Android hiện đại được cải tiến theo hướng nhất quán và cá nhân hóa. Giao diện hệ thống không chỉ thay đổi về mặt thẩm mỹ mà còn phản ánh triết lý thiết kế mới, trong đó hệ điều hành thích nghi với người dùng thay vì ngược lại. Các thành phần giao diện phản hồi nhanh hơn, chuyển động mượt hơn và độ trễ tương tác được giảm thiểu. Tuy nhiên, từ góc nhìn kỹ sư, điều quan trọng hơn là sự ổn định và dự đoán được hành vi hệ thống, giúp ứng dụng hoạt động nhất quán trên nhiều thiết bị khác nhau.
\ruby{ユーザー}{ゆーざー}\ruby{体験}{たいけん}は、\ruby{一貫}{いっかん}性と\ruby{個別}{こべつ}\ruby{化}{か}を\ruby{重視}{じゅうし}する\ruby{方向}{ほうこう}で\ruby{改善}{かいぜん}されている。\ruby{システム}{しすてむ}\ruby{UI}{ゆーあい}は\ruby{外観}{がいかん}の\ruby{刷新}{さっしん}にとどまらず、\ruby{OS}{おーえす}が\ruby{利用}{りよう}\ruby{者}{しゃ}に\ruby{適応}{てきおう}するという\ruby{新}{あら}たな\ruby{設計}{せっけい}\ruby{思想}{しそう}を\ruby{反映}{はんえい}する。\ruby{応答}{おうとう}は\ruby{高速}{こうそく}化し、\ruby{動作}{どうさ}は\ruby{滑}{なめ}らかになり、\ruby{操作}{そうさ}\ruby{遅延}{ちえん}は\ruby{最小}{さいしょう}化された。ただし、IT\ruby{技術}{ぎじゅつ}\ruby{者}{しゃ}の\ruby{視点}{してん}では、\ruby{外見}{がいけん}よりも、\ruby{挙動}{きょどう}の\ruby{安定}{あんてい}性と\ruby{予測}{よそく}\ruby{可能}{かのう}性が\ruby{重要}{じゅうよう}であり、これが\ruby{多様}{たよう}な\ruby{端末}{たんまつ}での\ruby{一貫}{いっかん}した\ruby{動作}{どうさ}を\ruby{支}{ささ}える。

Tổng thể, Android hiện đại cho thấy một xu hướng rõ ràng: giảm dần các quyền truy cập không kiểm soát, tăng cường các cơ chế bảo vệ mặc định và tối ưu hóa hiệu năng ở mức nền tảng. Điều này khiến việc phát triển ứng dụng trở nên khắt khe hơn, nhưng đổi lại là một hệ sinh thái bền vững, an toàn và phù hợp cho quy mô lớn. Với kỹ sư công nghệ thông tin, việc hiểu rõ các thay đổi này là điều kiện cần để xây dựng và duy trì các ứng dụng Android hiện đại một cách hiệu quả.
\ruby{総合}{そうごう}すると、\ruby{現代}{げんだい}Androidは、\ruby{無制限}{むせいげん}な\ruby{アクセス}{あくせす}\ruby{権限}{けんげん}を\ruby{縮小}{しゅくしょう}し、\ruby{既定}{きてい}の\ruby{防御}{ぼうぎょ}\ruby{機構}{きこう}を\ruby{強化}{きょうか}し、\ruby{基盤}{きばん}\ruby{レベル}{れべる}での\ruby{性能}{せいのう}\ruby{最適化}{さいてきか}を\ruby{進}{すす}めるという\ruby{明確}{めいかく}な\ruby{潮流}{ちょうりゅう}を\ruby{示}{しめ}している。その\ruby{結果}{けっか}、\ruby{開発}{かいはつ}は\ruby{厳格}{げんかく}になるが、\ruby{持続}{じぞく}\ruby{可能}{かのう}で\ruby{安全}{あんぜん}、かつ\ruby{大規模}{だいきぼ}に\ruby{適}{てき}した\ruby{エコシステム}{えこしすてむ}が\ruby{実現}{じつげん}される。IT\ruby{技術}{ぎじゅつ}\ruby{者}{しゃ}にとって、これらの\ruby{変化}{へんか}を\ruby{正確}{せいかく}に\ruby{理解}{りかい}することは、\ruby{現代}{げんだい}Android\ruby{アプリ}{あぷり}を\ruby{効率}{こうりつ}\ruby{的}{てき}に\ruby{構築}{こうちく}・\ruby{維持}{いじ}するための\ruby{必要}{ひつよう}\ruby{条件}{じょうけん}である。

\section{Modularization hệ thống: Project Treble và Mainline}
\ruby{システム}{しすてむ}\ruby{モジュール}{もじゅーる}\ruby{化}{か}:Project TrebleとMainline

Một trong những thay đổi mang tính kiến trúc quan trọng nhất của Android hiện đại là quá trình modularization hệ thống. Thay vì xem Android như một khối nguyên khép kín, Google chủ động tách hệ điều hành thành các thành phần độc lập, có thể phát triển và cập nhật riêng rẽ. Hai sáng kiến trung tâm cho hướng đi này là Project Treble và Project Mainline, trực tiếp giải quyết bài toán phân mảnh và chậm cập nhật vốn tồn tại nhiều năm trong hệ sinh thái Android.

\ruby{現代}{げんだい}Androidにおける\ruby{最}{もっと}も\ruby{重要}{じゅうよう}な\ruby{アーキテクチャ}{あーきてくちゃ}\ruby{変化}{へんか}の\ruby{一}{ひと}つが、\ruby{システム}{しすてむ}\ruby{モジュール}{もじゅーる}\ruby{化}{か}である。Androidを\ruby{単}{たん}なる\ruby{一体}{いったい}\ruby{型}{がた}の\ruby{塊}{かたまり}として\ruby{扱}{あつか}うのではなく、Googleは\ruby{オペレーティング}{おぺれーてぃんぐ}\ruby{システム}{しすてむ}を\ruby{独立}{どくりつ}した\ruby{構成}{こうせい}\ruby{要素}{ようそ}へと\ruby{分割}{ぶんかつ}した。これにより、\ruby{各}{かく}\ruby{要素}{ようそ}は\ruby{個別}{こべつ}に\ruby{開発}{かいはつ}・\ruby{更新}{こうしん}できる。中核となる\ruby{施策}{しさく}がProject TrebleとProject Mainlineであり、\ruby{長年}{ながねん}の\ruby{分断}{ぶんだん}と\ruby{更新}{こうしん}\ruby{遅延}{ちえん}という\ruby{課題}{かだい}に\ruby{直接}{ちょくせつ}\ruby{対処}{たいしょ}する。

Project Treble được giới thiệu từ Android 8.0 với mục tiêu tái cấu trúc ranh giới giữa Android framework và phần mềm phụ thuộc phần cứng. Trước Treble, các thành phần do nhà sản xuất chip và OEM cung cấp được gắn chặt vào framework Android, khiến mỗi lần nâng cấp phiên bản hệ điều hành đều đòi hỏi chỉnh sửa sâu ở nhiều lớp. Treble chuẩn hóa giao diện giữa framework và vendor layer thông qua các HAL (Hardware Abstraction Layer) ổn định, cho phép framework Android được nâng cấp mà không cần thay đổi phần cứng tương ứng.

Project TrebleはAndroid 8.0から\ruby{導入}{どうにゅう}され、Android\ruby{フレーム}{ふれーむ}\ruby{ワーク}{わーく}と\ruby{ハードウェア}{はーどうぇあ}\ruby{依存}{いぞん}\ruby{ソフトウェア}{そふとうぇあ}の\ruby{境界}{きょうかい}を\ruby{再構築}{さいこうちく}することを\ruby{目的}{もくてき}とした。Treble\ruby{以前}{いぜん}は、\ruby{チップ}{ちっぷ}\ruby{製造}{せいぞう}\ruby{業者}{ぎょうしゃ}やOEMが\ruby{提供}{ていきょう}する\ruby{部品}{ぶひん}がAndroid\ruby{フレーム}{ふれーむ}\ruby{ワーク}{わーく}に\ruby{密結合}{みつけつごう}しており、\ruby{版本}{ばんぽん}\ruby{更新}{こうしん}の\ruby{度}{たび}に\ruby{多層}{たそう}での\ruby{修正}{しゅうせい}が\ruby{必要}{ひつよう}だった。Trebleは\ruby{安定}{あんてい}したHAL(Hardware Abstraction Layer)を\ruby{通}{つう}じて\ruby{インターフェース}{いんたーふぇーす}を\ruby{標準}{ひょうじゅん}\ruby{化}{か}し、\ruby{ハードウェア}{はーどうぇあ}を\ruby{変更}{へんこう}せずに\ruby{フレーム}{ふれーむ}\ruby{ワーク}{わーく}を\ruby{更新}{こうしん}できるようにした。

Về mặt kỹ thuật, Treble chia hệ thống thành hai không gian rõ ràng: system partition và vendor partition. System partition chứa Android framework và các dịch vụ hệ thống do Google kiểm soát, trong khi vendor partition chứa driver, firmware và các thành phần đặc thù phần cứng. Sự tách biệt này giúp giảm đáng kể công sức tích hợp khi cập nhật Android mới, đặc biệt đối với các thiết bị đã ra mắt trên thị trường.

\ruby{技術}{ぎじゅつ}\ruby{的}{てき}にTrebleは\ruby{システム}{しすてむ}を\ruby{二}{に}つの\ruby{明確}{めいかく}な\ruby{領域}{りょういき}に\ruby{分}{わ}ける。system partitionにはAndroid\ruby{フレーム}{ふれーむ}\ruby{ワーク}{わーく}とGoogleが\ruby{管理}{かんり}する\ruby{システム}{しすてむ}\ruby{サービス}{さーびす}が\ruby{含}{ふく}まれ、vendor partitionには\ruby{ドライバ}{どらいば}、\ruby{ファーム}{ふぁーむ}\ruby{ウェア}{うぇあ}、\ruby{ハードウェア}{はーどうぇあ}\ruby{固有}{こゆう}の\ruby{要素}{ようそ}が\ruby{格納}{かくのう}される。この\ruby{分離}{ぶんり}により、\ruby{既存}{きそん}\ruby{端末}{たんまつ}でも\ruby{新}{あたら}しいAndroid\ruby{版本}{ばんぽん}への\ruby{統合}{とうごう}\ruby{負担}{ふたん}が\ruby{大幅}{おおはば}に\ruby{軽減}{けいげん}される。

Tuy nhiên, Treble không loại bỏ hoàn toàn vai trò của OEM. Các thành phần như kernel, bootloader và firmware vẫn do nhà sản xuất kiểm soát. Điều này có nghĩa là tốc độ cập nhật Android trên thực tế vẫn phụ thuộc vào cam kết hỗ trợ của từng OEM, nhưng rào cản kỹ thuật đã được hạ thấp đáng kể so với các thế hệ Android trước đó.

もっとも、TrebleはOEMの\ruby{役割}{やくわり}を\ruby{完全}{かんぜん}に\ruby{排除}{はいじょ}するものではない。kernel、bootloader、\ruby{ファーム}{ふぁーむ}\ruby{ウェア}{うぇあ}は\ruby{依然}{いぜん}として\ruby{製造}{せいぞう}\ruby{業者}{ぎょうしゃ}の\ruby{管理}{かんり}に\ruby{委}{ゆだ}ねられる。そのため、\ruby{実際}{じっさい}のAndroid\ruby{更新}{こうしん}\ruby{速度}{そくど}はOEMの\ruby{支援}{しえん}\ruby{姿勢}{しせい}に\ruby{左右}{さゆう}されるが、\ruby{技術}{ぎじゅつ}\ruby{的}{てき}な\ruby{障壁}{しょうへき}は\ruby{従来}{じゅうらい}より\ruby{大}{おお}きく\ruby{低下}{ていか}した。

Tiếp nối Treble, Project Mainline đẩy modularization đi xa hơn bằng cách tách một số thành phần hệ thống quan trọng thành các module độc lập có thể cập nhật trực tiếp thông qua Google Play. Các module này bao gồm thư viện media, networking, DNS resolver, timezone data và nhiều thành phần bảo mật cốt lõi. Thay vì chờ bản cập nhật OTA toàn hệ thống, các bản vá bảo mật và sửa lỗi có thể được phân phối nhanh chóng tới thiết bị người dùng.

Trebleに\ruby{続}{つづ}き、Project Mainlineは\ruby{モジュール}{もじゅーる}\ruby{化}{か}をさらに\ruby{推進}{すいしん}する。Mainlineでは、\ruby{重要}{じゅうよう}な\ruby{システム}{しすてむ}\ruby{部品}{ぶひん}を\ruby{独立}{どくりつ}した\ruby{モジュール}{もじゅーる}として\ruby{分離}{ぶんり}し、Google Playを\ruby{通}{つう}じて\ruby{直接}{ちょくせつ}\ruby{更新}{こうしん}できる。これには\ruby{メディア}{めでぃあ}\ruby{ライブラリ}{らいぶらり}、\ruby{ネットワーク}{ねっとわーく}、DNS\ruby{解決}{かいけつ}\ruby{機構}{きこう}、\ruby{タイム}{たいむ}\ruby{ゾーン}{ぞーん}\ruby{データ}{でーた}、および\ruby{中核}{ちゅうかく}\ruby{的}{てき}な\ruby{安全}{あんぜん}\ruby{部品}{ぶひん}が\ruby{含}{ふく}まれる。OTAを\ruby{待}{ま}たずに、\ruby{修正}{しゅうせい}や\ruby{パッチ}{ぱっち}を\ruby{迅速}{じんそく}に\ruby{配布}{はいふ}できる。

Ý nghĩa của Mainline nằm ở việc rút ngắn chu kỳ cập nhật và giảm phụ thuộc vào OEM trong các vấn đề bảo mật nghiêm trọng. Từ góc nhìn kỹ sư hệ thống, Android đang tiến gần hơn tới mô hình cập nhật liên tục, trong đó các thành phần quan trọng có thể được sửa lỗi độc lập mà không ảnh hưởng đến toàn bộ hệ điều hành. Điều này đặc biệt quan trọng trong bối cảnh Android vận hành trên hàng tỷ thiết bị với cấu hình phần cứng đa dạng.

Mainlineの\ruby{意義}{いぎ}は、\ruby{更新}{こうしん}\ruby{周期}{しゅうき}の\ruby{短縮}{たんしゅく}と、\ruby{深刻}{しんこく}な\ruby{安全}{あんぜん}\ruby{問題}{もんだい}におけるOEM\ruby{依存}{いぞん}の\ruby{低減}{ていげん}にある。\ruby{システム}{しすてむ}\ruby{技術}{ぎじゅつ}\ruby{者}{しゃ}の\ruby{視点}{してん}では、Androidは\ruby{継続}{けいぞく}\ruby{的}{てき}な\ruby{更新}{こうしん}\ruby{モデル}{もでる}へ\ruby{近}{ちか}づいており、\ruby{重要}{じゅうよう}な\ruby{部品}{ぶひん}を\ruby{全体}{ぜんたい}に\ruby{影響}{えいきょう}させず\ruby{修正}{しゅうせい}できる。これは、\ruby{多様}{たよう}な\ruby{構成}{こうせい}を\ruby{持}{も}つ\ruby{数十}{すうじゅう}\ruby{億}{おく}の\ruby{端末}{たんまつ}が\ruby{稼働}{かどう}する\ruby{現実}{げんじつ}において\ruby{極}{きわ}めて\ruby{重要}{じゅうよう}である。

Dù vậy, modularization cũng mang lại những thách thức mới. Việc chia nhỏ hệ thống đòi hỏi kiểm soát chặt chẽ về tương thích giữa các module và framework. Bên cạnh đó, các thiết bị giá rẻ hoặc đã cũ thường không được hỗ trợ đầy đủ các module Mainline, dẫn đến sự không đồng đều về khả năng cập nhật trong toàn hệ sinh thái.

もっとも、\ruby{モジュール}{もじゅーる}\ruby{化}{か}は\ruby{新}{あたら}しい\ruby{課題}{かだい}も\ruby{生}{しょう}む。\ruby{細分}{さいぶん}\ruby{化}{か}された\ruby{構成}{こうせい}では、\ruby{モジュール}{もじゅーる}と\ruby{フレーム}{ふれーむ}\ruby{ワーク}{わーく}\ruby{間}{かん}の\ruby{互換}{ごかん}\ruby{性}{せい}を\ruby{厳密}{げんみつ}に\ruby{管理}{かんり}する\ruby{必要}{ひつよう}がある。また、\ruby{低}{てい}\ruby{価格}{かかく}\ruby{帯}{たい}や\ruby{旧}{きゅう}\ruby{端末}{たんまつ}ではMainline\ruby{モジュール}{もじゅーる}が\ruby{十分}{じゅうぶん}に\ruby{提供}{ていきょう}されない\ruby{場合}{ばあい}があり、\ruby{更新}{こうしん}\ruby{能力}{のうりょく}の\ruby{不均一}{ふきんいつ}が\ruby{生}{しょう}じる。

Tổng kết lại, Project Treble và Mainline thể hiện rõ định hướng chiến lược của Android hiện đại: giảm phân mảnh, tăng khả năng cập nhật và kiểm soát tốt hơn các thành phần cốt lõi của hệ điều hành. Đối với kỹ sư công nghệ thông tin, việc hiểu rõ kiến trúc modular này là nền tảng quan trọng để đánh giá vòng đời thiết bị, chiến lược cập nhật và mức độ an toàn của các hệ thống Android trong thực tế triển khai.

\ruby{総括}{そうかつ}すると、Project TrebleとMainlineは、\ruby{分断}{ぶんだん}の\ruby{低減}{ていげん}、\ruby{更新}{こうしん}\ruby{性}{せい}の\ruby{向上}{こうじょう}、\ruby{中核}{ちゅうかく}\ruby{部品}{ぶひん}の\ruby{統制}{とうせい}という、\ruby{現代}{げんだい}Androidの\ruby{戦略}{せんりゃく}を\ruby{明確}{めいかく}に\ruby{示}{しめ}す。IT\ruby{技術}{ぎじゅつ}\ruby{者}{しゃ}にとって、この\ruby{モジュール}{もじゅーる}\ruby{化}{か}\ruby{構造}{こうぞう}の\ruby{理解}{りかい}は、\ruby{端末}{たんまつ}\ruby{寿命}{じゅみょう}、\ruby{更新}{こうしん}\ruby{戦略}{せんりゃく}、および\ruby{安全}{あんぜん}\ruby{性}{せい}を\ruby{評価}{ひょうか}するうえでの\ruby{基盤}{きばん}となる。

\section{Tích hợp AI và machine learning: cá nhân hóa, tối ưu tài nguyên và bảo mật}
\ruby{AI}{えーあい}と\ruby{機械}{きかい}\ruby{学習}{がくしゅう}の\ruby{統合}{とうごう}:\ruby{個人化}{こじんか}、\ruby{資源}{しげん}\ruby{最適化}{さいてきか}、および\ruby{セキュリティ}{せきゅりてぃ}

Trong Android hiện đại, trí tuệ nhân tạo và machine learning không còn được xem là các tính năng bổ sung ở tầng ứng dụng, mà đã trở thành một phần của kiến trúc hệ điều hành. Google định hướng AI như một lớp hạ tầng giúp Android thích nghi tốt hơn với người dùng, tối ưu việc sử dụng tài nguyên và nâng cao mức độ an toàn của toàn bộ hệ sinh thái.

\ruby{現代}{げんだい}のAndroidにおいて、\ruby{人工}{じんこう}\ruby{知能}{ちのう}や\ruby{機械}{きかい}\ruby{学習}{がくしゅう}は、もはや\ruby{アプリケーション}{あぷりけーしょん}\ruby{層}{そう}の\ruby{付加}{ふか}\ruby{機能}{きのう}ではなく、\ruby{OS}{おーえす}\ruby{アーキテクチャ}{あーきてくちゃ}の\ruby{一部}{いちぶ}となっている。Googleは、AIをAndroidが\ruby{利用者}{りようしゃ}に\ruby{適応}{てきおう}し、\ruby{資源}{しげん}の\ruby{利用}{りよう}を\ruby{最適化}{さいてきか}し、\ruby{生態系}{せいたいけい}\ruby{全体}{ぜんたい}の\ruby{安全性}{あんぜんせい}を\ruby{高}{たか}めるための\ruby{基盤}{きばん}\ruby{層}{そう}として\ruby{位置付}{いちづ}けている。

Một hướng tích hợp quan trọng là cá nhân hóa trải nghiệm người dùng dựa trên hành vi thực tế. Android sử dụng các mô hình machine learning để phân tích thói quen sử dụng ứng dụng, thời điểm hoạt động và ngữ cảnh sử dụng thiết bị. Từ đó, hệ thống đưa ra các gợi ý thông minh như ứng dụng được đề xuất, hành động nhanh phù hợp với tình huống, hoặc điều chỉnh giao diện và thiết lập hệ thống theo từng cá nhân. Cách tiếp cận này giúp giảm thao tác thủ công của người dùng và làm cho hệ điều hành trở nên “thích ứng” hơn theo thời gian.

\ruby{重要}{じゅうよう}な\ruby{統合}{とうごう}\ruby{領域}{りょういき}の\ruby{一}{ひと}つが、\ruby{実際}{じっさい}の\ruby{行動}{こうどう}に\ruby{基}{もと}づく\ruby{ユーザー}{ゆーざー}\ruby{体験}{たいけん}の\ruby{個人化}{こじんか}である。Androidは\ruby{機械}{きかい}\ruby{学習}{がくしゅう}\ruby{モデル}{もでる}を\ruby{用}{もち}いて、\ruby{アプリ}{あぷり}の\ruby{利用}{りよう}\ruby{習慣}{しゅうかん}、\ruby{利用}{りよう}\ruby{時間帯}{じかんたい}、および\ruby{端末}{たんまつ}\ruby{利用}{りよう}の\ruby{文脈}{ぶんみゃく}を\ruby{分析}{ぶんせき}する。これにより、\ruby{推奨}{すいしょう}\ruby{アプリ}{あぷり}、\ruby{状況}{じょうきょう}に\ruby{応}{おう}じた\ruby{即時}{そくじ}\ruby{操作}{そうさ}、あるいは\ruby{個々}{ここ}の\ruby{利用者}{りようしゃ}に\ruby{最適}{さいてき}な\ruby{UI}{ゆーあい}や\ruby{設定}{せってい}の\ruby{調整}{ちょうせい}が\ruby{行}{おこな}われる。この\ruby{手法}{しゅほう}は、\ruby{手動}{しゅどう}\ruby{操作}{そうさ}を\ruby{減}{へ}らし、OSを\ruby{時間}{じかん}とともに\ruby{適応的}{てきおうてき}な\ruby{存在}{そんざい}へと\ruby{進化}{しんか}させる。

Bên cạnh trải nghiệm, AI đóng vai trò trực tiếp trong việc tối ưu tài nguyên hệ thống. Các cơ chế như quản lý pin thích ứng, điều chỉnh độ sáng màn hình hay ưu tiên tiến trình đều dựa trên mô hình dự đoán hành vi người dùng. Thay vì phân bổ tài nguyên theo quy tắc tĩnh, Android sử dụng machine learning để dự đoán ứng dụng nào có khả năng được sử dụng tiếp theo, từ đó cấp phát CPU, bộ nhớ và quyền chạy nền một cách hợp lý. Với góc nhìn kỹ thuật, đây là sự chuyển dịch từ quản lý tài nguyên dựa trên luật cứng sang quản lý dựa trên dữ liệu và xác suất.

\ruby{体験}{たいけん}に\ruby{加}{くわ}え、AIは\ruby{システム}{しすてむ}\ruby{資源}{しげん}の\ruby{最適化}{さいてきか}にも\ruby{直接}{ちょくせつ}関与する。\ruby{適応}{てきおう}\ruby{型}{がた}の\ruby{バッテリー}{ばってりー}\ruby{管理}{かんり}、\ruby{画面}{がめん}\ruby{輝度}{きど}の\ruby{調整}{ちょうせい}、\ruby{プロセス}{ぷろせす}\ruby{優先}{ゆうせん}\ruby{制御}{せいぎょ}などは、\ruby{利用者}{りようしゃ}\ruby{行動}{こうどう}を\ruby{予測}{よそく}する\ruby{モデル}{もでる}に\ruby{基}{もと}づいている。\ruby{固定}{こてい}\ruby{的}{てき}な\ruby{規則}{きそく}による\ruby{資源}{しげん}\ruby{配分}{はいぶん}ではなく、\ruby{次}{つぎ}に\ruby{使}{つか}われる\ruby{可能性}{かのうせい}の\ruby{高}{たか}い\ruby{アプリ}{あぷり}を\ruby{予測}{よそく}し、CPUや\ruby{メモリ}{めもり}、\ruby{バックグラウンド}{ばっくぐらうんど}\ruby{実行}{じっこう}\ruby{権限}{けんげん}を\ruby{合理的}{ごうりてき}に\ruby{割}{わ}り\ruby{当}{あ}てる。これは\ruby{技術}{ぎじゅつ}\ruby{的}{てき}には、\ruby{規則}{きそく}\ruby{中心}{ちゅうしん}の\ruby{管理}{かんり}から、\ruby{データ}{でーた}と\ruby{確率}{かくりつ}に\ruby{基}{もと}づく\ruby{管理}{かんり}への\ruby{転換}{てんかん}を\ruby{意味}{いみ}する。

Một đặc điểm nổi bật của Android hiện đại là xu hướng xử lý AI trực tiếp trên thiết bị (on-device machine learning). Việc đưa mô hình ML xuống thiết bị giúp giảm độ trễ, tăng khả năng hoạt động ngoại tuyến và quan trọng hơn là hạn chế việc gửi dữ liệu nhạy cảm lên máy chủ. Cách tiếp cận này phản ánh rõ định hướng cân bằng giữa tiện ích AI và yêu cầu bảo vệ quyền riêng tư của người dùng, đặc biệt trong bối cảnh các quy định về dữ liệu ngày càng nghiêm ngặt.

\ruby{現代}{げんだい}Androidの\ruby{特徴}{とくちょう}として、\ruby{オン}{おん}\ruby{デバイス}{でばいす}\ruby{機械}{きかい}\ruby{学習}{がくしゅう}の\ruby{重視}{じゅうし}が\ruby{挙}{あ}げられる。ML\ruby{モデル}{もでる}を\ruby{端末}{たんまつ}\ruby{内}{ない}で\ruby{実行}{じっこう}することで、\ruby{遅延}{ちえん}を\ruby{低減}{ていげん}し、\ruby{オフライン}{おふらいん}\ruby{動作}{どうさ}を\ruby{可能}{かのう}にし、さらに\ruby{機微}{きび}な\ruby{データ}{でーた}を\ruby{サーバー}{さーばー}へ\ruby{送信}{そうしん}する\ruby{必要}{ひつよう}を\ruby{減}{へ}らす。この\ruby{方針}{ほうしん}は、AIの\ruby{利便性}{りべんせい}と\ruby{利用者}{りようしゃ}\ruby{プライバシー}{ぷらいばしー}\ruby{保護}{ほご}の\ruby{均衡}{きんこう}を\ruby{図}{はか}るというAndroidの\ruby{明確}{めいかく}な\ruby{方向性}{ほうこうせい}を\ruby{示}{しめ}している。

Trong lĩnh vực bảo mật, AI và machine learning được sử dụng để phát hiện các hành vi bất thường và mối đe dọa tiềm ẩn. Android có thể phân tích hành vi ứng dụng, mô hình truy cập tài nguyên và các dấu hiệu sử dụng bất thường để nhận diện phần mềm độc hại, gian lận hoặc hành vi xâm phạm quyền riêng tư. Thay vì chỉ dựa vào chữ ký tĩnh, hệ điều hành ngày càng phụ thuộc vào các mô hình học máy để phát hiện rủi ro theo thời gian thực.

\ruby{セキュリティ}{せきゅりてぃ}\ruby{分野}{ぶんや}では、AIと\ruby{機械}{きかい}\ruby{学習}{がくしゅう}が\ruby{異常}{いじょう}\ruby{行動}{こうどう}や\ruby{潜在的}{せんざいてき}\ruby{脅威}{きょうい}の\ruby{検知}{けんち}に\ruby{活用}{かつよう}されている。Androidは、\ruby{アプリ}{あぷり}の\ruby{挙動}{きょどう}、\ruby{資源}{しげん}\ruby{アクセス}{あくせす}の\ruby{パターン}{ぱたーん}、および\ruby{不自然}{ふしぜん}な\ruby{利用}{りよう}\ruby{兆候}{ちょうこう}を\ruby{分析}{ぶんせき}し、\ruby{マルウェア}{まるうぇあ}、\ruby{不正}{ふせい}、あるいは\ruby{プライバシー}{ぷらいばしー}\ruby{侵害}{しんがい}を\ruby{識別}{しきべつ}する。\ruby{静的}{せいてき}な\ruby{シグネチャ}{しぐねちゃ}に\ruby{依存}{いぞん}するのではなく、\ruby{リアル}{りある}\ruby{タイム}{たいむ}で\ruby{学習}{がくしゅう}\ruby{モデル}{もでる}に\ruby{基}{もと}づく\ruby{検出}{けんしゅつ}が\ruby{中核}{ちゅうかく}となりつつある。

Đối với lập trình viên, sự tích hợp sâu của AI trong Android đặt ra những yêu cầu mới. Ứng dụng cần tương thích với các cơ chế tối ưu và dự đoán của hệ thống, tránh các hành vi tiêu thụ tài nguyên không cần thiết hoặc cố tình vượt qua các giới hạn chạy nền. Đồng thời, developer có thể tận dụng các API và dịch vụ AI sẵn có của hệ điều hành để xây dựng tính năng thông minh mà không phải tự triển khai toàn bộ hạ tầng machine learning.

\ruby{開発者}{かいはつしゃ}にとって、AIの\ruby{深}{ふか}い\ruby{統合}{とうごう}は\ruby{新}{あたら}たな\ruby{要請}{ようせい}を\ruby{生}{しょう}む。\ruby{アプリ}{あぷり}は、\ruby{システム}{しすてむ}の\ruby{最適化}{さいてきか}や\ruby{予測}{よそく}\ruby{機構}{きこう}と\ruby{整合}{せいごう}し、\ruby{不要}{ふよう}な\ruby{資源}{しげん}\ruby{消費}{しょうひ}や\ruby{バックグラウンド}{ばっくぐらうんど}\ruby{制限}{せいげん}の\ruby{回避}{かいひ}を\ruby{避}{さ}ける\ruby{必要}{ひつよう}がある。その\ruby{一方}{いっぽう}で、\ruby{開発者}{かいはつしゃ}はAndroidが\ruby{提供}{ていきょう}する\ruby{AI}{えーあい}\ruby{API}{えーぴーあい}や\ruby{サービス}{さーびす}を\ruby{活用}{かつよう}し、\ruby{独自}{どくじ}に\ruby{機械}{きかい}\ruby{学習}{がくしゅう}\ruby{基盤}{きばん}を\ruby{構築}{こうちく}することなく、\ruby{高度}{こうど}な\ruby{機能}{きのう}を\ruby{実装}{じっそう}できる。

Tổng thể, AI và machine learning đang dần trở thành nền tảng vận hành của Android hiện đại. Chúng giúp hệ điều hành hoạt động hiệu quả hơn, cá nhân hóa sâu hơn và an toàn hơn, đồng thời thay đổi cách kỹ sư tiếp cận việc thiết kế và tối ưu ứng dụng. Hiểu rõ vai trò của AI trong Android không chỉ là hiểu công nghệ mới, mà là nắm được hướng phát triển cốt lõi của hệ điều hành trong dài hạn.

\ruby{総体的}{そうたいてき}に\ruby{見}{み}ると、AIと\ruby{機械}{きかい}\ruby{学習}{がくしゅう}は\ruby{現代}{げんだい}Androidの\ruby{運用}{うんよう}\ruby{基盤}{きばん}となりつつある。これらは、OSを\ruby{効率的}{こうりつてき}で、\ruby{高度}{こうど}に\ruby{個人化}{こじんか}され、より\ruby{安全}{あんぜん}な\ruby{存在}{そんざい}へと\ruby{変化}{へんか}させ、\ruby{エンジニア}{えんじにあ}の\ruby{設計}{せっけい}や\ruby{最適化}{さいてきか}の\ruby{発想}{はっそう}そのものを\ruby{変}{か}えている。AndroidにおけるAIの\ruby{役割}{やくわり}を\ruby{理解}{りかい}することは、\ruby{新技術}{しんぎじゅつ}を\ruby{知}{し}ることに\ruby{留}{とど}まらず、OSの\ruby{長期的}{ちょうきてき}\ruby{進化}{しんか}\ruby{方向}{ほうこう}を\ruby{把握}{はあく}することに\ruby{等}{ひと}しい。

\section{Xu hướng bảo mật nâng cao: kiểm soát dữ liệu, quyền truy cập và cập nhật nhanh}
\ruby{高度}{こうど}な\ruby{セキュリティ}{せきゅりてぃ}\ruby{動向}{どうこう}:\ruby{データ}{でーた}\ruby{制御}{せいぎょ}、\ruby{アクセス}{あくせす}\ruby{権限}{けんげん}、および\ruby{迅速}{じんそく}な\ruby{更新}{こうしん}

Bảo mật trong Android hiện đại không còn được tiếp cận như một lớp bổ sung bên ngoài, mà được tích hợp ngay từ kiến trúc nền tảng của hệ điều hành. Google định hướng xây dựng Android theo mô hình “secure by default”, trong đó các cơ chế bảo vệ được kích hoạt sẵn và người dùng chỉ cấp thêm quyền khi thật sự cần thiết. Xu hướng này phản ánh sự thay đổi căn bản trong cách Android cân bằng giữa tính mở và an toàn hệ thống.

\ruby{現代}{げんだい}のAndroidにおける\ruby{セキュリティ}{せきゅりてぃ}は、もはや\ruby{外部}{がいぶ}から\ruby{追加}{ついか}される\ruby{層}{そう}ではなく、\ruby{オペレーティングシステム}{おぺれーてぃんぐしすてむ}の\ruby{基盤}{きばん}\ruby{アーキテクチャ}{あーきてくちゃ}に\ruby{組}{く}み\ruby{込}{こ}まれている。Googleは「secure by default」という\ruby{方針}{ほうしん}のもと、\ruby{保護}{ほご}\ruby{機構}{きこう}を\ruby{初期}{しょき}\ruby{状態}{じょうたい}で\ruby{有効}{ゆうこう}にし、\ruby{必要}{ひつよう}な\ruby{場合}{ばあい}にのみ\ruby{利用者}{りようしゃ}が\ruby{権限}{けんげん}を\ruby{付与}{ふよ}する\ruby{設計}{せっけい}を\ruby{採用}{さいよう}している。この\ruby{動向}{どうこう}は、Androidが\ruby{開放性}{かいほうせい}と\ruby{システム}{しすてむ}\ruby{安全}{あんぜん}\ruby{性}{せい}の\ruby{均衡}{きんこう}を\ruby{再定義}{さいていぎ}していることを\ruby{示}{しめ}す。

Một trọng tâm quan trọng là kiểm soát dữ liệu người dùng. Android hiện đại giới hạn chặt chẽ khả năng truy cập dữ liệu nhạy cảm của ứng dụng, đặc biệt là dữ liệu cá nhân và dữ liệu hệ thống. Mỗi ứng dụng được cô lập trong một sandbox riêng, chỉ có thể truy cập tài nguyên được cấp phép rõ ràng. Các API truy cập dữ liệu ngày càng yêu cầu ngữ cảnh sử dụng cụ thể, giúp người dùng hiểu và kiểm soát tốt hơn việc dữ liệu của mình được sử dụng như thế nào.

\ruby{重要}{じゅうよう}な\ruby{焦点}{しょうてん}の一つは、\ruby{利用者}{りようしゃ}\ruby{データ}{でーた}の\ruby{制御}{せいぎょ}である。\ruby{現代}{げんだい}のAndroidは、\ruby{個人}{こじん}\ruby{情報}{じょうほう}や\ruby{システム}{しすてむ}\ruby{データ}{でーた}といった\ruby{機微}{きび}な\ruby{情報}{じょうほう}への\ruby{アクセス}{あくせす}を\ruby{厳格}{げんかく}に\ruby{制限}{せいげん}する。各\ruby{アプリケーション}{あぷりけーしょん}は\ruby{独立}{どくりつ}したsandboxに\ruby{隔離}{かくり}され、\ruby{明示}{めいじ}的に\ruby{許可}{きょか}された\ruby{資源}{しげん}のみに\ruby{アクセス}{あくせす}できる。\ruby{データ}{でーた}\ruby{取得}{しゅとく}APIは、\ruby{具体的}{ぐたいてき}な\ruby{利用}{りよう}\ruby{文脈}{ぶんみゃく}を\ruby{要求}{ようきゅう}するようになり、\ruby{利用者}{りようしゃ}が\ruby{自分}{じぶん}の\ruby{データ}{でーた}の\ruby{使}{つか}われ\ruby{方}{かた}を\ruby{理解}{りかい}しやすくなっている。

Quyền truy cập tài nguyên được quản lý theo hướng động và có thể thu hồi. Thay vì cấp quyền vĩnh viễn, Android cho phép người dùng cấp quyền tạm thời hoặc theo từng lần sử dụng. Hệ điều hành cũng tự động thu hồi quyền đối với các ứng dụng không được sử dụng trong thời gian dài, giảm nguy cơ lạm dụng dữ liệu ở trạng thái thụ động. Với góc nhìn kỹ sư, đây là một cơ chế quan trọng nhằm giảm bề mặt tấn công mà không cần phụ thuộc hoàn toàn vào ý thức người dùng.

\ruby{資源}{しげん}への\ruby{アクセス}{あくせす}\ruby{権限}{けんげん}は、\ruby{動的}{どうてき}かつ\ruby{回収}{かいしゅう}\ruby{可能}{かのう}な\ruby{方式}{ほうしき}で\ruby{管理}{かんり}される。\ruby{恒久的}{こうきゅうてき}な\ruby{権限}{けんげん}の\ruby{付与}{ふよ}に\ruby{代}{か}わり、Androidは\ruby{一時的}{いちじてき}または\ruby{使用}{しよう}\ruby{時}{じ}ごとの\ruby{許可}{きょか}を\ruby{認}{みと}める。また、\ruby{長期間}{ちょうきかん}\ruby{未使用}{みしよう}の\ruby{アプリケーション}{あぷりけーしょん}に\ruby{対}{たい}しては、\ruby{権限}{けんげん}を\ruby{自動}{じどう}で\ruby{回収}{かいしゅう}し、\ruby{受動的}{じゅどうてき}な\ruby{状態}{じょうたい}での\ruby{データ}{でーた}\ruby{乱用}{らんよう}リスクを\ruby{低減}{ていげん}する。\ruby{技術者}{ぎじゅつしゃ}の\ruby{視点}{してん}では、これは\ruby{利用者}{りようしゃ}の\ruby{意識}{いしき}に\ruby{全面的}{ぜんめんてき}に\ruby{依存}{いぞん}せず、\ruby{攻撃}{こうげき}\ruby{面}{めん}を\ruby{縮小}{しゅくしょう}する\ruby{重要}{じゅうよう}な\ruby{仕組}{しく}みである。

Ở cấp độ hệ thống, Android áp dụng nhiều lớp bảo vệ nhằm ngăn chặn việc can thiệp trái phép. Cơ chế Verified Boot đảm bảo tính toàn vẹn của hệ điều hành ngay từ quá trình khởi động, trong khi rollback protection ngăn chặn việc quay lại các phiên bản hệ thống kém an toàn hơn. SELinux được triển khai ở chế độ enforced trên toàn hệ thống, hạn chế nghiêm ngặt quyền hạn của tiến trình ngay cả khi chúng đã được cấp quyền cao hơn.

\ruby{システム}{しすてむ}\ruby{レベル}{れべる}では、Androidは\ruby{不正}{ふせい}な\ruby{介入}{かいにゅう}を\ruby{防}{ふせ}ぐため、\ruby{多層}{たそう}の\ruby{防御}{ぼうぎょ}を\ruby{適用}{てきよう}する。Verified Bootは\ruby{起動}{きどう}\ruby{段階}{だんかい}から\ruby{オペレーティングシステム}{おぺれーてぃんぐしすてむ}の\ruby{完全性}{かんぜんせい}を\ruby{保証}{ほしょう}し、rollback protectionは\ruby{安全性}{あんぜんせい}の\ruby{低}{ひく}い\ruby{過去}{かこ}の\ruby{バージョン}{ばーじょん}への\ruby{巻}{ま}き\ruby{戻}{もど}しを\ruby{防止}{ぼうし}する。SELinuxは\ruby{全体}{ぜんたい}\ruby{システム}{しすてむ}でenforced\ruby{モード}{もーど}として\ruby{動作}{どうさ}し、\ruby{高権限}{こうけんげん}が\ruby{付与}{ふよ}された\ruby{後}{のち}でも\ruby{プロセス}{ぷろせす}の\ruby{権限}{けんげん}を\ruby{厳}{きび}しく\ruby{制限}{せいげん}する。

Một xu hướng nổi bật khác là rút ngắn thời gian cập nhật bảo mật. Thông qua cơ chế modularization và các module hệ thống có thể cập nhật độc lập, Android cho phép vá lỗ hổng bảo mật quan trọng mà không cần chờ bản cập nhật toàn bộ hệ điều hành. Điều này giúp giảm đáng kể khoảng thời gian thiết bị tồn tại với các lỗ hổng đã được công bố, vốn là vấn đề nghiêm trọng trong các thế hệ Android trước.

もう一つの\ruby{顕著}{けんちょ}な\ruby{動向}{どうこう}は、\ruby{セキュリティ}{せきゅりてぃ}\ruby{更新}{こうしん}までの\ruby{時間}{じかん}を\ruby{短縮}{たんしゅく}することである。\ruby{モジュール}{もじゅーる}\ruby{化}{か}された\ruby{設計}{せっけい}と\ruby{独立}{どくりつ}して\ruby{更新}{こうしん}\ruby{可能}{かのう}な\ruby{システム}{しすてむ}\ruby{要素}{ようそ}により、Androidは\ruby{全面}{ぜんめん}の\ruby{OS}{おーえす}\ruby{更新}{こうしん}を\ruby{待}{ま}たずに\ruby{重要}{じゅうよう}な\ruby{脆弱性}{ぜいじゃくせい}を\ruby{修正}{しゅうせい}できる。これは、\ruby{公表}{こうひょう}された\ruby{脆弱性}{ぜいじゃくせい}を\ruby{抱}{かか}えたまま\ruby{端末}{たんまつ}が\ruby{稼働}{かどう}する\ruby{期間}{きかん}を\ruby{大幅}{おおはば}に\ruby{短}{みじか}くする。

Tuy nhiên, xu hướng bảo mật nâng cao cũng tạo ra áp lực không nhỏ cho lập trình viên và nhà sản xuất. Các ứng dụng cũ hoặc không tuân thủ chuẩn bảo mật mới dễ gặp lỗi, bị hạn chế chức năng hoặc bị loại khỏi hệ sinh thái. Developer buộc phải cập nhật cách tiếp cận, tuân thủ chặt chẽ các hướng dẫn về quyền truy cập và bảo vệ dữ liệu, thay vì tận dụng các lối đi tắt như trước đây.

しかし、\ruby{高度}{こうど}な\ruby{セキュリティ}{せきゅりてぃ}への\ruby{移行}{いこう}は、\ruby{開発者}{かいはつしゃ}や\ruby{製造業者}{せいぞうぎょうしゃ}にとって\ruby{小}{ちい}さくない\ruby{負担}{ふたん}ともなる。\ruby{旧}{きゅう}来の\ruby{アプリケーション}{あぷりけーしょん}や\ruby{新}{あたら}しい\ruby{安全}{あんぜん}\ruby{基準}{きじゅん}に\ruby{準拠}{じゅんきょ}しない\ruby{実装}{じっそう}は、\ruby{不具合}{ふぐあい}や\ruby{機能}{きのう}\ruby{制限}{せいげん}、さらには\ruby{エコシステム}{えこしすてむ}からの\ruby{排除}{はいじょ}につながりやすい。開発者は、\ruby{近道}{ちかみち}に\ruby{頼}{たよ}るのではなく、\ruby{権限}{けんげん}\ruby{管理}{かんり}と\ruby{データ}{でーた}\ruby{保護}{ほご}の\ruby{指針}{ししん}を\ruby{厳守}{げんしゅ}する\ruby{姿勢}{しせい}へと\ruby{転換}{てんかん}を\ruby{迫}{せま}られる。

Tổng kết lại, Android hiện đại coi bảo mật là yếu tố nền tảng, gắn chặt với kiến trúc hệ điều hành và vòng đời cập nhật. Việc kiểm soát dữ liệu, quản lý quyền truy cập chặt chẽ và rút ngắn thời gian vá lỗi giúp nâng cao đáng kể mức độ an toàn cho người dùng và doanh nghiệp. Với kỹ sư công nghệ thông tin, hiểu rõ các cơ chế bảo mật này là điều kiện tiên quyết để đánh giá rủi ro, thiết kế hệ thống và phát triển ứng dụng Android một cách bền vững.

\ruby{総括}{そうかつ}すると、\ruby{現代}{げんだい}のAndroidは\ruby{セキュリティ}{せきゅりてぃ}を\ruby{基盤}{きばん}\ruby{要素}{ようそ}として\ruby{位置}{いち}づけ、\ruby{アーキテクチャ}{あーきてくちゃ}と\ruby{更新}{こうしん}\ruby{ライフサイクル}{らいふさいくる}に\ruby{密接}{みっせつ}に\ruby{結}{むす}びつけている。\ruby{データ}{でーた}\ruby{制御}{せいぎょ}、\ruby{厳格}{げんかく}な\ruby{権限}{けんげん}\ruby{管理}{かんり}、および\ruby{迅速}{じんそく}な\ruby{修正}{しゅうせい}は、\ruby{利用者}{りようしゃ}と\ruby{企業}{きぎょう}の\ruby{安全}{あんぜん}\ruby{性}{せい}を\ruby{大幅}{おおはば}に\ruby{高}{たか}める。\ruby{情報}{じょうほう}\ruby{技術}{ぎじゅつ}の\ruby{技術者}{ぎじゅつしゃ}にとって、これらの\ruby{仕組}{しく}みを\ruby{正}{ただ}しく\ruby{理解}{りかい}することは、\ruby{リスク}{りすく}\ruby{評価}{ひょうか}、\ruby{システム}{しすてむ}\ruby{設計}{せっけい}、およびAndroid\ruby{アプリケーション}{あぷりけーしょん}の\ruby{持続}{じぞく}\ruby{的}{てき}な\ruby{開発}{かいはつ}における\ruby{前提}{ぜんてい}である。

\section{Dự đoán tương lai Android: vai trò trong hệ sinh thái số và cạnh tranh nền tảng}
\ruby{将来}{しょうらい}のAndroid\ruby{予測}{よそく}:\ruby{デジタル}{でじたる}\ruby{生態系}{せいたいけい}における\ruby{役割}{やくわり}と\ruby{基盤}{きばん}\ruby{競争}{きょうそう}

Trong bối cảnh công nghệ số phát triển nhanh và đa nền tảng, Android không còn chỉ được định vị là hệ điều hành dành cho điện thoại thông minh. Xu hướng hiện nay cho thấy Android đang dần trở thành một nền tảng cốt lõi trong hệ sinh thái số rộng lớn, bao phủ nhiều loại thiết bị và kịch bản sử dụng khác nhau. Dưới góc nhìn kỹ sư công nghệ thông tin, tương lai của Android được định hình bởi kiến trúc mở rộng, khả năng tích hợp sâu và sự cạnh tranh ở cấp độ hệ sinh thái.

\ruby{急速}{きゅうそく}に\ruby{発展}{はってん}する\ruby{デジタル}{でじたる}\ruby{技術}{ぎじゅつ}と\ruby{多}{た}\ruby{基盤}{きばん}\ruby{環境}{かんきょう}の\ruby{中}{なか}で、Androidはもはや\ruby{スマートフォン}{すまーとふぉん}\ruby{向}{む}けの\ruby{オペレーティングシステム}{おぺれーてぃんぐしすてむ}としてのみ\ruby{位置}{いち}づけられてはいない。\ruby{現在}{げんざい}の\ruby{潮流}{ちょうりゅう}は、Androidが\ruby{多様}{たよう}な\ruby{機器}{きき}や\ruby{利用}{りよう}\ruby{シナリオ}{しなりお}を\ruby{包含}{ほうがん}する\ruby{広範}{こうはん}な\ruby{デジタル}{でじたる}\ruby{生態系}{せいたいけい}の\ruby{中核}{ちゅうかく}\ruby{基盤}{きばん}へと\ruby{進化}{しんか}していることを\ruby{示}{しめ}している。\ruby{情報}{じょうほう}\ruby{技術}{ぎじゅつ}\ruby{技術者}{ぎじゅつしゃ}の\ruby{視点}{してん}から\ruby{見}{み}れば、Androidの\ruby{将来}{しょうらい}は\ruby{拡張}{かくちょう}し\ruby{続}{つづ}ける\ruby{アーキテクチャ}{あーきてくちゃ}、\ruby{深}{ふか}い\ruby{統合}{とうごう}\ruby{能力}{のうりょく}、および\ruby{生態系}{せいたいけい}\ruby{レベル}{れべる}での\ruby{競争}{きょうそう}によって\ruby{形作}{かたちづく}られる。

Một xu hướng rõ ràng là Android tiếp tục mở rộng phạm vi ứng dụng vượt ra ngoài smartphone. Hệ điều hành này đã và đang hiện diện trên thiết bị đeo, TV, hệ thống giải trí trên ô tô và các nền tảng nhúng. Việc tái sử dụng lõi hệ điều hành và các thành phần chung giúp giảm chi phí phát triển, đồng thời tạo ra trải nghiệm thống nhất cho người dùng trên nhiều loại thiết bị. Trong tương lai, Android có khả năng đóng vai trò như một lớp nền tảng chung cho các hệ thống thông minh, đặc biệt trong bối cảnh Internet of Things và thiết bị kết nối ngày càng phổ biến.

\ruby{明確}{めいかく}な\ruby{傾向}{けいこう}の\ruby{一}{ひと}つは、Androidが\ruby{スマートフォン}{すまーとふぉん}を\ruby{超}{こ}えて\ruby{適用}{てきよう}\ruby{範囲}{はんい}を\ruby{拡大}{かくだい}し\ruby{続}{つづ}けていることである。この\ruby{オペレーティングシステム}{おぺれーてぃんぐしすてむ}は、\ruby{ウェアラブル}{うぇあらぶる}\ruby{端末}{たんまつ}、\ruby{テレビ}{てれび}、\ruby{車載}{しゃさい}\ruby{エンターテインメント}{えんたーていんめんと}\ruby{システム}{しすてむ}、および\ruby{組込}{くみこみ}\ruby{基盤}{きばん}にも\ruby{導入}{どうにゅう}されてきた。\ruby{共通}{きょうつう}の\ruby{カーネル}{かーねる}や\ruby{構成}{こうせい}\ruby{要素}{ようそ}を\ruby{再利用}{さいりよう}することで、\ruby{開発}{かいはつ}\ruby{コスト}{こすと}を\ruby{削減}{さくげん}しつつ、\ruby{複数}{ふくすう}の\ruby{機器}{きき}において\ruby{統一}{とういつ}された\ruby{利用者}{りようしゃ}\ruby{体験}{たいけん}を\ruby{提供}{ていきょう}できる。\ruby{将来}{しょうらい}においてAndroidは、\ruby{特}{とく}にIoTや\ruby{接続}{せつぞく}\ruby{機器}{きき}が\ruby{普及}{ふきゅう}する\ruby{状況}{じょうきょう}の\ruby{中}{なか}で、\ruby{知的}{ちてき}\ruby{システム}{しすてむ}の\ruby{共通}{きょうつう}\ruby{基盤}{きばん}としての\ruby{役割}{やくわり}を\ruby{担}{にな}う\ruby{可能性}{かのうせい}が\ruby{高}{たか}い。

Song song với mở rộng phạm vi, Android tiếp tục củng cố vị thế trong hệ sinh thái dịch vụ số. Hệ điều hành ngày càng gắn chặt với các dịch vụ nền tảng như lưu trữ, đồng bộ dữ liệu, trí tuệ nhân tạo và bảo mật. Thay vì cạnh tranh đơn thuần ở mức hệ điều hành, Android tham gia vào cuộc cạnh tranh toàn diện giữa các hệ sinh thái, nơi trải nghiệm người dùng được quyết định bởi sự kết hợp giữa phần mềm, dịch vụ và phần cứng.

\ruby{適用}{てきよう}\ruby{範囲}{はんい}の\ruby{拡大}{かくだい}と\ruby{並行}{へいこう}して、Androidは\ruby{デジタル}{でじたる}\ruby{サービス}{さーびす}\ruby{生態系}{せいたいけい}における\ruby{地位}{ちい}を\ruby{一層}{いっそう}\ruby{強化}{きょうか}している。\ruby{保存}{ほぞん}、\ruby{データ}{でーた}\ruby{同期}{どうき}、\ruby{人工}{じんこう}\ruby{知能}{ちのう}、および\ruby{セキュリティ}{せきゅりてぃ}といった\ruby{基盤}{きばん}\ruby{サービス}{さーびす}との\ruby{結合}{けつごう}は\ruby{年々}{ねんねん}\ruby{深}{ふか}まっている。Androidは\ruby{単純}{たんじゅん}に\ruby{オペレーティングシステム}{おぺれーてぃんぐしすてむ}として\ruby{競争}{きょうそう}するのではなく、\ruby{ソフトウェア}{そふとうぇあ}、\ruby{サービス}{さーびす}、\ruby{ハードウェア}{はーどうぇあ}の\ruby{組合}{くみあ}せによって\ruby{利用者}{りようしゃ}\ruby{体験}{たいけん}が\ruby{決定}{けってい}される\ruby{生態系}{せいたいけい}\ruby{間}{かん}の\ruby{競争}{きょうそう}に\ruby{参画}{さんかく}している。

Về mặt kiến trúc, xu hướng modularization và cập nhật độc lập nhiều khả năng sẽ được mở rộng hơn nữa. Android trong tương lai có thể tiếp tục tách thêm các thành phần hệ thống để tăng tốc độ vá lỗi và giảm phụ thuộc vào nhà sản xuất. Điều này giúp Android tiến gần hơn tới mô hình vận hành linh hoạt, phù hợp với các yêu cầu bảo mật và ổn định của hệ thống quy mô lớn, bao gồm cả môi trường doanh nghiệp và hạ tầng số công cộng.

\ruby{アーキテクチャ}{あーきてくちゃ}の\ruby{観点}{かんてん}では、\ruby{モジュール}{もじゅーる}\ruby{化}{か}と\ruby{独立}{どくりつ}した\ruby{更新}{こうしん}の\ruby{流}{なが}れが、さらに\ruby{拡張}{かくちょう}される\ruby{可能性}{かのうせい}が\ruby{高}{たか}い。\ruby{将来}{しょうらい}のAndroidは、\ruby{システム}{しすてむ}\ruby{構成}{こうせい}\ruby{要素}{ようそ}を\ruby{一層}{いっそう}\ruby{分離}{ぶんり}し、\ruby{修正}{しゅうせい}の\ruby{迅速}{じんそく}な\ruby{適用}{てきよう}と\ruby{製造者}{せいぞうしゃ}への\ruby{依存}{いぞん}の\ruby{低減}{ていげん}を\ruby{図}{はか}るだろう。これは、\ruby{企業}{きぎょう}\ruby{環境}{かんきょう}や\ruby{公共}{こうきょう}\ruby{デジタル}{でじたる}\ruby{基盤}{きばん}を\ruby{含}{ふく}む\ruby{大規模}{だいきぼ}\ruby{システム}{しすてむ}に\ruby{求}{もと}められる\ruby{安全性}{あんぜんせい}と\ruby{安定性}{あんていせい}に\ruby{適合}{てきごう}した、\ruby{柔軟}{じゅうなん}な\ruby{運用}{うんよう}\ruby{モデル}{もでる}への\ruby{接近}{せっきん}を\ruby{意味}{いみ}する。

Trong bối cảnh cạnh tranh nền tảng, Android đối mặt trực tiếp với các hệ điều hành kiểm soát chặt chẽ hơn nhưng có mức độ đồng bộ cao. Lợi thế lớn nhất của Android vẫn là khả năng mở rộng, tùy biến và triển khai trên nhiều phân khúc phần cứng khác nhau. Tuy nhiên, để duy trì lợi thế này, Android buộc phải tiếp tục siết chặt tiêu chuẩn bảo mật, chất lượng ứng dụng và tính nhất quán của trải nghiệm người dùng.

\ruby{基盤}{きばん}\ruby{競争}{きょうそう}の\ruby{文脈}{ぶんみゃく}において、Androidは\ruby{厳格}{げんかく}な\ruby{制御}{せいぎょ}と\ruby{高}{たか}い\ruby{統合}{とうごう}\ruby{度}{ど}を\ruby{持}{も}つ\ruby{他}{ほか}の\ruby{オペレーティングシステム}{おぺれーてぃんぐしすてむ}と\ruby{正面}{しょうめん}から\ruby{競合}{きょうごう}している。Androidの\ruby{最大}{さいだい}の\ruby{強}{つよ}みは、\ruby{拡張性}{かくちょうせい}、\ruby{柔軟}{じゅうなん}な\ruby{カスタマイズ}{かすたまいず}、および\ruby{多様}{たよう}な\ruby{ハードウェア}{はーどうぇあ}\ruby{分野}{ぶんや}への\ruby{展開}{てんかい}にある。しかし、この\ruby{優位性}{ゆういせい}を\ruby{維持}{いじ}するためには、\ruby{セキュリティ}{せきゅりてぃ}\ruby{基準}{きじゅん}、\ruby{アプリケーション}{あぷりけーしょん}\ruby{品質}{ひんしつ}、および\ruby{利用者}{りようしゃ}\ruby{体験}{たいけん}の\ruby{一貫性}{いっかんせい}を\ruby{一層}{いっそう}\ruby{強化}{きょうか}し\ruby{続}{つづ}ける\ruby{必要}{ひつよう}がある。

Đối với lập trình viên và kỹ sư hệ thống, tương lai Android đòi hỏi sự thích nghi liên tục. Việc phát triển ứng dụng không còn chỉ xoay quanh API và giao diện, mà cần hiểu sâu kiến trúc hệ điều hành, chiến lược cập nhật và các ràng buộc về bảo mật, quyền riêng tư. Những kỹ sư có khả năng nắm bắt xu hướng nền tảng và thiết kế giải pháp phù hợp với hệ sinh thái Android sẽ có lợi thế rõ rệt trong dài hạn.

\ruby{開発者}{かいはつしゃ}および\ruby{システム}{しすてむ}\ruby{技術者}{ぎじゅつしゃ}にとって、Androidの\ruby{将来}{しょうらい}は\ruby{継続的}{けいぞくてき}な\ruby{適応}{てきおう}を\ruby{要求}{ようきゅう}する。\ruby{アプリケーション}{あぷりけーしょん}\ruby{開発}{かいはつ}は、もはやAPIや\ruby{インターフェース}{いんたーふぇーす}の\ruby{理解}{りかい}だけに\ruby{留}{とど}まらず、\ruby{オペレーティングシステム}{おぺれーてぃんぐしすてむ}の\ruby{構造}{こうぞう}、\ruby{更新}{こうしん}\ruby{戦略}{せんりゃく}、および\ruby{セキュリティ}{せきゅりてぃ}・\ruby{プライバシー}{ぷらいばしー}に\ruby{関}{かん}する\ruby{制約}{せいやく}を\ruby{深}{ふか}く\ruby{理解}{りかい}することが\ruby{求}{もと}められる。\ruby{基盤}{きばん}\ruby{動向}{どうこう}を\ruby{的確}{てきかく}に\ruby{把握}{はあく}し、Android\ruby{生態系}{せいたいけい}に\ruby{適合}{てきごう}した\ruby{解決策}{かいけつさく}を\ruby{設計}{せっけい}できる\ruby{技術者}{ぎじゅつしゃ}は、\ruby{長期的}{ちょうきてき}に\ruby{明確}{めいかく}な\ruby{優位}{ゆうい}を\ruby{持}{も}つ。

Tóm lại, Android trong tương lai không chỉ là một hệ điều hành di động, mà là một nền tảng hạ tầng số linh hoạt, đóng vai trò trung tâm trong nhiều lĩnh vực công nghệ. Sự cạnh tranh sẽ diễn ra ở cấp độ hệ sinh thái, nơi khả năng mở rộng, bảo mật và tích hợp sâu quyết định vị thế của Android trong kỷ nguyên số.

\ruby{総括}{そうかつ}すると、\ruby{将来}{しょうらい}のAndroidは\ruby{移動}{いどう}\ruby{オペレーティングシステム}{おぺれーてぃんぐしすてむ}に\ruby{留}{とど}まらず、\ruby{柔軟}{じゅうなん}な\ruby{デジタル}{でじたる}\ruby{基盤}{きばん}として\ruby{多}{おお}くの\ruby{技術}{ぎじゅつ}\ruby{分野}{ぶんや}で\ruby{中心的}{ちゅうしんてき}な\ruby{役割}{やくわり}を\ruby{果}{は}たす。\ruby{競争}{きょうそう}は\ruby{生態系}{せいたいけい}\ruby{レベル}{れべる}で\ruby{展開}{てんかい}され、\ruby{拡張性}{かくちょうせい}、\ruby{安全性}{あんぜんせい}、および\ruby{深}{ふか}い\ruby{統合}{とうごう}\ruby{能力}{のうりょく}が、\ruby{デジタル}{でじたる}\ruby{時代}{じだい}におけるAndroidの\ruby{位置}{いち}を\ruby{決定}{けってい}する。
