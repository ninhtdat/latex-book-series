\chapter{Lãnh đạo trong thay đổi và đổi mới}

Trong môi trường kinh doanh hiện đại, thay đổi và đổi mới không còn là các sáng kiến mang tính thời điểm mà đã trở thành yêu cầu thường trực đối với mọi tổ chức. Sự phát triển nhanh của công nghệ, áp lực cạnh tranh toàn cầu và kỳ vọng ngày càng cao của khách hàng buộc tổ chức phải liên tục điều chỉnh để tồn tại và phát triển. Trong bối cảnh đó, vai trò của lãnh đạo không chỉ dừng lại ở việc quản lý hoạt động hiện tại, mà còn phải dẫn dắt tổ chức thích ứng với thay đổi và xây dựng năng lực đổi mới bền vững.

\section{Bản chất của thay đổi trong tổ chức hiện đại}

Thay đổi trong tổ chức hiện đại mang những đặc điểm khác biệt rõ rệt so với cách hiểu truyền thống. Trước hết, thay đổi không còn là một sự kiện đơn lẻ có điểm bắt đầu và kết thúc rõ ràng, mà là một quá trình diễn ra liên tục. Tổ chức hiếm khi quay trở lại trạng thái ổn định kéo dài; thay vào đó, họ phải vận hành trong điều kiện biến động thường xuyên. Điều này đòi hỏi lãnh đạo phải chuyển từ tư duy “quản lý trong ổn định” sang tư duy “lãnh đạo trong bất định”.

Một đặc điểm quan trọng khác của thay đổi hiện đại là tính phức hợp. Các yếu tố công nghệ, con người, quy trình và văn hóa tổ chức liên kết chặt chẽ với nhau. Khi một yếu tố thay đổi, các yếu tố còn lại cũng bị tác động theo. Ví dụ, việc áp dụng một hệ thống công nghệ mới không chỉ là vấn đề kỹ thuật, mà còn kéo theo thay đổi trong cách phối hợp công việc, yêu cầu kỹ năng mới và điều chỉnh cách đánh giá hiệu quả. Nếu lãnh đạo tiếp cận thay đổi theo hướng đơn tuyến, tách rời từng phần, tổ chức sẽ gặp khó khăn trong triển khai và không đạt được giá trị mong muốn.

Thay đổi cũng ngày càng diễn ra trong bối cảnh thiếu chắc chắn. Thông tin không đầy đủ, thị trường biến động nhanh và các yếu tố rủi ro khó dự đoán khiến lãnh đạo không thể chờ đến khi mọi điều kiện trở nên rõ ràng mới hành động. Năng lực ra quyết định trong điều kiện không chắc chắn trở thành một yêu cầu cốt lõi. Lãnh đạo cần chấp nhận rằng sai sót là một phần của quá trình học hỏi, miễn là tổ chức có cơ chế điều chỉnh kịp thời.

Bên cạnh đó, thay đổi trong tổ chức hiện đại gắn chặt với yếu tố con người. Dù công nghệ hay chiến lược có được thiết kế tốt đến đâu, kết quả cuối cùng vẫn phụ thuộc vào việc con người có chấp nhận và thực thi thay đổi hay không. Thay đổi không chỉ ảnh hưởng đến cách làm việc, mà còn tác động đến cảm xúc, niềm tin và cảm nhận về giá trị cá nhân của nhân viên. Do đó, thay đổi mang tính kỹ thuật luôn song hành với thay đổi mang tính tâm lý và văn hóa.

Một điểm cần nhấn mạnh là thay đổi không đồng nghĩa với cải tiến tích cực trong mọi trường hợp. Có những thay đổi cần thiết nhưng gây xáo trộn trong ngắn hạn, thậm chí làm giảm hiệu quả tạm thời. Lãnh đạo cần phân biệt rõ giữa thay đổi mang tính phản ứng nhất thời và thay đổi có định hướng chiến lược. Việc thay đổi liên tục nhưng thiếu định hướng rõ ràng sẽ làm tổ chức mất tập trung và suy giảm niềm tin nội bộ.

Cuối cùng, thay đổi trong tổ chức hiện đại đòi hỏi sự cân bằng giữa tốc độ và tính bền vững. Hành động quá chậm khiến tổ chức tụt hậu, nhưng thay đổi quá nhanh, thiếu chuẩn bị sẽ tạo ra áp lực lớn lên con người và hệ thống. Hiểu đúng bản chất của thay đổi giúp lãnh đạo lựa chọn nhịp độ phù hợp, phân bổ nguồn lực hợp lý và xây dựng nền tảng cho đổi mới lâu dài.

\section{Nguyên nhân và biểu hiện của sự kháng cự thay đổi}

Kháng cự thay đổi là hiện tượng phổ biến và gần như không thể tránh khỏi trong mọi tổ chức. Đây không phải là vấn đề cá nhân hay biểu hiện của sự bảo thủ đơn thuần, mà là phản ứng tự nhiên của con người khi phải đối mặt với những điều không chắc chắn. Việc hiểu đúng nguyên nhân và biểu hiện của kháng cự giúp lãnh đạo có cách tiếp cận thực tế, giảm thiểu xung đột và nâng cao khả năng thành công của quá trình chuyển đổi.

Một trong những nguyên nhân cốt lõi của kháng cự là nỗi sợ mất mát. Thay đổi thường kéo theo nguy cơ mất vị trí, quyền lực, thu nhập hoặc vai trò quen thuộc. Ngay cả khi lãnh đạo khẳng định thay đổi mang lại lợi ích chung, nhiều cá nhân vẫn tập trung vào câu hỏi: “Tôi sẽ mất gì?”. Khi câu hỏi này không được trả lời thỏa đáng, sự kháng cự sẽ hình thành, dù công khai hay ngầm định.

Nguyên nhân thứ hai là sự thiếu hiểu biết hoặc thiếu thông tin rõ ràng về thay đổi. Khi mục tiêu, lộ trình và tác động của thay đổi không được truyền đạt đầy đủ, nhân viên dễ rơi vào trạng thái hoài nghi. Khoảng trống thông tin thường bị lấp đầy bằng suy đoán và tin đồn, làm gia tăng cảm giác bất an. Trong nhiều trường hợp, sự kháng cự không xuất phát từ bản thân thay đổi, mà từ cách thay đổi được truyền thông và triển khai.

Trải nghiệm tiêu cực từ các lần thay đổi trước cũng là một yếu tố quan trọng. Nếu tổ chức từng triển khai các sáng kiến thay đổi thất bại, thiếu nhất quán hoặc gây tổn hại đến niềm tin, nhân viên sẽ hình thành tâm lý phòng thủ. Họ có xu hướng coi thay đổi mới chỉ là một “phong trào nhất thời” và lựa chọn chờ đợi thay vì chủ động tham gia. Điều này làm suy yếu đáng kể động lực chuyển đổi.

Bên cạnh đó, áp lực gia tăng nhưng nguồn lực không tương xứng cũng dẫn đến kháng cự. Khi thay đổi đồng nghĩa với khối lượng công việc lớn hơn, yêu cầu cao hơn nhưng không đi kèm hỗ trợ phù hợp, nhân viên sẽ cảm thấy quá tải. Sự mệt mỏi tích lũy khiến họ phản ứng tiêu cực, dù về lý thuyết họ hiểu sự cần thiết của thay đổi.

Về mặt biểu hiện, kháng cự thay đổi không phải lúc nào cũng thể hiện bằng sự phản đối trực diện. Một biểu hiện phổ biến là sự trì hoãn và tuân thủ hình thức. Nhân viên làm theo yêu cầu nhưng thiếu cam kết thực chất, dẫn đến hiệu quả thấp. Đây là dạng kháng cự thụ động, khó nhận diện nhưng có tác động lâu dài.

Một biểu hiện khác là sự hoài nghi và lan truyền thái độ tiêu cực. Những câu hỏi mang tính nghi ngờ, so sánh bất lợi với quá khứ hoặc nhấn mạnh rủi ro thường xuất hiện trong các cuộc trao đổi không chính thức. Nếu không được xử lý kịp thời, những quan điểm này có thể lan rộng và ảnh hưởng đến tinh thần chung của tập thể.

Ngoài ra, kháng cự còn thể hiện qua việc né tránh trách nhiệm hoặc đổ lỗi cho hoàn cảnh. Cá nhân có thể cho rằng thay đổi là không khả thi, không phù hợp với thực tế, từ đó biện minh cho việc không hành động. Điều này làm chậm tiến độ chuyển đổi và gây căng thẳng trong nội bộ.

Đối với lãnh đạo, kháng cự thay đổi không nên được xem là vấn đề cần loại bỏ, mà là tín hiệu cần được lắng nghe. Mỗi biểu hiện kháng cự đều phản ánh một mối lo ngại cụ thể. Khi lãnh đạo tập trung vào việc hiểu nguyên nhân gốc rễ thay vì áp đặt mệnh lệnh, kháng cự có thể được chuyển hóa thành sự tham gia và cam kết.

\section{Vai trò của lãnh đạo trong dẫn dắt chuyển đổi}

Trong mọi nỗ lực thay đổi tổ chức, lãnh đạo giữ vai trò quyết định. Chiến lược có thể đúng, nguồn lực có thể đủ, nhưng nếu thiếu vai trò dẫn dắt hiệu quả của lãnh đạo, quá trình chuyển đổi rất dễ rơi vào trạng thái hình thức hoặc thất bại. Lãnh đạo trong chuyển đổi không đơn thuần là người ra quyết định, mà là người tạo định hướng, xây dựng niềm tin và huy động con người cùng hành động.

Trước hết, vai trò quan trọng nhất của lãnh đạo là xác lập định hướng rõ ràng cho thay đổi. Điều này bao gồm việc làm rõ lý do phải thay đổi, mục tiêu cần đạt được và những hệ quả nếu tổ chức không hành động. Khi định hướng không rõ ràng, thay đổi sẽ bị hiểu sai hoặc bị coi là mệnh lệnh mang tính áp đặt. Lãnh đạo cần truyền tải thông điệp một cách nhất quán, dễ hiểu và gắn với bối cảnh thực tế của tổ chức, thay vì chỉ dựa vào các khẩu hiệu chung chung.

Tiếp theo, lãnh đạo đóng vai trò trung tâm trong việc xây dựng và duy trì niềm tin. Thay đổi luôn đi kèm rủi ro và sự bất an, vì vậy con người thường quan sát hành vi của lãnh đạo để đánh giá mức độ nghiêm túc và đáng tin cậy của quá trình chuyển đổi. Sự nhất quán giữa lời nói và hành động của lãnh đạo có ý nghĩa quyết định. Khi lãnh đạo sẵn sàng chịu trách nhiệm, thừa nhận khó khăn và minh bạch trong giao tiếp, niềm tin sẽ được củng cố, ngay cả khi kết quả chưa đến ngay lập tức.

Một vai trò không thể thiếu khác là huy động và gắn kết con người. Thay đổi không thể được thực hiện chỉ bằng mệnh lệnh từ trên xuống. Lãnh đạo cần tạo điều kiện để nhân viên tham gia vào quá trình chuyển đổi, đặc biệt là ở những vấn đề ảnh hưởng trực tiếp đến công việc của họ. Việc lắng nghe ý kiến, phản hồi và mối quan ngại giúp lãnh đạo hiểu rõ thực tế triển khai, đồng thời làm tăng cảm giác được tôn trọng và sở hữu thay đổi trong đội ngũ.

Bên cạnh đó, lãnh đạo phải là hình mẫu trong việc thích ứng với thay đổi. Nhân viên sẽ khó chấp nhận thay đổi nếu lãnh đạo vẫn giữ cách làm cũ hoặc né tránh những điều chỉnh cần thiết. Việc chủ động học hỏi, thử nghiệm cách tiếp cận mới và sẵn sàng thay đổi chính mình gửi đi một thông điệp mạnh mẽ rằng chuyển đổi là nghiêm túc và không có ngoại lệ. Vai trò nêu gương này có tác động lớn hơn bất kỳ chỉ thị hành chính nào.

Lãnh đạo cũng cần quản trị nhịp độ và áp lực của chuyển đổi. Thay đổi quá nhanh có thể gây quá tải, trong khi thay đổi quá chậm làm mất động lực và niềm tin. Việc xác định các mốc chuyển đổi hợp lý, ưu tiên đúng trọng tâm và phân bổ nguồn lực phù hợp là trách nhiệm trực tiếp của lãnh đạo. Điều này đòi hỏi khả năng đánh giá thực tế tổ chức, thay vì chỉ bám vào kế hoạch ban đầu.

Cuối cùng, lãnh đạo đóng vai trò điều chỉnh và học hỏi liên tục trong suốt quá trình chuyển đổi. Không có kế hoạch thay đổi nào hoàn hảo ngay từ đầu. Những phản hồi từ thực tiễn triển khai cần được ghi nhận và sử dụng để điều chỉnh hướng đi. Lãnh đạo hiệu quả coi chuyển đổi là một quá trình học tập của toàn tổ chức, trong đó sai sót được xem là dữ liệu để cải tiến, không phải lý do để quy trách nhiệm cá nhân.

Tóm lại, vai trò của lãnh đạo trong dẫn dắt chuyển đổi không nằm ở việc kiểm soát mọi chi tiết, mà ở khả năng định hướng, tạo niềm tin, huy động con người và học hỏi liên tục. Khi lãnh đạo thực sự tham gia và thay đổi cùng tổ chức, quá trình chuyển đổi mới có cơ hội thành công bền vững.

\section{Tạo môi trường khuyến khích đổi mới và thử nghiệm}

Đổi mới không phải là kết quả của những ý tưởng cá nhân đơn lẻ, mà là sản phẩm của một môi trường tổ chức phù hợp. Vai trò của lãnh đạo trong giai đoạn này không phải là trực tiếp tạo ra mọi sáng kiến, mà là thiết kế và duy trì điều kiện để đổi mới có thể xảy ra một cách tự nhiên, liên tục và có kiểm soát. Nếu môi trường không ủng hộ, mọi lời kêu gọi đổi mới đều chỉ dừng lại ở mức khẩu hiệu.

Yếu tố nền tảng của môi trường đổi mới là an toàn tâm lý. Nhân viên chỉ sẵn sàng đề xuất ý tưởng mới và thử nghiệm cách làm khác khi họ không lo sợ bị trừng phạt nếu thất bại. An toàn tâm lý không có nghĩa là chấp nhận mọi sai sót, mà là phân biệt rõ giữa sai sót do thử nghiệm có chủ đích và sai sót do cẩu thả hay thiếu trách nhiệm. Lãnh đạo cần truyền đi thông điệp rõ ràng rằng thử nghiệm có kiểm soát là được khuyến khích, miễn là đi kèm tinh thần học hỏi và cải tiến.

Tiếp theo, tổ chức cần có không gian và cơ chế cho thử nghiệm. Đổi mới khó có thể diễn ra nếu mọi quy trình đều cứng nhắc và mọi quyết định đều phải qua nhiều tầng phê duyệt. Lãnh đạo cần cho phép các nhóm thử nghiệm ở quy mô nhỏ, với chi phí và rủi ro được giới hạn. Cách tiếp cận “thử nhanh – học nhanh – điều chỉnh nhanh” giúp tổ chức thu thập dữ liệu thực tế trước khi mở rộng sáng kiến, đồng thời giảm thiểu tổn thất nếu thử nghiệm không thành công.

Cơ chế ghi nhận và đánh giá cũng có ảnh hưởng trực tiếp đến hành vi đổi mới. Nếu hệ thống đánh giá chỉ tập trung vào kết quả ngắn hạn hoặc tránh rủi ro, nhân viên sẽ ưu tiên an toàn thay vì sáng tạo. Lãnh đạo cần điều chỉnh tiêu chí đánh giá để ghi nhận nỗ lực thử nghiệm, khả năng học hỏi và đóng góp ý tưởng, ngay cả khi kết quả chưa hoàn toàn thành công. Việc ghi nhận đúng giúp củng cố thông điệp rằng đổi mới là một phần chính thức trong công việc, không phải hoạt động bên lề.

Một yếu tố quan trọng khác là vai trò nêu gương của lãnh đạo. Môi trường đổi mới khó hình thành nếu lãnh đạo chỉ yêu cầu cấp dưới thay đổi trong khi bản thân vẫn giữ cách làm cũ. Khi lãnh đạo sẵn sàng thử nghiệm phương pháp quản lý mới, chấp nhận phản biện và điều chỉnh quyết định dựa trên phản hồi, họ tạo ra chuẩn mực hành vi cho toàn tổ chức. Hành động của lãnh đạo trong trường hợp này có tác động mạnh hơn mọi tuyên bố chính thức.

Ngoài ra, đổi mới cần được hỗ trợ bởi sự đa dạng về góc nhìn và hợp tác liên chức năng. Lãnh đạo cần khuyến khích sự trao đổi giữa các bộ phận, giảm bớt rào cản chức năng và tạo điều kiện để các nhóm có nền tảng khác nhau cùng làm việc. Nhiều ý tưởng đổi mới giá trị xuất hiện tại giao điểm giữa các lĩnh vực, nơi những cách nhìn khác nhau được kết nối và va chạm một cách xây dựng.

Cuối cùng, môi trường khuyến khích đổi mới cần có kỷ luật. Đổi mới không đồng nghĩa với tùy tiện hay thiếu định hướng. Lãnh đạo phải đảm bảo rằng các thử nghiệm gắn với mục tiêu chiến lược của tổ chức và được theo dõi, đánh giá một cách nghiêm túc. Khi đổi mới được đặt trong một khung kỷ luật rõ ràng, tổ chức vừa duy trì được sự linh hoạt, vừa đảm bảo hiệu quả dài hạn.

\section{Duy trì động lực đổi mới trong dài hạn}

Thách thức lớn nhất của đổi mới không nằm ở việc khởi động, mà ở khả năng duy trì trong dài hạn. Nhiều tổ chức có thể tạo ra làn sóng đổi mới ban đầu, nhưng sau một thời gian ngắn, động lực suy giảm, các sáng kiến bị gián đoạn và tổ chức quay trở lại lối vận hành cũ. Để tránh tình trạng này, lãnh đạo cần tiếp cận đổi mới như một năng lực cốt lõi, được nuôi dưỡng bằng kỷ luật và cam kết lâu dài.

Trước hết, đổi mới cần được gắn chặt với chiến lược tổng thể của tổ chức. Khi đổi mới bị tách rời khỏi mục tiêu chiến lược, nó dễ bị xem là hoạt động phụ hoặc phong trào nhất thời. Lãnh đạo phải làm rõ đổi mới phục vụ cho mục tiêu nào, đóng góp ra sao vào lợi thế cạnh tranh và giá trị dài hạn. Việc liên kết này giúp ưu tiên nguồn lực, tránh dàn trải và tạo cơ sở để đánh giá hiệu quả một cách thực tế.

Tiếp theo, lãnh đạo cần thiết lập nhịp độ đổi mới ổn định. Đổi mới bền vững không đòi hỏi những thay đổi đột phá liên tục, mà cần một chuỗi cải tiến đều đặn, có trọng tâm. Việc xác định các chu kỳ rà soát, đánh giá và điều chỉnh sáng kiến giúp tổ chức duy trì sự tập trung và tránh kiệt sức. Nhịp độ phù hợp cho phép nhân viên vừa thích ứng với thay đổi, vừa duy trì hiệu quả công việc thường nhật.

Đầu tư vào con người là điều kiện không thể thiếu để duy trì động lực đổi mới. Kỹ năng, tư duy và khả năng học hỏi của đội ngũ quyết định trực tiếp chất lượng và tính liên tục của các sáng kiến. Lãnh đạo cần tạo điều kiện cho việc đào tạo, chia sẻ tri thức và học tập từ thực tiễn. Khi nhân viên cảm nhận được rằng năng lực của họ được phát triển song song với yêu cầu đổi mới, mức độ cam kết sẽ được củng cố.

Một yếu tố quan trọng khác là cơ chế sàng lọc và kết thúc sáng kiến. Không phải mọi ý tưởng đổi mới đều mang lại giá trị lâu dài. Lãnh đạo cần có khả năng dừng lại đúng lúc những sáng kiến không còn phù hợp hoặc không đạt hiệu quả mong muốn. Việc kết thúc một sáng kiến không nên bị xem là thất bại, mà là một quyết định quản trị cần thiết để giải phóng nguồn lực cho các cơ hội khác. Cách lãnh đạo xử lý giai đoạn này ảnh hưởng lớn đến tinh thần đổi mới chung.

Cuối cùng, đổi mới bền vững đòi hỏi sự nhất quán từ lãnh đạo cấp cao. Khi ưu tiên của lãnh đạo thay đổi liên tục hoặc thông điệp không rõ ràng, tổ chức sẽ nhanh chóng mất phương hướng. Việc duy trì sự nhất quán trong cam kết, phân bổ nguồn lực và đánh giá kết quả gửi đi tín hiệu rõ ràng rằng đổi mới là một phần lâu dài trong cách tổ chức vận hành, không phải phản ứng ngắn hạn trước áp lực bên ngoài.

Tóm lại, duy trì động lực đổi mới trong dài hạn là kết quả của sự kết hợp giữa chiến lược rõ ràng, nhịp độ hợp lý, đầu tư vào con người và kỷ luật lãnh đạo. Khi đổi mới được tích hợp vào hệ thống vận hành và tư duy của tổ chức, nó trở thành lợi thế bền vững thay vì nỗ lực nhất thời.
