\chapter{Xây dựng và phát triển đội ngũ}

Xây dựng đội ngũ là nhiệm vụ trọng tâm của lãnh đạo, bởi mọi chiến lược chỉ có giá trị khi được triển khai hiệu quả bởi con người phù hợp. Một đội ngũ mạnh không hình thành một cách ngẫu nhiên, mà là kết quả của những quyết định có chủ đích trong tuyển chọn, tổ chức và phát triển nhân sự. Chương này tập trung vào việc giúp lãnh đạo tạo dựng nền tảng con người vững chắc, đảm bảo đội ngũ không chỉ đáp ứng yêu cầu hiện tại mà còn sẵn sàng cho các mục tiêu dài hạn của tổ chức.

\section{Tuyển chọn đúng người theo mục tiêu và giá trị chung}

Tuyển chọn nhân sự là bước đầu tiên và có ảnh hưởng lâu dài nhất đến chất lượng đội ngũ. Một quyết định tuyển dụng sai có thể làm suy giảm hiệu suất của cả tập thể, tạo ra xung đột nội bộ và tiêu tốn nhiều nguồn lực để khắc phục. Vì vậy, tuyển người không đơn thuần là lấp đầy vị trí trống, mà là lựa chọn những cá nhân phù hợp với định hướng phát triển và giá trị cốt lõi của tổ chức.

Nguyên tắc nền tảng trong tuyển chọn là bắt đầu từ mục tiêu. Trước khi tìm kiếm ứng viên, lãnh đạo cần xác định rõ vị trí này phục vụ cho mục tiêu nào và kết quả mong đợi là gì. Khi mục tiêu không rõ ràng, tiêu chí tuyển chọn sẽ trở nên mơ hồ, dẫn đến việc lựa chọn dựa trên cảm tính hoặc bằng cấp thay vì khả năng tạo ra giá trị thực tế. Một vị trí được định nghĩa rõ ràng về mục tiêu sẽ giúp sàng lọc ứng viên hiệu quả và giảm thiểu rủi ro sai lệch trong quá trình đánh giá.

Bên cạnh mục tiêu, sự phù hợp về giá trị là yếu tố không thể bỏ qua. Giá trị chung định hình cách con người hành xử, ra quyết định và hợp tác với nhau. Một cá nhân có năng lực chuyên môn cao nhưng không chia sẻ giá trị cốt lõi sẽ khó hòa nhập và có thể gây ảnh hưởng tiêu cực đến văn hóa đội ngũ. Trong dài hạn, sự lệch pha về giá trị thường nguy hiểm hơn sự thiếu hụt về kỹ năng.

Trong thực tế, quá trình tuyển chọn nên dựa trên ba nhóm tiêu chí chính. Thứ nhất là năng lực chuyên môn, nhằm đảm bảo ứng viên có khả năng thực hiện công việc ở mức yêu cầu. Thứ hai là thái độ và cách làm việc, bao gồm tinh thần trách nhiệm, khả năng học hỏi và mức độ chủ động. Thứ ba là sự phù hợp với văn hóa và giá trị tổ chức, thể hiện qua cách ứng viên nhìn nhận về hợp tác, kỷ luật và mục tiêu chung. Ba nhóm tiêu chí này cần được đánh giá đồng thời, không nên ưu tiên tuyệt đối một yếu tố và bỏ qua các yếu tố còn lại.

Lãnh đạo cũng cần nhận diện và tránh những sai lầm phổ biến trong tuyển dụng. Một là tuyển người vì áp lực thiếu nhân sự, chấp nhận hạ thấp tiêu chuẩn để giải quyết vấn đề trước mắt. Hai là tuyển người dựa trên tiềm năng chung chung mà không có kế hoạch phát triển cụ thể, dẫn đến việc nhân sự không được sử dụng đúng cách. Cả hai sai lầm này đều làm gia tăng chi phí quản lý và ảnh hưởng tiêu cực đến tinh thần đội ngũ.

Cuối cùng, tuyển chọn đúng người là trách nhiệm không thể ủy thác hoàn toàn. Bộ phận nhân sự có thể hỗ trợ về quy trình và công cụ, nhưng lãnh đạo phải là người chịu trách nhiệm cuối cùng cho quyết định tuyển dụng. Việc trực tiếp tham gia vào khâu đánh giá và lựa chọn giúp lãnh đạo đảm bảo sự thống nhất giữa con người được tuyển và mục tiêu dài hạn của tổ chức. Khi tuyển chọn được thực hiện đúng ngay từ đầu, các hoạt động phân công, huấn luyện và trao quyền sau này sẽ trở nên hiệu quả và bền vững hơn.

\section{Phân công công việc phù hợp với năng lực cá nhân}

Phân công công việc là bước chuyển hóa năng lực cá nhân thành kết quả tập thể. Một đội ngũ có nhân sự tốt nhưng phân công sai vẫn dẫn đến hiệu suất thấp, quá tải cục bộ và xung đột ngầm. Vì vậy, phân công không phải là hành động hành chính, mà là một quyết định quản trị đòi hỏi lãnh đạo hiểu rõ con người và mục tiêu công việc.

Nguyên tắc đầu tiên của phân công hiệu quả là đặt đúng người vào đúng việc. Mỗi cá nhân đều có điểm mạnh, điểm yếu và giới hạn riêng. Khi công việc phù hợp với năng lực, nhân sự có xu hướng chủ động, chịu trách nhiệm và duy trì động lực lâu dài. Ngược lại, khi bị giao việc vượt quá hoặc lệch với năng lực cốt lõi, họ dễ rơi vào trạng thái căng thẳng, phòng thủ hoặc làm việc đối phó.

Để phân công chính xác, lãnh đạo cần hiểu nhân sự trên ba khía cạnh. Thứ nhất là năng lực thực tế, không chỉ dựa vào bằng cấp hay mô tả công việc mà dựa vào kết quả đã chứng minh. Thứ hai là phong cách làm việc, bao gồm khả năng làm việc độc lập hay theo nhóm, mức độ kỷ luật và khả năng xử lý áp lực. Thứ ba là động lực cá nhân, bởi một nhiệm vụ phù hợp với mục tiêu cá nhân sẽ được thực hiện với chất lượng cao hơn.

Bên cạnh việc hiểu con người, lãnh đạo phải làm rõ yêu cầu của từng nhiệm vụ. Một sai lầm phổ biến là giao việc với mô tả chung chung, thiếu tiêu chí đánh giá và thời hạn cụ thể. Phân công hiệu quả cần đi kèm bốn yếu tố rõ ràng: mục tiêu công việc, kết quả đầu ra mong đợi, thời hạn hoàn thành và quyền hạn cần thiết để thực hiện. Khi các yếu tố này không được xác định, trách nhiệm trở nên mơ hồ và hiệu suất khó được kiểm soát.

Phân công cũng cần linh hoạt theo giai đoạn phát triển của đội ngũ. Với nhân sự mới hoặc chưa đủ kinh nghiệm, nhiệm vụ nên được chia nhỏ và có hướng dẫn cụ thể. Với nhân sự đã trưởng thành, lãnh đạo cần giao nhiệm vụ mang tính thách thức hơn, cho phép họ tự chủ trong cách triển khai. Sự linh hoạt này giúp đội ngũ vừa đảm bảo hiệu quả hiện tại, vừa tạo điều kiện phát triển năng lực lâu dài.

Một điểm quan trọng khác là tránh tình trạng dồn việc vào một số cá nhân chủ chốt. Việc này có thể mang lại kết quả ngắn hạn, nhưng về lâu dài sẽ gây quá tải, giảm chất lượng công việc và làm suy yếu tinh thần đội ngũ. Lãnh đạo cần phân bổ công việc có chủ đích, đồng thời tạo cơ hội cho các thành viên khác học hỏi và đảm nhận trách nhiệm lớn hơn.

Cuối cùng, phân công không phải là quyết định cố định. Lãnh đạo cần thường xuyên rà soát và điều chỉnh dựa trên kết quả thực tế. Khi một cá nhân không phù hợp với nhiệm vụ được giao, vấn đề không nhất thiết nằm ở con người, mà có thể nằm ở cách phân công. Khả năng điều chỉnh kịp thời thể hiện tư duy quản trị linh hoạt và giúp đội ngũ duy trì hiệu suất ổn định trong bối cảnh thay đổi.

\section{Huấn luyện và phát triển năng lực đội ngũ}

Huấn luyện và phát triển là yếu tố quyết định khả năng duy trì hiệu suất của đội ngũ trong dài hạn. Nếu tuyển chọn và phân công là điều kiện cần, thì huấn luyện là điều kiện đủ để đội ngũ không ngừng nâng cao năng lực và thích ứng với thay đổi. Lãnh đạo không thể kỳ vọng đội ngũ tự trưởng thành nếu thiếu định hướng, phản hồi và đầu tư phát triển có hệ thống.

Trước hết, cần xác định đúng vai trò của lãnh đạo trong huấn luyện. Huấn luyện không đồng nghĩa với làm thay hay giám sát vi mô. Nhiệm vụ của lãnh đạo là giúp nhân sự hiểu rõ mục tiêu, nhận diện điểm mạnh – điểm yếu của bản thân và từng bước cải thiện năng lực thông qua trải nghiệm thực tế. Một đội ngũ được huấn luyện tốt là đội ngũ có khả năng tự giải quyết vấn đề ngày càng hiệu quả mà không phụ thuộc hoàn toàn vào chỉ đạo trực tiếp.

Hoạt động huấn luyện cần được thực hiện theo từng cấp độ. Ở giai đoạn đầu, khi nhân sự mới hoặc nhiệm vụ còn mới, lãnh đạo cần hướng dẫn rõ ràng về cách làm, tiêu chuẩn và quy trình. Khi nhân sự đã nắm vững công việc, lãnh đạo nên chuyển sang vai trò gợi mở, đặt câu hỏi để khuyến khích họ tự suy nghĩ và đề xuất giải pháp. Ở cấp độ cao hơn, huấn luyện tập trung vào phản hồi và điều chỉnh, giúp nhân sự nhìn lại cách làm của mình và rút ra bài học để cải thiện hiệu quả trong tương lai.

Một yếu tố quan trọng trong huấn luyện là phản hồi kịp thời và cụ thể. Phản hồi hiệu quả không mang tính phán xét, mà tập trung vào hành vi và kết quả. Lãnh đạo cần chỉ rõ điều gì đang làm tốt, điều gì chưa đạt và cần thay đổi như thế nào. Phản hồi mơ hồ hoặc né tránh sẽ khiến nhân sự không hiểu rõ kỳ vọng, từ đó khó cải thiện năng lực thực sự.

Bên cạnh huấn luyện cá nhân, phát triển đội ngũ cần được thực hiện ở cấp độ tập thể. Lãnh đạo nên tạo cơ hội cho các thành viên học hỏi lẫn nhau thông qua làm việc nhóm, chia sẻ kinh nghiệm và luân chuyển nhiệm vụ phù hợp. Việc này không chỉ giúp nâng cao kỹ năng, mà còn tăng cường sự hiểu biết và phối hợp giữa các cá nhân trong đội ngũ.

Huấn luyện và phát triển cũng đòi hỏi sự kiên nhẫn và nhất quán. Kết quả không thể thấy ngay trong ngắn hạn, nhưng sẽ thể hiện rõ qua khả năng tự chủ, chất lượng quyết định và mức độ chịu trách nhiệm của nhân sự theo thời gian. Lãnh đạo cần xem đây là một khoản đầu tư dài hạn, thay vì chi phí ngắn hạn.

Cuối cùng, đội ngũ chỉ phát triển khi lãnh đạo sẵn sàng phát triển cùng họ. Việc liên tục học hỏi, cập nhật kiến thức và hoàn thiện kỹ năng lãnh đạo là cách gián tiếp nhưng hiệu quả nhất để tạo động lực cho đội ngũ. Khi lãnh đạo coi trọng huấn luyện và làm gương trong việc học tập, đội ngũ sẽ hình thành tinh thần cải tiến liên tục và sẵn sàng thích ứng với mọi thay đổi của tổ chức.

\section{Trao quyền đi kèm trách nhiệm và kiểm soát}

Trao quyền là bước chuyển quan trọng từ quản lý dựa trên mệnh lệnh sang quản trị dựa trên kết quả. Khi đội ngũ phát triển về năng lực, việc lãnh đạo ôm đồm quyết định và kiểm soát chi tiết không chỉ làm giảm hiệu suất mà còn kìm hãm sự trưởng thành của nhân sự. Tuy nhiên, trao quyền nếu không đi kèm trách nhiệm và cơ chế kiểm soát phù hợp sẽ nhanh chóng dẫn đến rủi ro và hỗn loạn.

Nguyên tắc cốt lõi của trao quyền là làm rõ phạm vi và giới hạn. Trao quyền không có nghĩa là giao việc rồi rút lui hoàn toàn, mà là xác định rõ những quyết định nào nhân sự được toàn quyền xử lý, những vấn đề nào cần tham vấn và những trường hợp nào bắt buộc phải báo cáo. Khi phạm vi không rõ ràng, nhân sự hoặc sẽ do dự, né tránh trách nhiệm, hoặc đưa ra quyết định vượt quá thẩm quyền cho phép.

Đi cùng với quyền hạn là trách nhiệm cụ thể. Mỗi nhiệm vụ được trao quyền cần gắn với kết quả đầu ra rõ ràng và tiêu chí đánh giá cụ thể. Nhân sự phải hiểu rằng quyền quyết định đi kèm với nghĩa vụ chịu trách nhiệm về kết quả, bao gồm cả thành công lẫn sai sót. Việc này giúp hình thành tư duy làm chủ công việc, thay vì tâm lý làm theo chỉ đạo để tránh rủi ro cá nhân.

Kiểm soát trong bối cảnh trao quyền cần được thực hiện thông qua hệ thống, không phải cảm tính. Lãnh đạo cần thiết lập các mốc kiểm tra, chỉ số đo lường và cơ chế báo cáo phù hợp với mức độ quan trọng của nhiệm vụ. Kiểm soát hiệu quả tập trung vào kết quả và xu hướng, thay vì soi xét từng hành động nhỏ. Điều này vừa đảm bảo an toàn cho tổ chức, vừa giữ được sự linh hoạt và chủ động của nhân sự.

Một sai lầm phổ biến là kiểm soát quá mức sau khi đã trao quyền. Việc liên tục can thiệp, thay đổi quyết định hoặc yêu cầu báo cáo chi tiết không cần thiết sẽ làm mất ý nghĩa của trao quyền và khiến nhân sự quay lại tâm lý phụ thuộc. Ngược lại, buông lỏng hoàn toàn cũng là một dạng thiếu trách nhiệm của lãnh đạo. Cân bằng giữa trao quyền và kiểm soát là kỹ năng quản trị cần được rèn luyện thông qua thực tiễn và phản hồi liên tục.

Trao quyền cũng cần được điều chỉnh theo mức độ trưởng thành của từng cá nhân. Với nhân sự mới hoặc chưa đủ kinh nghiệm, phạm vi quyền hạn nên hẹp và có sự theo dõi sát sao hơn. Khi năng lực và độ tin cậy tăng lên, lãnh đạo có thể mở rộng quyền hạn và giảm dần mức độ kiểm soát. Cách tiếp cận này giúp giảm thiểu rủi ro trong khi vẫn tạo điều kiện cho nhân sự phát triển.

Cuối cùng, trao quyền hiệu quả sẽ tạo ra một đội ngũ chủ động, có khả năng ra quyết định và chịu trách nhiệm độc lập. Lãnh đạo nhờ đó có thể tập trung vào các vấn đề chiến lược thay vì xử lý công việc sự vụ hàng ngày. Một tổ chức chỉ thực sự mạnh khi quyền quyết định được phân bổ hợp lý và mọi cá nhân đều hiểu rõ vai trò, trách nhiệm của mình trong hệ thống chung.

\section{Xây dựng văn hóa hợp tác và hiệu suất bền vững}

Văn hóa đội ngũ là yếu tố quyết định khả năng duy trì hiệu suất trong dài hạn. Nếu quy trình và cấu trúc tạo ra khung vận hành, thì văn hóa chính là lực chi phối cách con người hành xử khi không có sự giám sát trực tiếp. Một đội ngũ có kỹ năng tốt nhưng văn hóa yếu sẽ khó phối hợp, dễ xung đột và nhanh chóng suy giảm hiệu quả khi đối mặt với áp lực hoặc thay đổi.

Nguyên tắc đầu tiên trong xây dựng văn hóa là thống nhất về mục tiêu và cách làm việc. Văn hóa hợp tác chỉ hình thành khi các thành viên hiểu rõ họ đang cùng hướng tới điều gì và vì sao cần phối hợp với nhau. Lãnh đạo cần truyền đạt mục tiêu một cách nhất quán, đồng thời làm rõ các chuẩn mực hành vi được mong đợi trong quá trình làm việc chung, từ giao tiếp, phản hồi đến xử lý mâu thuẫn.

Niềm tin là nền tảng của hợp tác. Niềm tin không đến từ lời cam kết, mà được xây dựng qua hành động nhất quán theo thời gian. Lãnh đạo đóng vai trò trung tâm trong việc tạo dựng niềm tin bằng cách giữ lời hứa, xử lý vấn đề công bằng và minh bạch trong đánh giá kết quả. Khi niềm tin được củng cố, các thành viên sẵn sàng chia sẻ thông tin, hỗ trợ lẫn nhau và chấp nhận rủi ro cần thiết để cải thiện hiệu suất chung.

Giao tiếp thẳng thắn là yếu tố tiếp theo của văn hóa hiệu quả. Một môi trường làm việc lành mạnh cho phép các ý kiến khác biệt được nêu ra mà không sợ bị quy chụp hay trừng phạt. Lãnh đạo cần khuyến khích trao đổi trực tiếp, tập trung vào vấn đề thay vì cá nhân, và xử lý bất đồng một cách kịp thời. Tránh né xung đột không làm văn hóa tốt hơn, mà chỉ khiến vấn đề tích tụ và bùng phát nghiêm trọng hơn về sau.

Hiệu suất bền vững đòi hỏi sự cân bằng giữa kết quả và con người. Việc chỉ tập trung vào chỉ tiêu ngắn hạn có thể mang lại thành tích tức thời, nhưng sẽ làm đội ngũ kiệt sức và mất động lực trong dài hạn. Lãnh đạo cần thiết lập nhịp làm việc hợp lý, ghi nhận nỗ lực đúng lúc và tạo điều kiện để nhân sự phục hồi năng lượng. Một đội ngũ duy trì được sức bền sẽ có khả năng thích ứng tốt hơn với thay đổi và áp lực từ môi trường bên ngoài.

Khen thưởng và kỷ luật là công cụ định hình văn hóa mạnh mẽ nếu được sử dụng đúng cách. Khen thưởng cần gắn với hành vi phù hợp với giá trị và mục tiêu chung, không chỉ dựa trên kết quả cuối cùng. Kỷ luật cần rõ ràng, công bằng và nhất quán, nhằm bảo vệ chuẩn mực chung của đội ngũ. Sự thiếu nhất quán trong hai yếu tố này sẽ làm suy yếu niềm tin và làm mờ các giá trị cốt lõi.

Cuối cùng, văn hóa hợp tác và hiệu suất bền vững không thể được xây dựng trong thời gian ngắn. Đây là quá trình liên tục, đòi hỏi sự kiên định và cam kết lâu dài từ lãnh đạo. Khi lãnh đạo làm gương trong cách làm việc, giao tiếp và ra quyết định, văn hóa mong muốn sẽ dần trở thành thói quen chung của đội ngũ. Đến thời điểm đó, hiệu suất không còn phụ thuộc vào sự thúc ép, mà được duy trì một cách tự nhiên và ổn định.
