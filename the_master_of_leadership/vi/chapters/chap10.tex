\chapter{Phát triển bản thân của người lãnh đạo}

Phát triển bản thân là quá trình mang tính chiến lược, quyết định trực tiếp đến năng lực lãnh đạo trong dài hạn. Khi bối cảnh tổ chức và xã hội thay đổi nhanh chóng, lợi thế của người lãnh đạo không còn nằm ở chức danh hay quyền hạn, mà ở khả năng tự điều chỉnh, học hỏi và hoàn thiện chính mình. Chương này tập trung vào các yếu tố cốt lõi giúp người lãnh đạo duy trì hiệu quả, tính chính trực và sức bền nghề nghiệp, bắt đầu từ nền tảng quan trọng nhất: nhận thức bản thân.

\section{Nhận thức bản thân như nền tảng phát triển lãnh đạo}

Nhận thức bản thân là khả năng hiểu rõ chính mình một cách trung thực và có hệ thống, bao gồm giá trị cá nhân, niềm tin, động cơ, điểm mạnh, điểm yếu, cảm xúc và giới hạn của bản thân. Đối với người lãnh đạo, đây không phải là vấn đề mang tính nội tâm thuần túy, mà là năng lực nền tảng ảnh hưởng trực tiếp đến hành vi lãnh đạo, chất lượng quyết định và mức độ tạo ảnh hưởng lên tổ chức.

Trước hết, nhận thức bản thân giúp người lãnh đạo xác định rõ hệ giá trị và nguyên tắc dẫn dắt. Trong thực tế, người lãnh đạo thường xuyên đối mặt với những lựa chọn khó khăn, nơi không có đáp án hoàn hảo. Khi thiếu nền tảng giá trị rõ ràng, quyết định dễ bị chi phối bởi áp lực ngắn hạn, lợi ích cá nhân hoặc mong muốn né tránh xung đột. Ngược lại, một người hiểu rõ điều gì là cốt lõi đối với mình sẽ đưa ra quyết định nhất quán, tạo ra sự tin cậy và uy tín trong dài hạn.

Thứ hai, nhận thức đúng về điểm mạnh và điểm yếu cho phép người lãnh đạo sử dụng năng lực cá nhân một cách hiệu quả. Nhiều nhà lãnh đạo thất bại không phải vì thiếu năng lực, mà vì đánh giá sai bản thân: hoặc cố gắng kiểm soát những lĩnh vực không phải thế mạnh, hoặc bỏ qua những điểm yếu mang tính hệ thống. Khi hiểu rõ giới hạn cá nhân, người lãnh đạo biết cách phân quyền hợp lý, xây dựng đội ngũ bổ trợ và tập trung vào những giá trị mà mình tạo ra tốt nhất.

Thứ ba, nhận thức bản thân là điều kiện tiên quyết để quản lý cảm xúc và hành vi. Áp lực lãnh đạo thường kích hoạt các phản xạ cảm xúc mang tính bản năng như phòng thủ, nóng vội hoặc né tránh trách nhiệm. Nếu không nhận diện được những phản xạ này, người lãnh đạo dễ đưa ra quyết định cảm tính, làm tổn hại đến mối quan hệ và môi trường làm việc. Ngược lại, khi hiểu rõ trạng thái cảm xúc và nguyên nhân bên trong, họ có khả năng kiểm soát phản ứng, duy trì sự điềm tĩnh và hành xử có chủ đích.

Thứ tư, nhận thức bản thân giúp người lãnh đạo hiểu được tác động thực tế của mình lên người khác. Lãnh đạo không được đo bằng ý định, mà bằng ảnh hưởng. Khoảng cách giữa cách người lãnh đạo nghĩ mình đang thể hiện và cách nhân viên cảm nhận có thể rất lớn. Thông qua việc chủ động tiếp nhận phản hồi, quan sát phản ứng của người khác và tự đánh giá hành vi, người lãnh đạo có thể điều chỉnh phong cách giao tiếp và dẫn dắt để phù hợp hơn với bối cảnh và con người cụ thể.

Cuối cùng, nhận thức bản thân là điểm khởi đầu của mọi quá trình phát triển cá nhân. Không thể cải thiện điều mà bản thân không nhìn thấy hoặc không thừa nhận. Một người lãnh đạo có mức độ nhận thức bản thân cao thường có thái độ học hỏi, không phòng thủ trước phản hồi và sẵn sàng thay đổi. Đây chính là nền tảng để học tập suốt đời, tự phản tư hiệu quả và phát triển năng lực lãnh đạo một cách bền vững.

\section{Học tập suốt đời và nâng cao năng lực cá nhân}

Học tập suốt đời không còn là lựa chọn, mà là yêu cầu bắt buộc đối với người lãnh đạo trong môi trường biến động liên tục. Kiến thức, kỹ năng và mô hình quản trị từng hiệu quả trong quá khứ có thể nhanh chóng trở nên lỗi thời. Do đó, năng lực học tập liên tục chính là lợi thế cạnh tranh bền vững nhất của người lãnh đạo hiện đại.

Trước hết, người lãnh đạo cần thay đổi cách nhìn về học tập. Học tập không chỉ giới hạn ở các chương trình đào tạo chính quy, bằng cấp hay hội thảo chuyên môn. Đối với lãnh đạo, học tập là quá trình chủ động cập nhật tri thức, mở rộng góc nhìn và điều chỉnh tư duy thông qua nhiều nguồn khác nhau như thực tiễn công việc, phản hồi từ cấp dưới, trao đổi với đồng cấp và quan sát môi trường bên ngoài. Khi xem học tập là một phần của công việc lãnh đạo, người lãnh đạo sẽ duy trì được sự linh hoạt và khả năng thích nghi.

Thứ hai, học tập suốt đời đòi hỏi người lãnh đạo xác định rõ những năng lực cốt lõi cần phát triển. Không phải mọi kiến thức đều mang lại giá trị như nhau. Người lãnh đạo cần tập trung vào các nhóm năng lực có tác động trực tiếp đến vai trò của mình, bao gồm tư duy chiến lược, ra quyết định, quản trị con người, giao tiếp, và khả năng dẫn dắt thay đổi. Việc học có định hướng giúp tránh tình trạng học dàn trải nhưng không tạo ra cải thiện thực chất trong hiệu quả lãnh đạo.

Thứ ba, học tập hiệu quả gắn liền với khả năng học từ thực tiễn. Mỗi quyết định, mỗi dự án và mỗi thất bại đều là nguồn dữ liệu quan trọng nếu được khai thác đúng cách. Người lãnh đạo học tập suốt đời không chỉ hỏi “điều gì đã xảy ra”, mà còn đặt câu hỏi “tại sao lại xảy ra như vậy” và “lần sau cần làm khác đi điều gì”. Cách tiếp cận này giúp chuyển kinh nghiệm thành năng lực, thay vì chỉ tích lũy thâm niên.

Thứ tư, người lãnh đạo cần xây dựng thói quen học tập có kỷ luật. Áp lực công việc thường khiến việc học bị trì hoãn hoặc xem nhẹ. Tuy nhiên, nếu không có thời gian và cơ chế học tập rõ ràng, năng lực cá nhân sẽ dần tụt hậu so với yêu cầu thực tế. Việc dành thời gian cố định cho đọc, suy ngẫm, trao đổi chuyên môn hoặc huấn luyện cá nhân là biểu hiện của tư duy dài hạn và trách nhiệm lãnh đạo.

Thứ năm, học tập suốt đời đòi hỏi thái độ khiêm tốn trí tuệ. Người lãnh đạo càng ở vị trí cao càng dễ rơi vào cảm giác đã “biết đủ” hoặc ngại thừa nhận mình chưa biết. Đây là rào cản lớn nhất của quá trình học tập. Ngược lại, một người lãnh đạo sẵn sàng thừa nhận giới hạn hiểu biết, lắng nghe quan điểm khác biệt và học từ những người trẻ hơn hoặc cấp dưới sẽ duy trì được sự phát triển liên tục và tạo ra văn hóa học tập trong tổ chức.

Cuối cùng, việc nâng cao năng lực cá nhân của người lãnh đạo không chỉ phục vụ cho bản thân họ, mà còn lan tỏa đến toàn bộ tổ chức. Khi lãnh đạo thể hiện tinh thần học hỏi, tổ chức sẽ coi học tập là chuẩn mực, không phải ngoại lệ. Đây chính là nền tảng để xây dựng đội ngũ linh hoạt, có khả năng thích ứng và phát triển bền vững trong dài hạn.

\section{Tự phản tư và cải thiện từ trải nghiệm thực tế}

Tự phản tư là quá trình chủ động xem xét lại suy nghĩ, hành vi và quyết định của bản thân nhằm rút ra bài học có giá trị cho tương lai. Đối với người lãnh đạo, đây không phải là hoạt động mang tính cảm xúc hay hồi tưởng đơn thuần, mà là một công cụ phát triển năng lực có tính kỷ luật và định hướng rõ ràng. Trong thực tế, trải nghiệm chỉ trở thành kinh nghiệm khi được phản tư một cách nghiêm túc.

Trước hết, tự phản tư giúp người lãnh đạo hiểu rõ mối quan hệ giữa quyết định và kết quả. Nhiều lãnh đạo có xu hướng đánh giá thành công hoặc thất bại dựa trên kết quả cuối cùng mà bỏ qua quá trình ra quyết định. Cách tiếp cận này dễ dẫn đến ngộ nhận, bởi kết quả chịu ảnh hưởng của nhiều yếu tố ngoài tầm kiểm soát. Phản tư đúng cách đòi hỏi người lãnh đạo phân tích logic suy nghĩ, giả định ban đầu, thông tin sử dụng và cách triển khai hành động, từ đó đánh giá chất lượng quyết định một cách khách quan hơn.

Thứ hai, tự phản tư giúp nhận diện các khuôn mẫu hành vi lặp lại. Trong áp lực lãnh đạo, con người thường hành động theo thói quen và phản xạ đã hình thành từ trước. Nếu không phản tư, những khuôn mẫu này sẽ tiếp tục được củng cố, kể cả khi chúng không còn phù hợp. Thông qua việc nhìn lại các tình huống tương tự đã xảy ra, người lãnh đạo có thể phát hiện những điểm mạnh cần phát huy và những hạn chế cần điều chỉnh trong phong cách lãnh đạo của mình.

Thứ ba, tự phản tư tạo điều kiện để học từ sai lầm một cách có hệ thống. Sai lầm là điều không thể tránh khỏi trong lãnh đạo, đặc biệt khi phải ra quyết định trong điều kiện không chắc chắn. Vấn đề không nằm ở việc mắc sai lầm, mà ở việc có rút ra được bài học thực chất hay không. Phản tư giúp người lãnh đạo chuyển sai lầm thành tài sản học tập, tránh lặp lại cùng một lỗi dưới những hình thức khác nhau.

Thứ tư, tự phản tư cần được thực hiện một cách có cấu trúc. Phản tư hiệu quả không chỉ là suy nghĩ lan man, mà cần dựa trên những câu hỏi trọng tâm như: điều gì đã diễn ra, điều gì đã làm tốt, điều gì chưa hiệu quả, nguyên nhân cốt lõi là gì và cần thay đổi điều gì trong tương lai. Việc ghi chép, trao đổi với cố vấn hoặc thảo luận với đồng nghiệp đáng tin cậy có thể giúp quá trình phản tư trở nên sâu sắc và khách quan hơn.

Thứ năm, tự phản tư giúp người lãnh đạo phát triển sự trưởng thành cá nhân và nghề nghiệp. Khi thường xuyên đối diện với chính mình một cách trung thực, người lãnh đạo dần hình thành khả năng tự điều chỉnh, không đổ lỗi cho hoàn cảnh hay người khác. Điều này không chỉ nâng cao năng lực cá nhân mà còn tạo ra hình ảnh một người lãnh đạo có trách nhiệm, đáng tin cậy và sẵn sàng học hỏi.

Cuối cùng, tự phản tư là cầu nối giữa trải nghiệm và sự thay đổi thực sự. Nếu không có bước phản tư, trải nghiệm sẽ trôi qua mà không để lại giá trị lâu dài. Đối với người lãnh đạo, việc duy trì thói quen phản tư đều đặn chính là cách biến công việc hàng ngày thành nguồn học tập liên tục, từ đó nâng cao chất lượng lãnh đạo theo thời gian.

\section{Quản lý cảm xúc và áp lực lãnh đạo}

Quản lý cảm xúc và áp lực là năng lực thiết yếu đối với người lãnh đạo, bởi vai trò lãnh đạo luôn đi kèm với trách nhiệm cao, kỳ vọng lớn và mức độ bất định đáng kể. Cảm xúc của người lãnh đạo không chỉ ảnh hưởng đến bản thân họ, mà còn lan tỏa trực tiếp đến tinh thần, động lực và hành vi của toàn bộ tổ chức. Do đó, việc quản lý cảm xúc không phải là kìm nén hay né tránh, mà là nhận diện, điều tiết và sử dụng cảm xúc một cách có ý thức.

Trước hết, người lãnh đạo cần nhận diện được các nguồn áp lực đặc thù trong vai trò của mình. Áp lực có thể đến từ trách nhiệm ra quyết định, mâu thuẫn lợi ích, kỳ vọng từ cấp trên, nhân viên và các bên liên quan, cũng như từ chính tiêu chuẩn cá nhân mà người lãnh đạo tự đặt ra. Khi không nhận thức rõ nguồn gốc áp lực, người lãnh đạo dễ rơi vào trạng thái căng thẳng kéo dài, dẫn đến suy giảm hiệu quả và các phản ứng cảm xúc tiêu cực.

Thứ hai, quản lý cảm xúc bắt đầu từ khả năng nhận diện trạng thái cảm xúc của bản thân. Nhiều lãnh đạo chỉ chú ý đến hành động mà bỏ qua tín hiệu cảm xúc bên trong. Trong khi đó, cảm xúc thường là dấu hiệu sớm cho thấy sự quá tải, xung đột nội tâm hoặc rủi ro trong quyết định. Việc nhận diện kịp thời các trạng thái như tức giận, lo âu, thất vọng hay mệt mỏi giúp người lãnh đạo chủ động điều chỉnh hành vi trước khi cảm xúc chi phối quyết định.

Thứ ba, người lãnh đạo cần phát triển khả năng điều tiết cảm xúc trong những tình huống áp lực cao. Điều tiết cảm xúc không đồng nghĩa với việc phủ nhận cảm xúc, mà là lựa chọn cách phản ứng phù hợp. Trong các tình huống căng thẳng, khả năng tạm dừng, suy nghĩ có chủ đích và phản hồi một cách điềm tĩnh giúp người lãnh đạo giữ được sự sáng suốt và tạo cảm giác an toàn cho đội ngũ. Đây là yếu tố then chốt để duy trì niềm tin và sự ổn định trong tổ chức.

Thứ tư, quản lý cảm xúc của người lãnh đạo có tác động trực tiếp đến môi trường làm việc. Cách người lãnh đạo thể hiện cảm xúc sẽ thiết lập chuẩn mực hành vi cho toàn bộ đội ngũ. Một người lãnh đạo thường xuyên mất kiểm soát, cáu gắt hoặc bi quan sẽ tạo ra bầu không khí căng thẳng và phòng thủ. Ngược lại, sự điềm tĩnh, nhất quán và tôn trọng cảm xúc của người khác góp phần xây dựng môi trường làm việc tích cực, nơi nhân viên sẵn sàng chia sẻ và hợp tác.

Thứ năm, để quản lý áp lực một cách bền vững, người lãnh đạo cần chủ động xây dựng các cơ chế hỗ trợ cá nhân. Điều này bao gồm việc phân bổ công việc hợp lý, duy trì sức khỏe thể chất, có không gian trao đổi tin cậy với cố vấn hoặc đồng nghiệp, và thiết lập ranh giới rõ ràng giữa công việc và đời sống cá nhân. Quản lý áp lực không phải là xử lý khi đã quá tải, mà là phòng ngừa và điều chỉnh liên tục.

Tóm lại, quản lý cảm xúc và áp lực là biểu hiện của sự trưởng thành trong lãnh đạo. Một người lãnh đạo kiểm soát tốt cảm xúc không chỉ bảo vệ chính mình khỏi kiệt sức, mà còn tạo ra ảnh hưởng tích cực, giúp tổ chức duy trì hiệu quả và sự ổn định trong những giai đoạn nhiều thách thức.

\section{Cân bằng giữa hiệu quả công việc và cuộc sống cá nhân}

Cân bằng giữa hiệu quả công việc và cuộc sống cá nhân là một trong những thách thức lớn nhất đối với người lãnh đạo. Trách nhiệm cao, khối lượng công việc lớn và áp lực liên tục dễ khiến ranh giới giữa công việc và đời sống riêng bị xóa nhòa. Tuy nhiên, trong dài hạn, một người lãnh đạo không duy trì được sự cân bằng sẽ khó giữ được hiệu quả, sự tỉnh táo và sức bền cần thiết cho vai trò của mình.

Trước hết, cần hiểu rõ rằng cân bằng không đồng nghĩa với việc phân chia thời gian một cách cứng nhắc hay giảm mức độ cam kết với công việc. Đối với người lãnh đạo, vấn đề cốt lõi không phải là làm ít hơn, mà là quản trị năng lượng và sự tập trung hiệu quả hơn. Một người làm việc nhiều giờ nhưng trong trạng thái mệt mỏi, căng thẳng kéo dài sẽ kém hiệu quả hơn so với người biết phân bổ thời gian nghỉ ngơi, phục hồi và tập trung vào những việc có giá trị cao.

Thứ hai, cân bằng bắt đầu từ việc xác định lại khái niệm hiệu quả công việc. Hiệu quả không nên được đo bằng số giờ làm việc hay mức độ bận rộn, mà bằng kết quả tạo ra và giá trị mang lại cho tổ chức. Khi người lãnh đạo tập trung vào các ưu tiên chiến lược, ủy quyền hợp lý và loại bỏ những hoạt động không cần thiết, họ vừa nâng cao hiệu quả công việc, vừa giảm áp lực cá nhân. Đây là điều kiện quan trọng để duy trì sự cân bằng trong dài hạn.

Thứ ba, người lãnh đạo cần chủ động thiết lập ranh giới giữa công việc và đời sống cá nhân. Trong thực tế, công nghệ và văn hóa làm việc linh hoạt khiến ranh giới này ngày càng mờ nhạt. Nếu không có ranh giới rõ ràng, công việc sẽ liên tục xâm lấn thời gian nghỉ ngơi, gia đình và tái tạo năng lượng. Việc xác định thời điểm nghỉ ngơi, tạm ngắt kết nối và dành thời gian cho các hoạt động ngoài công việc không phải là dấu hiệu của thiếu cam kết, mà là biểu hiện của tư duy lãnh đạo bền vững.

Thứ tư, cân bằng công việc và cuộc sống cá nhân có tác động trực tiếp đến sức khỏe thể chất và tinh thần của người lãnh đạo. Sức khỏe suy giảm sẽ làm giảm khả năng tập trung, ra quyết định và kiểm soát cảm xúc. Trong khi đó, một người lãnh đạo có trạng thái thể chất và tinh thần tốt sẽ duy trì được sự tỉnh táo, điềm tĩnh và khả năng chịu áp lực cao hơn. Do đó, việc chăm sóc sức khỏe không nên được xem là vấn đề cá nhân, mà là trách nhiệm gắn liền với vai trò lãnh đạo.

Thứ năm, cách người lãnh đạo cân bằng công việc và cuộc sống cá nhân sẽ tạo ra chuẩn mực cho toàn bộ tổ chức. Khi lãnh đạo làm việc quá tải, hy sinh đời sống cá nhân và coi đó là tiêu chuẩn, tổ chức sẽ hình thành văn hóa kiệt sức, nơi hiệu quả ngắn hạn được đánh đổi bằng sự suy giảm dài hạn. Ngược lại, khi lãnh đạo thể hiện sự cân bằng hợp lý, tôn trọng thời gian cá nhân và khuyến khích làm việc hiệu quả thay vì làm việc quá mức, tổ chức sẽ phát triển một cách bền vững hơn.

Cuối cùng, cân bằng không phải là trạng thái cố định đạt được một lần, mà là quá trình điều chỉnh liên tục theo từng giai đoạn cuộc sống và sự nghiệp. Người lãnh đạo cần thường xuyên tự đánh giá mức độ căng thẳng, mức độ hài lòng và sức khỏe tổng thể của bản thân để kịp thời điều chỉnh. Khả năng duy trì sự cân bằng chính là yếu tố giúp người lãnh đạo giữ được hiệu quả lâu dài, sự minh mẫn trong quyết định và chất lượng ảnh hưởng tích cực đối với tổ chức.
