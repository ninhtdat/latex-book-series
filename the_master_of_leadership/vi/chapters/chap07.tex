\chapter{Ra quyết định và giải quyết vấn đề}

Ra quyết định là trung tâm của vai trò lãnh đạo. Mọi chiến lược, mục tiêu hay kế hoạch hành động đều chỉ có giá trị khi được chuyển hóa thành các quyết định cụ thể và được thực thi đúng thời điểm. Trong bối cảnh môi trường kinh doanh biến động, thông tin không đầy đủ và áp lực ngày càng cao, năng lực ra quyết định trở thành yếu tố quyết định hiệu quả và uy tín của người lãnh đạo. Chương này tập trung làm rõ bản chất của quyết định lãnh đạo, từ đó đặt nền tảng cho việc xác định vấn đề, sử dụng dữ liệu, cân bằng tư duy và quản trị rủi ro.

\section{Bản chất của quyết định lãnh đạo và hệ quả đi kèm}

Quyết định lãnh đạo khác căn bản với các quyết định tác nghiệp thông thường. Nếu quyết định tác nghiệp chủ yếu liên quan đến cách thực hiện một nhiệm vụ đã rõ, thì quyết định lãnh đạo liên quan đến việc lựa chọn hướng đi trong điều kiện không chắc chắn. Lãnh đạo phải quyết định khi thông tin còn thiếu, khi các phương án đều có rủi ro và khi thời gian không cho phép thử sai nhiều lần.

Bản chất đầu tiên của quyết định lãnh đạo là tính không hoàn hảo. Không có quyết định nào hội tụ đầy đủ thông tin, sự đồng thuận và độ an toàn tuyệt đối. Việc chờ đợi đến khi mọi dữ kiện trở nên rõ ràng thường dẫn đến bỏ lỡ cơ hội hoặc làm trầm trọng thêm vấn đề. Do đó, lãnh đạo cần chấp nhận rằng quyết định được đưa ra dựa trên xác suất và đánh giá hợp lý, chứ không phải trên sự chắc chắn tuyệt đối.

Thứ hai, quyết định lãnh đạo luôn đi kèm hệ quả rộng hơn phạm vi của chính quyết định đó. Một lựa chọn về chiến lược, nhân sự hay phân bổ nguồn lực không chỉ ảnh hưởng đến kết quả kinh doanh mà còn tác động đến tinh thần đội ngũ, niềm tin vào tổ chức và hình ảnh của người lãnh đạo. Quyết định thiếu nhất quán hoặc thay đổi liên tục làm suy giảm sự ổn định và tạo cảm giác bất an trong toàn bộ hệ thống.

Thứ ba, quyết định lãnh đạo gắn liền với trách nhiệm không thể chuyển giao. Lãnh đạo có thể tham khảo ý kiến, ủy quyền phân tích hoặc lắng nghe khuyến nghị từ đội ngũ, nhưng trách nhiệm cuối cùng vẫn thuộc về người đứng đầu. Khi kết quả không như mong đợi, việc đổ lỗi cho hoàn cảnh, thị trường hay cấp dưới chỉ làm xói mòn uy tín cá nhân và văn hóa chịu trách nhiệm của tổ chức.

Một khía cạnh quan trọng khác là nhận thức rằng không ra quyết định cũng là một quyết định. Sự trì hoãn thường xuất phát từ nỗi sợ sai hoặc sợ chịu trách nhiệm, nhưng trong nhiều trường hợp, trì hoãn gây ra thiệt hại lớn hơn so với việc lựa chọn một phương án chưa tối ưu. Trong môi trường cạnh tranh, tốc độ và thời điểm của quyết định có thể quan trọng không kém, thậm chí quan trọng hơn độ chính xác tuyệt đối.

Lãnh đạo hiệu quả không được đo bằng số lượng quyết định đúng, mà bằng cách họ xử lý hệ quả của quyết định. Một quyết định sai nhưng được nhận diện sớm, điều chỉnh kịp thời và rút ra bài học rõ ràng sẽ tạo ra giá trị lâu dài cho tổ chức. Ngược lại, một quyết định đúng nhưng được thực thi thiếu nhất quán hoặc không được theo dõi đến cùng vẫn có thể dẫn đến thất bại.

Hiểu rõ bản chất của quyết định lãnh đạo giúp người lãnh đạo thay đổi tư duy từ tìm kiếm sự an toàn sang quản trị rủi ro, từ né tránh trách nhiệm sang chủ động gánh vác. Đây là nền tảng cần thiết để bước sang các nội dung tiếp theo của chương, nơi quyết định không chỉ là hành động cá nhân mà là kết quả của tư duy hệ thống và kỷ luật lãnh đạo.

\section{Xác định đúng vấn đề cốt lõi cần giải quyết}

Trong thực tiễn lãnh đạo, nhiều quyết định thất bại không phải vì lựa chọn phương án sai, mà vì giải quyết sai vấn đề. Lãnh đạo thường bị cuốn vào việc xử lý các biểu hiện bề mặt thay vì dành đủ thời gian để xác định nguyên nhân cốt lõi. Khi vấn đề được xác định sai, mọi nỗ lực phía sau dù tốn kém và quyết liệt đến đâu cũng chỉ mang tính tạm thời.

Điểm cần phân biệt rõ ràng là giữa triệu chứng và vấn đề gốc. Triệu chứng là những gì dễ quan sát: doanh thu sụt giảm, dự án chậm tiến độ, nhân sự rời bỏ tổ chức, khách hàng phàn nàn. Đây là tín hiệu cho thấy có điều gì đó không ổn, nhưng bản thân chúng hiếm khi là vấn đề cần giải quyết. Vấn đề cốt lõi nằm sâu hơn, có thể là chiến lược không còn phù hợp, cấu trúc tổ chức bất hợp lý, mục tiêu thiếu rõ ràng hoặc năng lực lãnh đạo ở một cấp nào đó chưa đáp ứng yêu cầu.

Lãnh đạo cần kỷ luật tư duy để không phản ứng vội vàng trước triệu chứng. Phản ứng nhanh có thể tạo cảm giác đang hành động, nhưng nếu hành động dựa trên chẩn đoán sai, hệ quả là vấn đề sẽ lặp lại dưới hình thức khác. Một tổ chức liên tục “chữa cháy” thường là dấu hiệu của việc không dành đủ thời gian cho khâu xác định vấn đề.

Một nguyên tắc quan trọng là đặt đúng câu hỏi. Thay vì hỏi “chuyện gì đang xảy ra?”, lãnh đạo cần hỏi “tại sao chuyện này xảy ra?” và tiếp tục truy vấn cho đến khi chạm tới nguyên nhân có thể tác động được. Việc lặp lại câu hỏi “tại sao” nhiều lần giúp bóc tách dần các lớp bề mặt và tránh sa vào các giả định cảm tính. Tuy nhiên, mục tiêu không phải là tìm ra mọi nguyên nhân, mà là xác định nguyên nhân then chốt tạo ra tác động lớn nhất.

Xác định vấn đề cốt lõi cũng đòi hỏi khả năng tách cảm xúc khỏi phân tích. Khi tổ chức gặp khủng hoảng, áp lực thường khiến lãnh đạo quy trách nhiệm cho cá nhân hoặc phòng ban cụ thể. Cách tiếp cận này có thể thỏa mãn cảm xúc ngắn hạn nhưng hiếm khi giải quyết được vấn đề thực sự. Lãnh đạo cần nhìn vấn đề như một phần của hệ thống, nơi con người, quy trình và cấu trúc tương tác với nhau.

Một sai lầm phổ biến khác là cố gắng giải quyết quá nhiều vấn đề cùng lúc. Trong thực tế, nguồn lực lãnh đạo luôn có hạn. Việc xác định đúng vấn đề cốt lõi đồng nghĩa với việc chấp nhận rằng một số vấn đề thứ yếu sẽ tạm thời chưa được xử lý. Sự tập trung này cho phép tổ chức dồn nguồn lực vào điểm tạo đòn bẩy lớn nhất, thay vì phân tán và không tạo ra thay đổi rõ rệt.

Vấn đề cốt lõi cần được diễn đạt rõ ràng, cụ thể và có thể kiểm chứng. Một phát biểu vấn đề tốt thường ngắn gọn, trung tính và tập trung vào thực tế có thể quan sát. Việc này không chỉ giúp lãnh đạo tự rõ ràng trong tư duy, mà còn tạo nền tảng chung để đội ngũ cùng hiểu và hành động nhất quán. Khi vấn đề được mô tả mơ hồ hoặc mang tính phán xét, các quyết định phía sau sẽ thiếu định hướng rõ ràng.

Cuối cùng, xác định đúng vấn đề là trách nhiệm không thể né tránh của người lãnh đạo. Dù có thể tham vấn chuyên gia hay đội ngũ, lãnh đạo vẫn phải là người chịu trách nhiệm về cách đặt vấn đề. Đây là bước nền tảng cho mọi quyết định tiếp theo. Nếu bước này sai, các công cụ, dữ liệu hay kinh nghiệm ở những phần sau của chương sẽ không phát huy được giá trị thực sự.

\section{Sử dụng dữ liệu và thông tin một cách hiệu quả}

Trong môi trường lãnh đạo hiện đại, dữ liệu được xem là nền tảng cho ra quyết định. Tuy nhiên, dữ liệu chỉ thực sự có giá trị khi được sử dụng đúng cách. Nhiều tổ chức sở hữu lượng dữ liệu lớn nhưng vẫn ra quyết định kém hiệu quả, không phải vì thiếu thông tin, mà vì không biết thông tin nào cần dùng và dùng như thế nào.

Trước hết, lãnh đạo cần phân biệt rõ giữa dữ liệu và hiểu biết. Dữ liệu là các con số, báo cáo, chỉ số; hiểu biết là khả năng diễn giải dữ liệu trong bối cảnh cụ thể để hỗ trợ hành động. Việc sở hữu nhiều báo cáo không đồng nghĩa với việc có cái nhìn rõ ràng. Ngược lại, quá nhiều dữ liệu không liên quan có thể gây nhiễu, làm chậm quá trình ra quyết định và tạo cảm giác an toàn giả tạo.

Một nguyên tắc quan trọng là dữ liệu phải phục vụ câu hỏi, không phải ngược lại. Lãnh đạo cần xác định rõ mình đang cần quyết định điều gì, sau đó mới xác định loại dữ liệu nào thực sự cần thiết. Việc thu thập dữ liệu tràn lan, chỉ để “đủ hồ sơ” hoặc phục vụ báo cáo hình thức, thường không giúp cải thiện chất lượng quyết định.

Lãnh đạo cũng cần nhận thức rằng không phải dữ liệu nào cũng có độ tin cậy như nhau. Nguồn dữ liệu, cách thu thập, thời điểm và động cơ báo cáo đều ảnh hưởng đến chất lượng thông tin. Do đó, thay vì chỉ nhìn vào con số cuối cùng, lãnh đạo cần đặt câu hỏi về bối cảnh tạo ra dữ liệu đó. Dữ liệu tốt giúp soi sáng thực tế; dữ liệu kém có thể dẫn đến những kết luận sai lầm nhưng trông vẫn có vẻ hợp lý.

Một sai lầm phổ biến là sử dụng dữ liệu để hợp thức hóa quyết định đã có sẵn. Khi lãnh đạo chỉ chọn những con số ủng hộ quan điểm cá nhân và bỏ qua các tín hiệu ngược chiều, dữ liệu trở thành công cụ biện minh thay vì công cụ hỗ trợ tư duy. Cách tiếp cận này làm giảm giá trị của dữ liệu và khiến tổ chức đánh mất cơ hội nhìn thấy rủi ro tiềm ẩn.

Bên cạnh đó, lãnh đạo cần chấp nhận thực tế rằng dữ liệu luôn có độ trễ và không bao giờ hoàn hảo. Trong nhiều tình huống, đặc biệt là khi thị trường biến động nhanh, chờ đủ dữ liệu đồng nghĩa với việc chấp nhận hành động muộn. Do đó, vấn đề không phải là có đầy đủ dữ liệu hay không, mà là dữ liệu hiện có đã đủ để đưa ra quyết định ở mức rủi ro chấp nhận được hay chưa.

Sử dụng dữ liệu hiệu quả cũng đòi hỏi khả năng đơn giản hóa. Lãnh đạo cần tập trung vào một số chỉ số then chốt phản ánh bản chất vấn đề, thay vì theo dõi quá nhiều chỉ số chi tiết. Việc này giúp giữ được góc nhìn chiến lược và tránh bị cuốn vào các biến động nhỏ không mang tính quyết định.

Cuối cùng, dữ liệu không thay thế vai trò của lãnh đạo, mà chỉ hỗ trợ lãnh đạo ra quyết định tốt hơn. Trách nhiệm của lãnh đạo là kết hợp dữ liệu với hiểu biết về con người, bối cảnh và mục tiêu dài hạn của tổ chức. Khi được sử dụng đúng cách, dữ liệu giúp giảm rủi ro và tăng tính nhất quán trong quyết định. Khi bị lạm dụng hoặc sử dụng sai mục đích, dữ liệu có thể trở thành rào cản cho hành động và đổi mới.

\section{Cân bằng giữa phân tích logic, kinh nghiệm và trực giác}

Ra quyết định lãnh đạo không phải là bài toán thuần túy kỹ thuật. Nếu chỉ dựa vào phân tích logic và mô hình số liệu, quyết định có thể đúng về mặt lý thuyết nhưng thiếu tính thực tiễn. Ngược lại, nếu chỉ dựa vào kinh nghiệm hay trực giác, quyết định dễ rơi vào cảm tính và thiên kiến cá nhân. Thách thức của lãnh đạo nằm ở việc cân bằng ba yếu tố này trong từng bối cảnh cụ thể.

Phân tích logic là nền tảng tối thiểu của mọi quyết định nghiêm túc. Logic giúp lãnh đạo so sánh phương án, đánh giá chi phí – lợi ích, dự đoán hệ quả và loại bỏ những lựa chọn rõ ràng là không khả thi. Phân tích tốt giúp giảm xác suất sai lầm lớn và tạo cơ sở minh bạch để giải thích quyết định với tổ chức. Tuy nhiên, logic chỉ hiệu quả khi các giả định đầu vào phản ánh đúng thực tế. Logic dựa trên giả định sai sẽ cho ra kết luận sai một cách rất thuyết phục.

Kinh nghiệm đóng vai trò như bộ lọc thực tế cho phân tích logic. Kinh nghiệm giúp lãnh đạo nhận ra những mẫu hình lặp lại, những rủi ro khó thể hiện trên báo cáo và những yếu tố con người không thể lượng hóa. Một lãnh đạo từng trải thường biết đâu là điểm cần cảnh giác, đâu là tín hiệu sớm của vấn đề và đâu là giới hạn thực sự của tổ chức. Tuy nhiên, kinh nghiệm cũng có mặt trái. Kinh nghiệm quá khứ có thể trở thành định kiến, khiến lãnh đạo áp dụng các giải pháp cũ cho bối cảnh mới mà không nhận ra môi trường đã thay đổi.

Trực giác là yếu tố gây tranh cãi nhất nhưng không thể loại bỏ trong quyết định lãnh đạo. Trực giác không phải là cảm xúc bộc phát, mà là kết quả của việc não bộ tổng hợp nhanh những trải nghiệm đã tích lũy. Trong các tình huống thiếu dữ liệu, áp lực thời gian cao hoặc khi các phương án đều có rủi ro tương đương, trực giác giúp lãnh đạo đưa ra lựa chọn dứt khoát. Tuy nhiên, trực giác chỉ đáng tin khi được xây dựng trên nền tảng trải nghiệm đủ sâu và khả năng tự nhận thức cao. Trực giác thiếu kiểm chứng dễ bị chi phối bởi cái tôi và cảm xúc cá nhân.

Sai lầm phổ biến là tuyệt đối hóa một yếu tố và xem nhẹ các yếu tố còn lại. Có lãnh đạo chỉ tin vào số liệu và mô hình, dẫn đến quyết định chậm và thiếu linh hoạt. Có người lại viện dẫn kinh nghiệm để bác bỏ mọi phân tích mới, khiến tổ chức trì trệ. Cũng có trường hợp lạm dụng trực giác để hợp thức hóa các quyết định chủ quan. Cả ba cách tiếp cận này đều tiềm ẩn rủi ro lớn.

Cân bằng hiệu quả đòi hỏi lãnh đạo phải ý thức rõ mình đang thiên về yếu tố nào. Khi logic cho thấy một phương án hợp lý nhưng trực giác cảnh báo rủi ro, đó là tín hiệu cần xem xét lại giả định hoặc tìm thêm góc nhìn. Khi kinh nghiệm thúc đẩy một lựa chọn quen thuộc, lãnh đạo cần tự hỏi liệu bối cảnh hiện tại còn tương đồng với quá khứ hay không. Khi trực giác thôi thúc quyết định nhanh, cần xác định rõ mức độ rủi ro có thể chấp nhận.

Trong thực tế, quyết định lãnh đạo tốt thường bắt đầu bằng logic, được kiểm chứng bằng kinh nghiệm và được chốt lại bằng trực giác. Sự kết hợp này không bảo đảm luôn đúng, nhưng giúp quyết định có chiều sâu, phù hợp bối cảnh và đủ dứt khoát để hành động. Đây là năng lực chỉ có thể rèn luyện qua thời gian, va chạm và sự phản tỉnh liên tục của người lãnh đạo.

\section{Quản lý rủi ro và chịu trách nhiệm với quyết định}

Không có quyết định lãnh đạo nào hoàn toàn không rủi ro. Sự khác biệt giữa lãnh đạo hiệu quả và lãnh đạo kém không nằm ở việc tránh rủi ro, mà ở cách nhận diện, quản lý và chịu trách nhiệm đối với rủi ro đó. Một tổ chức chỉ có thể vận hành ổn định khi lãnh đạo nhìn thẳng vào rủi ro thay vì né tránh hoặc che giấu.

Quản lý rủi ro bắt đầu từ việc thừa nhận rủi ro là yếu tố tất yếu của ra quyết định. Mọi phương án đều có mặt được và mặt mất. Lãnh đạo cần xác định rõ rủi ro trọng yếu, tức là những rủi ro có khả năng gây tác động lớn nhất đến mục tiêu cốt lõi của tổ chức. Việc liệt kê quá nhiều rủi ro nhỏ thường làm loãng sự tập trung và không giúp cải thiện chất lượng quyết định.

Một nguyên tắc quan trọng là phân biệt giữa rủi ro có thể chấp nhận và rủi ro không thể chấp nhận. Rủi ro có thể chấp nhận là những tổn thất mà tổ chức có khả năng chịu đựng và phục hồi. Rủi ro không thể chấp nhận là những rủi ro đe dọa sự tồn tại, uy tín hoặc nền tảng đạo đức của tổ chức. Việc xác định ranh giới này giúp lãnh đạo ra quyết định rõ ràng hơn và tránh những lựa chọn vượt quá năng lực chịu đựng thực tế.

Quản lý rủi ro không đồng nghĩa với việc phải có phương án dự phòng cho mọi tình huống. Thay vào đó, lãnh đạo cần chuẩn bị cho các kịch bản xấu nhất có khả năng xảy ra và xác định trước các điểm kích hoạt để điều chỉnh quyết định. Điều này cho phép tổ chức phản ứng nhanh khi bối cảnh thay đổi, thay vì bị động và lúng túng.

Trách nhiệm là phần không thể tách rời của quản lý rủi ro. Khi quyết định dẫn đến kết quả không mong muốn, lãnh đạo cần chịu trách nhiệm một cách rõ ràng và minh bạch. Việc nhận trách nhiệm không làm suy yếu vị thế lãnh đạo, mà ngược lại, củng cố niềm tin của đội ngũ và tạo nền tảng cho văn hóa học hỏi. Đổ lỗi cho cá nhân hoặc hoàn cảnh chỉ tạo ra sự phòng thủ và che giấu vấn đề.

Chịu trách nhiệm cũng đồng nghĩa với việc theo dõi và đánh giá quyết định sau khi triển khai. Nhiều quyết định thất bại không phải vì lựa chọn ban đầu sai, mà vì thiếu sự giám sát và điều chỉnh kịp thời. Lãnh đạo cần thiết lập các mốc kiểm tra rõ ràng để đánh giá liệu quyết định còn phù hợp hay không, và sẵn sàng điều chỉnh khi giả định ban đầu không còn đúng.

Cuối cùng, quản lý rủi ro và chịu trách nhiệm đòi hỏi sự can đảm lãnh đạo. Can đảm để đưa ra quyết định trong điều kiện không chắc chắn, can đảm để chấp nhận hậu quả, và can đảm để sửa sai khi cần thiết. Một quyết định sai nhưng được xử lý có trách nhiệm sẽ tạo ra giá trị lâu dài cho tổ chức hơn một quyết định đúng nhưng bị bỏ mặc. Đây là điểm kết thúc logic của chương, khẳng định rằng ra quyết định không chỉ là hành động lựa chọn, mà là cam kết lãnh đạo đến cùng với lựa chọn đó.
