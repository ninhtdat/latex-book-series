\chapter{Bản chất của lãnh đạo}

Lãnh đạo là yếu tố quyết định khả năng tồn tại và phát triển dài hạn của tổ chức. Trong thực tế, nhiều tổ chức được quản lý tốt nhưng vẫn thất bại vì thiếu lãnh đạo đúng nghĩa. Nguyên nhân chủ yếu xuất phát từ việc hiểu sai bản chất của lãnh đạo, coi lãnh đạo là quyền lực hành chính hoặc tập hợp các kỹ năng điều hành. Chương này tập trung làm rõ nền tảng khái niệm, giúp người đọc hình thành tư duy đúng về lãnh đạo trước khi tiếp cận các nội dung chuyên sâu hơn.

\section{Khái niệm lãnh đạo và mục đích tồn tại của lãnh đạo}

Lãnh đạo không phải là một chức danh, mà là một quá trình tạo ảnh hưởng. Quá trình này diễn ra khi một cá nhân hoặc một nhóm có khả năng định hướng nhận thức, hành vi và quyết định của người khác nhằm đạt được mục tiêu chung. Điểm phân biệt quan trọng của lãnh đạo nằm ở tính tự nguyện: người khác đi theo không vì bị ép buộc, mà vì họ tin vào phương hướng và người dẫn dắt.

Từ góc nhìn này, lãnh đạo có thể xuất hiện ở bất kỳ đâu trong tổ chức, không phụ thuộc hoàn toàn vào cơ cấu quyền lực chính thức. Một cá nhân không giữ vị trí quản lý vẫn có thể đóng vai trò lãnh đạo nếu họ tạo được ảnh hưởng tích cực và bền vững. Ngược lại, một người nắm quyền nhưng không tạo ra ảnh hưởng thực chất thì chỉ đang thực thi quyền hạn, chưa thực hiện đúng vai trò lãnh đạo.

Mục đích tồn tại của lãnh đạo gắn liền với nhu cầu căn bản của con người và tổ chức. Trước hết là nhu cầu về phương hướng. Trong môi trường nhiều biến động, tổ chức luôn đứng trước các lựa chọn chiến lược và rủi ro. Lãnh đạo tồn tại để trả lời câu hỏi tổ chức đang đi đâu và vì sao phải đi theo hướng đó. Khi thiếu phương hướng rõ ràng, nguồn lực bị phân tán, con người mất niềm tin và tổ chức rơi vào trạng thái phản ứng thụ động.

Tiếp theo là nhu cầu về sự gắn kết. Một tổ chức không chỉ là tập hợp các cá nhân làm việc cùng nhau, mà là một cộng đồng cùng theo đuổi mục tiêu chung. Lãnh đạo có nhiệm vụ kết nối mục tiêu tổ chức với động cơ cá nhân, giúp mỗi người hiểu vai trò của mình trong bức tranh tổng thể. Khi con người hiểu được ý nghĩa công việc, họ sẵn sàng cam kết và nỗ lực vượt mức yêu cầu tối thiểu.

Ngoài ra, lãnh đạo còn tồn tại để thúc đẩy sự thay đổi và phát triển. Môi trường kinh doanh và xã hội không ngừng biến đổi, khiến những mô hình thành công trong quá khứ nhanh chóng trở nên lỗi thời. Lãnh đạo không chỉ duy trì trật tự hiện có, mà còn phải thách thức các giả định cũ, khuyến khích học hỏi và dẫn dắt tổ chức thích nghi. Đây là chức năng mà quản lý thuần túy không thể thay thế.

Một khía cạnh quan trọng khác của mục đích lãnh đạo là trách nhiệm tạo ra chuẩn mực. Thông qua hành vi, quyết định và cách ứng xử, người lãnh đạo xác lập những giá trị được chấp nhận trong tổ chức. Những giá trị này ảnh hưởng trực tiếp đến văn hóa, đạo đức và cách tổ chức phản ứng trước áp lực. Lãnh đạo vì vậy không chỉ chịu trách nhiệm về kết quả, mà còn về cách thức đạt được kết quả đó.

Tóm lại, lãnh đạo tồn tại vì tổ chức cần phương hướng, sự gắn kết và khả năng phát triển bền vững. Khi lãnh đạo được hiểu đúng và thực hiện đúng, nó trở thành lực đẩy giúp tổ chức vượt qua giới hạn hiện tại. Khi lãnh đạo bị hiểu sai hoặc xem nhẹ, tổ chức dù vận hành trơn tru vẫn có nguy cơ mất phương hướng và suy yếu trong dài hạn.

\section{Sự khác biệt cốt lõi giữa lãnh đạo và quản lý}

Lãnh đạo và quản lý thường được sử dụng thay thế cho nhau trong thực tế, nhưng về bản chất đây là hai chức năng khác nhau, phục vụ những mục tiêu khác nhau trong tổ chức. Việc không phân biệt rõ hai khái niệm này dẫn đến hệ quả phổ biến: tổ chức vận hành ổn định trong ngắn hạn nhưng thiếu sức bật trong dài hạn, hoặc ngược lại, có nhiều ý tưởng nhưng không thể triển khai hiệu quả.

Quản lý tập trung vào việc duy trì trật tự và hiệu quả vận hành. Chức năng cốt lõi của quản lý bao gồm lập kế hoạch, phân bổ nguồn lực, tổ chức công việc, giám sát tiến độ và kiểm soát kết quả. Mục tiêu của quản lý là đảm bảo hệ thống hoạt động đúng như thiết kế, giảm thiểu sai lệch và rủi ro trong quá trình thực thi. Quản lý trả lời câu hỏi “làm như thế nào” và “làm theo quy trình nào”.

Ngược lại, lãnh đạo tập trung vào việc tạo ra phương hướng và thay đổi. Lãnh đạo đặt ra tầm nhìn, xác định ưu tiên chiến lược và truyền cảm hứng để con người sẵn sàng đi theo hướng mới. Nếu quản lý nhấn mạnh vào tính ổn định và khả năng dự đoán, thì lãnh đạo chấp nhận sự bất định và chủ động đối diện với rủi ro. Lãnh đạo trả lời câu hỏi “đi đâu”, “vì sao phải đi” và “điều gì thực sự quan trọng”.

Một khác biệt cốt lõi khác nằm ở cách sử dụng quyền lực. Quản lý chủ yếu dựa vào quyền lực chính thức gắn với chức vụ, quy định và quy trình. Lãnh đạo, trong khi đó, dựa nhiều hơn vào ảnh hưởng cá nhân, uy tín và niềm tin. Người quản lý có thể khiến nhân viên tuân thủ, nhưng người lãnh đạo khiến họ cam kết. Sự khác biệt này đặc biệt quan trọng trong bối cảnh tri thức, nơi hiệu quả công việc phụ thuộc nhiều vào động lực nội tại hơn là sự giám sát.

Về thời gian và tầm nhìn, quản lý thường tập trung vào ngắn hạn và trung hạn, với các mục tiêu cụ thể, đo lường được và gắn với hiệu suất. Lãnh đạo hướng nhiều hơn đến dài hạn, đặt câu hỏi về tính bền vững và giá trị cốt lõi của tổ chức. Khi chỉ có quản lý mà thiếu lãnh đạo, tổ chức dễ rơi vào trạng thái tối ưu hóa cái hiện có nhưng bỏ lỡ cơ hội tương lai.

Tuy nhiên, cần nhấn mạnh rằng lãnh đạo và quản lý không đối lập, mà bổ sung cho nhau. Một tổ chức chỉ có lãnh đạo mà thiếu quản lý sẽ rơi vào tình trạng hỗn loạn, ý tưởng không được hiện thực hóa. Ngược lại, một tổ chức chỉ có quản lý mà thiếu lãnh đạo sẽ trở nên cứng nhắc, chậm thích nghi và mất dần sức cạnh tranh. Vấn đề không nằm ở việc chọn một trong hai, mà ở việc hiểu đúng vai trò của từng chức năng và kết hợp chúng một cách hợp lý.

Trong thực tế, nhiều người giữ vị trí quản lý bị cuốn vào công việc vận hành hàng ngày đến mức không còn thời gian và năng lượng cho vai trò lãnh đạo. Điều này dẫn đến hiện tượng “quản lý quá mức, lãnh đạo thiếu hụt”. Nhận thức rõ sự khác biệt giữa lãnh đạo và quản lý là bước đầu để người quản lý chuyển dịch tư duy, từng bước đảm nhận vai trò lãnh đạo đúng nghĩa.

\section{Ảnh hưởng, quyền lực và trách nhiệm của người lãnh đạo}

Ảnh hưởng là nền tảng cốt lõi của lãnh đạo. Không có ảnh hưởng, mọi quyền hạn chính thức đều trở nên hạn chế và ngắn hạn. Ảnh hưởng trong lãnh đạo được hiểu là khả năng tác động đến suy nghĩ, thái độ và hành vi của người khác theo hướng tự nguyện, không cần đến ép buộc hay giám sát liên tục. Mức độ ảnh hưởng càng lớn thì năng lực lãnh đạo càng cao, bất kể chức danh hay vị trí trong cơ cấu tổ chức.

Quyền lực của người lãnh đạo là nguồn tạo ra ảnh hưởng, nhưng không phải mọi quyền lực đều có giá trị như nhau. Quyền lực chính thức xuất phát từ chức vụ, vai trò và thẩm quyền được trao. Loại quyền lực này cần thiết để đảm bảo trật tự và kỷ luật, nhưng chỉ đủ để yêu cầu sự tuân thủ, không đủ để tạo ra cam kết. Trong khi đó, quyền lực cá nhân xuất phát từ năng lực chuyên môn, uy tín, giá trị đạo đức và cách ứng xử nhất quán. Đây là nguồn quyền lực bền vững, tạo ra sự tin tưởng và sẵn sàng đi theo trong dài hạn.

Một người lãnh đạo hiệu quả hiểu rằng quyền lực không phải là công cụ để kiểm soát con người, mà là phương tiện để phục vụ mục tiêu chung. Việc lạm dụng quyền lực chính thức có thể mang lại kết quả nhanh trong ngắn hạn, nhưng sẽ làm xói mòn niềm tin và động lực nội tại của đội ngũ. Ngược lại, việc xây dựng quyền lực cá nhân đòi hỏi thời gian và kỷ luật, nhưng tạo ra ảnh hưởng sâu và ổn định.

Đi liền với ảnh hưởng và quyền lực là trách nhiệm. Trách nhiệm đầu tiên của người lãnh đạo là chịu trách nhiệm cho các quyết định và hệ quả của chúng. Trong bối cảnh phức tạp, không phải mọi quyết định đều mang lại kết quả như mong đợi, nhưng người lãnh đạo không được né tránh hay đổ lỗi. Việc dám chịu trách nhiệm là nền tảng của uy tín và niềm tin.

Trách nhiệm thứ hai là trách nhiệm đối với con người. Lãnh đạo không chỉ tối ưu hóa hiệu suất, mà còn phải phát triển năng lực và tiềm năng của đội ngũ. Điều này bao gồm việc tạo môi trường an toàn để học hỏi, cho phép sai lầm có kiểm soát và khuyến khích phản biện. Một người lãnh đạo chỉ tập trung vào kết quả mà bỏ qua con người sẽ tạo ra tổ chức kiệt sức và mong manh.

Trách nhiệm thứ ba là trách nhiệm đối với giá trị và chuẩn mực. Thông qua hành vi hàng ngày, người lãnh đạo xác lập ranh giới giữa điều được chấp nhận và không được chấp nhận trong tổ chức. Những quyết định trong tình huống khó khăn, đặc biệt khi phải lựa chọn giữa lợi ích ngắn hạn và giá trị dài hạn, sẽ bộc lộ rõ nhất bản chất lãnh đạo. Văn hóa tổ chức, xét đến cùng, phản ánh trực tiếp chuẩn mực mà lãnh đạo dung túng hoặc bảo vệ.

Cuối cùng, người lãnh đạo có trách nhiệm cân bằng giữa quyền lực và sự khiêm tốn. Quyền lực nếu không đi kèm tự nhận thức sẽ dẫn đến chủ quan và xa rời thực tế. Khiêm tốn không phải là yếu đuối, mà là khả năng lắng nghe, học hỏi và điều chỉnh. Lãnh đạo trưởng thành là người biết sử dụng quyền lực một cách có giới hạn, đặt lợi ích tổ chức lên trên cái tôi cá nhân.

\section{Vai trò của lãnh đạo trong việc định hướng và tạo ý nghĩa}

Một trong những vai trò quan trọng nhất của lãnh đạo là định hướng. Định hướng không chỉ là việc xác định mục tiêu hay chiến lược, mà là làm rõ tổ chức đang theo đuổi điều gì, ưu tiên điều gì và sẵn sàng đánh đổi điều gì. Trong bối cảnh nguồn lực luôn hữu hạn và môi trường nhiều biến động, việc thiếu định hướng rõ ràng sẽ khiến tổ chức rơi vào tình trạng phân tán, mỗi bộ phận theo đuổi một mục tiêu riêng, làm suy yếu sức mạnh tổng thể.

Lãnh đạo chịu trách nhiệm chuyển hóa tầm nhìn trừu tượng thành định hướng cụ thể và có thể hành động. Điều này đòi hỏi khả năng đơn giản hóa các vấn đề phức tạp, xác định trọng tâm và truyền đạt một cách nhất quán. Một định hướng tốt không cần quá chi tiết, nhưng phải đủ rõ để mọi người hiểu mình cần làm gì và không cần làm gì. Khi định hướng liên tục thay đổi hoặc mâu thuẫn, niềm tin của đội ngũ sẽ suy giảm và hiệu quả tổ chức bị ảnh hưởng nghiêm trọng.

Bên cạnh định hướng, lãnh đạo còn giữ vai trò tạo ra ý nghĩa cho công việc. Con người không chỉ làm việc vì lương thưởng hay nghĩa vụ, mà còn vì mong muốn công việc của mình có giá trị. Lãnh đạo giúp trả lời câu hỏi “vì sao công việc này quan trọng” và “đóng góp của tôi tạo ra tác động gì”. Khi ý nghĩa được làm rõ, động lực nội tại được kích hoạt, giúp con người duy trì nỗ lực ngay cả trong điều kiện khó khăn.

Việc tạo ý nghĩa không đến từ những khẩu hiệu chung chung, mà từ sự liên kết nhất quán giữa mục tiêu, hành động và giá trị. Người lãnh đạo cần chỉ ra mối liên hệ giữa công việc hàng ngày và mục tiêu dài hạn của tổ chức, đồng thời công nhận những đóng góp cụ thể của từng cá nhân. Sự công nhận đúng lúc và đúng mức giúp củng cố cảm nhận về giá trị và vai trò của mỗi người trong tập thể.

Một khía cạnh quan trọng khác của vai trò này là giúp con người hiểu và chấp nhận sự thay đổi. Thay đổi thường đi kèm với bất ổn và kháng cự. Lãnh đạo không thể loại bỏ hoàn toàn cảm giác này, nhưng có thể giảm thiểu nó bằng cách giải thích rõ lý do thay đổi, lợi ích dài hạn và tác động cụ thể đến từng nhóm liên quan. Khi con người hiểu được ý nghĩa của thay đổi, họ sẵn sàng hợp tác thay vì chống đối.

Cuối cùng, việc định hướng và tạo ý nghĩa đòi hỏi sự nhất quán giữa lời nói và hành động của người lãnh đạo. Mọi thông điệp sẽ trở nên vô nghĩa nếu bị phủ nhận bởi hành vi thực tế. Người lãnh đạo không chỉ nói về giá trị và mục tiêu, mà phải thể hiện chúng trong các quyết định khó khăn, đặc biệt khi phải đánh đổi lợi ích ngắn hạn. Chính sự nhất quán này tạo ra niềm tin, và niềm tin là nền tảng để định hướng và ý nghĩa được duy trì trong dài hạn.

\section{Lãnh đạo trước những thay đổi của môi trường hiện đại}

Môi trường hiện đại đặc trưng bởi tốc độ thay đổi nhanh, mức độ phức tạp cao và tính bất định ngày càng lớn. Những yếu tố như toàn cầu hóa, tiến bộ công nghệ, sự thay đổi trong hành vi và kỳ vọng của con người khiến các mô hình quản lý truyền thống dần mất hiệu quả. Trong bối cảnh này, vai trò của lãnh đạo không còn là duy trì trạng thái ổn định, mà là dẫn dắt tổ chức thích nghi và tiến hóa liên tục.

Một yêu cầu then chốt đối với lãnh đạo hiện đại là năng lực tư duy linh hoạt. Những kế hoạch dài hạn cứng nhắc dễ nhanh chóng trở nên lỗi thời. Người lãnh đạo cần chấp nhận rằng không phải mọi vấn đề đều có dữ liệu đầy đủ hay câu trả lời chắc chắn. Thay vì tìm kiếm sự hoàn hảo, lãnh đạo cần đưa ra quyết định dựa trên thông tin tốt nhất hiện có, đồng thời sẵn sàng điều chỉnh khi bối cảnh thay đổi. Khả năng học hỏi nhanh và thay đổi quan điểm khi cần thiết trở thành năng lực cốt lõi.

Bên cạnh đó, lãnh đạo trước thay đổi đòi hỏi sự quyết đoán có trách nhiệm. Sự chần chừ trong môi trường biến động thường gây ra rủi ro lớn hơn so với việc hành động và điều chỉnh sau. Tuy nhiên, quyết đoán không đồng nghĩa với độc đoán. Người lãnh đạo cần cân bằng giữa việc lắng nghe đa chiều và việc chịu trách nhiệm đưa ra quyết định cuối cùng. Khi trách nhiệm được gánh vác rõ ràng, tổ chức có thể tiến lên thay vì mắc kẹt trong tranh luận kéo dài.

Thay đổi cũng đặt ra thách thức lớn về con người. Áp lực thích nghi liên tục dễ dẫn đến căng thẳng, mệt mỏi và kháng cự. Lãnh đạo hiện đại phải nhận thức được tác động tâm lý của thay đổi, từ đó xây dựng môi trường an toàn về mặt tinh thần, nơi con người được phép đặt câu hỏi, bày tỏ lo ngại và học hỏi từ sai lầm. Việc bỏ qua yếu tố con người sẽ khiến mọi nỗ lực thay đổi trở nên mong manh và khó bền vững.

Ngoài ra, lãnh đạo cần tái định nghĩa khái niệm kiểm soát. Trong môi trường phức tạp, việc kiểm soát chi tiết không còn khả thi và phản tác dụng. Thay vào đó, lãnh đạo cần tập trung vào việc thiết lập nguyên tắc, giá trị và mục tiêu rõ ràng, trao quyền cho đội ngũ tự đưa ra quyết định trong phạm vi cho phép. Cách tiếp cận này giúp tổ chức phản ứng nhanh hơn và khai thác tốt hơn trí tuệ tập thể.

Cuối cùng, lãnh đạo trước những thay đổi của môi trường hiện đại là lãnh đạo bằng tầm nhìn dài hạn và sự kiên định về giá trị. Công cụ, mô hình và chiến lược có thể thay đổi, nhưng giá trị cốt lõi và mục tiêu căn bản của tổ chức cần được giữ vững. Chính sự ổn định về giá trị này tạo ra điểm tựa cho con người trong bối cảnh bất định, đồng thời giúp tổ chức duy trì bản sắc và hướng phát triển bền vững.
