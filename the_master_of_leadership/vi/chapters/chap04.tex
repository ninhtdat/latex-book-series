\chapter{Tầm nhìn và định hướng chiến lược}

Trong bối cảnh môi trường kinh doanh và tổ chức thay đổi nhanh chóng, năng lực lãnh đạo không còn được đánh giá chủ yếu qua khả năng điều hành công việc hàng ngày, mà qua khả năng định hướng tương lai. Tầm nhìn trở thành nền tảng cho mọi quyết định chiến lược, giúp tổ chức duy trì phương hướng, tạo động lực và phát triển bền vững. Chương này tập trung làm rõ vai trò cốt lõi của tầm nhìn trong lãnh đạo và cách tầm nhìn chi phối toàn bộ quá trình hoạch định và thực thi chiến lược.

\section{Vai trò của tầm nhìn trong lãnh đạo}

Tầm nhìn là hình ảnh rõ ràng về trạng thái mong muốn của tổ chức trong tương lai, phản ánh khát vọng phát triển và định hướng dài hạn. Đối với nhà lãnh đạo, tầm nhìn không chỉ là một tuyên bố mang tính truyền thông, mà là công cụ quản trị chiến lược có ảnh hưởng trực tiếp đến tư duy, hành vi và quyết định ở mọi cấp độ.

Vai trò đầu tiên của tầm nhìn là định hướng ra quyết định. Trong thực tế, lãnh đạo phải liên tục đối mặt với các lựa chọn phức tạp, thường xuyên chịu áp lực từ thời gian, nguồn lực và lợi ích ngắn hạn. Một tầm nhìn rõ ràng giúp lãnh đạo có tiêu chí nhất quán để đánh giá các phương án, từ đó lựa chọn những quyết định phù hợp với mục tiêu dài hạn, ngay cả khi phải chấp nhận hy sinh lợi ích trước mắt. Không có tầm nhìn, quyết định dễ trở nên cảm tính, phản ứng bị động và thiếu sự liên kết chiến lược.

Thứ hai, tầm nhìn tạo ra ý nghĩa và động lực cho tổ chức. Con người có xu hướng cam kết mạnh mẽ hơn khi họ hiểu vì sao công việc của mình có giá trị và đóng góp vào điều gì lớn hơn bản thân họ. Tầm nhìn giúp chuyển đổi công việc hàng ngày từ những nhiệm vụ rời rạc thành một phần của hành trình chung. Khi nhân sự nhìn thấy tương lai mà tổ chức đang hướng tới và vai trò của mình trong bức tranh đó, mức độ gắn kết, tinh thần trách nhiệm và sự chủ động sẽ được nâng cao đáng kể.

Thứ ba, tầm nhìn là nền tảng để xây dựng và lựa chọn chiến lược. Chiến lược trả lời câu hỏi “làm thế nào để đạt được tầm nhìn”, do đó không thể tách rời khỏi tầm nhìn. Một tổ chức có tầm nhìn rõ ràng sẽ dễ dàng xác định các ưu tiên chiến lược, tập trung nguồn lực vào những lĩnh vực tạo ra giá trị dài hạn và tránh sa đà vào các sáng kiến không phục vụ mục tiêu chung. Ngược lại, khi tầm nhìn mơ hồ hoặc thay đổi liên tục, chiến lược sẽ thiếu nhất quán, dẫn đến lãng phí nguồn lực và suy giảm hiệu quả.

Thứ tư, tầm nhìn đóng vai trò gắn kết và thống nhất hành động trong toàn tổ chức. Trong các tổ chức lớn hoặc đa dạng về chức năng, nguy cơ xung đột mục tiêu giữa các bộ phận là rất cao. Tầm nhìn chung giúp các đơn vị có cùng điểm tham chiếu khi phối hợp, giảm mâu thuẫn lợi ích và tăng cường tính đồng bộ trong triển khai. Khi mọi cấp độ đều hiểu và chia sẻ tầm nhìn, tổ chức có khả năng vận hành như một chỉnh thể thống nhất thay vì tập hợp rời rạc các bộ phận.

Cuối cùng, tầm nhìn phản ánh năng lực và bản lĩnh của nhà lãnh đạo. Lãnh đạo không chỉ chịu trách nhiệm cho kết quả hiện tại, mà còn cho tương lai của tổ chức. Một tầm nhìn rõ ràng, thực tế và có khả năng truyền cảm hứng cho thấy lãnh đạo hiểu sâu sắc bối cảnh, năng lực cốt lõi và hướng phát triển dài hạn. Ngược lại, việc thiếu tầm nhìn hoặc chỉ dừng ở những khẩu hiệu chung chung thường là dấu hiệu của tư duy ngắn hạn và hạn chế trong vai trò lãnh đạo chiến lược.

Tuy nhiên, cần nhấn mạnh rằng tầm nhìn chỉ thực sự có giá trị khi gắn liền với hành động. Một tầm nhìn hay trên giấy nhưng không được sử dụng làm cơ sở cho quyết định, chiến lược và phân bổ nguồn lực sẽ nhanh chóng mất đi ý nghĩa. Do đó, vai trò của tầm nhìn trong lãnh đạo không nằm ở việc tuyên bố, mà ở việc lãnh đạo kiên định sử dụng tầm nhìn như kim chỉ nam cho toàn bộ hoạt động của tổ chức.

\section{Phương pháp xây dựng tầm nhìn rõ ràng và thực tế}

Xây dựng tầm nhìn là một quá trình mang tính chiến lược, đòi hỏi tư duy hệ thống và sự hiểu biết sâu sắc về tổ chức, môi trường và con người. Một tầm nhìn hiệu quả không xuất phát từ cảm hứng nhất thời hay mong muốn chủ quan của lãnh đạo, mà phải được hình thành trên cơ sở phân tích thực tế và định hướng dài hạn rõ ràng.

Bước đầu tiên trong xây dựng tầm nhìn là phân tích bối cảnh một cách toàn diện. Nhà lãnh đạo cần đánh giá môi trường bên ngoài, bao gồm xu hướng thị trường, công nghệ, cạnh tranh, chính sách và các yếu tố xã hội có ảnh hưởng trực tiếp đến tổ chức. Song song với đó là việc nhìn thẳng vào nội lực: năng lực cốt lõi, điểm mạnh, điểm yếu, văn hóa tổ chức và mức độ sẵn sàng thay đổi. Tầm nhìn không thể tách rời bối cảnh; nếu bỏ qua yếu tố này, tầm nhìn sẽ hoặc quá viển vông, hoặc quá an toàn và thiếu sức dẫn dắt.

Bước thứ hai là xác định rõ giá trị cốt lõi và bản sắc của tổ chức. Tầm nhìn không chỉ nói về việc tổ chức sẽ đạt được điều gì, mà còn phản ánh tổ chức là ai và theo đuổi điều gì. Việc làm rõ giá trị cốt lõi giúp tầm nhìn có chiều sâu và tính nhất quán, đồng thời tạo ra ranh giới rõ ràng cho các lựa chọn chiến lược trong tương lai. Một tầm nhìn tốt phải phù hợp với bản sắc tổ chức, nếu không sẽ khó được chấp nhận và duy trì trong dài hạn.

Bước thứ ba là xác định trạng thái tương lai mong muốn một cách cụ thể. Thay vì những câu chữ chung chung, lãnh đạo cần trả lời rõ ràng các câu hỏi: tổ chức muốn đứng ở vị trí nào, phục vụ đối tượng nào, tạo ra giá trị gì khác biệt và ở quy mô ra sao trong một khoảng thời gian xác định. Việc mô tả tương lai càng cụ thể thì tầm nhìn càng dễ chuyển hóa thành mục tiêu và chiến lược. Tầm nhìn không cần quá chi tiết, nhưng phải đủ rõ để mọi người có thể hình dung và định hướng hành động.

Bước thứ tư là kiểm tra tính khả thi của tầm nhìn. Một tầm nhìn thực tế không có nghĩa là dễ dàng, mà là có khả năng đạt được nếu tổ chức tập trung và thực thi đúng cách. Lãnh đạo cần đánh giá khoảng cách giữa hiện tại và tương lai mong muốn, từ đó xác định những thay đổi lớn về năng lực, cấu trúc, con người hoặc nguồn lực cần thiết. Việc kiểm tra này giúp điều chỉnh tầm nhìn ở mức thách thức nhưng không vượt quá khả năng thực tế của tổ chức.

Bước thứ năm là chuẩn hóa tầm nhìn thành thông điệp ngắn gọn, rõ ràng và nhất quán. Một tầm nhìn hiệu quả thường được thể hiện bằng một hoặc hai câu súc tích, dễ nhớ và có tính định hướng hành động. Ngôn ngữ sử dụng cần đơn giản, tránh thuật ngữ mơ hồ hoặc quá kỹ thuật. Mục tiêu của bước này là đảm bảo tầm nhìn có thể được truyền đạt nhất quán ở mọi cấp độ, từ lãnh đạo cấp cao đến nhân sự tuyến đầu.

Cuối cùng, cần nhấn mạnh rằng xây dựng tầm nhìn không phải là công việc làm một lần rồi kết thúc. Tầm nhìn cần được rà soát định kỳ để đảm bảo vẫn phù hợp với bối cảnh và định hướng phát triển, nhưng không nên thay đổi tùy tiện. Sự ổn định của tầm nhìn tạo ra niềm tin và sự nhất quán, trong khi việc điều chỉnh có kiểm soát giúp tổ chức thích ứng với thay đổi. Nhà lãnh đạo hiệu quả là người biết cân bằng giữa tính kiên định và sự linh hoạt trong quá trình xây dựng và duy trì tầm nhìn.

\section{Truyền đạt tầm nhìn để tạo sự đồng thuận}

Một tầm nhìn dù được xây dựng tốt đến đâu cũng sẽ không tạo ra giá trị nếu không được truyền đạt hiệu quả. Trong thực tế, thất bại phổ biến của nhiều tổ chức không nằm ở việc thiếu tầm nhìn, mà ở việc tầm nhìn không được hiểu đúng, không được tin tưởng và không được chuyển hóa thành hành động thống nhất. Vì vậy, truyền đạt tầm nhìn là một năng lực lãnh đạo mang tính quyết định.

Nguyên tắc đầu tiên trong truyền đạt tầm nhìn là làm rõ ý nghĩa, không chỉ nội dung. Nhân sự không cam kết với những câu chữ trừu tượng, họ cam kết với ý nghĩa đằng sau những câu chữ đó. Nhà lãnh đạo cần trả lời rõ ràng câu hỏi “vì sao tầm nhìn này quan trọng” và “nếu đạt được, tổ chức và cá nhân sẽ thay đổi như thế nào”. Khi tầm nhìn được gắn với lợi ích cụ thể, cả ở cấp tổ chức lẫn cấp cá nhân, mức độ đồng thuận sẽ cao hơn đáng kể.

Nguyên tắc thứ hai là lặp lại có chủ đích và nhất quán. Truyền đạt tầm nhìn không phải là một sự kiện, mà là một quá trình liên tục. Lãnh đạo cần lặp lại tầm nhìn trong nhiều bối cảnh khác nhau: họp chiến lược, đánh giá kết quả, ra quyết định quan trọng, và cả trong giao tiếp thường ngày. Sự lặp lại không nhằm nhồi nhét thông tin, mà nhằm củng cố nhận thức và tạo sự quen thuộc. Khi nhân sự có thể tự diễn đạt lại tầm nhìn bằng ngôn ngữ của họ, quá trình truyền đạt mới thực sự hiệu quả.

Nguyên tắc thứ ba là sử dụng nhiều kênh và hình thức truyền thông phù hợp. Mỗi nhóm trong tổ chức có đặc điểm, mối quan tâm và mức độ tiếp nhận khác nhau. Truyền đạt tầm nhìn chỉ qua văn bản hoặc bài phát biểu một chiều thường không đủ. Lãnh đạo cần kết hợp giữa giao tiếp trực tiếp, đối thoại hai chiều, ví dụ thực tế và câu chuyện cụ thể. Việc minh họa tầm nhìn bằng các tình huống gần gũi giúp nhân sự hiểu rõ hơn mối liên hệ giữa tầm nhìn và công việc hàng ngày của họ.

Nguyên tắc thứ tư là gắn tầm nhìn với hành vi và quyết định cụ thể của lãnh đạo. Nhân sự không đánh giá mức độ nghiêm túc của tầm nhìn qua lời nói, mà qua hành động. Nếu các quyết định quan trọng về nhân sự, đầu tư hay ưu tiên công việc không phản ánh tầm nhìn đã công bố, niềm tin sẽ nhanh chóng suy giảm. Ngược lại, khi lãnh đạo nhất quán trong hành động, tầm nhìn sẽ trở thành chuẩn mực ngầm định chi phối hành vi trong toàn tổ chức.

Nguyên tắc thứ năm là khuyến khích sự tham gia và phản hồi. Truyền đạt tầm nhìn không đồng nghĩa với áp đặt. Để tạo sự đồng thuận thực sự, lãnh đạo cần tạo không gian cho đối thoại, lắng nghe phản hồi và giải đáp những băn khoăn, nghi ngại. Quá trình này giúp làm rõ những hiểu lầm, đồng thời cho phép điều chỉnh cách diễn đạt tầm nhìn sao cho phù hợp hơn với thực tế vận hành. Khi nhân sự cảm thấy họ được lắng nghe và có vai trò trong việc hiện thực hóa tầm nhìn, mức độ cam kết sẽ tăng lên rõ rệt.

Cuối cùng, cần phân biệt rõ giữa sự đồng thuận và sự đồng ý hình thức. Đồng thuận thực sự thể hiện ở việc các cá nhân chủ động điều chỉnh hành vi và quyết định của mình theo tầm nhìn chung, ngay cả khi không có sự giám sát trực tiếp. Đạt được mức độ này đòi hỏi thời gian, sự kiên trì và tính nhất quán cao từ phía lãnh đạo. Truyền đạt tầm nhìn, vì vậy, không phải là nhiệm vụ phụ trợ, mà là một phần không thể tách rời của vai trò lãnh đạo chiến lược.

\section{Chuyển hóa tầm nhìn thành mục tiêu chiến lược}

Tầm nhìn chỉ thực sự có giá trị khi được chuyển hóa thành các mục tiêu chiến lược cụ thể và có thể đo lường. Đây là bước trung gian quan trọng, nối giữa tư duy định hướng dài hạn và hoạt động quản trị hàng ngày. Nếu bỏ qua hoặc làm sơ sài bước này, tầm nhìn sẽ dừng lại ở mức khẩu hiệu, không đủ sức chi phối hành động của tổ chức.

Nguyên tắc đầu tiên trong chuyển hóa tầm nhìn là phân rã tầm nhìn thành các chủ đề chiến lược. Tầm nhìn thường mang tính khái quát, phản ánh trạng thái mong muốn trong tương lai. Để triển khai, lãnh đạo cần xác định các trụ cột lớn quyết định việc đạt được tầm nhìn, chẳng hạn như tăng trưởng thị trường, chất lượng sản phẩm, năng lực nhân sự, hiệu quả vận hành hoặc trải nghiệm khách hàng. Các chủ đề chiến lược này đóng vai trò làm cầu nối, giúp tầm nhìn trở nên gần với thực tiễn quản trị hơn.

Nguyên tắc thứ hai là xác định mục tiêu chiến lược rõ ràng cho từng chủ đề. Mục tiêu chiến lược trả lời câu hỏi “chúng ta cần đạt được điều gì trong một giai đoạn xác định để tiến gần hơn đến tầm nhìn”. Các mục tiêu này phải đủ cụ thể để định hướng hành động, nhưng vẫn mang tính chiến lược, tránh sa vào các chỉ tiêu tác nghiệp ngắn hạn. Một mục tiêu chiến lược tốt thường tập trung vào kết quả then chốt, có tác động lớn và tạo ra sự khác biệt.

Nguyên tắc thứ ba là đảm bảo mục tiêu chiến lược có thể đo lường và theo dõi. Việc lượng hóa mục tiêu không nhằm mục đích kiểm soát cứng nhắc, mà để tạo ra sự rõ ràng và minh bạch trong đánh giá tiến độ. Các chỉ số đo lường giúp lãnh đạo và tổ chức biết mình đang ở đâu trên hành trình thực hiện tầm nhìn, từ đó kịp thời điều chỉnh chiến lược khi cần thiết. Mục tiêu không đo lường được thường dẫn đến tranh cãi, cảm tính và thiếu trách nhiệm giải trình.

Nguyên tắc thứ tư là gắn mục tiêu chiến lược với thời hạn và trách nhiệm cụ thể. Mỗi mục tiêu cần có khung thời gian rõ ràng và cá nhân hoặc bộ phận chịu trách nhiệm chính. Điều này giúp tránh tình trạng mục tiêu tồn tại trên giấy nhưng không ai thực sự sở hữu. Trách nhiệm rõ ràng không nhằm tạo áp lực cá nhân, mà nhằm đảm bảo có người chủ động điều phối nguồn lực, theo dõi tiến độ và báo cáo kết quả.

Nguyên tắc thứ năm là đảm bảo tính liên kết dọc và ngang của mục tiêu. Mục tiêu chiến lược cấp tổ chức cần được liên kết với mục tiêu của các đơn vị và cá nhân, đồng thời tránh mâu thuẫn giữa các mục tiêu khác nhau. Khi mục tiêu ở các cấp độ không đồng bộ, tổ chức sẽ gặp tình trạng mỗi bộ phận tối ưu theo cách riêng, nhưng tổng thể lại không tiến gần hơn đến tầm nhìn chung.

Cuối cùng, lãnh đạo cần hiểu rằng chuyển hóa tầm nhìn thành mục tiêu chiến lược không phải là bước làm một lần rồi kết thúc. Đây là quá trình cần được rà soát định kỳ để đảm bảo mục tiêu vẫn phù hợp với bối cảnh và năng lực thực tế. Sự linh hoạt trong điều chỉnh mục tiêu, trên nền tảng tầm nhìn ổn định, là yếu tố then chốt giúp tổ chức vừa kiên định định hướng dài hạn, vừa thích ứng hiệu quả với thay đổi.

\section{Gắn kết chiến lược với nguồn lực và hành động cụ thể}

Chiến lược chỉ có giá trị khi được bảo đảm bằng nguồn lực và được triển khai thông qua hành động cụ thể. Một trong những nguyên nhân phổ biến khiến chiến lược thất bại là khoảng cách lớn giữa mục tiêu chiến lược và thực tế phân bổ nguồn lực. Khi chiến lược không được phản ánh trong ngân sách, nhân sự và ưu tiên công việc hàng ngày, tầm nhìn và mục tiêu chiến lược sẽ nhanh chóng mất đi tính thực thi.

Nguyên tắc đầu tiên là bảo đảm sự phù hợp giữa chiến lược và phân bổ nguồn lực. Nguồn lực ở đây bao gồm tài chính, con người, thời gian và sự chú ý của lãnh đạo. Những ưu tiên chiến lược quan trọng nhất phải được cấp đủ nguồn lực tương xứng. Nếu mọi sáng kiến đều được phân bổ nguồn lực ngang nhau, tổ chức sẽ rơi vào tình trạng dàn trải, không tạo ra đột phá. Lãnh đạo cần sẵn sàng đưa ra lựa chọn khó khăn, tập trung đầu tư cho một số ít ưu tiên then chốt và chấp nhận cắt giảm hoặc dừng lại những hoạt động không phục vụ chiến lược.

Nguyên tắc thứ hai là chuyển mục tiêu chiến lược thành các kế hoạch hành động cụ thể. Mỗi mục tiêu chiến lược cần được cụ thể hóa thành các chương trình, dự án hoặc sáng kiến với phạm vi rõ ràng, mốc thời gian cụ thể và kết quả đầu ra xác định. Việc này giúp chiến lược đi từ cấp độ định hướng xuống cấp độ thực thi, nơi các cá nhân và nhóm có thể hành động trực tiếp. Kế hoạch hành động càng rõ ràng thì khả năng triển khai đồng bộ và hiệu quả càng cao.

Nguyên tắc thứ ba là xác định rõ trách nhiệm và cơ chế phối hợp. Mỗi hành động chiến lược cần có người chịu trách nhiệm cuối cùng, đồng thời làm rõ vai trò phối hợp của các bộ phận liên quan. Sự mơ hồ trong trách nhiệm thường dẫn đến trì hoãn, né tránh và đùn đẩy công việc. Lãnh đạo cần thiết lập cơ chế phối hợp rõ ràng, trong đó trách nhiệm cá nhân đi kèm với quyền hạn tương ứng, nhằm bảo đảm tiến độ và chất lượng thực hiện.

Nguyên tắc thứ tư là theo dõi, đánh giá và điều chỉnh việc thực thi chiến lược. Gắn kết chiến lược với hành động không phải là quá trình tuyến tính, mà là vòng lặp liên tục giữa lập kế hoạch, thực hiện và đánh giá. Lãnh đạo cần thiết lập các điểm kiểm soát định kỳ để theo dõi tiến độ, đánh giá kết quả và kịp thời điều chỉnh khi bối cảnh thay đổi hoặc khi giả định ban đầu không còn phù hợp. Việc đánh giá cần tập trung vào cả kết quả đạt được và cách thức triển khai, nhằm rút ra bài học cho các giai đoạn tiếp theo.

Nguyên tắc thứ năm là tạo sự liên kết giữa chiến lược và hệ thống quản trị nhân sự. Các cơ chế đánh giá hiệu quả công việc, khen thưởng và phát triển nhân sự cần phản ánh rõ các ưu tiên chiến lược. Khi hành vi và kết quả phù hợp với chiến lược được ghi nhận và khuyến khích, tổ chức sẽ hình thành động lực thực thi tự nhiên. Ngược lại, nếu hệ thống đánh giá và khen thưởng tách rời chiến lược, nhân sự sẽ ưu tiên những mục tiêu ngắn hạn hoặc cá nhân, làm suy yếu định hướng chung.

Kết luận, gắn kết chiến lược với nguồn lực và hành động cụ thể là bước quyết định để biến tầm nhìn thành kết quả thực tế. Đây là thước đo rõ ràng nhất cho năng lực lãnh đạo chiến lược. Lãnh đạo hiệu quả không chỉ biết xác định hướng đi, mà còn bảo đảm tổ chức có đủ nguồn lực, cơ chế và kỷ luật thực thi để đi đến đích đã lựa chọn.
