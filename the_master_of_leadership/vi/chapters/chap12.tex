\chapter{Hành trình trở thành bậc thầy lãnh đạo}

Lãnh đạo không phải là một trạng thái đạt được một lần rồi dừng lại, mà là một hành trình phát triển liên tục. Mỗi giai đoạn trong hành trình đó đòi hỏi người lãnh đạo phải tự nhìn lại, điều chỉnh tư duy và nâng cấp cách hành động của mình. Chương cuối cùng này không nhằm giới thiệu thêm khái niệm mới, mà tập trung hệ thống hóa những nguyên tắc cốt lõi đã được đề cập xuyên suốt cuốn sách, từ đó giúp người đọc có nền tảng vững chắc để tiếp tục rèn luyện và trưởng thành trên con đường lãnh đạo dài hạn.

\section{Tổng hợp các nguyên tắc lãnh đạo cốt lõi}

Mọi hình thức lãnh đạo bền vững đều được xây dựng trên một số nguyên tắc nền tảng. Những nguyên tắc này không phụ thuộc vào chức danh, ngành nghề hay quy mô tổ chức, mà gắn trực tiếp với bản chất của việc dẫn dắt con người và tạo ra kết quả.

Nguyên tắc đầu tiên là lãnh đạo bắt đầu từ trách nhiệm cá nhân. Một nhà lãnh đạo thực thụ không đổ lỗi cho hoàn cảnh, cấp dưới hay hệ thống khi kết quả không đạt kỳ vọng. Họ chấp nhận trách nhiệm cuối cùng cho các quyết định của mình, kể cả khi quyết định đó được đưa ra trong điều kiện thông tin không đầy đủ. Trách nhiệm không chỉ là nhận lỗi khi sai, mà còn là chủ động hành động trước khi vấn đề trở nên nghiêm trọng.

Nguyên tắc thứ hai là tính nhất quán giữa lời nói và hành động. Uy tín lãnh đạo không đến từ những phát biểu hùng hồn, mà từ việc người khác quan sát thấy sự nhất quán lâu dài trong cách một người đưa ra quyết định và cư xử. Khi lời nói và hành động không đồng nhất, niềm tin sẽ bị xói mòn rất nhanh. Ngược lại, sự nhất quán tạo ra cảm giác an toàn, giúp đội nhóm yên tâm hành động mà không phải đoán ý lãnh đạo.

Nguyên tắc thứ ba là con người luôn là trung tâm của lãnh đạo. Kết quả kinh doanh, hiệu suất hay chỉ số đều được tạo ra thông qua con người. Do đó, một nhà lãnh đạo giỏi cần hiểu động cơ, năng lực và giới hạn của từng cá nhân trong đội nhóm. Quan tâm đến con người không đồng nghĩa với dễ dãi, mà là đặt đúng người vào đúng vai trò, tạo điều kiện để họ phát huy tốt nhất khả năng của mình.

Nguyên tắc thứ tư là ra quyết định dựa trên cả dữ liệu và phán đoán. Dữ liệu giúp giảm rủi ro và tránh cảm tính, nhưng dữ liệu không bao giờ phản ánh toàn bộ bức tranh. Kinh nghiệm, trực giác và hiểu biết về bối cảnh đóng vai trò quan trọng trong những tình huống phức tạp. Một nhà lãnh đạo hiệu quả biết cân bằng giữa phân tích và quyết đoán, thay vì trì hoãn vô hạn để chờ thông tin hoàn hảo.

Nguyên tắc cuối cùng là học hỏi liên tục từ thực tiễn. Lãnh đạo không thể phát triển nếu chỉ dựa vào lý thuyết. Mỗi quyết định, mỗi thành công hay thất bại đều là một nguồn học tập. Việc thường xuyên phản tư, đánh giá lại cách làm và sẵn sàng điều chỉnh giúp nhà lãnh đạo ngày càng trưởng thành. Những người ngừng học hỏi thường bị tụt lại phía sau, ngay cả khi họ từng rất thành công.

Năm nguyên tắc trên tạo thành nền móng cho toàn bộ hành trình lãnh đạo. Chúng không phải là danh sách để ghi nhớ, mà là tiêu chuẩn để soi chiếu hành vi hàng ngày của người lãnh đạo trong thực tế công việc.

\section{Đánh giá vị trí hiện tại trên hành trình lãnh đạo}

Trước khi đặt ra mục tiêu hay xây dựng kế hoạch phát triển, người lãnh đạo cần trả lời một câu hỏi nền tảng: mình đang đứng ở đâu. Việc không đánh giá đúng vị trí hiện tại dẫn đến hai rủi ro phổ biến: hoặc đặt mục tiêu quá xa rời thực tế, hoặc tự mãn và dậm chân tại chỗ. Đánh giá bản thân trong lãnh đạo không phải là hoạt động mang tính cảm xúc, mà là một quá trình phân tích nghiêm túc dựa trên hành vi và kết quả cụ thể.

Bước đầu tiên là phân biệt rõ giữa chức danh và năng lực lãnh đạo. Nhiều người giữ vị trí quản lý nhưng chưa thực sự dẫn dắt được đội nhóm. Ngược lại, có những cá nhân không có chức danh chính thức nhưng lại tạo ra ảnh hưởng mạnh mẽ đến người khác. Do đó, việc đánh giá không nên dựa vào cấp bậc trong sơ đồ tổ chức, mà dựa vào mức độ ảnh hưởng, khả năng ra quyết định và năng lực tạo ra kết quả thông qua người khác.

Một cách tiếp cận hiệu quả là tự đánh giá dựa trên các nhóm năng lực cốt lõi. Thứ nhất là năng lực định hướng: người lãnh đạo có hiểu rõ mục tiêu chung và truyền đạt được mục tiêu đó cho đội nhóm hay không. Nếu cấp dưới không thể giải thích rõ họ đang hướng tới điều gì, đó là dấu hiệu cho thấy vai trò định hướng đang gặp vấn đề. Thứ hai là năng lực thực thi: quyết định có được chuyển hóa thành hành động cụ thể, đúng thời hạn và đúng chất lượng hay không. Thứ ba là năng lực quản trị con người: khả năng giao việc, phản hồi, phát triển năng lực và xử lý mâu thuẫn trong đội nhóm.

Tuy nhiên, tự đánh giá luôn tiềm ẩn sự thiên lệch. Con người thường có xu hướng đánh giá cao điểm mạnh và xem nhẹ điểm yếu của chính mình. Vì vậy, phản hồi từ người khác là nguồn thông tin không thể thiếu. Một nhà lãnh đạo nghiêm túc cần chủ động thu thập phản hồi từ nhiều chiều: cấp trên, đồng nghiệp ngang cấp và cấp dưới trực tiếp. Điểm quan trọng là không tìm kiếm lời khen, mà tìm kiếm sự thật. Những phản hồi lặp đi lặp lại, dù tích cực hay tiêu cực, thường phản ánh khá chính xác vị trí hiện tại.

Bên cạnh phản hồi trực tiếp, kết quả hoạt động của đội nhóm cũng là một chỉ báo quan trọng. Tỷ lệ hoàn thành mục tiêu, mức độ chủ động của nhân sự, khả năng tự ra quyết định của cấp dưới và mức độ ổn định của đội ngũ đều phản ánh chất lượng lãnh đạo. Một đội nhóm phụ thuộc quá nhiều vào lãnh đạo trong các quyết định nhỏ thường cho thấy lãnh đạo chưa trao quyền hiệu quả hoặc chưa xây dựng được niềm tin.

Sau khi thu thập thông tin, bước tiếp theo là xác định các điểm mù lãnh đạo. Điểm mù là những hạn chế mà bản thân người lãnh đạo không nhận ra, nhưng lại ảnh hưởng rõ rệt đến người khác. Ví dụ, một người có thể cho rằng mình thẳng thắn, trong khi đội nhóm cảm nhận đó là thiếu tinh tế. Việc nhận diện điểm mù đòi hỏi sự khiêm tốn và sẵn sàng đối diện với những thông tin không dễ chịu.

Cuối cùng, việc đánh giá vị trí hiện tại cần được xem như một hoạt động định kỳ, không phải làm một lần rồi bỏ qua. Môi trường thay đổi, quy mô đội nhóm thay đổi và bản thân người lãnh đạo cũng thay đổi theo thời gian. Đánh giá định kỳ giúp người lãnh đạo điều chỉnh kịp thời, tránh việc tiếp tục sử dụng những cách làm đã không còn phù hợp. Chỉ khi hiểu rõ mình đang ở đâu, người lãnh đạo mới có thể lựa chọn đúng con đường để tiếp tục tiến lên trên hành trình trở thành bậc thầy lãnh đạo.

\section{Xác định mục tiêu phát triển cá nhân rõ ràng}

Sau khi đã đánh giá tương đối chính xác vị trí hiện tại, bước tiếp theo trên hành trình lãnh đạo là xác định mục tiêu phát triển cá nhân. Đây là giai đoạn mang tính quyết định, bởi mục tiêu sai hoặc mơ hồ sẽ kéo theo những nỗ lực dàn trải, thiếu trọng tâm và khó tạo ra thay đổi thực chất. Đối với lãnh đạo, mục tiêu phát triển không nên dừng lại ở mong muốn chung chung, mà phải gắn chặt với hành vi cụ thể và tác động đo lường được.

Trước hết, mục tiêu lãnh đạo cần xuất phát từ những khoảng trống đã được nhận diện trong quá trình đánh giá. Thay vì cố gắng cải thiện mọi khía cạnh cùng lúc, người lãnh đạo nên chọn một đến hai năng lực then chốt có ảnh hưởng lớn nhất đến hiệu quả của đội nhóm. Ví dụ, nếu phản hồi cho thấy cấp dưới thiếu chủ động và thường xuyên chờ chỉ đạo, mục tiêu phát triển không nên là “trở thành lãnh đạo truyền cảm hứng hơn”, mà là “nâng cao năng lực giao quyền và trao trách nhiệm”.

Một mục tiêu phát triển cá nhân hiệu quả cần đáp ứng ba tiêu chí cơ bản. Thứ nhất, mục tiêu phải cụ thể và quan sát được. Người lãnh đạo cần mô tả rõ hành vi nào sẽ thay đổi, trong bối cảnh nào và với ai. Thứ hai, mục tiêu phải đo lường được thông qua kết quả hoặc phản hồi. Điều này giúp tránh việc tự đánh giá cảm tính. Thứ ba, mục tiêu phải thực tế trong khung thời gian và nguồn lực hiện có. Một mục tiêu quá tham vọng thường dẫn đến nản chí và bỏ dở giữa chừng.

Bên cạnh tính rõ ràng, mục tiêu lãnh đạo cần gắn trực tiếp với kết quả của đội nhóm, không chỉ với sự phát triển cá nhân. Lãnh đạo không phải là hành trình tự hoàn thiện đơn lẻ, mà là quá trình tạo ra ảnh hưởng tích cực đến người khác. Do đó, khi đặt mục tiêu, người lãnh đạo cần tự hỏi: nếu tôi đạt được mục tiêu này, đội nhóm sẽ thay đổi như thế nào, làm việc hiệu quả hơn ra sao, và tổ chức được lợi gì. Nếu không trả lời được câu hỏi này, mục tiêu đó có thể chưa thực sự phù hợp.

Một sai lầm phổ biến là đặt mục tiêu dựa trên hình mẫu lý tưởng thay vì bối cảnh thực tế. Mỗi tổ chức, mỗi giai đoạn phát triển và mỗi đội nhóm đòi hỏi những ưu tiên lãnh đạo khác nhau. Việc sao chép mục tiêu của người khác, dù họ thành công, không đảm bảo mang lại kết quả tương tự. Người lãnh đạo cần căn cứ vào vai trò hiện tại, mức độ trưởng thành của đội ngũ và thách thức cụ thể đang đối mặt để xác định mục tiêu phù hợp.

Sau khi xác định mục tiêu, việc viết ra mục tiêu một cách rõ ràng là bước không nên bỏ qua. Việc này giúp chuyển ý định mơ hồ thành một cam kết cụ thể với chính bản thân. Một mục tiêu được viết rõ ràng cũng là cơ sở để theo dõi tiến độ và tự đánh giá sau này. Ngoài ra, việc chia sẻ mục tiêu với một người đáng tin cậy, như cấp trên hoặc cố vấn, có thể tạo thêm áp lực tích cực để duy trì kỷ luật rèn luyện.

Cuối cùng, cần nhìn nhận rằng mục tiêu phát triển lãnh đạo không phải là đích đến cố định. Khi môi trường thay đổi hoặc khi bản thân đã đạt được một mức độ nhất định, mục tiêu cần được điều chỉnh cho phù hợp. Việc xác định mục tiêu rõ ràng không chỉ giúp người lãnh đạo tập trung nỗ lực, mà còn tạo ra cảm giác tiến bộ rõ rệt, từ đó duy trì động lực trên hành trình dài hạn trở thành bậc thầy lãnh đạo.

\section{Xây dựng lộ trình hành động cụ thể và thực tế}

Mục tiêu phát triển lãnh đạo chỉ có giá trị khi được chuyển hóa thành hành động cụ thể. Nhiều người lãnh đạo hiểu rõ mình cần cải thiện điều gì, nhưng lại không tiến bộ vì thiếu một lộ trình hành động rõ ràng và khả thi. Lộ trình đóng vai trò như cầu nối giữa ý định và kết quả, giúp biến mong muốn phát triển thành những thay đổi quan sát được trong công việc hàng ngày.

Nguyên tắc đầu tiên khi xây dựng lộ trình là đơn giản và tập trung. Một lộ trình hiệu quả không cần phức tạp, nhưng phải tập trung vào đúng trọng tâm đã xác định ở phần mục tiêu. Thay vì liệt kê quá nhiều hoạt động, người lãnh đạo nên chọn một số hành động then chốt có tác động lớn nhất. Ví dụ, nếu mục tiêu là cải thiện năng lực giao quyền, lộ trình có thể tập trung vào việc xác định rõ các đầu việc có thể giao, tiêu chuẩn kết quả và mức độ kiểm soát phù hợp, thay vì tham gia hàng loạt khóa học chung chung.

Tiếp theo, lộ trình cần được chia theo mốc thời gian cụ thể. Việc chia nhỏ hành trình thành các giai đoạn ngắn giúp người lãnh đạo dễ theo dõi tiến độ và điều chỉnh kịp thời. Một khung thời gian thường được sử dụng là 30–60–90 ngày. Giai đoạn đầu tập trung quan sát và thử nghiệm hành vi mới, giai đoạn tiếp theo củng cố và mở rộng phạm vi áp dụng, giai đoạn cuối đánh giá hiệu quả và chuẩn hóa cách làm. Cách tiếp cận này giúp tránh tình trạng hứng khởi ban đầu nhưng nhanh chóng quay lại thói quen cũ.

Một yếu tố quan trọng khác của lộ trình là gắn hành động phát triển với công việc thực tế. Phát triển lãnh đạo không nên tách rời khỏi nhiệm vụ hàng ngày. Mỗi cuộc họp, mỗi lần giao việc, mỗi tình huống xử lý mâu thuẫn đều là cơ hội để rèn luyện. Khi hành động phát triển được tích hợp trực tiếp vào công việc, người lãnh đạo vừa học vừa làm, đồng thời tạo ra giá trị thực cho tổ chức.

Bên cạnh hành động, lộ trình cần có cơ chế theo dõi và phản hồi. Người lãnh đạo nên xác định rõ những chỉ báo cho thấy mình đang đi đúng hướng, chẳng hạn mức độ chủ động của cấp dưới, chất lượng quyết định được giao quyền hay phản hồi trực tiếp từ đội nhóm. Việc tự đánh giá định kỳ, kết hợp với phản hồi từ người khác, giúp phát hiện sớm những lệch hướng và điều chỉnh kịp thời trước khi thói quen cũ quay trở lại.

Cuối cùng, một lộ trình thực tế phải tính đến giới hạn về thời gian, năng lượng và áp lực công việc. Nhiều kế hoạch phát triển thất bại không phải vì mục tiêu sai, mà vì lộ trình không phù hợp với thực tế. Người lãnh đạo cần trung thực với chính mình về khả năng duy trì kỷ luật và khối lượng công việc hiện tại. Một lộ trình vừa sức, được thực hiện đều đặn, sẽ mang lại kết quả bền vững hơn nhiều so với những kế hoạch quá tham vọng nhưng nhanh chóng bị bỏ dở.

Xây dựng lộ trình hành động không phải là bước kết thúc, mà là nền tảng cho việc rèn luyện lâu dài. Khi có lộ trình rõ ràng, người lãnh đạo không còn phụ thuộc vào cảm hứng nhất thời, mà có một hệ thống hành động giúp họ tiến bộ từng bước vững chắc trên hành trình trở thành bậc thầy lãnh đạo.

\section{Cam kết rèn luyện để trở thành bậc thầy lãnh đạo}

Không có lộ trình hay phương pháp nào mang lại kết quả nếu thiếu cam kết rèn luyện nghiêm túc. Lãnh đạo, ở cấp độ cao nhất, là sự lựa chọn mang tính dài hạn: lựa chọn không ngừng cải thiện bản thân dù áp lực công việc, môi trường thay đổi hay kết quả ngắn hạn chưa như mong đợi. Phần cuối của hành trình này không nói về kỹ thuật, mà nói về thái độ và kỷ luật cá nhân của người lãnh đạo.

Cam kết đầu tiên là cam kết với việc rèn luyện mỗi ngày. Phát triển lãnh đạo không diễn ra trong những khoảnh khắc đặc biệt, mà trong các quyết định và hành vi nhỏ lặp đi lặp lại hàng ngày. Cách lắng nghe trong một cuộc họp, cách phản hồi khi cấp dưới mắc sai sót, hay cách đưa ra quyết định trong điều kiện thiếu thông tin đều phản ánh mức độ rèn luyện của người lãnh đạo. Khi việc rèn luyện trở thành thói quen, sự tiến bộ sẽ diễn ra một cách tự nhiên và bền vững.

Cam kết thứ hai là cam kết chịu trách nhiệm trọn vẹn cho kết quả. Một bậc thầy lãnh đạo không tìm kiếm lý do biện minh khi đối mặt với thất bại. Thay vào đó, họ tập trung phân tích nguyên nhân, rút ra bài học và điều chỉnh cách làm. Việc thừa nhận sai lầm không làm suy yếu vai trò lãnh đạo, mà ngược lại, tạo ra sự tin cậy và khuyến khích đội nhóm học hỏi từ sai lầm thay vì che giấu.

Cam kết thứ ba là cam kết phát triển người khác. Lãnh đạo đích thực không được đo bằng những gì cá nhân đạt được, mà bằng những gì đội nhóm có thể làm khi không có sự hiện diện trực tiếp của người lãnh đạo. Việc đầu tư thời gian huấn luyện, giao quyền và xây dựng đội ngũ kế cận là dấu hiệu rõ ràng của tư duy lãnh đạo trưởng thành. Một người chỉ tập trung vào thành tích cá nhân sẽ sớm chạm trần phát triển.

Cam kết tiếp theo là duy trì tinh thần học hỏi liên tục. Môi trường kinh doanh, công nghệ và kỳ vọng của con người thay đổi không ngừng. Những cách làm hiệu quả trong quá khứ có thể trở thành rào cản trong tương lai. Người lãnh đạo cần chủ động cập nhật kiến thức, học từ trải nghiệm thực tế và sẵn sàng điều chỉnh niềm tin của mình khi bối cảnh thay đổi. Sự cứng nhắc là kẻ thù của lãnh đạo bền vững.

Cuối cùng, cam kết trở thành bậc thầy lãnh đạo là cam kết đi đường dài. Không có điểm kết thúc rõ ràng cho hành trình này. Mỗi giai đoạn phát triển sẽ đặt ra những thách thức mới, đòi hỏi người lãnh đạo tiếp tục nâng cấp tư duy và kỹ năng. Việc kiên trì rèn luyện, ngay cả khi không có sự công nhận tức thời, chính là yếu tố phân biệt giữa người giữ vai trò lãnh đạo và người thực sự sống với vai trò đó.

Khi khép lại chương này, người lãnh đạo cần tự đặt cho mình một câu hỏi mang tính cam kết: nếu trong mười hai tháng tới không có bất kỳ thay đổi nào trong cách lãnh đạo, hậu quả sẽ là gì đối với bản thân, đội nhóm và tổ chức. Câu hỏi đó không nhằm tạo áp lực, mà nhằm nhắc nhở rằng lãnh đạo luôn là một lựa chọn chủ động. Việc tiếp tục rèn luyện hay dừng lại hoàn toàn nằm trong tay mỗi người. Chọn con đường trở thành bậc thầy lãnh đạo cũng đồng nghĩa với việc chấp nhận trách nhiệm phát triển bản thân suốt đời.
