\chapter{Khởi nguồn của Công nghệ Thông tin}

Công nghệ thông tin không xuất hiện một cách đột ngột cùng với máy tính hay Internet, mà là kết quả của một quá trình phát triển rất dài trong lịch sử nhân loại. Trước khi có máy móc, con người đã phải đối mặt với một bài toán mang tính nền tảng: làm thế nào để lưu trữ, tổ chức và xử lý thông tin phục vụ cho đời sống xã hội ngày càng phức tạp. Việc tìm lời giải cho bài toán đó chính là điểm khởi đầu của toàn bộ lĩnh vực công nghệ thông tin ngày nay.

\section{Nhu cầu lưu trữ và xử lý thông tin trong các xã hội cổ đại}

Ngay từ khi con người bắt đầu hình thành các cộng đồng ổn định, thông tin đã trở thành yếu tố sống còn đối với sự tồn tại và phát triển của xã hội. Ở quy mô nhỏ, thông tin có thể chỉ là số lượng lương thực, công cụ lao động hoặc thành viên trong bộ tộc. Tuy nhiên, khi xã hội mở rộng về dân số, lãnh thổ và hoạt động kinh tế, khối lượng thông tin cần quản lý tăng lên nhanh chóng và vượt xa khả năng ghi nhớ của cá nhân.

Trong các xã hội cổ đại, nhà cầm quyền cần nắm được thông tin về dân cư để tổ chức lao động, thu thuế và duy trì trật tự. Hoạt động thương mại đòi hỏi ghi nhận số lượng hàng hóa, giá trị trao đổi và các cam kết giữa các bên. Nông nghiệp cần theo dõi mùa vụ, sản lượng và phân phối lương thực. Tôn giáo và nghi lễ cần lưu giữ các quy tắc, truyền thuyết và tri thức mang tính biểu tượng. Tất cả những nhu cầu này đều đặt ra yêu cầu chung: thông tin phải được lưu trữ và xử lý một cách có hệ thống.

Ban đầu, thông tin được lưu giữ hoàn toàn dựa trên trí nhớ con người. Các câu chuyện truyền miệng, quy ước và kinh nghiệm được truyền từ thế hệ này sang thế hệ khác. Phương pháp này có ưu điểm là linh hoạt và không cần công cụ hỗ trợ, nhưng tồn tại nhiều hạn chế nghiêm trọng. Thông tin dễ bị sai lệch theo thời gian, phụ thuộc mạnh vào người ghi nhớ, và không thể mở rộng khi xã hội trở nên phức tạp hơn. Khi quy mô thông tin tăng lên, sai sót trong ghi nhớ có thể dẫn đến hậu quả nghiêm trọng trong quản lý và ra quyết định.

Chính từ những hạn chế đó, con người bắt đầu tìm cách “ngoại hóa” trí nhớ, tức là đưa thông tin ra khỏi bộ não cá nhân và gắn nó với các vật mang tin bên ngoài. Việc đánh dấu bằng sỏi, khắc ký hiệu lên gỗ, xương hoặc đá là những bước đi đầu tiên nhằm cố định thông tin theo một cách bền vững hơn. Dù còn rất thô sơ, các phương pháp này đã thể hiện một tư duy quan trọng: thông tin có thể được biểu diễn, lưu trữ và truy xuất thông qua các phương tiện vật chất.

Cùng với lưu trữ, nhu cầu xử lý thông tin cũng ngày càng rõ rệt. Không chỉ cần biết “có bao nhiêu”, con người còn cần so sánh, cộng dồn, phân chia và dự đoán. Việc quản lý kho lương thực, phân bổ tài nguyên hay tính toán thuế khóa đều đòi hỏi những thao tác xử lý thông tin có tính lặp lại và tuân theo quy tắc. Đây chính là mầm mống của tư duy xử lý dữ liệu và thuật toán, dù chưa được gọi tên như trong công nghệ thông tin hiện đại.

Điểm then chốt ở giai đoạn này là sự chuyển biến trong nhận thức: thông tin không còn là khái niệm trừu tượng gắn chặt với con người, mà trở thành một đối tượng có thể thao tác được. Khi thông tin được xem như một thực thể có thể lưu trữ, sao chép và xử lý, con người bắt đầu xây dựng các phương thức và công cụ chuyên biệt để làm việc với nó. Từ góc nhìn hiện đại, đây chính là bước khởi đầu của tư duy hệ thống thông tin.

Tóm lại, nhu cầu lưu trữ và xử lý thông tin trong các xã hội cổ đại không chỉ xuất phát từ sự tò mò hay tiến bộ trí tuệ, mà là kết quả tất yếu của quá trình tổ chức xã hội. Những nhu cầu này đã tạo ra áp lực liên tục buộc con người phải tìm kiếm các giải pháp ngày càng hiệu quả hơn. Áp lực đó chính là động lực sâu xa dẫn đến sự ra đời của các công cụ tính toán, hệ thống ghi chép và cuối cùng là công nghệ thông tin như một lĩnh vực độc lập.

\section{Các công cụ tính toán sơ khai}

Khi nhu cầu xử lý thông tin định lượng ngày càng gia tăng, con người buộc phải tìm ra những công cụ hỗ trợ cho việc đếm và tính toán. Việc dựa hoàn toàn vào trí nhớ và tính nhẩm nhanh chóng bộc lộ giới hạn, đặc biệt trong các hoạt động như thương mại, phân phối tài nguyên và quản lý sản xuất. Từ đó, các công cụ tính toán sơ khai ra đời như một bước tiến quan trọng trong lịch sử xử lý thông tin.

Những công cụ ban đầu mang tính vật lý và trực quan cao. Con người sử dụng sỏi, hạt, que gỗ hoặc nút thắt trên dây để biểu diễn số lượng. Mỗi vật thể đại diện cho một đơn vị thông tin. Cách làm này tuy đơn giản nhưng có giá trị thực tiễn lớn, vì nó cho phép “nhìn thấy” thông tin thay vì chỉ ghi nhớ trong đầu. Đây là bước chuyển quan trọng từ thông tin trừu tượng sang thông tin được biểu diễn bằng vật chất.

Cùng với đó, các hệ đếm được hình thành nhằm chuẩn hóa cách biểu diễn số lượng. Hệ đếm không chỉ là công cụ toán học mà còn là một quy ước thông tin, giúp mọi người trong cùng một cộng đồng hiểu và xử lý dữ liệu theo cùng một cách. Một số nền văn minh phát triển các hệ đếm riêng, tiêu biểu như hệ cơ số 10 gắn với số ngón tay, hay hệ cơ số 60 của người Babylon phục vụ cho thiên văn học và đo lường thời gian. Việc lựa chọn hệ đếm phản ánh nhu cầu thực tế và tư duy xử lý thông tin của từng xã hội.

Trong số các công cụ tính toán sơ khai, bàn tính là một phát minh có ảnh hưởng sâu rộng và lâu dài. Bàn tính cho phép thực hiện các phép toán cơ bản một cách có hệ thống, nhanh và chính xác hơn so với tính nhẩm. Quan trọng hơn, bàn tính tách biệt rõ ràng giữa dữ liệu đầu vào, quá trình xử lý và kết quả đầu ra. Người sử dụng thao tác theo các quy tắc cố định, gần với khái niệm “thuật toán” trong công nghệ thông tin hiện đại.

Các công cụ này cho thấy một đặc điểm chung: chúng không tự động, nhưng đã thể hiện rõ tư duy cơ giới hóa việc xử lý thông tin. Con người không còn xử lý từng con số một cách ngẫu nhiên, mà tuân theo các bước xác định, có thể lặp lại và truyền dạy cho người khác. Điều này làm giảm sai sót, tăng hiệu suất và tạo điều kiện cho việc mở rộng quy mô xử lý dữ liệu.

Tuy nhiên, các công cụ tính toán sơ khai vẫn tồn tại nhiều hạn chế. Chúng phụ thuộc hoàn toàn vào kỹ năng của người sử dụng, tốc độ xử lý còn chậm và khó đáp ứng khi khối lượng dữ liệu tăng lớn. Việc tính toán phức tạp đòi hỏi nhiều bước thủ công, dễ dẫn đến nhầm lẫn. Dù vậy, chính những hạn chế này lại đóng vai trò quan trọng, vì chúng làm lộ rõ nhu cầu về các phương thức xử lý thông tin hiệu quả hơn.

Từ góc nhìn của công nghệ thông tin, các công cụ tính toán sơ khai không chỉ là tiền thân của máy tính, mà còn đặt nền móng cho nhiều khái niệm cốt lõi: biểu diễn dữ liệu, quy trình xử lý, tính lặp và tính chính xác. Chúng chứng minh rằng thông tin có thể được thao tác theo quy tắc và kết quả có thể dự đoán trước nếu tuân thủ đúng quy trình.

Như vậy, sự ra đời của các công cụ tính toán sơ khai đánh dấu một bước tiến quan trọng trong lịch sử công nghệ thông tin. Chúng thể hiện nỗ lực đầu tiên của con người nhằm tách hoạt động xử lý thông tin khỏi khả năng cá nhân, hướng tới các hệ thống hỗ trợ có tính chuẩn hóa và mở rộng. Đây chính là tiền đề trực tiếp cho sự phát triển của các hệ thống ghi chép và lưu trữ dữ liệu ở giai đoạn tiếp theo.

\section{Sự phát triển của chữ viết và hệ thống ghi chép dữ liệu}

Khi quy mô xã hội và mức độ phức tạp của thông tin vượt quá khả năng xử lý bằng ghi nhớ và các công cụ đếm đơn giản, con người cần một phương thức lưu trữ thông tin có tính bền vững, chính xác và có thể truyền tải qua không gian và thời gian. Chữ viết ra đời trong bối cảnh đó, không phải như một phát minh mang tính nghệ thuật, mà trước hết là một giải pháp kỹ thuật cho bài toán quản lý thông tin.

Các hình thức chữ viết sớm nhất xuất hiện gắn liền với nhu cầu hành chính và kinh tế. Tại các nền văn minh cổ đại, chữ viết được dùng để ghi chép thuế khóa, giao dịch thương mại, phân phối lương thực và các quy định pháp lý. Điều này cho thấy chữ viết ngay từ đầu đã mang bản chất của một hệ thống ghi chép dữ liệu, phục vụ cho quản lý và điều hành xã hội hơn là biểu đạt cảm xúc cá nhân.

Về mặt bản chất, chữ viết là một phương thức mã hóa thông tin. Các sự vật, hành động và khái niệm được biểu diễn bằng ký hiệu thống nhất, cho phép con người lưu trữ thông tin dưới dạng vật chất như bảng đất sét, giấy papyrus hay thẻ tre. Việc mã hóa này giúp thông tin tách khỏi người tạo ra nó, có thể được đọc, hiểu và sử dụng bởi những người khác, thậm chí ở các thế hệ sau. Đây là một bước tiến mang tính nền tảng trong lịch sử thông tin.

Cùng với sự phát triển của chữ viết, các hệ thống ghi chép ngày càng trở nên có cấu trúc hơn. Thông tin không còn được ghi lại một cách rời rạc, mà được tổ chức theo danh sách, bảng biểu, sổ sách và kho lưu trữ. Việc phân loại dữ liệu theo mục đích, thời gian hoặc đối tượng giúp tăng khả năng tra cứu và giảm sai sót trong quản lý. Từ góc nhìn hiện đại, đây chính là những hình thức sơ khai của cơ sở dữ liệu.

Một điểm quan trọng khác là sự xuất hiện của các chuẩn mực trong ghi chép. Ngôn ngữ viết, đơn vị đo lường và cách trình bày thông tin dần được thống nhất trong phạm vi một xã hội hoặc quốc gia. Chuẩn hóa giúp đảm bảo rằng cùng một dữ liệu sẽ được hiểu theo cùng một cách, hạn chế sự nhập nhằng và mâu thuẫn. Đây là yếu tố cốt lõi trong mọi hệ thống thông tin, kể cả các hệ thống công nghệ thông tin ngày nay.

Tuy nhiên, các hệ thống ghi chép dữ liệu thời kỳ này vẫn mang tính thủ công và phụ thuộc nhiều vào con người. Việc tạo lập, sao chép và cập nhật thông tin tốn nhiều thời gian và công sức. Sai sót có thể xảy ra do nhầm lẫn, gian lận hoặc hư hỏng vật liệu lưu trữ. Dù vậy, so với việc chỉ dựa vào trí nhớ, chữ viết đã nâng năng lực xử lý thông tin của xã hội lên một mức hoàn toàn mới.

Từ góc độ công nghệ thông tin, sự phát triển của chữ viết và hệ thống ghi chép dữ liệu đánh dấu sự hình thành rõ ràng của khái niệm “dữ liệu”. Thông tin đã được biểu diễn dưới dạng ký hiệu, lưu trữ trên vật mang tin và có thể được truy xuất khi cần thiết. Đây chính là tiền đề trực tiếp cho các khái niệm như lưu trữ, truy vấn và quản lý dữ liệu trong các hệ thống CNTT hiện đại.

Như vậy, chữ viết không chỉ là thành tựu văn hóa hay ngôn ngữ, mà còn là một bước ngoặt kỹ thuật trong lịch sử xử lý thông tin. Nó đặt nền móng cho việc xây dựng các hệ thống thông tin có tổ chức, tạo điều kiện để con người tiến tới những phương thức xử lý thông tin ngày càng tự động và hiệu quả hơn ở các giai đoạn tiếp theo.

\section{Vai trò của toán học và logic học trong nền tảng công nghệ thông tin}

Song song với sự phát triển của công cụ tính toán và hệ thống ghi chép dữ liệu, toán học và logic học dần hình thành như những nền tảng tư duy cốt lõi cho việc xử lý thông tin. Nếu các công cụ vật lý giúp con người thao tác với dữ liệu, thì toán học và logic học cung cấp các nguyên tắc trừu tượng để tổ chức, phân tích và suy luận trên dữ liệu đó. Đây là bước chuyển từ xử lý thông tin mang tính kinh nghiệm sang xử lý thông tin có cơ sở lý thuyết.

Toán học ra đời từ nhu cầu đo lường, tính toán và so sánh trong đời sống thực tiễn. Tuy nhiên, theo thời gian, toán học không chỉ dừng lại ở các phép tính số học đơn giản mà mở rộng sang đại số, hình học và lý thuyết tập hợp. Những khái niệm này cho phép con người mô hình hóa thế giới thực thành các biểu diễn trừu tượng, từ đó xử lý thông tin một cách chính xác và có hệ thống hơn. Việc trừu tượng hóa này là yếu tố then chốt trong mọi hệ thống công nghệ thông tin.

Logic học, đặc biệt là logic mệnh đề và logic suy diễn, đóng vai trò xác định tính đúng sai trong quá trình xử lý thông tin. Thay vì dựa vào cảm tính hay kinh nghiệm cá nhân, logic học cung cấp các quy tắc suy luận chặt chẽ, đảm bảo rằng nếu dữ liệu đầu vào đúng và quy trình xử lý tuân thủ quy tắc, thì kết quả đầu ra sẽ nhất quán và đáng tin cậy. Nguyên tắc này trở thành nền tảng cho tư duy lập trình và thiết kế hệ thống sau này.

Một đóng góp quan trọng của toán học và logic học là hình thành khái niệm thuật toán. Thuật toán có thể hiểu là một tập hợp hữu hạn các bước được xác định rõ ràng nhằm giải quyết một bài toán cụ thể. Ngay cả trước khi máy tính ra đời, con người đã sử dụng các thuật toán trong tính toán, đo đạc và quản lý. Việc mô tả quy trình xử lý thông tin bằng các bước logic là tiền đề trực tiếp cho sự xuất hiện của chương trình máy tính.

Toán học và logic học cũng góp phần tạo ra tư duy hệ thống. Thông tin không còn được nhìn nhận một cách rời rạc, mà được xem xét trong mối quan hệ với các thông tin khác, tuân theo các quy luật và cấu trúc nhất định. Tư duy này giúp con người xây dựng các mô hình xử lý thông tin có thể mở rộng, kiểm soát và kiểm chứng. Đây là yêu cầu bắt buộc đối với các hệ thống thông tin quy mô lớn.

Từ góc nhìn của công nghệ thông tin hiện đại, nhiều khái niệm cốt lõi như cấu trúc dữ liệu, thuật toán, ngôn ngữ lập trình và hệ thống logic đều có nguồn gốc trực tiếp từ toán học và logic học. Không có các nền tảng này, việc tự động hóa xử lý thông tin sẽ thiếu cơ sở lý luận và khó đảm bảo tính chính xác.

Như vậy, toán học và logic học không chỉ là công cụ hỗ trợ, mà là trụ cột tư duy của công nghệ thông tin. Chúng cung cấp ngôn ngữ, quy tắc và phương pháp để con người có thể biểu diễn, phân tích và xử lý thông tin một cách khoa học. Chính nền tảng này đã tạo điều kiện cho sự ra đời của các hệ thống xử lý thông tin tự động ở các giai đoạn phát triển tiếp theo.

\section{Giới hạn của phương pháp thủ công và yêu cầu tự động hóa thông tin}

Mặc dù các công cụ tính toán sơ khai, hệ thống chữ viết và nền tảng toán học – logic học đã giúp con người nâng cao đáng kể năng lực xử lý thông tin, toàn bộ quá trình này vẫn mang tính thủ công và phụ thuộc nặng nề vào con người. Khi xã hội bước sang giai đoạn phát triển cao hơn về kinh tế, khoa học và quản trị, những hạn chế của phương pháp thủ công ngày càng bộc lộ rõ ràng và trở thành lực cản cho sự tiến bộ.

Giới hạn đầu tiên là tốc độ xử lý. Việc ghi chép, tính toán và tổng hợp dữ liệu bằng tay đòi hỏi nhiều thời gian, đặc biệt khi khối lượng thông tin tăng lên nhanh chóng. Trong các hoạt động như quản lý dân số, thu thuế hay thương mại quy mô lớn, độ trễ trong xử lý thông tin có thể dẫn đến quyết định chậm trễ hoặc sai lầm. Tốc độ xử lý thủ công không còn đáp ứng được yêu cầu của xã hội đang mở rộng.

Giới hạn thứ hai là độ chính xác và tính nhất quán. Phương pháp thủ công phụ thuộc trực tiếp vào kỹ năng, kinh nghiệm và trạng thái của con người. Sai sót trong tính toán, nhầm lẫn khi ghi chép hoặc diễn giải không thống nhất dữ liệu là điều khó tránh khỏi. Khi dữ liệu được sao chép nhiều lần, lỗi có xu hướng tích lũy và lan rộng, làm giảm độ tin cậy của toàn bộ hệ thống thông tin.

Giới hạn thứ ba là khả năng mở rộng. Các hệ thống thủ công hoạt động hiệu quả ở quy mô nhỏ, nhưng gặp khó khăn nghiêm trọng khi khối lượng dữ liệu và số lượng người tham gia tăng lên. Việc bổ sung thêm thông tin đồng nghĩa với việc tăng nhân lực, thời gian và chi phí, trong khi hiệu quả không tăng tương ứng. Điều này khiến các phương pháp truyền thống không phù hợp với nhu cầu quản lý phức tạp của xã hội hiện đại.

Trước những giới hạn đó, yêu cầu tự động hóa xử lý thông tin trở nên tất yếu. Tự động hóa nhằm giảm sự phụ thuộc vào con người trong các thao tác lặp đi lặp lại, chuẩn hóa quy trình xử lý và nâng cao tốc độ cũng như độ chính xác. Ý tưởng sử dụng máy móc để hỗ trợ hoặc thay thế con người trong xử lý thông tin bắt đầu hình thành, trước hết trong lĩnh vực tính toán, sau đó mở rộng sang lưu trữ và truyền tải dữ liệu.

Yêu cầu tự động hóa không chỉ xuất phát từ nhu cầu kỹ thuật, mà còn từ yêu cầu quản trị và ra quyết định. Khi thông tin được xử lý nhanh hơn và đáng tin cậy hơn, các tổ chức có thể phản ứng kịp thời với biến động của môi trường kinh tế – xã hội. Điều này tạo ra lợi thế cạnh tranh và thúc đẩy đổi mới trong nhiều lĩnh vực.

Từ góc nhìn lịch sử, chính sự bất cập của phương pháp thủ công đã mở đường cho các phát minh mang tính đột phá trong xử lý thông tin, dẫn tới sự ra đời của các máy tính cơ học, điện tử và cuối cùng là công nghệ thông tin hiện đại. Quá trình này không phải là sự phủ nhận các phương pháp truyền thống, mà là sự kế thừa và phát triển trên nền tảng đã được xây dựng trước đó.

Kết luận lại, giới hạn của xử lý thông tin thủ công đã tạo ra một bước ngoặt trong lịch sử phát triển của nhân loại. Nhu cầu tự động hóa thông tin không chỉ là hệ quả của tiến bộ kỹ thuật, mà là phản ứng tất yếu trước sự gia tăng về quy mô và độ phức tạp của thông tin. Đây chính là điểm nối giữa giai đoạn tiền công nghệ thông tin và sự hình thành của công nghệ thông tin như một lĩnh vực độc lập.
