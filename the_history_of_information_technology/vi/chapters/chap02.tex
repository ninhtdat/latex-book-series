\chapter{Sự ra đời của máy tính cơ học và điện tử}

Lịch sử máy tính không bắt đầu từ các linh kiện điện tử hiện đại mà hình thành từ rất sớm, gắn liền với nhu cầu tính toán của con người trong thương mại, khoa học và quản lý xã hội. Trước khi điện năng được khai thác cho xử lý thông tin, nhiều thiết bị cơ học đã ra đời nhằm hỗ trợ và tự động hóa các phép tính. Giai đoạn này đặt nền móng tư duy và kỹ thuật cho khái niệm máy tính theo nghĩa hiện đại.

\section{Máy tính cơ học và các phát minh tiên phong trong tính toán tự động}

Máy tính cơ học là kết quả trực tiếp của nhu cầu giảm sai sót, tăng tốc độ và chuẩn hóa hoạt động tính toán. Trong các xã hội tiền công nghiệp, phần lớn phép tính được thực hiện thủ công, phụ thuộc vào kỹ năng cá nhân, dễ sai lệch và khó mở rộng khi quy mô dữ liệu tăng lên. Điều này tạo áp lực lớn trong các lĩnh vực như thuế khóa, thương mại, thiên văn học và kỹ thuật.

Thiết bị tính toán cơ học sớm nhất mang tính hỗ trợ hơn là thay thế con người. Bàn tính là ví dụ điển hình, cho phép biểu diễn số và thực hiện phép toán dựa trên thao tác vật lý có quy tắc. Dù đơn giản, bàn tính thể hiện một nguyên lý quan trọng: phép tính có thể được ánh xạ sang trạng thái vật lý, từ đó mở ra khả năng tự động hóa.

Bước tiến quan trọng xuất hiện khi các nhà phát minh tìm cách cơ giới hóa hoàn toàn quá trình tính toán. Các máy cộng trừ và nhân chia sử dụng bánh răng, trục quay và cơ cấu truyền động ra đời nhằm thực hiện phép toán mà không cần can thiệp liên tục của con người. Điểm then chốt của các thiết bị này không nằm ở tốc độ vượt trội, mà ở khả năng thực hiện phép tính một cách nhất quán theo cơ chế đã định sẵn. Đây là tiền đề cho khái niệm “thuật toán được hiện thực hóa bằng máy móc”.

Trong thế kỷ XIX, tư duy về máy tính cơ học đạt bước phát triển mang tính khái niệm sâu sắc. Một số thiết kế không chỉ thực hiện phép tính đơn lẻ mà còn hướng tới khả năng lập trình, tức là thay đổi chuỗi thao tác thông qua cấu hình ban đầu. Dù nhiều thiết bị trong số này không được chế tạo hoàn chỉnh do hạn chế kỹ thuật và chi phí, chúng đã xác lập những ý tưởng cốt lõi: phân tách dữ liệu và thao tác, sử dụng cơ chế điều khiển, và coi tính toán là một quy trình có thể tự động hóa toàn phần.

Tuy nhiên, máy tính cơ học tồn tại những giới hạn rõ rệt. Do phụ thuộc vào chuyển động vật lý, chúng có tốc độ thấp, dễ hao mòn và khó mở rộng về độ phức tạp. Khi số lượng phép toán tăng, kích thước và độ tinh vi của cơ cấu cơ khí tăng theo cấp số lớn, dẫn đến chi phí bảo trì và sai số ngày càng cao. Điều này khiến máy tính cơ học chỉ phù hợp với một số bài toán nhất định và không thể đáp ứng nhu cầu tính toán ngày càng tăng của khoa học và công nghiệp.

Dù vậy, giá trị lớn nhất của máy tính cơ học không nằm ở hiệu năng mà ở tư duy mà chúng mang lại. Lần đầu tiên trong lịch sử, con người nhìn nhận tính toán như một hoạt động có thể được chuẩn hóa, tự động hóa và tách khỏi năng lực cá nhân. Máy móc không còn chỉ là công cụ khuếch đại sức lao động vật lý, mà trở thành phương tiện xử lý thông tin.

Chính tư duy này đã tạo tiền đề trực tiếp cho sự chuyển đổi sang máy tính điện tử sau này. Khi các linh kiện điện cho phép thay thế chuyển động cơ học bằng tín hiệu điện, những nguyên lý đã được hình thành từ thời máy tính cơ học nhanh chóng được kế thừa và phát triển. Do đó, máy tính cơ học giữ vai trò nền tảng, không phải vì thành tựu kỹ thuật vượt trội, mà vì đã định hình cách con người thiết kế và sử dụng hệ thống tính toán tự động.

\section{Quá trình chuyển đổi từ cơ học sang điện tử trong xử lý thông tin}

Mặc dù máy tính cơ học đã đặt nền móng quan trọng cho tư duy tự động hóa tính toán, các giới hạn vật lý của cơ cấu cơ khí sớm trở thành rào cản đối với sự phát triển tiếp theo. Khi khoa học và kỹ thuật bước sang giai đoạn mới vào cuối thế kỷ XIX và đầu thế kỷ XX, nhu cầu xử lý khối lượng dữ liệu lớn với tốc độ cao vượt xa khả năng của các hệ thống cơ học truyền thống. Điều này buộc giới nghiên cứu phải tìm kiếm một phương thức xử lý thông tin hoàn toàn khác.

Hạn chế cốt lõi của máy tính cơ học nằm ở chuyển động vật lý. Bánh răng, trục quay và các cơ cấu liên kết không chỉ chậm mà còn dễ mài mòn, gây sai số và đòi hỏi bảo trì thường xuyên. Khi số lượng phép toán tăng lên, hệ thống trở nên cồng kềnh và kém tin cậy. Trong khi đó, các lĩnh vực như vật lý, hóa học, thiên văn học và quân sự bắt đầu phát sinh những bài toán yêu cầu hàng nghìn, thậm chí hàng triệu phép tính lặp lại trong thời gian ngắn. Máy tính cơ học không còn đáp ứng được yêu cầu thực tiễn.

Sự phát triển của điện học và điện tử đã mở ra hướng đi mới. Thay vì sử dụng chuyển động cơ học để biểu diễn và xử lý thông tin, các nhà khoa học nhận ra rằng trạng thái điện có thể đảm nhiệm vai trò tương tự nhưng với tốc độ vượt trội. Dòng điện có thể bật hoặc tắt gần như tức thời, cho phép biểu diễn dữ liệu dưới dạng trạng thái nhị phân và thực hiện các phép toán logic nhanh hơn nhiều so với cơ khí.

Giai đoạn chuyển tiếp này không diễn ra đột ngột mà thông qua các hệ thống lai giữa cơ học và điện. Một số máy sử dụng rơ-le điện từ để thay thế một phần bánh răng cơ học. Rơ-le cho phép điều khiển mạch điện bằng tín hiệu điện khác, từ đó giảm số lượng bộ phận chuyển động và tăng độ tin cậy. Tuy nhiên, rơ-le vẫn có độ trễ nhất định và không hoàn toàn loại bỏ được các hạn chế về tốc độ.

Bước ngoặt thực sự xuất hiện khi linh kiện điện tử thuần túy được đưa vào sử dụng. Việc thay thế các thành phần cơ khí bằng linh kiện điện tử cho phép thiết kế các hệ thống xử lý thông tin hoàn toàn mới, trong đó phép tính được thực hiện thông qua các mạch logic thay vì chuyển động vật lý. Điều này làm thay đổi bản chất của máy tính: từ một cỗ máy cơ khí thực hiện phép tính đơn lẻ sang một hệ thống điện tử có khả năng xử lý dữ liệu liên tục và linh hoạt.

Sự chuyển đổi từ cơ học sang điện tử không chỉ là thay đổi về công nghệ, mà còn là thay đổi về tư duy thiết kế. Máy tính bắt đầu được nhìn nhận như một hệ thống xử lý thông tin tổng quát, có thể lập trình để giải quyết nhiều loại bài toán khác nhau. Khái niệm chương trình trở nên rõ ràng hơn, tách biệt với phần cứng, cho phép cùng một hệ thống máy tính có thể thực hiện nhiều nhiệm vụ khác nhau chỉ bằng cách thay đổi cách điều khiển.

Tác động của sự chuyển đổi này là rất sâu rộng. Tốc độ xử lý tăng lên nhiều bậc độ lớn, độ chính xác được cải thiện đáng kể, và khả năng mở rộng hệ thống trở nên khả thi. Quan trọng hơn, điện tử hóa xử lý thông tin đã tạo điều kiện cho việc chuẩn hóa và công nghiệp hóa máy tính, mở đường cho sự ra đời của các thế hệ máy tính điện tử đầu tiên.

Như vậy, quá trình chuyển đổi từ máy tính cơ học sang máy tính điện tử không chỉ giải quyết các hạn chế kỹ thuật tồn tại trước đó, mà còn đặt nền tảng cho cách tiếp cận hiện đại đối với xử lý thông tin. Đây là bước ngoặt mang tính quyết định, dẫn trực tiếp tới sự hình thành của máy tính điện tử thế hệ đầu và sự bùng nổ của công nghệ thông tin trong các giai đoạn tiếp theo.

\section{Đặc điểm kỹ thuật của máy tính điện tử thế hệ đầu}

Máy tính điện tử thế hệ đầu đánh dấu bước chuyển mang tính cách mạng trong lịch sử xử lý thông tin, khi các nguyên lý điện tử được áp dụng trực tiếp để thực hiện phép tính. Giai đoạn này chủ yếu diễn ra trong khoảng từ giữa thập niên 1940 đến cuối thập niên 1950, với các hệ thống được thiết kế nhằm giải quyết những bài toán mà máy tính cơ học và bán điện tử không còn đáp ứng được.

Đặc trưng kỹ thuật nổi bật nhất của máy tính điện tử thế hệ đầu là việc sử dụng đèn chân không làm linh kiện cơ bản để xây dựng các mạch logic. Đèn chân không cho phép khuếch đại và chuyển mạch tín hiệu điện với tốc độ cao hơn nhiều so với rơ-le điện từ. Nhờ đó, máy tính có thể thực hiện hàng nghìn phép toán mỗi giây, một con số vượt trội so với các hệ thống trước đó. Tuy nhiên, đèn chân không tiêu thụ nhiều năng lượng, sinh nhiệt lớn và có độ bền thấp, dẫn đến tỷ lệ hỏng hóc cao.

Về mặt cấu trúc, máy tính điện tử thế hệ đầu có kích thước rất lớn, thường chiếm toàn bộ một căn phòng hoặc thậm chí cả một tòa nhà. Hệ thống bao gồm hàng nghìn đến hàng chục nghìn đèn chân không, cùng với các mạch dây phức tạp để kết nối và điều khiển. Việc vận hành đòi hỏi điều kiện môi trường nghiêm ngặt, đặc biệt là làm mát và nguồn điện ổn định, khiến chi phí triển khai và duy trì ở mức rất cao.

Khả năng lưu trữ dữ liệu của các máy tính thế hệ đầu còn hạn chế. Bộ nhớ chính thường sử dụng các công nghệ sơ khai như ống thủy ngân, trống từ hoặc các dạng bộ nhớ điện tử thô sơ. Dung lượng nhỏ và tốc độ truy xuất chưa cao khiến việc tối ưu hóa chương trình trở thành yêu cầu bắt buộc. Mỗi byte bộ nhớ đều có giá trị, buộc các nhà lập trình phải hiểu sâu cấu trúc phần cứng để khai thác hiệu quả.

Về phương thức lập trình, máy tính điện tử thế hệ đầu chưa có khái niệm ngôn ngữ lập trình bậc cao. Chương trình thường được viết trực tiếp bằng ngôn ngữ máy hoặc thông qua bảng cắm dây và công tắc điều khiển. Điều này khiến việc lập trình trở nên phức tạp, tốn thời gian và dễ xảy ra sai sót. Mỗi thay đổi nhỏ trong bài toán có thể đòi hỏi cấu hình lại toàn bộ hệ thống, làm giảm tính linh hoạt trong sử dụng.

Dù tồn tại nhiều hạn chế, máy tính điện tử thế hệ đầu đã chứng minh được giá trị thực tiễn rõ rệt. Chúng có khả năng giải quyết các bài toán mà trước đây gần như không thể thực hiện trong thời gian chấp nhận được, đặc biệt trong các lĩnh vực khoa học, kỹ thuật và quân sự. Quan trọng hơn, các hệ thống này xác lập mô hình cơ bản của máy tính hiện đại: sử dụng linh kiện điện tử, xử lý dữ liệu nhị phân, và thực thi chương trình theo trình tự logic.

Tóm lại, máy tính điện tử thế hệ đầu không phải là giải pháp hoàn hảo về mặt kỹ thuật, nhưng chúng đóng vai trò nền tảng không thể thay thế. Những đặc điểm kỹ thuật của giai đoạn này vừa phản ánh giới hạn công nghệ đương thời, vừa mở đường cho các cải tiến mang tính đột phá ở những thế hệ máy tính tiếp theo.

\section{Vai trò của chiến tranh và nghiên cứu khoa học đối với CNTT}

Sự phát triển nhanh chóng của máy tính điện tử trong nửa đầu thế kỷ XX không thể tách rời bối cảnh chiến tranh và các chương trình nghiên cứu khoa học quy mô lớn. Trên thực tế, chính các yêu cầu cấp bách về quân sự và khoa học đã tạo ra động lực mạnh mẽ, cả về tài chính lẫn tổ chức, thúc đẩy sự ra đời và hoàn thiện của công nghệ máy tính.

Trong thời chiến, đặc biệt là các cuộc chiến tranh quy mô toàn cầu, việc xử lý thông tin nhanh và chính xác trở thành yếu tố sống còn. Các bài toán như tính quỹ đạo đạn pháo, dự báo đường bay, phân tích tín hiệu, hay giải mã thông tin liên lạc đều vượt quá khả năng tính toán thủ công. Nhu cầu này buộc các quốc gia phải tìm kiếm những giải pháp tính toán mới, có thể rút ngắn thời gian xử lý từ hàng tuần xuống còn vài giờ hoặc vài phút.

Chiến tranh cũng tạo ra môi trường chấp nhận rủi ro công nghệ cao. Những hệ thống máy tính điện tử đầu tiên có chi phí rất lớn, độ tin cậy chưa cao và chưa có tiền lệ sử dụng rộng rãi. Trong điều kiện hòa bình, các dự án như vậy khó có thể được phê duyệt. Tuy nhiên, trong bối cảnh chiến tranh, lợi thế chiến lược mà công nghệ mang lại vượt xa các rủi ro kỹ thuật, khiến chính phủ sẵn sàng đầu tư mạnh vào nghiên cứu và thử nghiệm.

Song song với động lực quân sự, nghiên cứu khoa học đóng vai trò then chốt trong việc hình thành nền tảng lý thuyết cho máy tính. Các lĩnh vực như toán học, vật lý và logic học cung cấp khung tư duy để mô hình hóa quá trình tính toán. Những nghiên cứu về thuật toán, logic nhị phân và cấu trúc điều khiển đã chuyển khái niệm máy tính từ một cỗ máy tính toán chuyên dụng sang một hệ thống xử lý thông tin tổng quát.

Một điểm quan trọng khác là sự kết nối giữa các nhà khoa học, kỹ sư và tổ chức nghiên cứu. Chiến tranh buộc các quốc gia phải huy động nguồn lực trí tuệ trên quy mô lớn, tập hợp các chuyên gia từ nhiều lĩnh vực khác nhau vào cùng một mục tiêu. Mô hình hợp tác liên ngành này tạo điều kiện cho sự giao thoa giữa lý thuyết và thực tiễn, giúp các ý tưởng khoa học nhanh chóng được hiện thực hóa thành hệ thống kỹ thuật.

Sau khi chiến tranh kết thúc, các thành tựu công nghệ không biến mất mà được chuyển hóa sang mục đích dân sự. Máy tính bắt đầu được ứng dụng trong nghiên cứu khoa học, quản lý hành chính và công nghiệp. Nhiều chương trình nghiên cứu ban đầu phục vụ quân sự trở thành nền tảng cho sự phát triển lâu dài của công nghệ thông tin, từ phương pháp tổ chức dữ liệu đến cách thiết kế hệ thống tính toán.

Tóm lại, chiến tranh và nghiên cứu khoa học không chỉ là bối cảnh lịch sử mà còn là động lực quyết định đối với sự hình thành của máy tính điện tử. Chính sự kết hợp giữa nhu cầu thực tiễn khắc nghiệt và nền tảng khoa học vững chắc đã thúc đẩy công nghệ máy tính vượt qua giai đoạn thử nghiệm, tiến tới vai trò trung tâm trong khoa học và xã hội hiện đại.

\section{Nền móng cho sự phát triển của ngành công nghiệp máy tính}

Từ những máy tính điện tử thế hệ đầu, một nền móng kỹ thuật và tư duy mới đã được hình thành, tạo điều kiện cho sự phát triển của ngành công nghiệp máy tính trong các thập niên tiếp theo. Dù còn nhiều hạn chế về kích thước, chi phí và độ tin cậy, các hệ thống ban đầu đã chứng minh rằng máy tính không chỉ là công cụ thử nghiệm trong phòng thí nghiệm, mà là phương tiện xử lý thông tin có giá trị ứng dụng rộng rãi.

Trước hết, các nguyên lý cốt lõi của máy tính hiện đại được xác lập trong giai đoạn này. Máy tính được định nghĩa như một hệ thống có thể lập trình, thực hiện chuỗi lệnh để xử lý dữ liệu theo trình tự logic. Khái niệm phân tách giữa phần cứng và chương trình bắt đầu hình thành, cho phép cùng một hệ thống máy tính có thể giải quyết nhiều bài toán khác nhau. Đây là bước tiến mang tính quyết định, giúp máy tính vượt khỏi vai trò thiết bị chuyên dụng để trở thành công cụ đa năng.

Bên cạnh đó, kinh nghiệm thiết kế và vận hành máy tính điện tử thế hệ đầu đã cung cấp bài học quan trọng cho việc cải tiến công nghệ. Những vấn đề về tiêu thụ năng lượng, độ bền linh kiện và khả năng mở rộng thúc đẩy các nhà nghiên cứu tìm kiếm giải pháp thay thế hiệu quả hơn. Chính từ nhu cầu khắc phục các hạn chế này, các thế hệ linh kiện mới và kiến trúc máy tính tiên tiến hơn đã ra đời, tạo động lực cho tiến trình đổi mới liên tục.

Sự xuất hiện của máy tính điện tử cũng làm thay đổi cách tổ chức sản xuất và nghiên cứu. Thay vì các dự án đơn lẻ mang tính thử nghiệm, máy tính dần được chuẩn hóa và sản xuất theo quy mô lớn. Quá trình này đánh dấu sự chuyển dịch từ nghiên cứu thuần túy sang hình thành một ngành công nghiệp, nơi máy tính được xem là sản phẩm có thể thương mại hóa. Các tổ chức và doanh nghiệp bắt đầu nhìn thấy tiềm năng kinh tế lâu dài của công nghệ xử lý thông tin.

Ngoài ra, máy tính điện tử đặt nền móng cho sự phát triển của đội ngũ chuyên môn mới. Các vai trò như kỹ sư phần cứng, lập trình viên và nhà phân tích hệ thống dần hình thành, kéo theo sự thay đổi trong đào tạo và nghiên cứu khoa học. Công nghệ thông tin bắt đầu trở thành một lĩnh vực độc lập, với hệ thống kiến thức, phương pháp và chuẩn mực riêng.

Tóm lại, từ những máy tính điện tử sơ khai, nền móng của ngành công nghiệp máy tính đã được xây dựng trên ba trụ cột chính: nguyên lý xử lý thông tin có thể lập trình, động lực cải tiến công nghệ liên tục và khả năng ứng dụng rộng rãi trong thực tiễn. Chính nền móng này đã tạo điều kiện cho sự bùng nổ của máy tính trong nửa sau thế kỷ XX và đặt cơ sở cho vai trò trung tâm của công nghệ thông tin trong xã hội hiện đại.
