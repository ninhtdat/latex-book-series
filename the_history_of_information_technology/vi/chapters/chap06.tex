\chapter{Mạng máy tính và Internet}

Sự phát triển của công nghệ thông tin không thể tách rời quá trình hình thành và mở rộng của mạng máy tính. Từ những hệ thống máy tính hoạt động độc lập, con người dần đối mặt với nhu cầu kết nối để chia sẻ tài nguyên, dữ liệu và phối hợp xử lý công việc. Chương này tập trung phân tích bối cảnh ra đời của mạng máy tính, các mô hình ban đầu và những động lực kỹ thuật – kinh tế đã dẫn tới sự xuất hiện của Internet, đặt nền móng cho xã hội số hiện đại.

\section{Nhu cầu kết nối và chia sẻ tài nguyên máy tính}

Trong giai đoạn đầu của lịch sử điện toán, máy tính là những hệ thống cồng kềnh, đắt đỏ và được sử dụng chủ yếu cho các bài toán tính toán khoa học hoặc quân sự. Mỗi máy tính hoạt động như một thực thể độc lập, phục vụ một nhóm người dùng hạn chế. Cách tiếp cận này nhanh chóng bộc lộ nhiều hạn chế khi nhu cầu sử dụng máy tính tăng lên cùng với sự mở rộng của các tổ chức và doanh nghiệp.

Vấn đề đầu tiên là chi phí. Việc đầu tư nhiều máy tính riêng lẻ để phục vụ các bộ phận khác nhau dẫn đến lãng phí tài nguyên, đặc biệt là năng lực xử lý và thiết bị ngoại vi như máy in, thiết bị lưu trữ. Không phải lúc nào các tài nguyên này cũng được sử dụng hết công suất, trong khi chi phí đầu tư và bảo trì vẫn rất cao. Điều này thúc đẩy nhu cầu chia sẻ tài nguyên giữa nhiều người dùng và nhiều hệ thống.

Vấn đề thứ hai là hiệu quả làm việc và cộng tác. Khi dữ liệu nằm rải rác trên các máy tính độc lập, việc trao đổi thông tin trở nên chậm chạp và dễ sai sót. Các quy trình xử lý dữ liệu thủ công, sao chép bằng phương tiện vật lý hoặc nhập liệu lặp lại làm giảm năng suất và tăng rủi ro mất mát thông tin. Nhu cầu kết nối các máy tính để chia sẻ dữ liệu theo thời gian gần thực trở thành yêu cầu cấp thiết, đặc biệt trong các lĩnh vực nghiên cứu, quản lý và kinh doanh.

Ngoài ra, sự phát triển của phần mềm cũng góp phần làm rõ nhu cầu kết nối. Nhiều ứng dụng đòi hỏi dữ liệu tập trung, khả năng truy cập đồng thời và phối hợp xử lý giữa nhiều người dùng. Điều này đặt ra yêu cầu về một hạ tầng cho phép các máy tính giao tiếp với nhau một cách tin cậy, có kiểm soát và hiệu quả.

Từ những nhu cầu trên, khái niệm chia sẻ tài nguyên máy tính dần hình thành. Chia sẻ tài nguyên không chỉ giới hạn ở phần cứng như bộ xử lý, bộ nhớ hay thiết bị ngoại vi, mà còn bao gồm dữ liệu, phần mềm và dịch vụ. Một hệ thống mạng cho phép nhiều người dùng truy cập vào cùng một tài nguyên, theo các cơ chế phân quyền và quản lý nhất định, từ đó tối ưu hóa chi phí và nâng cao hiệu quả vận hành.

Bên cạnh yếu tố kinh tế và kỹ thuật, yếu tố tổ chức và quản trị cũng đóng vai trò quan trọng. Các tổ chức lớn cần một cách thức thống nhất để quản lý thông tin, đảm bảo tính nhất quán và khả năng kiểm soát. Mạng máy tính trở thành công cụ hỗ trợ quản lý tập trung, giúp lãnh đạo và người vận hành có cái nhìn toàn diện hơn về hệ thống thông tin của mình.

Như vậy, nhu cầu kết nối và chia sẻ tài nguyên máy tính xuất phát từ nhiều nguyên nhân: chi phí, hiệu quả, cộng tác và quản trị. Đây chính là nền tảng tư duy dẫn tới sự ra đời của các mô hình mạng máy tính đầu tiên, mở đường cho những bước phát triển sâu rộng hơn trong lĩnh vực truyền thông dữ liệu và Internet sau này.

\section{Các mô hình mạng máy tính ban đầu}

Sau khi nhu cầu kết nối và chia sẻ tài nguyên máy tính trở nên rõ ràng, vấn đề tiếp theo đặt ra là tổ chức việc kết nối các hệ thống như thế nào cho phù hợp với khả năng công nghệ của từng giai đoạn. Các mô hình mạng máy tính ban đầu ra đời trong bối cảnh hạ tầng phần cứng còn hạn chế, tốc độ truyền dữ liệu thấp và chi phí triển khai cao. Tuy còn đơn giản so với các hệ thống hiện đại, những mô hình này đã đặt nền móng quan trọng cho sự phát triển của mạng máy tính và Internet sau này.

Mô hình đầu tiên có thể kể đến là mô hình điện toán tập trung. Trong mô hình này, một máy tính trung tâm có năng lực xử lý lớn đảm nhiệm toàn bộ việc tính toán và lưu trữ dữ liệu, trong khi các thiết bị đầu cuối chỉ đóng vai trò nhập liệu và hiển thị kết quả. Các thiết bị đầu cuối hầu như không có khả năng xử lý độc lập. Ưu điểm của mô hình này là dễ quản lý và kiểm soát dữ liệu, nhưng nhược điểm rõ ràng là chi phí cao, khả năng mở rộng kém và phụ thuộc hoàn toàn vào hệ thống trung tâm.

Khi công nghệ phần cứng phát triển, mô hình máy chủ – máy trạm (client–server) dần hình thành. Trong mô hình này, các máy trạm có khả năng xử lý độc lập ở mức nhất định, trong khi máy chủ đảm nhiệm vai trò cung cấp tài nguyên và dịch vụ dùng chung như lưu trữ dữ liệu, ứng dụng hoặc xác thực người dùng. Mô hình client–server giúp phân tán tải xử lý, cải thiện hiệu năng và tăng tính linh hoạt so với điện toán tập trung. Tuy nhiên, nó vẫn tồn tại điểm yếu là sự phụ thuộc vào máy chủ; khi máy chủ gặp sự cố, toàn bộ hệ thống có thể bị gián đoạn.

Song song với đó, mô hình mạng ngang hàng (peer-to-peer) cũng xuất hiện trong các môi trường quy mô nhỏ. Trong mô hình này, các máy tính kết nối với nhau một cách tương đối bình đẳng, mỗi nút mạng vừa có thể cung cấp vừa có thể sử dụng tài nguyên. Mô hình ngang hàng đơn giản, chi phí thấp và không yêu cầu hạ tầng máy chủ phức tạp. Tuy nhiên, việc quản lý, bảo mật và mở rộng trở nên khó khăn khi số lượng nút mạng tăng lên, khiến mô hình này khó đáp ứng nhu cầu của các tổ chức lớn.

Về mặt phạm vi kết nối, các mạng cục bộ (LAN) là hình thức phổ biến trong giai đoạn đầu. Mạng LAN cho phép kết nối các máy tính trong một khu vực địa lý hạn chế như phòng làm việc, tòa nhà hoặc khuôn viên cơ quan. Công nghệ truyền dẫn ban đầu chủ yếu dựa trên cáp đồng, với tốc độ và độ tin cậy còn hạn chế. Dù vậy, mạng LAN đã chứng minh hiệu quả rõ rệt trong việc chia sẻ tài nguyên và hỗ trợ làm việc nhóm.

Các mô hình mạng máy tính ban đầu tuy còn nhiều hạn chế về hiệu năng, độ tin cậy và khả năng mở rộng, nhưng chúng đã giải quyết được bài toán cốt lõi của thời kỳ đó: kết nối các hệ thống độc lập thành một môi trường làm việc chung. Quan trọng hơn, những mô hình này hình thành các khái niệm nền tảng như nút mạng, liên kết, dịch vụ và phân quyền truy cập. Đây chính là cơ sở lý thuyết và thực tiễn để các hệ thống mạng phức tạp hơn, đặc biệt là Internet, được xây dựng và phát triển trong các giai đoạn tiếp theo.

\section{Sự ra đời của Internet và các giao thức cốt lõi}

Sự phát triển của các mô hình mạng máy tính ban đầu đã giải quyết nhu cầu kết nối trong phạm vi hẹp, nhưng đồng thời cũng làm bộc lộ một giới hạn lớn: các mạng này thường hoạt động biệt lập, khó liên thông với nhau. Khi quy mô tổ chức mở rộng và nhu cầu trao đổi thông tin vượt ra ngoài phạm vi địa lý cục bộ, bài toán kết nối liên mạng trở thành thách thức trung tâm của ngành công nghệ thông tin.

Trong bối cảnh đó, ý tưởng về một mạng máy tính có khả năng kết nối nhiều mạng nhỏ khác nhau đã hình thành. Mục tiêu không chỉ là mở rộng phạm vi kết nối, mà còn đảm bảo hệ thống có thể tiếp tục hoạt động ngay cả khi một phần hạ tầng gặp sự cố. Điều này đặc biệt quan trọng trong các ứng dụng nghiên cứu và quốc phòng, nơi tính liên tục của thông tin đóng vai trò sống còn.

Giải pháp mang tính đột phá là nguyên lý truyền thông theo gói. Thay vì gửi toàn bộ dữ liệu như một khối liên tục, thông tin được chia thành các gói nhỏ, mỗi gói mang theo địa chỉ nguồn và đích. Các gói này có thể đi qua những tuyến đường khác nhau trong mạng và được ghép lại tại nơi nhận. Cách tiếp cận này giúp mạng linh hoạt hơn, tận dụng tốt tài nguyên truyền dẫn và có khả năng tự thích nghi khi một tuyến đường bị gián đoạn.

Trên nền tảng đó, bộ giao thức TCP/IP ra đời và nhanh chóng trở thành chuẩn kỹ thuật cho việc truyền thông giữa các mạng. Giao thức IP chịu trách nhiệm định địa chỉ và định tuyến các gói dữ liệu, cho phép dữ liệu đi từ nguồn đến đích qua nhiều mạng trung gian. Giao thức TCP đảm nhiệm việc kiểm soát truyền thông đầu cuối, đảm bảo dữ liệu được truyền đầy đủ, đúng thứ tự và có khả năng phát hiện cũng như xử lý lỗi. Sự phân tách rõ ràng vai trò giữa các giao thức này tạo nên một kiến trúc linh hoạt và dễ mở rộng.

Một điểm quan trọng của Internet là tính phi tập trung. Không tồn tại một trung tâm điều khiển duy nhất cho toàn bộ mạng. Thay vào đó, mỗi mạng con và mỗi nút mạng chịu trách nhiệm cho phần hoạt động của mình, tuân thủ các chuẩn giao thức chung. Chính đặc điểm này giúp Internet có khả năng mở rộng gần như vô hạn, khi bất kỳ mạng mới nào cũng có thể tham gia nếu tuân thủ các quy tắc kỹ thuật đã được thống nhất.

Khái niệm địa chỉ IP đóng vai trò then chốt trong việc định danh các thiết bị trên Internet. Mỗi thiết bị được gán một địa chỉ duy nhất, cho phép các gói dữ liệu xác định chính xác điểm đến. Cùng với đó, cơ chế định tuyến cho phép các bộ định tuyến lựa chọn đường đi tối ưu cho dữ liệu dựa trên trạng thái mạng tại từng thời điểm. Đây là nền tảng cho khả năng truyền thông toàn cầu của Internet.

Ban đầu, Internet chủ yếu phục vụ cho mục đích nghiên cứu và trao đổi học thuật. Tuy nhiên, nhờ thiết kế mở và khả năng mở rộng cao, Internet nhanh chóng thu hút sự tham gia của các tổ chức, doanh nghiệp và sau đó là người dùng cá nhân. Các giao thức cốt lõi không gắn chặt với một ứng dụng cụ thể, cho phép nhiều loại dịch vụ khác nhau được xây dựng phía trên, từ thư điện tử, truyền tệp đến các ứng dụng thời gian thực.

Như vậy, sự ra đời của Internet không phải là một sự kiện đơn lẻ, mà là kết quả của quá trình tích lũy nhu cầu và đổi mới kỹ thuật. Các giao thức cốt lõi như TCP/IP đã giải quyết bài toán liên mạng một cách tổng quát, tạo ra một hạ tầng truyền thông chung cho toàn cầu. Đây chính là bước ngoặt quan trọng, đặt nền móng cho World Wide Web và sự bùng nổ của các dịch vụ số trong các giai đoạn tiếp theo.

\section{World Wide Web và quá trình toàn cầu hóa thông tin}

Sau khi Internet hình thành và ổn định về mặt hạ tầng truyền thông, một vấn đề mới xuất hiện: làm thế nào để người dùng có thể truy cập, khai thác và chia sẻ thông tin trên Internet một cách thuận tiện và thống nhất. Các dịch vụ ban đầu như thư điện tử hay truyền tệp đòi hỏi kiến thức kỹ thuật nhất định, chưa phù hợp với số đông người dùng. Trong bối cảnh đó, World Wide Web ra đời như một lớp ứng dụng giúp Internet trở nên dễ tiếp cận và có tính phổ cập cao hơn.

World Wide Web không đồng nhất với Internet. Internet là hạ tầng mạng toàn cầu cho phép các thiết bị kết nối và trao đổi dữ liệu, trong khi Web là một hệ thống dịch vụ hoạt động trên hạ tầng đó. Web cung cấp một mô hình truy cập thông tin dựa trên các tài nguyên được liên kết với nhau, cho phép người dùng di chuyển giữa các nội dung một cách trực quan thông qua các liên kết siêu văn bản. Sự phân biệt này mang ý nghĩa quan trọng về mặt kiến trúc, giúp Internet có thể hỗ trợ nhiều loại ứng dụng khác nhau, không chỉ riêng Web.

Nền tảng kỹ thuật của World Wide Web dựa trên ba thành phần cốt lõi: ngôn ngữ đánh dấu để mô tả nội dung, giao thức truyền thông để trao đổi dữ liệu và phần mềm phía người dùng để hiển thị thông tin. Ngôn ngữ HTML cho phép mô tả cấu trúc và nội dung của tài liệu một cách thống nhất, trong khi giao thức HTTP quy định cách thức yêu cầu và phản hồi dữ liệu giữa máy khách và máy chủ. Trình duyệt web đóng vai trò trung gian, giúp người dùng tương tác với các tài nguyên trên Internet mà không cần quan tâm đến chi tiết kỹ thuật bên dưới.

Một đặc điểm quan trọng của Web là cơ chế liên kết. Thông qua các siêu liên kết, thông tin không còn tồn tại như những tài liệu độc lập mà trở thành một mạng lưới tri thức có thể mở rộng liên tục. Người dùng có thể dễ dàng truy cập từ một trang thông tin sang các nguồn liên quan khác, vượt qua rào cản địa lý và tổ chức. Điều này làm thay đổi căn bản cách thức con người tiếp cận và tiêu thụ thông tin.

World Wide Web cũng đóng vai trò trung tâm trong quá trình toàn cầu hóa thông tin. Nhờ Web, thông tin có thể được công bố và tiếp cận gần như tức thời trên phạm vi toàn thế giới. Các tổ chức, doanh nghiệp và cá nhân có thể truyền tải thông điệp của mình đến một lượng lớn người dùng mà không cần hạ tầng truyền thông truyền thống phức tạp. Chi phí xuất bản và phân phối thông tin giảm mạnh, tạo điều kiện cho sự đa dạng hóa nguồn nội dung.

Sự phổ biến của Web còn thúc đẩy quá trình chuẩn hóa và chia sẻ tri thức. Các tiêu chuẩn mở giúp nội dung có thể được truy cập trên nhiều nền tảng và thiết bị khác nhau. Đồng thời, Web tạo ra môi trường thuận lợi cho hợp tác quốc tế trong nghiên cứu, giáo dục và kinh doanh, khi ranh giới về không gian và thời gian dần bị xóa nhòa.

Tuy nhiên, quá trình toàn cầu hóa thông tin cũng đặt ra những thách thức mới. Sự gia tăng nhanh chóng của nội dung số làm nảy sinh các vấn đề về độ tin cậy, bản quyền và kiểm soát thông tin. Dù vậy, không thể phủ nhận rằng World Wide Web đã trở thành yếu tố then chốt trong việc biến Internet từ một hạ tầng kỹ thuật thành một không gian thông tin toàn cầu, tác động sâu sắc đến mọi lĩnh vực của đời sống hiện đại.

\section{Tác động của Internet đến giao tiếp, kinh doanh và xã hội hiện đại}

Khi Internet và World Wide Web trở nên phổ biến, tác động của chúng nhanh chóng vượt ra ngoài phạm vi kỹ thuật, ảnh hưởng sâu rộng đến cách con người giao tiếp, tổ chức hoạt động kinh tế và vận hành xã hội. Internet không chỉ là một công cụ truyền thông mới, mà đã trở thành hạ tầng nền tảng của đời sống hiện đại.

Trong lĩnh vực giao tiếp, Internet làm thay đổi căn bản phương thức trao đổi thông tin giữa con người. Các hình thức liên lạc truyền thống vốn phụ thuộc vào không gian và thời gian dần được thay thế hoặc bổ sung bởi thư điện tử, nhắn tin tức thời, hội nghị trực tuyến và mạng xã hội. Thông tin có thể được truyền đi gần như tức thời với chi phí thấp, cho phép cá nhân và tổ chức duy trì kết nối liên tục trên phạm vi toàn cầu. Điều này làm mờ ranh giới địa lý, đồng thời thay đổi kỳ vọng của con người về tốc độ và tính sẵn sàng của thông tin.

Trong lĩnh vực kinh doanh, Internet tạo ra một môi trường hoàn toàn mới cho hoạt động thương mại và quản trị. Thương mại điện tử cho phép doanh nghiệp tiếp cận khách hàng trực tiếp mà không cần hệ thống phân phối vật lý truyền thống. Các mô hình kinh doanh dựa trên nền tảng số, dữ liệu và dịch vụ trực tuyến xuất hiện và phát triển nhanh chóng. Internet cũng hỗ trợ tối ưu hóa chuỗi cung ứng, quản lý quan hệ khách hàng và ra quyết định dựa trên dữ liệu thời gian thực. Đối với doanh nghiệp, khả năng khai thác hiệu quả Internet trở thành yếu tố cạnh tranh then chốt.

Ở cấp độ xã hội, Internet góp phần định hình lại cách thức tiếp cận tri thức và tham gia vào đời sống cộng đồng. Giáo dục trực tuyến, tài nguyên học tập mở và các cộng đồng tri thức số giúp mở rộng cơ hội tiếp cận kiến thức cho nhiều nhóm đối tượng. Đồng thời, Internet tạo ra không gian cho các hoạt động xã hội, văn hóa và chính trị, nơi cá nhân có thể bày tỏ quan điểm và tham gia thảo luận công khai.

Tuy nhiên, bên cạnh những lợi ích rõ rệt, Internet cũng đặt ra nhiều thách thức. Các vấn đề về an toàn thông tin, quyền riêng tư và bảo mật dữ liệu ngày càng trở nên nghiêm trọng khi lượng thông tin cá nhân được số hóa và lưu chuyển trên mạng tăng lên. Sự phụ thuộc quá mức vào Internet cũng có thể dẫn đến rủi ro về gián đoạn dịch vụ và tác động tiêu cực đến hành vi xã hội.

Nhìn tổng thể, Internet đã trở thành một phần không thể thiếu của hạ tầng kinh tế – xã hội hiện đại. Tác động của nó không chỉ nằm ở việc tăng tốc độ và quy mô giao tiếp, mà còn ở việc tái cấu trúc cách con người làm việc, kinh doanh và tương tác với nhau. Hiểu rõ những tác động này là cơ sở quan trọng để khai thác Internet một cách hiệu quả và bền vững trong tương lai.
